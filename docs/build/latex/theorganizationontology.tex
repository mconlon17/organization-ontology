%% Generated by Sphinx.
\def\sphinxdocclass{report}
\documentclass[letterpaper,10pt,english]{sphinxmanual}
\ifdefined\pdfpxdimen
   \let\sphinxpxdimen\pdfpxdimen\else\newdimen\sphinxpxdimen
\fi \sphinxpxdimen=.75bp\relax

\PassOptionsToPackage{warn}{textcomp}
\usepackage[utf8]{inputenc}
\ifdefined\DeclareUnicodeCharacter
% support both utf8 and utf8x syntaxes
  \ifdefined\DeclareUnicodeCharacterAsOptional
    \def\sphinxDUC#1{\DeclareUnicodeCharacter{"#1}}
  \else
    \let\sphinxDUC\DeclareUnicodeCharacter
  \fi
  \sphinxDUC{00A0}{\nobreakspace}
  \sphinxDUC{2500}{\sphinxunichar{2500}}
  \sphinxDUC{2502}{\sphinxunichar{2502}}
  \sphinxDUC{2514}{\sphinxunichar{2514}}
  \sphinxDUC{251C}{\sphinxunichar{251C}}
  \sphinxDUC{2572}{\textbackslash}
\fi
\usepackage{cmap}
\usepackage[T1]{fontenc}
\usepackage{amsmath,amssymb,amstext}
\usepackage{babel}



\usepackage{times}
\expandafter\ifx\csname T@LGR\endcsname\relax
\else
% LGR was declared as font encoding
  \substitutefont{LGR}{\rmdefault}{cmr}
  \substitutefont{LGR}{\sfdefault}{cmss}
  \substitutefont{LGR}{\ttdefault}{cmtt}
\fi
\expandafter\ifx\csname T@X2\endcsname\relax
  \expandafter\ifx\csname T@T2A\endcsname\relax
  \else
  % T2A was declared as font encoding
    \substitutefont{T2A}{\rmdefault}{cmr}
    \substitutefont{T2A}{\sfdefault}{cmss}
    \substitutefont{T2A}{\ttdefault}{cmtt}
  \fi
\else
% X2 was declared as font encoding
  \substitutefont{X2}{\rmdefault}{cmr}
  \substitutefont{X2}{\sfdefault}{cmss}
  \substitutefont{X2}{\ttdefault}{cmtt}
\fi


\usepackage[Bjarne]{fncychap}
\usepackage{sphinx}

\fvset{fontsize=\small}
\usepackage{geometry}


% Include hyperref last.
\usepackage{hyperref}
% Fix anchor placement for figures with captions.
\usepackage{hypcap}% it must be loaded after hyperref.
% Set up styles of URL: it should be placed after hyperref.
\urlstyle{same}

\addto\captionsenglish{\renewcommand{\contentsname}{Contents}}

\usepackage{sphinxmessages}
\setcounter{tocdepth}{0}



\title{The Organization Ontology}
\date{Jun 06, 2021}
\release{}
\author{The VIVO Ontology Interest Group}
\newcommand{\sphinxlogo}{\vbox{}}
\renewcommand{\releasename}{}
\makeindex
\begin{document}

\pagestyle{empty}
\sphinxmaketitle
\pagestyle{plain}
\sphinxtableofcontents
\pagestyle{normal}
\phantomsection\label{\detokenize{index::doc}}


\sphinxAtStartPar
The Organization Ontology (ORG) is an ontology for
representing organizations of all kinds.

\sphinxAtStartPar
The Organization Ontology uses \sphinxhref{http://www.ontobee.org/ontology/BFO}{Basic Formal Ontology (BFO)} as an upper level ontology, and conforms to
\sphinxhref{http://www.obofoundry.org/principles/fp-000-summary.html}{Open Biomedical Ontologies (OBO) Principles} for constructing interoperable ontologies.

\sphinxAtStartPar
The Organization
Ontology is a one of several ontologies developed for the representation of
scholarship by the \sphinxhref{https://vivoweb.org}{VIVO Project}. The Organization Ontology is
not limited to the representation of
scholarship \textendash{} it can be used to represent organizations in any setting.

\sphinxAtStartPar
In {\hyperref[\detokenize{glossary:glossary}]{\sphinxcrossref{\DUrole{std,std-ref}{VIVO 1}}}}, organizational representation was part of the VIVO
ontology.  In the new VIVO ontology, organizational
representation has been removed in favor of this new Organizational Ontology.  The
Organizational Ontology
is independent of VIVO and
can be used in any setting where information about organizations needs to be
represented.

\sphinxAtStartPar
An organization is a group of people with a purpose. It is not merely a group of
people \textendash{} that would be a collection of people, not an organization. The purpose
may be explicit or implicit. Organizations may be legally constituted or informal.
Organizations may be parts of other organizations.

\sphinxAtStartPar
See {\hyperref[\detokenize{organizations:organizations}]{\sphinxcrossref{\DUrole{std,std-ref}{Organizations}}}} for subsumption and subtypes.

\sphinxAtStartPar
The Organization Ontology is designed to insure it can represent
\sphinxhref{http://ror.org}{Research Organization Registry} data. ROR is a curated, CC0
collection of facts regarding over 97,000 research organizations in the world. The
Organization Ontology can represent these facts for use in graph\sphinxhyphen{}based systems such
as VIVO.

\sphinxAtStartPar
See {\hyperref[\detokenize{domain-definition:domain-definition}]{\sphinxcrossref{\DUrole{std,std-ref}{the domain definition}}}} for an extended defintion with
competency questions,
and consequences.

\sphinxAtStartPar
Tools are provided with the Organization Ontology for retrieving ROR data and
creating individuals with assertions as RDF triples using the Organization Ontology.
\phantomsection\label{\detokenize{domain-definition:domain-definition}}
\index{domain definition@\spxentry{domain definition}}\ignorespaces 

\chapter{Organization Ontology Domain Definition}
\label{\detokenize{domain-definition:organization-ontology-domain-definition}}\label{\detokenize{domain-definition:index-0}}\label{\detokenize{domain-definition::doc}}
\sphinxAtStartPar
The Organization Ontology is used to represent basic facts regarding organizations,
their structure, functions, interests, locations, and
their associations with other organizations and people.

\index{competency questions@\spxentry{competency questions}}\ignorespaces 

\section{Ontology Competency Questions \sphinxfootnotemark[1]}
\label{\detokenize{domain-definition:ontology-competency-questions-1}}\label{\detokenize{domain-definition:index-1}}%
\begin{footnotetext}[1]\phantomsection\label{\thesphinxscope.1}%
\sphinxAtStartFootnote
Competency questions are questions the ontology, including individuals
represented using the ontology, should be able to answer. They are indications of
the purpose of the ontology. Questions are likely to be nuanced.
%
\end{footnotetext}\ignorespaces \begin{enumerate}
\sphinxsetlistlabels{\arabic}{enumi}{enumii}{}{.}%
\item {} 
\sphinxAtStartPar
What organizations, in this region, have this interest?  Have this disposition? How
can I
get in contact with these organizations?

\item {} 
\sphinxAtStartPar
What is the organizational chart of this organization?  How many departments does
this university have?  How many branches does that company have?

\item {} 
\sphinxAtStartPar
Which organizations are members of this association?

\item {} 
\sphinxAtStartPar
What people have memberships, affiliations, or other roles in which organizations?

\item {} 
\sphinxAtStartPar
How can I learn more about this organization?  What is its home page, its Wikipedia
page?

\item {} 
\sphinxAtStartPar
How did this organization come to be and/or come to end?  What documents, people,
other organizations were involved in the creation, change, or end of this
organization?

\item {} 
\sphinxAtStartPar
How is this organization identified in registries of organizations?

\end{enumerate}


\section{Consequences and Observations}
\label{\detokenize{domain-definition:consequences-and-observations}}\begin{enumerate}
\sphinxsetlistlabels{\arabic}{enumi}{enumii}{}{.}%
\item {} 
\sphinxAtStartPar
Associating scholarly works, outputs, and projects with organizations is beyond the
scope of this ontology.  Other ontologies represent scholarly works.

\item {} 
\sphinxAtStartPar
Similarly, associations of organizations with performances and events are beyond
the scope of this ontology.  Other ontologies represented performances and
events.

\end{enumerate}


\chapter{Organizations}
\label{\detokenize{organizations:organizations}}\label{\detokenize{organizations:id1}}\label{\detokenize{organizations::doc}}
\sphinxAtStartPar
An organization is any collection of people with a purpose.  Organizations may be
formal/legal, as in the case of universities and corporations, or they may be informal,
as, for example, clubs.  Organizations may be parts of other organizations.


\section{Subsumption}
\label{\detokenize{organizations:subsumption}}
\sphinxAtStartPar
Organizations are {\hyperref[\detokenize{glossary:glossary}]{\sphinxcrossref{\DUrole{std,std-ref}{generically dependent continuants}}}} %
\begin{footnote}[1]\sphinxAtStartFootnote
By OBO\sphinxhyphen{}complaint, we mean the ORG ontology has been developed in accordance with
\sphinxhref{http://www.obofoundry.org/principles/fp-000-summary.html}{Open Biomedical Ontologies (OBO) Principles}.
%
\end{footnote} since they
depend
on the people and documents which define them. All the people and documents may be
replaced with
other people and documents, and the organization continues to exist.


\section{Overview}
\label{\detokenize{organizations:overview}}
\sphinxAtStartPar
{\hyperref[\detokenize{organizations:figure-1}]{\sphinxcrossref{Figure 1}}} shows the classes and properties used to represent organizations in ORG.
An overview of the classes and properties follows the figure.

\begin{figure}[htbp]
\centering
\capstart

\noindent\sphinxincludegraphics{{org-overview}.png}
\caption{Figure 1.  Representation of organizations.  The organization of interest is at the
center of the figure.  See notes below.}\label{\detokenize{organizations:id5}}\label{\detokenize{organizations:figure-1}}\end{figure}

\sphinxAtStartPar
At the center of the figure note that an organization has a name (rdfs:label).

\sphinxAtStartPar
An organization has a type.  In the figure, the type of the organization is
org:nonprofit.  See below for a further discussion of types.

\sphinxAtStartPar
Now proceeding clockwise from type:
\begin{itemize}
\item {} 
\sphinxAtStartPar
An organization may be denoted by one or more identifiers.  Identifiers are represented
using the Identifier Ontology (IDO).  Note that the identifier is an entity.  It exists
independently of the organization to denotes.

\item {} 
\sphinxAtStartPar
Orgs may be related to other orgs.  An org may be part of another organization.

\item {} 
\sphinxAtStartPar
An org may be affiliated with another organization.

\item {} 
\sphinxAtStartPar
An org may be denoted by a postal address.  See {\hyperref[\detokenize{addresses::doc}]{\sphinxcrossref{\DUrole{doc}{Addresses}}}} for
details.  Addresses have
properties that indicate how they are to be used.

\item {} 
\sphinxAtStartPar
An org may have a predecessor organization.  Organizations undergo change.  The
resulting
organization may be a new organization of a different type, different people, different
purpose.

\item {} 
\sphinxAtStartPar
An organization may be a member of another organization.

\item {} 
\sphinxAtStartPar
Organizations often have web sites.  Web sites are information content entities that are
about the organization.  Note that the web site is an entity that exists with or without
the organization it is about.

\item {} 
\sphinxAtStartPar
Organizations have one or more dispositions.  Dispositions identify the purpose of an
organization.  Dispositions of an organization may change over time.  See below for a
further discussion of dispositions.  A disposition is dependent the entity which
has the disposition.  In BFO, a disposition is a specifically dependent continuant,
dependent on the entity which has the specific disposition.

\item {} 
\sphinxAtStartPar
An org may occupy zero or more facilities, such as an office building, or university
campus. A facility
is typically a man\sphinxhyphen{}made structure attached to the ground.  As such, facilities have
geographical locations \textendash{} in cities, for example.  See {\hyperref[\detokenize{facilities::doc}]{\sphinxcrossref{\DUrole{doc}{Facilities}}}}
for more detail.

\item {} 
\sphinxAtStartPar
Organizations come into being as the result of founding processes which have associated
dates.  See {\hyperref[\detokenize{datetimes::doc}]{\sphinxcrossref{\DUrole{doc}{Dates and Times}}}} for
a further discussion of the representation of dates and times related to organizations.

\item {} 
\sphinxAtStartPar
Organizations may be denoted by one or more email addresses.  See {\hyperref[\detokenize{addresses::doc}]{\sphinxcrossref{\DUrole{doc}{Addresses}}}} for details.  As with postal addresses, email addresses may have
properties describing their purpose.

\end{itemize}


\section{Types}
\label{\detokenize{organizations:types}}
\sphinxAtStartPar
Organizations have one of the types in the table below. These are mutually exclusive.
An organization can
not be more than one type, just as an animal cannot be more than one species.

\sphinxAtStartPar
See {\hyperref[\detokenize{organizations:table-1}]{\sphinxcrossref{Table 1}}}.


\begin{savenotes}\sphinxattablestart
\centering
\sphinxcapstartof{table}
\sphinxthecaptionisattop
\sphinxcaption{Table 1 Types of Organizations}\label{\detokenize{organizations:id6}}\label{\detokenize{organizations:table-1}}
\sphinxaftertopcaption
\begin{tabulary}{\linewidth}[t]{|T|T|}
\hline
\sphinxstyletheadfamily 
\sphinxAtStartPar
Term ID \sphinxhyphen{} Label
&\sphinxstyletheadfamily 
\sphinxAtStartPar
Definition
\\
\hline
\sphinxAtStartPar
{\hyperref[\detokenize{doc-ORG_0000002::doc}]{\sphinxcrossref{\DUrole{doc}{ORG\_0000002 \sphinxhyphen{} government organization}}}}
&
\sphinxAtStartPar
An organization which is the body of persons that

\sphinxAtStartPar
constitutes the governing authority of a political

\sphinxAtStartPar
unit
\\
\hline
\sphinxAtStartPar
{\hyperref[\detokenize{doc-ORG_0000003::doc}]{\sphinxcrossref{\DUrole{doc}{ORG\_0000003 \sphinxhyphen{} company}}}}
&
\sphinxAtStartPar
A legal entity of associated persons created for a

\sphinxAtStartPar
specific purpose, typically commercial, in which

\sphinxAtStartPar
excess revenue may be distributed to the company’s

\sphinxAtStartPar
owners.
\\
\hline
\sphinxAtStartPar
{\hyperref[\detokenize{doc-ORG_0000004::doc}]{\sphinxcrossref{\DUrole{doc}{ORG\_0000004 \sphinxhyphen{} nonprofit organization}}}}
&
\sphinxAtStartPar
A legal entity of associated persons created for a

\sphinxAtStartPar
specific purpose, typically a mission, in which

\sphinxAtStartPar
excess revenue is reinvested to serve the entity’s

\sphinxAtStartPar
mission
\\
\hline
\sphinxAtStartPar
{\hyperref[\detokenize{doc-ORG_0000005::doc}]{\sphinxcrossref{\DUrole{doc}{ORG\_0000005 \sphinxhyphen{} informal organization}}}}
&
\sphinxAtStartPar
A group of people recognized as such by people

\sphinxAtStartPar
outside the group. Without legal standing.
\\
\hline
\sphinxAtStartPar
{\hyperref[\detokenize{doc-ORG_0000006::doc}]{\sphinxcrossref{\DUrole{doc}{ORG\_0000006 \sphinxhyphen{} organization part}}}}
&
\sphinxAtStartPar
An organization which exists as part of another

\sphinxAtStartPar
organization.  Implies a part\_of relationship to

\sphinxAtStartPar
another organization
\\
\hline
\end{tabulary}
\par
\sphinxattableend\end{savenotes}

\sphinxAtStartPar
{\hyperref[\detokenize{organizations:figure-2}]{\sphinxcrossref{\DUrole{std,std-ref}{Figure 2.  Subclasses of organization and subsumption hierarchy.  The subclasses are
mutually exclusive.}}}} shows the subsumption hierarchy for organization and its subclasses.

\begin{figure}[htbp]
\centering
\capstart

\noindent\sphinxincludegraphics{{org-types}.png}
\caption{Figure 2.  Subclasses of organization and subsumption hierarchy.  The subclasses are
mutually exclusive.}\label{\detokenize{organizations:id7}}\label{\detokenize{organizations:figure-2}}\end{figure}


\section{Dispositions}
\label{\detokenize{organizations:dispositions}}
\sphinxAtStartPar
Organizations have dispositions which indicate the purposes organizations have.  An
organization might have a disposition of \sphinxstyleemphasis{library} or \sphinxstyleemphasis{healthcare} or \sphinxstyleemphasis{military}.
Dispositions are shown in {\hyperref[\detokenize{organizations:table-2}]{\sphinxcrossref{\DUrole{std,std-ref}{Table 2 Dispositions}}}}  An organization may have any number
of dispositions.

\sphinxAtStartPar
See {\hyperref[\detokenize{organizations:table-2}]{\sphinxcrossref{Table 2}}}.


\begin{savenotes}\sphinxattablestart
\centering
\sphinxcapstartof{table}
\sphinxthecaptionisattop
\sphinxcaption{Table 2 Dispositions}\label{\detokenize{organizations:id8}}\label{\detokenize{organizations:table-2}}
\sphinxaftertopcaption
\begin{tabulary}{\linewidth}[t]{|T|T|}
\hline
\sphinxstyletheadfamily 
\sphinxAtStartPar
Term ID \sphinxhyphen{} Label
&\sphinxstyletheadfamily 
\sphinxAtStartPar
Definition
\\
\hline
\sphinxAtStartPar
{\hyperref[\detokenize{doc-ORG_0000007::doc}]{\sphinxcrossref{\DUrole{doc}{ORG\_0000007 \sphinxhyphen{} university disposition}}}}
&
\sphinxAtStartPar
A disposition to award academic degrees and

\sphinxAtStartPar
conduct research in a variety of academic

\sphinxAtStartPar
disciplines
\\
\hline
\sphinxAtStartPar
{\hyperref[\detokenize{doc-ORG_0000008::doc}]{\sphinxcrossref{\DUrole{doc}{ORG\_0000008 \sphinxhyphen{} association disposition}}}}
&
\sphinxAtStartPar
A disposition to organize organizations or

\sphinxAtStartPar
individuals along and industry or academic lines
\\
\hline
\sphinxAtStartPar
{\hyperref[\detokenize{doc-ORG_0000009::doc}]{\sphinxcrossref{\DUrole{doc}{ORG\_0000009 \sphinxhyphen{} consortium disposition}}}}
&
\sphinxAtStartPar
A disposition to organize organizations along

\sphinxAtStartPar
industry or academic lines
\\
\hline
\sphinxAtStartPar
{\hyperref[\detokenize{doc-ORG_0000010::doc}]{\sphinxcrossref{\DUrole{doc}{ORG\_0000010 \sphinxhyphen{} service provider disposition}}}}
&
\sphinxAtStartPar
A disposition to provide service with or without a

\sphinxAtStartPar
fee
\\
\hline
\sphinxAtStartPar
{\hyperref[\detokenize{doc-ORG_0000012::doc}]{\sphinxcrossref{\DUrole{doc}{ORG\_0000012 \sphinxhyphen{} extension provider disposition}}}}
&
\sphinxAtStartPar
A disposition to provide extension services,

\sphinxAtStartPar
typically in agriculture.  Extension provides

\sphinxAtStartPar
access to university research findings and advice

\sphinxAtStartPar
to agriculturalists.
\\
\hline
\sphinxAtStartPar
{\hyperref[\detokenize{doc-ORG_0000013::doc}]{\sphinxcrossref{\DUrole{doc}{ORG\_0000013 \sphinxhyphen{} technology transfer disposition}}}}
&
\sphinxAtStartPar
A disposition to create licenses for intellectual

\sphinxAtStartPar
property for use by these beyond the creators
\\
\hline
\sphinxAtStartPar
{\hyperref[\detokenize{doc-ORG_0000014::doc}]{\sphinxcrossref{\DUrole{doc}{ORG\_0000014 \sphinxhyphen{} philanthropy disposition}}}}
&
\sphinxAtStartPar
A disposition to donate charitable causes,

\sphinxAtStartPar
sometimes in the form of grants involving

\sphinxAtStartPar
contracts regarding the use of the donated funds

\sphinxAtStartPar
or effort.
\\
\hline
\sphinxAtStartPar
{\hyperref[\detokenize{doc-ORG_0000015::doc}]{\sphinxcrossref{\DUrole{doc}{ORG\_0000015 \sphinxhyphen{} funding disposition}}}}
&
\sphinxAtStartPar
A disposition to fund proposals, often is response

\sphinxAtStartPar
to a call for proposals by the entity with the

\sphinxAtStartPar
funding disposition
\\
\hline
\sphinxAtStartPar
{\hyperref[\detokenize{doc-ORG_0000017::doc}]{\sphinxcrossref{\DUrole{doc}{ORG\_0000017 \sphinxhyphen{} hospital service provider disposition}}}}
&
\sphinxAtStartPar
A disposition to provide hospital\sphinxhyphen{}based health

\sphinxAtStartPar
care services to humans
\\
\hline
\sphinxAtStartPar
{\hyperref[\detokenize{doc-ORG_0000018::doc}]{\sphinxcrossref{\DUrole{doc}{ORG\_0000018 \sphinxhyphen{} archive disposition}}}}
&
\sphinxAtStartPar
A disposition to collect, store, and provide

\sphinxAtStartPar
access to inanimate material entities, and/or

\sphinxAtStartPar
information content entitites
\\
\hline
\sphinxAtStartPar
{\hyperref[\detokenize{doc-ORG_0000019::doc}]{\sphinxcrossref{\DUrole{doc}{ORG\_0000019 \sphinxhyphen{} museum disposition}}}}
&
\sphinxAtStartPar
A disposition to collect, store, and provide

\sphinxAtStartPar
access to inanimate material entities in a

\sphinxAtStartPar
facility
\\
\hline
\sphinxAtStartPar
{\hyperref[\detokenize{doc-ORG_0000020::doc}]{\sphinxcrossref{\DUrole{doc}{ORG\_0000020 \sphinxhyphen{} gallery disposition}}}}
&
\sphinxAtStartPar
A disposition to display collected works from an

\sphinxAtStartPar
archive
\\
\hline
\sphinxAtStartPar
{\hyperref[\detokenize{doc-ORG_0000021::doc}]{\sphinxcrossref{\DUrole{doc}{ORG\_0000021 \sphinxhyphen{} publishing disposition}}}}
&
\sphinxAtStartPar
A disposition to publish information content

\sphinxAtStartPar
entities
\\
\hline
\sphinxAtStartPar
{\hyperref[\detokenize{doc-ORG_0000022::doc}]{\sphinxcrossref{\DUrole{doc}{ORG\_0000022 \sphinxhyphen{} research disposition}}}}
&
\sphinxAtStartPar
A disposition to conduct research
\\
\hline
\sphinxAtStartPar
{\hyperref[\detokenize{doc-ORG_0000023::doc}]{\sphinxcrossref{\DUrole{doc}{ORG\_0000023 \sphinxhyphen{} education disposition}}}}
&
\sphinxAtStartPar
A disposition to teach, and provide experiential

\sphinxAtStartPar
opprtunities for students
\\
\hline
\sphinxAtStartPar
{\hyperref[\detokenize{doc-ORG_0000024::doc}]{\sphinxcrossref{\DUrole{doc}{ORG\_0000024 \sphinxhyphen{} training disposition}}}}
&
\sphinxAtStartPar
A disposition to train, and provide experiential

\sphinxAtStartPar
opportunities for trainees
\\
\hline
\sphinxAtStartPar
{\hyperref[\detokenize{doc-ORG_0000025::doc}]{\sphinxcrossref{\DUrole{doc}{ORG\_0000025 \sphinxhyphen{} research administration disposition}}}}
&
\sphinxAtStartPar
A disposition to provide resources and oversight

\sphinxAtStartPar
for those conducting research
\\
\hline
\sphinxAtStartPar
{\hyperref[\detokenize{doc-ORG_0000026::doc}]{\sphinxcrossref{\DUrole{doc}{ORG\_0000026 \sphinxhyphen{} library disposition}}}}
&
\sphinxAtStartPar
A disposition to provide library services
\\
\hline
\sphinxAtStartPar
{\hyperref[\detokenize{doc-ORG_0000027::doc}]{\sphinxcrossref{\DUrole{doc}{ORG\_0000027 \sphinxhyphen{} commerce disposition}}}}
&
\sphinxAtStartPar
A disposition to sell things
\\
\hline
\sphinxAtStartPar
{\hyperref[\detokenize{doc-ORG_0000028::doc}]{\sphinxcrossref{\DUrole{doc}{ORG\_0000028 \sphinxhyphen{} military disposition}}}}
&
\sphinxAtStartPar
A disposition to engage in warfare
\\
\hline
\sphinxAtStartPar
{\hyperref[\detokenize{doc-ORG_0000029::doc}]{\sphinxcrossref{\DUrole{doc}{ORG\_0000029 \sphinxhyphen{} religious disposition}}}}
&
\sphinxAtStartPar
A disposition to engage in matters of spirtuality

\sphinxAtStartPar
and faith
\\
\hline
\sphinxAtStartPar
{\hyperref[\detokenize{doc-ORG_0000030::doc}]{\sphinxcrossref{\DUrole{doc}{ORG\_0000030 \sphinxhyphen{} governing disposition}}}}
&
\sphinxAtStartPar
A disposition to provide governance
\\
\hline
\sphinxAtStartPar
{\hyperref[\detokenize{doc-ORG_0000031::doc}]{\sphinxcrossref{\DUrole{doc}{ORG\_0000031 \sphinxhyphen{} manufacturing disposition}}}}
&
\sphinxAtStartPar
A dispositon to construct material entities
\\
\hline
\sphinxAtStartPar
{\hyperref[\detokenize{doc-ORG_0000032::doc}]{\sphinxcrossref{\DUrole{doc}{ORG\_0000032 \sphinxhyphen{} project team disposition}}}}
&
\sphinxAtStartPar
A disposition to execute and finish a project.
\\
\hline
\sphinxAtStartPar
{\hyperref[\detokenize{doc-ORG_0000033::doc}]{\sphinxcrossref{\DUrole{doc}{ORG\_0000033 \sphinxhyphen{} sports disposition}}}}
&
\sphinxAtStartPar
A disposition to engage in sports activites,

\sphinxAtStartPar
typically competitive.
\\
\hline
\sphinxAtStartPar
{\hyperref[\detokenize{doc-ORG_0000079::doc}]{\sphinxcrossref{\DUrole{doc}{ORG\_0000079 \sphinxhyphen{} airline disposition}}}}
&
\sphinxAtStartPar
The disposition of an organization that operates

\sphinxAtStartPar
airplanes carrying frieght or passengers
\\
\hline
\sphinxAtStartPar
{\hyperref[\detokenize{doc-ORG_0000080::doc}]{\sphinxcrossref{\DUrole{doc}{ORG\_0000080 \sphinxhyphen{} media disposition}}}}
&
\sphinxAtStartPar
The disposition of an organization that creates,

\sphinxAtStartPar
transmits, and/or licenses live or recorded

\sphinxAtStartPar
material for viewing by others
\\
\hline
\sphinxAtStartPar
{\hyperref[\detokenize{doc-ORG_0000081::doc}]{\sphinxcrossref{\DUrole{doc}{ORG\_0000081 \sphinxhyphen{} performing disposition}}}}
&
\sphinxAtStartPar
The disposition of an organization to perform live

\sphinxAtStartPar
or recorded music, theatre, or dance
\\
\hline
\sphinxAtStartPar
{\hyperref[\detokenize{doc-ORG_0000082::doc}]{\sphinxcrossref{\DUrole{doc}{ORG\_0000082 \sphinxhyphen{} labor union disposition}}}}
&
\sphinxAtStartPar
The disposition of an organization to organize

\sphinxAtStartPar
workers for the purpose of negotiations with

\sphinxAtStartPar
employers of the workers
\\
\hline
\end{tabulary}
\par
\sphinxattableend\end{savenotes}


\section{Examples}
\label{\detokenize{organizations:examples}}
\begin{sphinxShadowBox}
\sphinxstyletopictitle{Duke University}

\sphinxAtStartPar
Duke is a nonprofit organization
with a dispositions of university, education, and research

\sphinxAtStartPar
Duke has an organization part, Duke Health, which has a disposition of
healthcare.  Duke Health has an organizational part, Duke University Hospital,
which has a disposition of hospital.
\end{sphinxShadowBox}

\begin{sphinxShadowBox}
\sphinxstyletopictitle{United States Navy}

\sphinxAtStartPar
The United States Navy is an organization part of the US Department of Defense
with disposition of military.
\end{sphinxShadowBox}

\begin{sphinxShadowBox}
\sphinxstyletopictitle{BASF}

\sphinxAtStartPar
BASF is a company with a disposition of commerce.
\end{sphinxShadowBox}


\section{Qualities}
\label{\detokenize{organizations:qualities}}
\sphinxAtStartPar
Qualities are realized entities that do not require a process, and can be added or
subtracted from an entity without altering the entity.  Use the {\hyperref[\detokenize{doc-RO_0000086::doc}]{\sphinxcrossref{\DUrole{doc}{has quality}}}} property to associate a quality with an entity.

\sphinxAtStartPar
For example, to assert organization x is student\sphinxhyphen{}led:

\begin{sphinxVerbatim}[commandchars=\\\{\}]
\PYG{n}{x} \PYG{n}{has\PYGZus{}quality} \PYG{n}{y}
\PYG{n}{y} \PYG{n}{a} \PYG{n}{student\PYGZus{}led\PYGZus{}organization\PYGZus{}quality}
\end{sphinxVerbatim}

\sphinxAtStartPar
See {\hyperref[\detokenize{organizations:table-3}]{\sphinxcrossref{Table 3}}}.


\begin{savenotes}\sphinxattablestart
\centering
\sphinxcapstartof{table}
\sphinxthecaptionisattop
\sphinxcaption{Table 3 Qualities}\label{\detokenize{organizations:id9}}\label{\detokenize{organizations:table-3}}
\sphinxaftertopcaption
\begin{tabulary}{\linewidth}[t]{|T|T|}
\hline
\sphinxstyletheadfamily 
\sphinxAtStartPar
Term ID \sphinxhyphen{} Label
&\sphinxstyletheadfamily 
\sphinxAtStartPar
Definition
\\
\hline
\sphinxAtStartPar
{\hyperref[\detokenize{doc-ORG_0000034::doc}]{\sphinxcrossref{\DUrole{doc}{ORG\_0000034 \sphinxhyphen{} information address quality}}}}
&
\sphinxAtStartPar
A quality of an address to be used for information

\sphinxAtStartPar
inquiries
\\
\hline
\sphinxAtStartPar
{\hyperref[\detokenize{doc-ORG_0000035::doc}]{\sphinxcrossref{\DUrole{doc}{ORG\_0000035 \sphinxhyphen{} billing address quality}}}}
&
\sphinxAtStartPar
A quality of an address to be used to receive

\sphinxAtStartPar
bills
\\
\hline
\sphinxAtStartPar
{\hyperref[\detokenize{doc-ORG_0000036::doc}]{\sphinxcrossref{\DUrole{doc}{ORG\_0000036 \sphinxhyphen{} shipping address quality}}}}
&
\sphinxAtStartPar
A quality of an address to be used to receive

\sphinxAtStartPar
shipped goods
\\
\hline
\sphinxAtStartPar
{\hyperref[\detokenize{doc-ORG_0000037::doc}]{\sphinxcrossref{\DUrole{doc}{ORG\_0000037 \sphinxhyphen{} preferred address quality}}}}
&
\sphinxAtStartPar
A quality of an address to be displayed in most

\sphinxAtStartPar
settings
\\
\hline
\sphinxAtStartPar
{\hyperref[\detokenize{doc-ORG_0000038::doc}]{\sphinxcrossref{\DUrole{doc}{ORG\_0000038 \sphinxhyphen{} homepage quality}}}}
&
\sphinxAtStartPar
A quality to be the primary website for an entity.
\\
\hline
\sphinxAtStartPar
{\hyperref[\detokenize{doc-ORG_0000039::doc}]{\sphinxcrossref{\DUrole{doc}{ORG\_0000039 \sphinxhyphen{} wikipedia quality}}}}
&
\sphinxAtStartPar
A quality to be the webpage within WikiPedia

\sphinxAtStartPar
regarding the entity
\\
\hline
\sphinxAtStartPar
{\hyperref[\detokenize{doc-ORG_0000063::doc}]{\sphinxcrossref{\DUrole{doc}{ORG\_0000063 \sphinxhyphen{} student led organization quality}}}}
&
\sphinxAtStartPar
The quality of an organization that is led by a

\sphinxAtStartPar
student
\\
\hline
\sphinxAtStartPar
{\hyperref[\detokenize{doc-ORG_0000064::doc}]{\sphinxcrossref{\DUrole{doc}{ORG\_0000064 \sphinxhyphen{} woman led organization quality}}}}
&
\sphinxAtStartPar
The quality of an organization that is led by a

\sphinxAtStartPar
woman
\\
\hline
\sphinxAtStartPar
{\hyperref[\detokenize{doc-ORG_0000065::doc}]{\sphinxcrossref{\DUrole{doc}{ORG\_0000065 \sphinxhyphen{} minority led organization quality}}}}
&
\sphinxAtStartPar
The quality of an organiztion that is led by a

\sphinxAtStartPar
designated minority
\\
\hline
\sphinxAtStartPar
{\hyperref[\detokenize{doc-ORG_0000066::doc}]{\sphinxcrossref{\DUrole{doc}{ORG\_0000066 \sphinxhyphen{} registered address quality}}}}
&
\sphinxAtStartPar
The quality of a location that is the

\sphinxAtStartPar
legal/registered location for the organization
\\
\hline
\sphinxAtStartPar
{\hyperref[\detokenize{doc-ORG_0000067::doc}]{\sphinxcrossref{\DUrole{doc}{ORG\_0000067 \sphinxhyphen{} primary address quality}}}}
&
\sphinxAtStartPar
The quality of a location that is the

\sphinxAtStartPar
primary/preferred location for the organization
\\
\hline
\end{tabulary}
\par
\sphinxattableend\end{savenotes}
\phantomsection\label{\detokenize{identifiers:identifiers}}
\index{identifiers@\spxentry{identifiers}}\ignorespaces 

\chapter{Identifiers}
\label{\detokenize{identifiers:index-0}}\label{\detokenize{identifiers:id1}}\label{\detokenize{identifiers::doc}}
\sphinxAtStartPar
An identifier is a string or symbol, assigned to an organization by a {\hyperref[\detokenize{glossary:glossary}]{\sphinxcrossref{\DUrole{std,std-ref}{dubbing
process}}}}

\sphinxAtStartPar
The Organization Ontology uses \sphinxhref{https://github.com/mconlon17/identifier-ontology}{The Identifier Ontology} to represent identifiers for
organizations.

\sphinxAtStartPar
The Identifier Ontology is a small set of terms in \sphinxhref{http://www.ontobee.org/ontology/IAO}{Information Artifact Ontology (IAO)} to represent identifiers,
and in particular, persistent identifiers, often called PIDs.  Persistent identifiers
are maintained by one or more maintainers interested in the persistence of the
identifier and its assignment to an entity over time.

\sphinxAtStartPar
The table below lists identifiers available in the Organization Ontology \sphinxstepexplicit %
\begin{footnote}[1]\phantomsection\label{\thesphinxscope.1}%
\sphinxAtStartFootnote
If a needed organization identifier is not in the table, please open a
Github issue with the name and source of the identifier for inclusion in subsequent
releases of the Organization Ontology.
%
\end{footnote}

\sphinxAtStartPar
See {\hyperref[\detokenize{identifiers:table-4}]{\sphinxcrossref{Table 4}}}.


\begin{savenotes}\sphinxattablestart
\centering
\sphinxcapstartof{table}
\sphinxthecaptionisattop
\sphinxcaption{Table 4 Identifiers}\label{\detokenize{identifiers:id4}}\label{\detokenize{identifiers:table-4}}
\sphinxaftertopcaption
\begin{tabulary}{\linewidth}[t]{|T|T|}
\hline
\sphinxstyletheadfamily 
\sphinxAtStartPar
Term ID \sphinxhyphen{} Label
&\sphinxstyletheadfamily 
\sphinxAtStartPar
Definition
\\
\hline
\sphinxAtStartPar
{\hyperref[\detokenize{doc-IAO_0022003::doc}]{\sphinxcrossref{\DUrole{doc}{IAO\_0022003 \sphinxhyphen{} crossref funder identifier}}}}
&
\sphinxAtStartPar
An identifier assigned by CrossRef to an

\sphinxAtStartPar
organization which has funded a project resulting

\sphinxAtStartPar
in a published work
\\
\hline
\sphinxAtStartPar
{\hyperref[\detokenize{doc-IAO_0022006::doc}]{\sphinxcrossref{\DUrole{doc}{IAO\_0022006 \sphinxhyphen{} dbpedia identifier}}}}
&
\sphinxAtStartPar
A URL used by DBpedia to identify an entity
\\
\hline
\sphinxAtStartPar
{\hyperref[\detokenize{doc-IAO_0022010::doc}]{\sphinxcrossref{\DUrole{doc}{IAO\_0022010 \sphinxhyphen{} global research organization identifier}}}}
&
\sphinxAtStartPar
An identifier assigned and managed by Digital

\sphinxAtStartPar
Science for the purpose of denoting research

\sphinxAtStartPar
organizations
\\
\hline
\sphinxAtStartPar
{\hyperref[\detokenize{doc-IAO_0022014::doc}]{\sphinxcrossref{\DUrole{doc}{IAO\_0022014 \sphinxhyphen{} international standard name identifier}}}}
&
\sphinxAtStartPar
An identifier for persons and organizations which

\sphinxAtStartPar
may be assigned by matching algorithms based on

\sphinxAtStartPar
records provided by publishers
\\
\hline
\sphinxAtStartPar
{\hyperref[\detokenize{doc-IAO_0022022::doc}]{\sphinxcrossref{\DUrole{doc}{IAO\_0022022 \sphinxhyphen{} research organization registry identifier}}}}
&
\sphinxAtStartPar
An identifier assigned by ROR to research

\sphinxAtStartPar
organizations in the world
\\
\hline
\sphinxAtStartPar
{\hyperref[\detokenize{doc-IAO_0022027::doc}]{\sphinxcrossref{\DUrole{doc}{IAO\_0022027 \sphinxhyphen{} wikidata q number}}}}
&
\sphinxAtStartPar
QID (or Q number) is the unique identifier of a

\sphinxAtStartPar
data item on Wikidata, comprising the letter “Q”

\sphinxAtStartPar
followed by one or more digits.
\\
\hline
\sphinxAtStartPar
{\hyperref[\detokenize{doc-IAO_0022057::doc}]{\sphinxcrossref{\DUrole{doc}{IAO\_0022057 \sphinxhyphen{} ringgold identifier}}}}
&
\sphinxAtStartPar
The Ringgold Identifier is a unique numerical

\sphinxAtStartPar
identifier applied to organizations in the

\sphinxAtStartPar
scholarly supply chain
\\
\hline
\end{tabulary}
\par
\sphinxattableend\end{savenotes}


\section{Usage}
\label{\detokenize{identifiers:usage}}
\sphinxAtStartPar
To assert that an organization has an identifier, we assert the existence of the
identifier of a particular type, its value/representation, and its association to the
organization.  We say:

\begin{sphinxVerbatim}[commandchars=\\\{\}]
\PYG{n}{x} \PYG{n}{denoted\PYGZus{}by} \PYG{n}{y}
\PYG{n}{y} \PYG{n}{a} \PYG{n}{research\PYGZus{}orgnization\PYGZus{}registry\PYGZus{}identifier}
\PYG{n}{y} \PYG{n}{has\PYGZus{}representation} \PYG{l+s+s2}{\PYGZdq{}}\PYG{l+s+s2}{ror\PYGZhy{}value}\PYG{l+s+s2}{\PYGZdq{}}
\end{sphinxVerbatim}
\phantomsection\label{\detokenize{addresses:addresses}}
\index{addresses@\spxentry{addresses}}\ignorespaces 

\chapter{Addresses}
\label{\detokenize{addresses:index-0}}\label{\detokenize{addresses:id1}}\label{\detokenize{addresses::doc}}
\sphinxAtStartPar
The Organization Ontology represents addresses (postal and email) as “things” that denote
organizations.


\section{Email Addresses}
\label{\detokenize{addresses:email-addresses}}
\sphinxAtStartPar
\sphinxcode{\sphinxupquote{IAO\_0000429}} is the term id for the class email address. It is an information artifact
that denotes an
organization and has a text string representation.

\sphinxAtStartPar
If y is the URI of an organization we can say y has email address \sphinxhref{mailto:info@abc.com}{info@abc.com} by
asserting:

\begin{sphinxVerbatim}[commandchars=\\\{\}]
\PYG{n}{y} \PYG{n}{denoted\PYGZus{}by} \PYG{n}{x}
\PYG{n}{x} \PYG{n}{a} \PYG{n}{email\PYGZus{}address}
\PYG{n}{x} \PYG{n}{has\PYGZus{}email\PYGZus{}representation} \PYG{l+s+s2}{\PYGZdq{}}\PYG{l+s+s2}{info@abc.com}\PYG{l+s+s2}{\PYGZdq{}}
\end{sphinxVerbatim}

\sphinxAtStartPar
Because the email address is a thing, we can assign qualities to it. Email addresses may
have purposes (dispositions) to help users route email effectively. We might say:

\begin{sphinxVerbatim}[commandchars=\\\{\}]
\PYG{n}{x} \PYG{n}{has\PYGZus{}disposition} \PYG{n}{z}
\PYG{n}{z} \PYG{n}{a} \PYG{n}{information\PYGZus{}address\PYGZus{}disposition}
\end{sphinxVerbatim}

\sphinxAtStartPar
to indicate that x is an email address that can be used by people to ask questions and
get help.


\section{Postal Addresses}
\label{\detokenize{addresses:postal-addresses}}
\sphinxAtStartPar
Postal addresses are represented in a manner analogous to email addresses \textendash{} postal
addresses are information artifacts that denote an organization. Like email
addresses they may have qualities (dispositions) that help people use the postal
address effectively. Postal addresses are things. The term ID is \sphinxcode{\sphinxupquote{IAO\_0000422}}.

\sphinxAtStartPar
Postal addresses are text strings in which the “parts” of a postal address are
delimited by semicolons. Applications can parse these strings into parts needed
by the application. Parts and parsing vary by jurisdiction but should conform to
\sphinxhref{https://www.upu.int}{Universal Postal Union} standards in implementations.

\sphinxAtStartPar
To say org y has a billing postal address, we can assert:

\begin{sphinxVerbatim}[commandchars=\\\{\}]
\PYG{n}{y} \PYG{n}{denoted\PYGZus{}by} \PYG{n}{x}
\PYG{n}{x} \PYG{n}{a} \PYG{n}{postal\PYGZus{}address}
\PYG{n}{x} \PYG{n}{has\PYGZus{}disposition} \PYG{n}{z}
\PYG{n}{z} \PYG{n}{a} \PYG{n}{billing\PYGZus{}address\PYGZus{}disposition}
\PYG{n}{x} \PYG{n}{has\PYGZus{}postal\PYGZus{}address\PYGZus{}representation} \PYG{l+s+s2}{\PYGZdq{}}\PYG{l+s+s2}{line 1; line 2; city; region; country; postal\PYGZhy{}code}\PYG{l+s+s2}{\PYGZdq{}}
\end{sphinxVerbatim}


\section{Terms used to represent Addresses}
\label{\detokenize{addresses:terms-used-to-represent-addresses}}
\sphinxAtStartPar
{\hyperref[\detokenize{addresses:table-10}]{\sphinxcrossref{\DUrole{std,std-ref}{Table 10 Terms used to represent addresses}}}} lists term ids used in the representation of addresses


\begin{savenotes}\sphinxattablestart
\centering
\sphinxcapstartof{table}
\sphinxthecaptionisattop
\sphinxcaption{Table 10 Terms used to represent addresses}\label{\detokenize{addresses:id2}}\label{\detokenize{addresses:table-10}}
\sphinxaftertopcaption
\begin{tabulary}{\linewidth}[t]{|T|T|}
\hline
\sphinxstyletheadfamily 
\sphinxAtStartPar
Term
&\sphinxstyletheadfamily 
\sphinxAtStartPar
Notes
\\
\hline
\sphinxAtStartPar
{\hyperref[\detokenize{doc-IAO_0000235::doc}]{\sphinxcrossref{\DUrole{doc}{IAO\_0000235 \sphinxhyphen{} denoted by}}}}
&
\sphinxAtStartPar
An organization is denoted by an address
\\
\hline
\sphinxAtStartPar
{\hyperref[\detokenize{doc-IAO_0000429::doc}]{\sphinxcrossref{\DUrole{doc}{IAO\_0000429 \sphinxhyphen{} email address}}}}
&
\sphinxAtStartPar
An entity with properties and a value
\\
\hline
\sphinxAtStartPar
{\hyperref[\detokenize{doc-ORG_3000002::doc}]{\sphinxcrossref{\DUrole{doc}{ORG\_3000002 \sphinxhyphen{} has email representation}}}}
&
\sphinxAtStartPar
A datatype property to contain an email address string
\\
\hline
\sphinxAtStartPar
{\hyperref[\detokenize{doc-RO_0000091::doc}]{\sphinxcrossref{\DUrole{doc}{RO\_0000091 \sphinxhyphen{} has disposition}}}}
&
\sphinxAtStartPar
Object property relating an entity to a disposition
\\
\hline
\sphinxAtStartPar
{\hyperref[\detokenize{doc-ORG_0000031::doc}]{\sphinxcrossref{\DUrole{doc}{ORG\_0000031 \sphinxhyphen{} manufacturing disposition}}}}
&
\sphinxAtStartPar
A quality of an adress to obtain information
\\
\hline
\sphinxAtStartPar
{\hyperref[\detokenize{doc-IAO_0000422::doc}]{\sphinxcrossref{\DUrole{doc}{IAO\_0000422 \sphinxhyphen{} postal address}}}}
&
\sphinxAtStartPar
An entity with properties and a value for postal delivery
\\
\hline
\sphinxAtStartPar
{\hyperref[\detokenize{doc-ORG_0000032::doc}]{\sphinxcrossref{\DUrole{doc}{ORG\_0000032 \sphinxhyphen{} project team disposition}}}}
&
\sphinxAtStartPar
An address used to send bills to an entity
\\
\hline
\sphinxAtStartPar
{\hyperref[\detokenize{doc-ORG_3000003::doc}]{\sphinxcrossref{\DUrole{doc}{ORG\_3000003 \sphinxhyphen{} has postal address representation}}}}
&
\sphinxAtStartPar
An datatype property to contain a postal address string
\\
\hline
\end{tabulary}
\par
\sphinxattableend\end{savenotes}
\phantomsection\label{\detokenize{facilities:facilities}}
\index{Facilities@\spxentry{Facilities}}\ignorespaces 

\chapter{Facilities}
\label{\detokenize{facilities:index-0}}\label{\detokenize{facilities:id1}}\label{\detokenize{facilities::doc}}
\sphinxAtStartPar
A facility is a human\sphinxhyphen{}made structure, attached to the ground.  Examples include
\begin{itemize}
\item {} 
\sphinxAtStartPar
buildings, including special purpose building such as hospitals and libraries

\item {} 
\sphinxAtStartPar
campuses and other collections of building in contiguous space

\item {} 
\sphinxAtStartPar
bridges, monuments, parks, parking lots, towers, dams, and all other human\sphinxhyphen{}made
structures on the ground

\end{itemize}

\sphinxAtStartPar
Note that we exclude structures in space, non human made structures such as ant hills,
and geological “structures” such as caves.

\sphinxAtStartPar
We also exclude spaces in facilities that that may have a specific purpose.  We may say
“the gene sequencing facility located in Building 42,” but the gene sequencing
“facility” in this sentence is not a facility in the sense described here.


\section{Facilities in the Organization Ontology}
\label{\detokenize{facilities:facilities-in-the-organization-ontology}}
\sphinxAtStartPar
The Organization Ontology is focused on organizations.  Organizations have relations
to Facilities \textendash{} they may occupy, own, lease, or otherwise be related.  It is
not the purpose of the Organization Ontology to provide extensive representation of
facilities.  The Organization Ontology has simple representations that appear to cover
important use cases, particularly in the representation of organizations in scholarship.


\section{Types of Facilities}
\label{\detokenize{facilities:types-of-facilities}}\begin{itemize}
\item {} 
\sphinxAtStartPar
building

\item {} 
\sphinxAtStartPar
campus

\end{itemize}

\sphinxAtStartPar
Perhaps we do not need more than these to start.


\section{Properties of Facilities}
\label{\detokenize{facilities:properties-of-facilities}}\begin{itemize}
\item {} 
\sphinxAtStartPar
have names, abbreviations, nicknames, and acronyms.

\item {} 
\sphinxAtStartPar
have locations.  Facilities may be “located in” a city, or may have
a geolocation with a latitude/longitude representation.

\item {} 
\sphinxAtStartPar
have identifiers. These are represented using the Identifier Ontology (IDO).

\item {} 
\sphinxAtStartPar
A facility may be part of a campus.

\item {} 
\sphinxAtStartPar
A room may be located in a building.

\end{itemize}


\section{Relation of Organizations and Facilities}
\label{\detokenize{facilities:relation-of-organizations-and-facilities}}\begin{itemize}
\item {} 
\sphinxAtStartPar
occupies.  The organization has zero or more of its people residing in or working at
or regularly visiting the facility.  Occupies can be used when the ownership
of a facility is not of interest, ambiguous, or unknown.

\end{itemize}

\sphinxAtStartPar
No other relations are anticipated for the Organization Ontology.
\phantomsection\label{\detokenize{locations:locations}}
\index{Locations@\spxentry{Locations}}\ignorespaces 

\chapter{Locations}
\label{\detokenize{locations:index-0}}\label{\detokenize{locations:id1}}\label{\detokenize{locations::doc}}
\sphinxAtStartPar
The Organization Ontology represents locations as places on the earth.  The following
entities have locations:
\begin{itemize}
\item {} 
\sphinxAtStartPar
continents

\item {} 
\sphinxAtStartPar
countries.  Including disputed countries \sphinxstepexplicit %
\begin{footnote}[1]\phantomsection\label{\thesphinxscope.1}%
\sphinxAtStartFootnote
definition of “countries” is a matter of dispute and controversy.  Any list of
countries is subject to dispute.
%
\end{footnote}.

\item {} 
\sphinxAtStartPar
regions of countries.  These may have many different names based on the local
jurisdiction, such as territory, state, region, province, or even “kingdom” in the
case of the United Kingdom.

\item {} 
\sphinxAtStartPar
populated places, which may be cities \sphinxstepexplicit %
\begin{footnote}[2]\phantomsection\label{\thesphinxscope.2}%
\sphinxAtStartFootnote
A city often means a governed place, or the government of the place, “The City
of New York”  For our purposes we do not distinguish between city, town, village or
other possibly formal, legal designations.
%
\end{footnote}.  These need not be legally recognized,
merely recognized by people outside the populated place.

\item {} 
\sphinxAtStartPar
{\hyperref[\detokenize{facilities::doc}]{\sphinxcrossref{\DUrole{doc}{facilities}}}}

\end{itemize}


\section{Properties of Locations}
\label{\detokenize{locations:properties-of-locations}}\begin{itemize}
\item {} 
\sphinxAtStartPar
located in.  The Louvre is located in Paris.  Paris is located in France.  Metropolitan
France is located in Europe.

\item {} 
\sphinxAtStartPar
has geographic representation \textendash{} a text string of latitude and longitude of (hopefully)
the centroid of the location.  For example, Paris has geographical representation
“48.864716,2.349014”  Note there are no compass designations (E, W, N. S) in the
representation. A negative latitude is south of the equator, a positive latitude is
north of the equator.  A negative longitude is east of the prime meridian,a positive
longitude is west of the prime meridian.

\end{itemize}


\section{Relations of Locations to Organizations and Facilities}
\label{\detokenize{locations:relations-of-locations-to-organizations-and-facilities}}
\sphinxAtStartPar
Organizations occupy locations.  They are not “located in” locations for two reasons:
\begin{enumerate}
\sphinxsetlistlabels{\arabic}{enumi}{enumii}{}{.}%
\item {} 
\sphinxAtStartPar
Organizations are not material.  Only material things have locations.  An Organization
such as a chess club may meet in a variety of locations, but they are not located
in a location.  An organization such as Amazon has a presence in many locations.

\item {} 
\sphinxAtStartPar
“located in” means all of something located wholly within something else.

\end{enumerate}

\sphinxAtStartPar
Organizations occupy locations.  This means they have some legal right to the location
(own, lease, title, other) or they have one or more persons affiliated with the
organization who is at the location (all or some of the time). While
occupation may involve disputes, most do not.

\sphinxAtStartPar
We can then say

\begin{sphinxVerbatim}[commandchars=\\\{\}]
\PYG{n}{The} \PYG{n}{University} \PYG{n}{of} \PYG{n}{Florida} \PYG{l+s+s1}{\PYGZsq{}}\PYG{l+s+s1}{occupies}\PYG{l+s+s1}{\PYGZsq{}} \PYG{n}{The} \PYG{n}{University} \PYG{n}{of} \PYG{n}{Florida} \PYG{n}{Gainesville} \PYG{n}{campus}
\PYG{n}{The} \PYG{n}{University} \PYG{n}{of} \PYG{n}{Florida} \PYG{n}{Gainesville} \PYG{n}{campus} \PYG{l+s+s1}{\PYGZsq{}}\PYG{l+s+s1}{is located in}\PYG{l+s+s1}{\PYGZsq{}} \PYG{n}{Gainesville}
\PYG{n}{The} \PYG{n}{University} \PYG{n}{of} \PYG{n}{Florida} \PYG{n}{Gainesville} \PYG{n}{campus} \PYG{l+s+s1}{\PYGZsq{}}\PYG{l+s+s1}{has geolocation representation}\PYG{l+s+s1}{\PYGZsq{}} \PYG{l+s+s2}{\PYGZdq{}}\PYG{l+s+s2}{29.6436325,\PYGZhy{}82.3571242}\PYG{l+s+s2}{\PYGZdq{}}
\end{sphinxVerbatim}

\sphinxAtStartPar
Note that ‘located in’ is transitive.  Gainesville is located in Florida.  Florida is
located in the United States.  We can infer that the University of Florida campus is
located in the United States.

\sphinxAtStartPar
{\hyperref[\detokenize{organizations::doc}]{\sphinxcrossref{\DUrole{doc}{Organizations}}}} do not have locations.  Facilities, and buildings
have locations. Campuses have locations.

\sphinxAtStartPar
{\hyperref[\detokenize{locations:table-14}]{\sphinxcrossref{\DUrole{std,std-ref}{Table 14 Terms used to represent locations}}}} lists terms used in the representation of locations


\begin{savenotes}\sphinxattablestart
\centering
\sphinxcapstartof{table}
\sphinxthecaptionisattop
\sphinxcaption{Table 14 Terms used to represent locations}\label{\detokenize{locations:id6}}\label{\detokenize{locations:table-14}}
\sphinxaftertopcaption
\begin{tabulary}{\linewidth}[t]{|T|T|}
\hline
\sphinxstyletheadfamily 
\sphinxAtStartPar
Term
&\sphinxstyletheadfamily 
\sphinxAtStartPar
Notes
\\
\hline
\sphinxAtStartPar
{\hyperref[\detokenize{doc-ORG_0000040::doc}]{\sphinxcrossref{\DUrole{doc}{ORG\_0000040 \sphinxhyphen{} architectural structure}}}}
&
\sphinxAtStartPar
A man\sphinxhyphen{}made construction attached to the ground, a bauwerk
\\
\hline
\sphinxAtStartPar
{\hyperref[\detokenize{doc-ORG_0000041::doc}]{\sphinxcrossref{\DUrole{doc}{ORG\_0000041 \sphinxhyphen{} campus}}}}
&
\sphinxAtStartPar
The grounds of a business, university, or other
\\
\hline
\sphinxAtStartPar
{\hyperref[\detokenize{doc-ORG_0000042::doc}]{\sphinxcrossref{\DUrole{doc}{ORG\_0000042 \sphinxhyphen{} facility}}}}
&
\sphinxAtStartPar
An architectural structure with a function
\\
\hline
\sphinxAtStartPar
{\hyperref[\detokenize{doc-ORG_0000043::doc}]{\sphinxcrossref{\DUrole{doc}{ORG\_0000043 \sphinxhyphen{} building}}}}
&
\sphinxAtStartPar
A permanent walled and roofed construction
\\
\hline
\sphinxAtStartPar
{\hyperref[\detokenize{doc-ORG_0000044::doc}]{\sphinxcrossref{\DUrole{doc}{ORG\_0000044 \sphinxhyphen{} room}}}}
&
\sphinxAtStartPar
A space delineated by partitions in a building
\\
\hline
\sphinxAtStartPar
{\hyperref[\detokenize{doc-ORG_0000047::doc}]{\sphinxcrossref{\DUrole{doc}{ORG\_0000047 \sphinxhyphen{} continent}}}}
&
\sphinxAtStartPar
One of the seven major land masses of the earth
\\
\hline
\sphinxAtStartPar
{\hyperref[\detokenize{doc-ORG_0000048::doc}]{\sphinxcrossref{\DUrole{doc}{ORG\_0000048 \sphinxhyphen{} country}}}}
&
\sphinxAtStartPar
The territory occupied by a sovereign state
\\
\hline
\sphinxAtStartPar
{\hyperref[\detokenize{doc-ORG_0000049::doc}]{\sphinxcrossref{\DUrole{doc}{ORG\_0000049 \sphinxhyphen{} region}}}}
&
\sphinxAtStartPar
Any subdivision of the territory of a country
\\
\hline
\sphinxAtStartPar
{\hyperref[\detokenize{doc-ORG_0000050::doc}]{\sphinxcrossref{\DUrole{doc}{ORG\_0000050 \sphinxhyphen{} populated place}}}}
&
\sphinxAtStartPar
Any named place on the earth occupied by people
\\
\hline
\sphinxAtStartPar
{\hyperref[\detokenize{doc-ORG_2000002::doc}]{\sphinxcrossref{\DUrole{doc}{ORG\_2000002 \sphinxhyphen{} has occurent part}}}}
&
\sphinxAtStartPar
The relation indicating an organization occupies a location
\\
\hline
\sphinxAtStartPar
{\hyperref[\detokenize{doc-RO_0001015::doc}]{\sphinxcrossref{\DUrole{doc}{RO\_0001015 \sphinxhyphen{} location of}}}}
&
\sphinxAtStartPar
Location of
\\
\hline
\sphinxAtStartPar
{\hyperref[\detokenize{doc-RO_0001025::doc}]{\sphinxcrossref{\DUrole{doc}{RO\_0001025 \sphinxhyphen{} located in}}}}
&
\sphinxAtStartPar
Located in
\\
\hline
\sphinxAtStartPar
{\hyperref[\detokenize{doc-ORG_0000045::doc}]{\sphinxcrossref{\DUrole{doc}{ORG\_0000045 \sphinxhyphen{} geographic region}}}}
&
\sphinxAtStartPar
A geographical location on the earth
\\
\hline
\sphinxAtStartPar
{\hyperref[\detokenize{doc-ORG_0000046::doc}]{\sphinxcrossref{\DUrole{doc}{ORG\_0000046 \sphinxhyphen{} geographic point}}}}
&
\sphinxAtStartPar
A point on the earth
\\
\hline
\sphinxAtStartPar
{\hyperref[\detokenize{doc-ORG_3000004::doc}]{\sphinxcrossref{\DUrole{doc}{ORG\_3000004 \sphinxhyphen{} has geolocation representation}}}}
&
\sphinxAtStartPar
A geolocation representation as lat,long
\\
\hline
\end{tabulary}
\par
\sphinxattableend\end{savenotes}
\phantomsection\label{\detokenize{datetimes:datetimes}}
\index{Dates@\spxentry{Dates}}\index{Times@\spxentry{Times}}\index{Datetimes@\spxentry{Datetimes}}\ignorespaces 

\chapter{Dates and Times}
\label{\detokenize{datetimes:dates-and-times}}\label{\detokenize{datetimes:index-0}}\label{\detokenize{datetimes::doc}}
\sphinxAtStartPar
The Organization Ontology uses the W3C Time Ontology \sphinxstepexplicit %
\begin{footnote}[1]\phantomsection\label{\thesphinxscope.1}%
\sphinxAtStartFootnote
\sphinxurl{https://www.w3.org/TR/owl-time/}
%
\end{footnote} for representation of
dates and times.  The key entity is time:Instant, which may have a precision to
indicate whether we know the instant to a particular day, month, year, or with
more precision such as hour, minute, second, and so on.  The Time Ontology has sophisticated
semantics for many kinds of calendars.  The examples for the Organization
Ontology assume Gregorian calendar.  Most time instants related to organizations are
at the day or year precision.  Assertions such as “This org was founded in that year”
are common.


\section{BFO Date Semantics}
\label{\detokenize{datetimes:bfo-date-semantics}}
\sphinxAtStartPar
BFO has “occurents” \textendash{} entities which occur in time.  The most important BFO entity
for organizations is \sphinxcode{\sphinxupquote{BFO\_000015}} \textendash{} process.  A process is an occurrent which
has proper temporal parts and involves an entity as participant or output.

\sphinxAtStartPar
In the ORG ontology, most processes of interest \textendash{} the creation of organizations,
the dissolution of organizations, are
processes in which rights and privileges are obtained.  All
organizations have rights and privileges, either formally recognized in founding
documents, or informally by word of mouth and personal agreements.  These are not
currently represented in the Organization Ontology, but are readily added.

\sphinxAtStartPar
Processes often have process boundaries (\sphinxcode{\sphinxupquote{BFO\_000035}}) as demarcations in a process, points
in time at which something occurred.  Process boundaries can have an
associated time instant to indicate when the process boundary occurred.

\sphinxAtStartPar
{\hyperref[\detokenize{datetimes:figure-3}]{\sphinxcrossref{\DUrole{std,std-ref}{Figure 3.  General time pattern.  An organization is the output of a founding process.}}}} shows the general pattern.  An organization is the output of a founding
process.  The founding process has a process boundary which is
the moment in the process at which the organization comes into existence (the founding).
The founding has an associated time instant.  The instant has a datetime value and
a time precision.

\begin{figure}[htbp]
\centering
\capstart

\noindent\sphinxincludegraphics{{general-time-pattern}.png}
\caption{Figure 3.  General time pattern.  An organization is the output of a founding process.}\label{\detokenize{datetimes:id3}}\label{\detokenize{datetimes:figure-3}}\end{figure}

\sphinxAtStartPar
\sphinxstyleemphasis{Implementation note:}  The colors in the figure represent dependency of entities.  The
dark blue organization is the
entity of interest.  The light blue founding process and founding process boundary are
dependent on
the object of interest.  In typical data systems, if the organization was removed from
the system,
its founding process and founding process boundary would also be removed.  The datetime
instant would not be removed, it may be associated with other occurrents in the data
system.

\sphinxAtStartPar
This representation achieves several goals:
\begin{enumerate}
\sphinxsetlistlabels{\arabic}{enumi}{enumii}{}{.}%
\item {} 
\sphinxAtStartPar
Uses BFO to represent time semantics, clarifying the nature of dates and what
they represent using a consistent upper level ontology.

\item {} 
\sphinxAtStartPar
Uses W3C Time Ontology to represent time values gaining richness of expression
in the representation of date and time values.

\item {} 
\sphinxAtStartPar
Provides the framework necessary to add additional detail regarding processes
related to organizations, including additional processes, rights and privileges
resulting from processes, additional milestones in processes,
locations and participants of processes, documents and other participants and
outputs of processes.

\end{enumerate}


\section{Example}
\label{\detokenize{datetimes:example}}
\sphinxAtStartPar
To assert that an organization x was founded in the year 1853, we would say:

\begin{sphinxVerbatim}[commandchars=\\\{\}]
\PYG{n}{x} \PYG{n}{a} \PYG{n}{organization}
\PYG{n}{x} \PYG{n}{output\PYGZus{}of} \PYG{n}{y}
\PYG{n}{y} \PYG{n}{a} \PYG{n}{founding\PYGZus{}process}
\PYG{n}{y} \PYG{n}{has\PYGZus{}occurent\PYGZus{}part} \PYG{n}{z}
\PYG{n}{z} \PYG{n}{a} \PYG{n}{founding\PYGZus{}process\PYGZus{}boundary}
\PYG{n}{z} \PYG{n}{has\PYGZus{}instant} \PYG{n}{t}
\PYG{n}{t} \PYG{n}{a} \PYG{n}{instant}
\PYG{n}{t} \PYG{n}{unit\PYGZus{}type} \PYG{n}{unit\PYGZus{}year}
\PYG{n}{t} \PYG{n}{has\PYGZus{}xsd\PYGZus{}datetme\PYGZus{}stamp} \PYG{l+s+s2}{\PYGZdq{}}\PYG{l+s+s2}{1853\PYGZhy{}01\PYGZhy{}01T00:00:00Z}\PYG{l+s+s2}{\PYGZdq{}}\PYG{o}{\PYGZca{}}\PYG{o}{\PYGZca{}}\PYG{n}{xsd}\PYG{p}{:}\PYG{n}{dateTimeStamp}
\end{sphinxVerbatim}

\sphinxAtStartPar
\sphinxstyleemphasis{Implementation note 1:}  Some data systems may be pre\sphinxhyphen{}populated with instants
of year precision for years of interest.  In such a case the above example would
end with z has\_instant t, where t is the pre\sphinxhyphen{}existing instant representing 1853 with
year precision.

\sphinxAtStartPar
\sphinxstyleemphasis{Implementation note 2:}  Some data system may require that time representations
be formatted in a particular way, for example W3C datetime standard, with a time zone,
even if the time precision is year only.  Elements of the time representation other
than the year would be ignored by the data system.  In such cases, the time representation
might be appear as “1853\sphinxhyphen{}01\sphinxhyphen{}01T00:00:01+00:00” for example.


\section{Terms used to represent dates and times}
\label{\detokenize{datetimes:terms-used-to-represent-dates-and-times}}
\sphinxAtStartPar
{\hyperref[\detokenize{datetimes:table-13}]{\sphinxcrossref{\DUrole{std,std-ref}{Table 13 Terms used to represent dates and times}}}} lists terms used to represent dates and times


\begin{savenotes}\sphinxattablestart
\centering
\sphinxcapstartof{table}
\sphinxthecaptionisattop
\sphinxcaption{Table 13 Terms used to represent dates and times}\label{\detokenize{datetimes:id4}}\label{\detokenize{datetimes:table-13}}
\sphinxaftertopcaption
\begin{tabulary}{\linewidth}[t]{|T|T|}
\hline
\sphinxstyletheadfamily 
\sphinxAtStartPar
Term
&\sphinxstyletheadfamily 
\sphinxAtStartPar
Notes
\\
\hline
\sphinxAtStartPar
{\hyperref[\detokenize{doc-RO_0002353::doc}]{\sphinxcrossref{\DUrole{doc}{RO\_0002353 \sphinxhyphen{} output of}}}}
&
\sphinxAtStartPar
output of
\\
\hline
\sphinxAtStartPar
{\hyperref[\detokenize{doc-RO_0002234::doc}]{\sphinxcrossref{\DUrole{doc}{RO\_0002234 \sphinxhyphen{} has output}}}}
&
\sphinxAtStartPar
has output
\\
\hline
\sphinxAtStartPar
{\hyperref[\detokenize{doc-BFO_0000015::doc}]{\sphinxcrossref{\DUrole{doc}{BFO\_0000015 \sphinxhyphen{} process}}}}
&
\sphinxAtStartPar
process
\\
\hline
\sphinxAtStartPar
{\hyperref[\detokenize{doc-ORG_0000040::doc}]{\sphinxcrossref{\DUrole{doc}{ORG\_0000040 \sphinxhyphen{} architectural structure}}}}
&
\sphinxAtStartPar
founding process
\\
\hline
\sphinxAtStartPar
{\hyperref[\detokenize{doc-ORG_2000003::doc}]{\sphinxcrossref{\DUrole{doc}{ORG\_2000003 \sphinxhyphen{} has time instant}}}}
&
\sphinxAtStartPar
has occurent part
\\
\hline
\sphinxAtStartPar
{\hyperref[\detokenize{doc-RO_0002012::doc}]{\sphinxcrossref{\DUrole{doc}{RO\_0002012 \sphinxhyphen{} occurent part of}}}}
&
\sphinxAtStartPar
occurent part of
\\
\hline
\sphinxAtStartPar
{\hyperref[\detokenize{doc-BFO_0000035::doc}]{\sphinxcrossref{\DUrole{doc}{BFO\_0000035 \sphinxhyphen{} process boundary}}}}
&
\sphinxAtStartPar
process boundary
\\
\hline
\sphinxAtStartPar
{\hyperref[\detokenize{doc-ORG_0000041::doc}]{\sphinxcrossref{\DUrole{doc}{ORG\_0000041 \sphinxhyphen{} campus}}}}
&
\sphinxAtStartPar
founding process boundary
\\
\hline
\sphinxAtStartPar
{\hyperref[\detokenize{doc-ORG_0000042::doc}]{\sphinxcrossref{\DUrole{doc}{ORG\_0000042 \sphinxhyphen{} facility}}}}
&
\sphinxAtStartPar
dissolution process
\\
\hline
\sphinxAtStartPar
{\hyperref[\detokenize{doc-ORG_0000043::doc}]{\sphinxcrossref{\DUrole{doc}{ORG\_0000043 \sphinxhyphen{} building}}}}
&
\sphinxAtStartPar
dissolution process boundary
\\
\hline
\sphinxAtStartPar
{\hyperref[\detokenize{doc-ORG_2000003::doc}]{\sphinxcrossref{\DUrole{doc}{ORG\_2000003 \sphinxhyphen{} has time instant}}}}
&
\sphinxAtStartPar
has instant.  Process boundaries have instants.
\\
\hline
\sphinxAtStartPar
{\hyperref[\detokenize{doc-Instant::doc}]{\sphinxcrossref{\DUrole{doc}{Instant \sphinxhyphen{} time instant}}}}
&
\sphinxAtStartPar
Instant
\\
\hline
\sphinxAtStartPar
{\hyperref[\detokenize{doc-inXSDDateTimeStamp::doc}]{\sphinxcrossref{\DUrole{doc}{inXSDDateTimeStamp \sphinxhyphen{} in XSD Date\sphinxhyphen{}Time\sphinxhyphen{}Stamp}}}}
&
\sphinxAtStartPar
An xsd:datetimestamp string associated with an Instant
\\
\hline
\sphinxAtStartPar
{\hyperref[\detokenize{doc-unitType::doc}]{\sphinxcrossref{\DUrole{doc}{unitType \sphinxhyphen{} temporal unit type}}}}
&
\sphinxAtStartPar
has datetime precision
\\
\hline
\sphinxAtStartPar
{\hyperref[\detokenize{doc-unitYear::doc}]{\sphinxcrossref{\DUrole{doc}{unitYear \sphinxhyphen{} year (unit of temporal duration)}}}}
&
\sphinxAtStartPar
year precision
\\
\hline
\sphinxAtStartPar
{\hyperref[\detokenize{doc-unitMonth::doc}]{\sphinxcrossref{\DUrole{doc}{unitMonth \sphinxhyphen{} month (unit of temporal duration)}}}}
&
\sphinxAtStartPar
month precision
\\
\hline
\sphinxAtStartPar
{\hyperref[\detokenize{doc-unitDay::doc}]{\sphinxcrossref{\DUrole{doc}{unitDay \sphinxhyphen{} day (unit of temporal duration)}}}}
&
\sphinxAtStartPar
day precision
\\
\hline
\sphinxAtStartPar
{\hyperref[\detokenize{doc-unitHour::doc}]{\sphinxcrossref{\DUrole{doc}{unitHour \sphinxhyphen{} hour (unit of temporal duration)}}}}
&
\sphinxAtStartPar
hour precision
\\
\hline
\sphinxAtStartPar
{\hyperref[\detokenize{doc-unitMinute::doc}]{\sphinxcrossref{\DUrole{doc}{unitMinute \sphinxhyphen{} minute (unit of temporal duration)}}}}
&
\sphinxAtStartPar
minute precision
\\
\hline
\sphinxAtStartPar
{\hyperref[\detokenize{doc-unitSecond::doc}]{\sphinxcrossref{\DUrole{doc}{unitSecond \sphinxhyphen{} second (unit of temporal duration)}}}}
&
\sphinxAtStartPar
second precision
\\
\hline
\end{tabulary}
\par
\sphinxattableend\end{savenotes}
\phantomsection\label{\detokenize{associations:associations}}
\index{associations@\spxentry{associations}}\ignorespaces 

\chapter{Associations}
\label{\detokenize{associations:index-0}}\label{\detokenize{associations:id1}}\label{\detokenize{associations::doc}}

\section{Associations of Organizations with other Organizations}
\label{\detokenize{associations:associations-of-organizations-with-other-organizations}}
\sphinxAtStartPar
Organizations can be associated with each other in a variety of manners.  Here we
describe how to represent various associations between organizations.


\subsection{Affiliation}
\label{\detokenize{associations:affiliation}}
\sphinxAtStartPar
An organization can be affiliated with another organization.  In the Organization
Ontology,
affiliation is a broad term and may mean and association that is not further specified.
The only restriction on affiliation is that it is \sphinxstyleemphasis{symmetric}.  If organization x is
affiliated with organization y, then organization x is affiliated with organization y.

\sphinxAtStartPar
We say simply

\begin{sphinxVerbatim}[commandchars=\\\{\}]
\PYG{n}{x} \PYG{n}{a} \PYG{n}{organization}
\PYG{n}{y} \PYG{n}{a} \PYG{n}{organization}
\PYG{n}{y} \PYG{n}{affiliated\PYGZus{}with} \PYG{n}{x}
\end{sphinxVerbatim}

\sphinxAtStartPar
for which we can infer \sphinxstepexplicit %
\begin{footnote}[1]\phantomsection\label{\thesphinxscope.1}%
\sphinxAtStartFootnote
Some inferencers are able to create/materialize symmetric assertions.  Some query
engines will satisfy query requirements as if the assertion had been materialized.
%
\end{footnote}:

\begin{sphinxVerbatim}[commandchars=\\\{\}]
\PYG{n}{x} \PYG{n}{affiliated\PYGZus{}with} \PYG{n}{y}
\end{sphinxVerbatim}


\subsection{Structure}
\label{\detokenize{associations:structure}}
\sphinxAtStartPar
Representing organizational structure is a common need.  The Organization Ontology
has two properties, has\_organizational\_part and organizational\_part\_of, to specify the
relationship between an organizational part and its parent organization (which may also
be an organizational part).

\sphinxAtStartPar
For example, if x represents Baylor University, and y represents its College of Medicine
we would assert:

\begin{sphinxVerbatim}[commandchars=\\\{\}]
\PYG{n}{x} \PYG{n}{has\PYGZus{}organizational\PYGZus{}part} \PYG{n}{y}
\end{sphinxVerbatim}

\sphinxAtStartPar
from which we could infer \sphinxstyleemphasis{y organizational\_part\_of x} \sphinxstepexplicit %
\begin{footnote}[2]\phantomsection\label{\thesphinxscope.2}%
\sphinxAtStartFootnote
Some inferencers are able to create/materialize inverse assertions.  Some query
engines will satisfy query requirements as if the assertion had been materialized.
%
\end{footnote}.


\subsection{Spin\sphinxhyphen{}offs}
\label{\detokenize{associations:spin-offs}}
\sphinxAtStartPar
Representing organizations that spin\sphinxhyphen{}off from other organizations is done
using \sphinxstyleemphasis{has spin\sphinxhyphen{}off organization” and *spin\sphinxhyphen{}off organization off} properties.

\sphinxAtStartPar
For example, to represent that y spun\sphinxhyphen{}off of x, we would assert:

\begin{sphinxVerbatim}[commandchars=\\\{\}]
\PYG{n}{y} \PYG{n}{spin}\PYG{o}{\PYGZhy{}}\PYG{n}{off\PYGZus{}organization\PYGZus{}of} \PYG{n}{x}
\end{sphinxVerbatim}

\sphinxAtStartPar
from which we could infer \sphinxstyleemphasis{x has\_spin\sphinxhyphen{}off\_organization y} \sphinxfootnotemark[2].

\sphinxAtStartPar
In some cases, we want to know more about the nature of the spin\sphinxhyphen{}off process, including
dates of various milestones, people who participated, documents involved, and more.
\sphinxstyleemphasis{spin\sphinxhyphen{}off process} is an entity in which other entities may participate, and as an
occurrent, may have one or more \sphinxstyleemphasis{spin\sphinxhyphen{}off process boundaries} as occurent parts. See
{\hyperref[\detokenize{datetimes::doc}]{\sphinxcrossref{\DUrole{doc}{Dates and Times}}}} for representation of time\sphinxhyphen{}based entities.


\subsection{History}
\label{\detokenize{associations:history}}
\sphinxAtStartPar
In some cases, we may wish to assert that organization y is the successor of organization
x, meaning that x has ceased to exist, y now exists, and that the result of the
dissolution
of x was the formation of y.  The inverse of \sphinxstyleemphasis{successor\_of} is \sphinxstyleemphasis{has\_successor}.  We can
say
either \sphinxstyleemphasis{y successor\_of x} or \sphinxstyleemphasis{x has\_successor y} \sphinxfootnotemark[2].

\sphinxAtStartPar
We say:

\begin{sphinxVerbatim}[commandchars=\\\{\}]
\PYG{n}{x} \PYG{n}{a} \PYG{n}{organization}
\PYG{n}{y} \PYG{n}{a} \PYG{n}{organization}
\PYG{n}{y} \PYG{n}{successor\PYGZus{}of} \PYG{n}{x}
\end{sphinxVerbatim}

\sphinxAtStartPar
In some cases, we want to know more about the nature of the succession process, including
dates of various milestones, people who participated, documents involved, and more.
\sphinxstyleemphasis{succession process} is an entity in which other entities may participate, and as an
occurrent, may have one or more \sphinxstyleemphasis{succession process boundaries} as occurent parts. See
{\hyperref[\detokenize{datetimes::doc}]{\sphinxcrossref{\DUrole{doc}{Dates and Times}}}} for representation of time\sphinxhyphen{}based entities.


\subsection{Membership}
\label{\detokenize{associations:membership}}
\sphinxAtStartPar
Organizations may be members of other organizations.  If x is a member of y, we can
simple say

\begin{sphinxVerbatim}[commandchars=\\\{\}]
\PYG{n}{x} \PYG{n}{member\PYGZus{}of} \PYG{n}{y}
\end{sphinxVerbatim}

\sphinxAtStartPar
and we can equivalently say \sphinxstyleemphasis{y has\_member x} \sphinxfootnotemark[2].

\sphinxAtStartPar
In some cases, we may need to know more about the membership, which is an
asymmetric relationship of two organizations \textendash{} one organization is the member
and the other organization has granted membership to the member organization.

\sphinxAtStartPar
This is shown in {\hyperref[\detokenize{associations:figure-4}]{\sphinxcrossref{Figure 4}}}.  Since Membership is an occurent, it can have
time\sphinxhyphen{}related assertions.

\begin{figure}[htbp]
\centering
\capstart

\noindent\sphinxincludegraphics{{membership-pattern}.png}
\caption{Figure 4.  General membership pattern.  Organization 1 has a member role in a
membership.  Organization 2 has a grantor role in the membership.}\label{\detokenize{associations:id9}}\label{\detokenize{associations:figure-4}}\end{figure}

\sphinxAtStartPar
To say that v was a member of z through a membership x, we would assert:

\begin{sphinxVerbatim}[commandchars=\\\{\}]
\PYG{n}{v} \PYG{n}{bearer\PYGZus{}of} \PYG{n}{w}
\PYG{n}{w} \PYG{n}{a} \PYG{n}{organizational\PYGZus{}member\PYGZus{}role}
\PYG{n}{w} \PYG{n}{realized\PYGZus{}in} \PYG{n}{x}
\PYG{n}{x} \PYG{n}{a} \PYG{n}{organizational\PYGZus{}membership}
\PYG{n}{x} \PYG{n}{realizes} \PYG{n}{y}
\PYG{n}{y} \PYG{n}{a} \PYG{n}{organizational\PYGZus{}membership\PYGZus{}grantor\PYGZus{}role}
\PYG{n}{y} \PYG{n}{inheres\PYGZus{}in} \PYG{n}{z}
\end{sphinxVerbatim}

\sphinxAtStartPar
Since each of these properties has an inverse, we could equivalently have asserted:

\begin{sphinxVerbatim}[commandchars=\\\{\}]
\PYG{n}{z} \PYG{n}{bearer\PYGZus{}of} \PYG{n}{y}
\PYG{n}{y} \PYG{n}{a} \PYG{n}{organizational\PYGZus{}membership\PYGZus{}grantor\PYGZus{}role}
\PYG{n}{y} \PYG{n}{realized\PYGZus{}in} \PYG{n}{z}
\PYG{n}{x} \PYG{n}{a} \PYG{n}{organizational\PYGZus{}membership}
\PYG{n}{x} \PYG{n}{realizes} \PYG{n}{w}
\PYG{n}{w} \PYG{n}{a} \PYG{n}{organizational\PYGZus{}member\PYGZus{}role}
\PYG{n}{w} \PYG{n}{inheres\PYGZus{}in} \PYG{n}{x}
\end{sphinxVerbatim}

\sphinxAtStartPar
Note that this detail is typically only needed in cases where we wanted to say more
about the roles, or membership entity.


\subsection{Future work \textendash{} Associations of Organizations to Occurents}
\label{\detokenize{associations:future-work-associations-of-organizations-to-occurents}}
\sphinxAtStartPar
Organizations have associations to occurent such as projects, performances, and
events.  Future ontological work in one or more of these domains should
create the properties and classes needed to represent how organizations
are associated with them.


\section{Associations of Organizations and People}
\label{\detokenize{associations:associations-of-organizations-and-people}}
\sphinxAtStartPar
In a similar manner to the associations between organizations and organizations, we can
associate organizations and people.  And in a similar manner, we can simply
say the organization has an association with a person, or we can use an intermediate
entity and roles to describe how the organization and person are related, providing
detail regarding dates, documents, processes, and participants in the association
between an organization and a person.


\subsection{Has Employee / Employer Of}
\label{\detokenize{associations:has-employee-employer-of}}
\sphinxAtStartPar
To assert a person is an employee of an organization, we can simply say

\begin{sphinxVerbatim}[commandchars=\\\{\}]
\PYG{n}{x} \PYG{n}{has\PYGZus{}employee} \PYG{n}{y}
\end{sphinxVerbatim}

\sphinxAtStartPar
or, equivalently, we can say:

\begin{sphinxVerbatim}[commandchars=\\\{\}]
\PYG{n}{y} \PYG{n}{has\PYGZus{}employee} \PYG{n}{x}
\end{sphinxVerbatim}

\sphinxAtStartPar
The entity \sphinxstyleemphasis{position} is a relationship between a person and an organization.  A person
may one of several roles with respect to the position.  An organization may have
one of several roles with respect to the position.  See {\hyperref[\detokenize{associations:figure-5}]{\sphinxcrossref{Figure 5}}}.

\begin{figure}[htbp]
\centering
\capstart

\noindent\sphinxincludegraphics{{employee-pattern}.png}
\caption{Figure 5.  General employee pattern.  Organization 1 has a member role in a
membership.  Organization 2 has a grantor role in the membership.}\label{\detokenize{associations:id10}}\label{\detokenize{associations:figure-5}}\end{figure}

\sphinxAtStartPar
To say that v was an employee of z through a position x, we would assert:

\begin{sphinxVerbatim}[commandchars=\\\{\}]
\PYG{n}{v} \PYG{n}{bearer\PYGZus{}of} \PYG{n}{w}
\PYG{n}{w} \PYG{n}{a} \PYG{n}{organizational\PYGZus{}employee\PYGZus{}role}
\PYG{n}{w} \PYG{n}{realized\PYGZus{}in} \PYG{n}{x}
\PYG{n}{x} \PYG{n}{a} \PYG{n}{organizational\PYGZus{}position}
\PYG{n}{x} \PYG{n}{realizes} \PYG{n}{y}
\PYG{n}{y} \PYG{n}{a} \PYG{n}{organizational\PYGZus{}employer\PYGZus{}role}
\PYG{n}{y} \PYG{n}{inheres\PYGZus{}in} \PYG{n}{z}
\end{sphinxVerbatim}

\sphinxAtStartPar
Since each of these properties has an inverse, we could equivalently have asserted:

\begin{sphinxVerbatim}[commandchars=\\\{\}]
\PYG{n}{z} \PYG{n}{bearer\PYGZus{}of} \PYG{n}{y}
\PYG{n}{y} \PYG{n}{a} \PYG{n}{organizational\PYGZus{}employer\PYGZus{}role}
\PYG{n}{y} \PYG{n}{realized\PYGZus{}in} \PYG{n}{z}
\PYG{n}{x} \PYG{n}{a} \PYG{n}{organizational\PYGZus{}position}
\PYG{n}{x} \PYG{n}{realizes} \PYG{n}{w}
\PYG{n}{w} \PYG{n}{a} \PYG{n}{organizational\PYGZus{}employee\PYGZus{}role}
\PYG{n}{w} \PYG{n}{inheres\PYGZus{}in} \PYG{n}{x}
\end{sphinxVerbatim}

\sphinxAtStartPar
Note that this detail is typically only needed in cases where we wanted to say more
about the roles, or membership entity.


\subsection{Additional Roles and Associations between Organizations and People}
\label{\detokenize{associations:additional-roles-and-associations-between-organizations-and-people}}
\sphinxAtStartPar
Additional roles and properties are available to assert other associations
between organizations and people.

\sphinxAtStartPar
\sphinxstylestrong{has\_member/ member\_of} can be used to assert a person is a member of an
organization.  \sphinxstylestrong{person\_member\_role} and
\sphinxstylestrong{person\_member\_grantor\_role} can be used as in {\hyperref[\detokenize{associations:figure-5}]{\sphinxcrossref{Figure 5}}} to assert
that a person has a member role in an organization granted by the organization.
A membership denotes the association.

\begin{sphinxShadowBox}
\sphinxstylesidebartitle{Persons are not Organizations}

\sphinxAtStartPar
Separate object properties are used to represent associations between organizations
and other organizations, and associations between persons and organizations.  Separate
roles and entities
are used to represent person roles in memberships and organizational roles in
memberships.  Separate entities are used to describe a membership of an
organization in another organization, and a membership of a person in an
organization.
\end{sphinxShadowBox}

\sphinxAtStartPar
\sphinxstylestrong{has\_associate / associate\_of} can be used to assert a person is an associate of an
organization.  \sphinxstylestrong{organizational\_associate\_role} and
\sphinxstylestrong{organizational\_associate\_grantor\_role} can be used as in {\hyperref[\detokenize{associations:figure-5}]{\sphinxcrossref{Figure 5}}} to assert
that a person has an associate role in an organization granted by the organization.
The position denotes the association.

\sphinxAtStartPar
\sphinxstylestrong{head\_of / has\_head} can be used to assert a person is the head of an
organization.  \sphinxstylestrong{organizational\_head\_role} and
\sphinxstylestrong{organizational\_head\_grantor\_role} can be used as in {\hyperref[\detokenize{associations:figure-5}]{\sphinxcrossref{Figure 5}}} to assert
that a person has a head role in an organization granted by the organization.
The position denotes the relationship.

\sphinxAtStartPar
\sphinxstylestrong{has\_volunteer / volunteer\_of} can be used to assert a person is a volunteer of an
organization.  \sphinxstylestrong{organizational\_volunteer\_role} and
\sphinxstylestrong{organizational\_volunteer\_grantor\_role} can be used as in {\hyperref[\detokenize{associations:figure-5}]{\sphinxcrossref{Figure 5}}} to assert
that a person has a volunteer role in an organization granted by the organization.
The position denotes the relationship.

\sphinxAtStartPar
\sphinxstylestrong{has\_appointee / appointee\_of} can be used to assert a person is an appointee of an
organization.  \sphinxstylestrong{organizational\_appointee\_role} and
\sphinxstylestrong{organizational\_appointee\_grantor\_role} can be used as in {\hyperref[\detokenize{associations:figure-5}]{\sphinxcrossref{Figure 5}}} to assert
that a person has an appointee role in an organization granted by the organization.
The position denotes the association.


\chapter{Annotation properties}
\label{\detokenize{annotation-properties:annotation-properties}}\label{\detokenize{annotation-properties::doc}}
\sphinxAtStartPar
Annotation properties provide text for readers of ontologies to explain the
use of terms.  \sphinxhref{http://www.ontobee.org/ontology/IAO}{Information Artifact Ontology (IAO)} defines annotation properties used throughout the VIVO ontologies
for defining terms.  {\hyperref[\detokenize{classes:table-5}]{\sphinxcrossref{\DUrole{std,std-ref}{Table 5 Classes}}}} lists the IAO properties used to annotate terms in the
ontologies.  {\hyperref[\detokenize{annotation-properties:table-6}]{\sphinxcrossref{\DUrole{std,std-ref}{Table 6 Annotation Properties}}}} lists the terms in the controlled vocabulary for curation
status.  To assert that a term’s metadata is complete, the assertion is:
\begin{quote}

\sphinxAtStartPar
\textless{}term\textgreater{} IAO:0000114 IAO\_0000120
\end{quote}

\sphinxAtStartPar
Alternative terms (IAO\_0000118) are not common in the VIVO ontologies.  All other
annotations are expected for all terms.

\sphinxAtStartPar
See {\hyperref[\detokenize{annotation-properties:table-6}]{\sphinxcrossref{Table 6}}}.


\begin{savenotes}\sphinxattablestart
\centering
\sphinxcapstartof{table}
\sphinxthecaptionisattop
\sphinxcaption{Table 6 Annotation Properties}\label{\detokenize{annotation-properties:id1}}\label{\detokenize{annotation-properties:table-6}}
\sphinxaftertopcaption
\begin{tabulary}{\linewidth}[t]{|T|T|}
\hline
\sphinxstyletheadfamily 
\sphinxAtStartPar
Term ID \sphinxhyphen{} Label
&\sphinxstyletheadfamily 
\sphinxAtStartPar
Definition
\\
\hline
\sphinxAtStartPar
{\hyperref[\detokenize{doc-IAO_0000111::doc}]{\sphinxcrossref{\DUrole{doc}{IAO\_0000111 \sphinxhyphen{} editor preferred label}}}}
&
\sphinxAtStartPar
The concise, meaningful, and human\sphinxhyphen{}friendly name

\sphinxAtStartPar
for a class or property preferred by the ontology

\sphinxAtStartPar
developers. (US\sphinxhyphen{}English)
\\
\hline
\sphinxAtStartPar
{\hyperref[\detokenize{doc-IAO_0000112::doc}]{\sphinxcrossref{\DUrole{doc}{IAO\_0000112 \sphinxhyphen{} example of usage}}}}
&
\sphinxAtStartPar
A phrase describing how a term should be used

\sphinxAtStartPar
and/or a citation to a work which uses it. May

\sphinxAtStartPar
also include other kinds of examples that

\sphinxAtStartPar
facilitate immediate understanding, such as widely

\sphinxAtStartPar
know prototypes or instances of a class, or cases

\sphinxAtStartPar
where a relation is said to hold.
\\
\hline
\sphinxAtStartPar
{\hyperref[\detokenize{doc-IAO_0000114::doc}]{\sphinxcrossref{\DUrole{doc}{IAO\_0000114 \sphinxhyphen{} has curation status}}}}
&
\sphinxAtStartPar
A specification of the state of the metadata for a

\sphinxAtStartPar
term
\\
\hline
\sphinxAtStartPar
{\hyperref[\detokenize{doc-IAO_0000115::doc}]{\sphinxcrossref{\DUrole{doc}{IAO\_0000115 \sphinxhyphen{} definition}}}}
&
\sphinxAtStartPar
The official definition, explaining the meaning of

\sphinxAtStartPar
a class or property. Shall be Aristotelian,

\sphinxAtStartPar
formalized and normalized. Can be augmented with

\sphinxAtStartPar
colloquial definitions.
\\
\hline
\sphinxAtStartPar
{\hyperref[\detokenize{doc-IAO_0000116::doc}]{\sphinxcrossref{\DUrole{doc}{IAO\_0000116 \sphinxhyphen{} editor note}}}}
&
\sphinxAtStartPar
An administrative note intended for its editor. It

\sphinxAtStartPar
may not be included in the publication version of

\sphinxAtStartPar
the ontology, so it should contain nothing

\sphinxAtStartPar
necessary for end users to understand the

\sphinxAtStartPar
ontology.
\\
\hline
\sphinxAtStartPar
{\hyperref[\detokenize{doc-IAO_0000117::doc}]{\sphinxcrossref{\DUrole{doc}{IAO\_0000117 \sphinxhyphen{} term editor}}}}
&
\sphinxAtStartPar
Name of editor entering the term in the file. The

\sphinxAtStartPar
term editor is a point of contact for information

\sphinxAtStartPar
regarding the term. The term editor may be, but is

\sphinxAtStartPar
not always, the author of the definition, which

\sphinxAtStartPar
may have been worked upon by several people
\\
\hline
\sphinxAtStartPar
{\hyperref[\detokenize{doc-IAO_0000118::doc}]{\sphinxcrossref{\DUrole{doc}{IAO\_0000118 \sphinxhyphen{} alternative term}}}}
&
\sphinxAtStartPar
An alternative name for a class or property which

\sphinxAtStartPar
means the same thing as the preferred name

\sphinxAtStartPar
(semantically equivalent)
\\
\hline
\sphinxAtStartPar
{\hyperref[\detokenize{doc-IAO_0000119::doc}]{\sphinxcrossref{\DUrole{doc}{IAO\_0000119 \sphinxhyphen{} definition source}}}}
&
\sphinxAtStartPar
Formal citation, e.g. identifier in external

\sphinxAtStartPar
database to indicate / attribute source(s) for the

\sphinxAtStartPar
definition. Free text indicate / attribute

\sphinxAtStartPar
source(s) for the definition. EXAMPLE: Author

\sphinxAtStartPar
Name, URI, MeSH Term C04, PUBMED ID, Wiki uri on

\sphinxAtStartPar
31.01.2007
\\
\hline
\sphinxAtStartPar
{\hyperref[\detokenize{doc-IAO_0000232::doc}]{\sphinxcrossref{\DUrole{doc}{IAO\_0000232 \sphinxhyphen{} curator note}}}}
&
\sphinxAtStartPar
An administrative note of use for a curator but of

\sphinxAtStartPar
no use for a user
\\
\hline
\sphinxAtStartPar
{\hyperref[\detokenize{doc-IAO_0000233::doc}]{\sphinxcrossref{\DUrole{doc}{IAO\_0000233 \sphinxhyphen{} term tracker item}}}}
&
\sphinxAtStartPar
An IRI or similar locator for a request or

\sphinxAtStartPar
discussion of an ontology term.
\\
\hline
\sphinxAtStartPar
{\hyperref[\detokenize{doc-IAO_0000412::doc}]{\sphinxcrossref{\DUrole{doc}{IAO\_0000412 \sphinxhyphen{} imported from}}}}
&
\sphinxAtStartPar
For external terms/classes, the ontology from

\sphinxAtStartPar
which the term was imported
\\
\hline
\sphinxAtStartPar
{\hyperref[\detokenize{doc-ORG_1000001::doc}]{\sphinxcrossref{\DUrole{doc}{ORG\_1000001 \sphinxhyphen{} vivo 1 ontology reference}}}}
&
\sphinxAtStartPar
The term in the original VIVO ontology that is

\sphinxAtStartPar
most like the annotated term.
\\
\hline
\sphinxAtStartPar
{\hyperref[\detokenize{doc-created::doc}]{\sphinxcrossref{\DUrole{doc}{created \sphinxhyphen{} No label}}}}
&
\sphinxAtStartPar
None
\\
\hline
\sphinxAtStartPar
{\hyperref[\detokenize{doc-creator::doc}]{\sphinxcrossref{\DUrole{doc}{creator \sphinxhyphen{} No label}}}}
&
\sphinxAtStartPar
None
\\
\hline
\sphinxAtStartPar
{\hyperref[\detokenize{doc-description::doc}]{\sphinxcrossref{\DUrole{doc}{description \sphinxhyphen{} No label}}}}
&
\sphinxAtStartPar
None
\\
\hline
\sphinxAtStartPar
{\hyperref[\detokenize{doc-title::doc}]{\sphinxcrossref{\DUrole{doc}{title \sphinxhyphen{} No label}}}}
&
\sphinxAtStartPar
None
\\
\hline
\sphinxAtStartPar
{\hyperref[\detokenize{doc-license::doc}]{\sphinxcrossref{\DUrole{doc}{license \sphinxhyphen{} No label}}}}
&
\sphinxAtStartPar
None
\\
\hline
\sphinxAtStartPar
{\hyperref[\detokenize{doc-inverseOf::doc}]{\sphinxcrossref{\DUrole{doc}{inverseOf \sphinxhyphen{} No label}}}}
&
\sphinxAtStartPar
None
\\
\hline
\end{tabulary}
\par
\sphinxattableend\end{savenotes}


\begin{savenotes}\sphinxattablestart
\centering
\sphinxcapstartof{table}
\sphinxthecaptionisattop
\sphinxcaption{Table 11 Common Annotation Properties}\label{\detokenize{annotation-properties:id2}}\label{\detokenize{annotation-properties:table-11}}
\sphinxaftertopcaption
\begin{tabulary}{\linewidth}[t]{|T|T|T|}
\hline
\sphinxstyletheadfamily 
\sphinxAtStartPar
Property
&\sphinxstyletheadfamily 
\sphinxAtStartPar
Label
&\sphinxstyletheadfamily 
\sphinxAtStartPar
Notes
\\
\hline
\sphinxAtStartPar
\sphinxcode{\sphinxupquote{IAO\_0000112}}
&
\sphinxAtStartPar
example of usage
&
\sphinxAtStartPar
A phrase describing how a term should be used
\\
\hline
\sphinxAtStartPar
\sphinxcode{\sphinxupquote{IAO\_0000114}}
&
\sphinxAtStartPar
has curation status
&
\sphinxAtStartPar
A term from a controlled vocabulary
\\
\hline
\sphinxAtStartPar
\sphinxcode{\sphinxupquote{IAO\_0000115}}
&
\sphinxAtStartPar
definition
&
\sphinxAtStartPar
Explains the meaning of a term or property
\\
\hline
\sphinxAtStartPar
\sphinxcode{\sphinxupquote{IAO\_0000116}}
&
\sphinxAtStartPar
editor note
&
\sphinxAtStartPar
An administrative note intended for the term editor
\\
\hline
\sphinxAtStartPar
\sphinxcode{\sphinxupquote{IAO\_0000117}}
&
\sphinxAtStartPar
term editor
&
\sphinxAtStartPar
Name of the editor
\\
\hline
\sphinxAtStartPar
\sphinxcode{\sphinxupquote{IAO\_0000118}}
&
\sphinxAtStartPar
alternative term
&
\sphinxAtStartPar
Alternative name for the term
\\
\hline
\sphinxAtStartPar
\sphinxcode{\sphinxupquote{IAO\_0000119}}
&
\sphinxAtStartPar
definition source
&
\sphinxAtStartPar
Definition citation, may be a link to definition
\\
\hline
\end{tabulary}
\par
\sphinxattableend\end{savenotes}


\begin{savenotes}\sphinxattablestart
\centering
\sphinxcapstartof{table}
\sphinxthecaptionisattop
\sphinxcaption{Table 12 Curation Status}\label{\detokenize{annotation-properties:id3}}\label{\detokenize{annotation-properties:table-12}}
\sphinxaftertopcaption
\begin{tabulary}{\linewidth}[t]{|T|T|T|}
\hline
\sphinxstyletheadfamily 
\sphinxAtStartPar
Property
&\sphinxstyletheadfamily 
\sphinxAtStartPar
Label
&\sphinxstyletheadfamily 
\sphinxAtStartPar
Notes
\\
\hline
\sphinxAtStartPar
\sphinxcode{\sphinxupquote{IAO\_0000120}}
&
\sphinxAtStartPar
metadata complete
&
\sphinxAtStartPar
Term has all metadata, but may not be final
\\
\hline
\sphinxAtStartPar
\sphinxcode{\sphinxupquote{IAO\_0000121}}
&
\sphinxAtStartPar
organizational term
&
\sphinxAtStartPar
Tags used to aid ontology development
\\
\hline
\sphinxAtStartPar
\sphinxcode{\sphinxupquote{IAO\_0000122}}
&
\sphinxAtStartPar
ready for release
&
\sphinxAtStartPar
No further edits needed for term
\\
\hline
\sphinxAtStartPar
\sphinxcode{\sphinxupquote{IAO\_0000123}}
&
\sphinxAtStartPar
metadata incomplete
&
\sphinxAtStartPar
Term is under development
\\
\hline
\sphinxAtStartPar
\sphinxcode{\sphinxupquote{IAO\_0000124}}
&
\sphinxAtStartPar
uncurated
&
\sphinxAtStartPar
Name and class ID, little else
\\
\hline
\sphinxAtStartPar
\sphinxcode{\sphinxupquote{IAO\_0000125}}
&
\sphinxAtStartPar
pending final vetting
&
\sphinxAtStartPar
Complete, awaiting final review
\\
\hline
\sphinxAtStartPar
\sphinxcode{\sphinxupquote{IAO\_0000423}}
&
\sphinxAtStartPar
to be replaced with ext
&
\sphinxAtStartPar
The term is a placeholder and belongs elsewhere
\\
\hline
\end{tabulary}
\par
\sphinxattableend\end{savenotes}


\section{Non IAO Annotation Properties}
\label{\detokenize{annotation-properties:non-iao-annotation-properties}}
\sphinxAtStartPar
The VIVO ontologies use other annotation properties to describe terms and the ontologies.

\sphinxAtStartPar
\sphinxcode{\sphinxupquote{rdfs:label}} is required for all terms and for the ontology itself.  All labels must be
unique.  This greatly simplifies
the use of the ontologies \textendash{} one can search for the label and unambiguously find the
corresponding term.


\subsection{Ontology annotations}
\label{\detokenize{annotation-properties:ontology-annotations}}
\sphinxAtStartPar
Several annotation are used to describe the ontology and are not used further.

\sphinxAtStartPar
\sphinxcode{\sphinxupquote{terms:license}} is required for all ontologies.  The license should be CC0 or CC\sphinxhyphen{}BY, no
other restrictions are acceptable for use in the VIVO ontologies.

\sphinxAtStartPar
\sphinxcode{\sphinxupquote{owl:versionIRI}} a URL identifying the ontology version

\sphinxAtStartPar
\sphinxcode{\sphinxupquote{owl:versionInfo}} a text string identifying the ontology version

\sphinxAtStartPar
\sphinxcode{\sphinxupquote{dc:created}} a date string specifying the date the ontology was originally created

\sphinxAtStartPar
\sphinxcode{\sphinxupquote{dc:creator}} a text string with the name and URL of the creator of the ontology

\sphinxAtStartPar
\sphinxcode{\sphinxupquote{dc:description}} a text description of the ontology, its domain, and purpose

\sphinxAtStartPar
\sphinxcode{\sphinxupquote{dc:title}} the name of the ontology to be used in citations

\sphinxAtStartPar
\sphinxcode{\sphinxupquote{rdfs:comment}} additional text describing the context of the ontology


\subsection{Term annotations}
\label{\detokenize{annotation-properties:term-annotations}}
\sphinxAtStartPar
The following annotations are used to describe terms that are imported to the ORG
ontology.  Most of these are substitutes for the standardized annotation
properties describe above in {\hyperref[\detokenize{classes:table-5}]{\sphinxcrossref{\DUrole{std,std-ref}{Table 5 Classes}}}}.

\sphinxAtStartPar
\sphinxcode{\sphinxupquote{skos:altLabel}}
\sphinxcode{\sphinxupquote{skos:definition}}
\sphinxcode{\sphinxupquote{skos:example}}
\sphinxcode{\sphinxupquote{skos:prefLabel}}
\sphinxcode{\sphinxupquote{skos:scopeNote}}
\begin{quote}

\index{IAO\_0000111@\spxentry{IAO\_0000111}!editor preferred label@\spxentry{editor preferred label}}\index{editor preferred label@\spxentry{editor preferred label}!IAO\_0000111@\spxentry{IAO\_0000111}}\ignorespaces \end{quote}


\subsubsection{IAO\_0000111 \sphinxhyphen{} editor preferred label}
\label{\detokenize{doc-IAO_0000111:iao-0000111-editor-preferred-label}}\label{\detokenize{doc-IAO_0000111:index-0}}\label{\detokenize{doc-IAO_0000111::doc}}
\begin{sphinxShadowBox}
\sphinxstyletopictitle{Label}

\sphinxAtStartPar
editor preferred label
\end{sphinxShadowBox}

\begin{sphinxShadowBox}
\sphinxstyletopictitle{Definition}

\sphinxAtStartPar
The concise, meaningful, and human\sphinxhyphen{}friendly name for a class or property preferred by the ontology developers. (US\sphinxhyphen{}English)
\end{sphinxShadowBox}

\begin{sphinxShadowBox}
\sphinxstyletopictitle{Definition source}

\sphinxAtStartPar
GROUP:OBI:\textless{}http://purl.obolibrary.org/obo/obi\textgreater{}
\end{sphinxShadowBox}

\begin{sphinxShadowBox}
\sphinxstyletopictitle{Term editor}

\sphinxAtStartPar
PERSON:Daniel Schober
\end{sphinxShadowBox}
\begin{quote}

\index{IAO\_0000112@\spxentry{IAO\_0000112}!example of usage@\spxentry{example of usage}}\index{example of usage@\spxentry{example of usage}!IAO\_0000112@\spxentry{IAO\_0000112}}\ignorespaces \end{quote}


\subsubsection{IAO\_0000112 \sphinxhyphen{} example of usage}
\label{\detokenize{doc-IAO_0000112:iao-0000112-example-of-usage}}\label{\detokenize{doc-IAO_0000112:index-0}}\label{\detokenize{doc-IAO_0000112::doc}}
\begin{sphinxShadowBox}
\sphinxstyletopictitle{Label}

\sphinxAtStartPar
example of usage
\end{sphinxShadowBox}

\begin{sphinxShadowBox}
\sphinxstyletopictitle{Definition}

\sphinxAtStartPar
A phrase describing how a term should be used and/or a citation to a work which uses it. May also include other kinds of examples that facilitate immediate understanding, such as widely know prototypes or instances of a class, or cases where a relation is said to hold.
\end{sphinxShadowBox}

\begin{sphinxShadowBox}
\sphinxstyletopictitle{Definition source}

\sphinxAtStartPar
GROUP:OBI:\textless{}http://purl.obolibrary.org/obo/obi\textgreater{}
\end{sphinxShadowBox}

\begin{sphinxShadowBox}
\sphinxstyletopictitle{Term editor}

\sphinxAtStartPar
PERSON:Daniel Schober
\end{sphinxShadowBox}
\begin{quote}

\index{IAO\_0000114@\spxentry{IAO\_0000114}!has curation status@\spxentry{has curation status}}\index{has curation status@\spxentry{has curation status}!IAO\_0000114@\spxentry{IAO\_0000114}}\ignorespaces \end{quote}


\subsubsection{IAO\_0000114 \sphinxhyphen{} has curation status}
\label{\detokenize{doc-IAO_0000114:iao-0000114-has-curation-status}}\label{\detokenize{doc-IAO_0000114:index-0}}\label{\detokenize{doc-IAO_0000114::doc}}
\begin{sphinxShadowBox}
\sphinxstyletopictitle{Label}

\sphinxAtStartPar
has curation status
\end{sphinxShadowBox}

\begin{sphinxShadowBox}
\sphinxstyletopictitle{Definition}

\sphinxAtStartPar
A specification of the state of the metadata for a term
\end{sphinxShadowBox}

\begin{sphinxShadowBox}
\sphinxstyletopictitle{Definition source}

\sphinxAtStartPar
OBI\_0000281
\end{sphinxShadowBox}

\begin{sphinxShadowBox}
\sphinxstyletopictitle{Term editor}

\sphinxAtStartPar
PERSON:Alan Ruttenberg

\sphinxAtStartPar
PERSON:Bill Bug

\sphinxAtStartPar
PERSON:Melanie Courtot
\end{sphinxShadowBox}
\begin{quote}

\index{IAO\_0000115@\spxentry{IAO\_0000115}!definition@\spxentry{definition}}\index{definition@\spxentry{definition}!IAO\_0000115@\spxentry{IAO\_0000115}}\ignorespaces \end{quote}


\subsubsection{IAO\_0000115 \sphinxhyphen{} definition}
\label{\detokenize{doc-IAO_0000115:iao-0000115-definition}}\label{\detokenize{doc-IAO_0000115:index-0}}\label{\detokenize{doc-IAO_0000115::doc}}
\begin{sphinxShadowBox}
\sphinxstyletopictitle{Label}

\sphinxAtStartPar
definition
\end{sphinxShadowBox}

\begin{sphinxShadowBox}
\sphinxstyletopictitle{Definition}

\sphinxAtStartPar
The official definition, explaining the meaning of a class or property. Shall be Aristotelian, formalized and normalized. Can be augmented with colloquial definitions.
\end{sphinxShadowBox}

\begin{sphinxShadowBox}
\sphinxstyletopictitle{Definition source}

\sphinxAtStartPar
GROUP:OBI:\textless{}http://purl.obolibrary.org/obo/obi\textgreater{}
\end{sphinxShadowBox}

\begin{sphinxShadowBox}
\sphinxstyletopictitle{Editor’s note}

\sphinxAtStartPar
2012\sphinxhyphen{}04\sphinxhyphen{}05:
Barry Smith

\sphinxAtStartPar
The official OBI definition, explaining the meaning of a class or property: ‘Shall be Aristotelian, formalized and normalized. Can be augmented with colloquial definitions’  is terrible.

\sphinxAtStartPar
Can you fix to something like:

\sphinxAtStartPar
A statement of necessary and sufficient conditions explaining the meaning of an expression referring to a class or property.

\sphinxAtStartPar
Alan Ruttenberg

\sphinxAtStartPar
Your proposed definition is a reasonable candidate, except that it is very common that necessary and sufficient conditions are not given. Mostly they are necessary, occasionally they are necessary and sufficient or just sufficient. Often they use terms that are not themselves defined and so they effectively can’t be evaluated by those criteria.

\sphinxAtStartPar
On the specifics of the proposed definition:

\sphinxAtStartPar
We don’t have definitions of ‘meaning’ or ‘expression’ or ‘property’. For ‘reference’ in the intended sense I think we use the term ‘denotation’. For ‘expression’, I think we you mean symbol, or identifier. For ‘meaning’ it differs for class and property. For class we want documentation that let’s the intended reader determine whether an entity is instance of the class, or not. For property we want documentation that let’s the intended reader determine, given a pair of potential relata, whether the assertion that the relation holds is true. The ‘intended reader’ part suggests that we also specify who, we expect, would be able to understand the definition, and also generalizes over human and computer reader to include textual and logical definition.

\sphinxAtStartPar
Personally, I am more comfortable weakening definition to documentation, with instructions as to what is desirable.

\sphinxAtStartPar
We also have the outstanding issue of how to aim different definitions to different audiences. A clinical audience reading chebi wants a different sort of definition documentation/definition from a chemistry trained audience, and similarly there is a need for a definition that is adequate for an ontologist to work with.
\end{sphinxShadowBox}

\begin{sphinxShadowBox}
\sphinxstyletopictitle{Term editor}

\sphinxAtStartPar
PERSON:Daniel Schober
\end{sphinxShadowBox}
\begin{quote}

\index{IAO\_0000116@\spxentry{IAO\_0000116}!editor note@\spxentry{editor note}}\index{editor note@\spxentry{editor note}!IAO\_0000116@\spxentry{IAO\_0000116}}\ignorespaces \end{quote}


\subsubsection{IAO\_0000116 \sphinxhyphen{} editor note}
\label{\detokenize{doc-IAO_0000116:iao-0000116-editor-note}}\label{\detokenize{doc-IAO_0000116:index-0}}\label{\detokenize{doc-IAO_0000116::doc}}
\begin{sphinxShadowBox}
\sphinxstyletopictitle{Label}

\sphinxAtStartPar
editor note
\end{sphinxShadowBox}

\begin{sphinxShadowBox}
\sphinxstyletopictitle{Definition}

\sphinxAtStartPar
An administrative note intended for its editor. It may not be included in the publication version of the ontology, so it should contain nothing necessary for end users to understand the ontology.
\end{sphinxShadowBox}

\begin{sphinxShadowBox}
\sphinxstyletopictitle{Definition source}

\sphinxAtStartPar
GROUP:OBI:\textless{}http://purl.obfoundry.org/obo/obi\textgreater{}
\end{sphinxShadowBox}

\begin{sphinxShadowBox}
\sphinxstyletopictitle{Term editor}

\sphinxAtStartPar
PERSON:Daniel Schober
\end{sphinxShadowBox}
\begin{quote}

\index{IAO\_0000117@\spxentry{IAO\_0000117}!term editor@\spxentry{term editor}}\index{term editor@\spxentry{term editor}!IAO\_0000117@\spxentry{IAO\_0000117}}\ignorespaces \end{quote}


\subsubsection{IAO\_0000117 \sphinxhyphen{} term editor}
\label{\detokenize{doc-IAO_0000117:iao-0000117-term-editor}}\label{\detokenize{doc-IAO_0000117:index-0}}\label{\detokenize{doc-IAO_0000117::doc}}
\begin{sphinxShadowBox}
\sphinxstyletopictitle{Label}

\sphinxAtStartPar
term editor
\end{sphinxShadowBox}

\begin{sphinxShadowBox}
\sphinxstyletopictitle{Definition}

\sphinxAtStartPar
Name of editor entering the term in the file. The term editor is a point of contact for information regarding the term. The term editor may be, but is not always, the author of the definition, which may have been worked upon by several people
\end{sphinxShadowBox}

\begin{sphinxShadowBox}
\sphinxstyletopictitle{Definition source}

\sphinxAtStartPar
GROUP:OBI:\textless{}http://purl.obolibrary.org/obo/obi\textgreater{}
\end{sphinxShadowBox}

\begin{sphinxShadowBox}
\sphinxstyletopictitle{Editor’s note}

\sphinxAtStartPar
20110707, MC: label update to term editor and definition modified accordingly. See \sphinxurl{https://github.com/information-artifact-ontology/IAO/issues/115}.
\end{sphinxShadowBox}

\begin{sphinxShadowBox}
\sphinxstyletopictitle{Term editor}

\sphinxAtStartPar
PERSON:Daniel Schober
\end{sphinxShadowBox}
\begin{quote}

\index{IAO\_0000118@\spxentry{IAO\_0000118}!alternative term@\spxentry{alternative term}}\index{alternative term@\spxentry{alternative term}!IAO\_0000118@\spxentry{IAO\_0000118}}\ignorespaces \end{quote}


\subsubsection{IAO\_0000118 \sphinxhyphen{} alternative term}
\label{\detokenize{doc-IAO_0000118:iao-0000118-alternative-term}}\label{\detokenize{doc-IAO_0000118:index-0}}\label{\detokenize{doc-IAO_0000118::doc}}
\begin{sphinxShadowBox}
\sphinxstyletopictitle{Label}

\sphinxAtStartPar
alternative term
\end{sphinxShadowBox}

\begin{sphinxShadowBox}
\sphinxstyletopictitle{Definition}

\sphinxAtStartPar
An alternative name for a class or property which means the same thing as the preferred name (semantically equivalent)
\end{sphinxShadowBox}

\begin{sphinxShadowBox}
\sphinxstyletopictitle{Definition source}

\sphinxAtStartPar
GROUP:OBI:\textless{}http://purl.obolibrary.org/obo/obi\textgreater{}
\end{sphinxShadowBox}

\begin{sphinxShadowBox}
\sphinxstyletopictitle{Term editor}

\sphinxAtStartPar
PERSON:Daniel Schober
\end{sphinxShadowBox}
\begin{quote}

\index{IAO\_0000119@\spxentry{IAO\_0000119}!definition source@\spxentry{definition source}}\index{definition source@\spxentry{definition source}!IAO\_0000119@\spxentry{IAO\_0000119}}\ignorespaces \end{quote}


\subsubsection{IAO\_0000119 \sphinxhyphen{} definition source}
\label{\detokenize{doc-IAO_0000119:iao-0000119-definition-source}}\label{\detokenize{doc-IAO_0000119:index-0}}\label{\detokenize{doc-IAO_0000119::doc}}
\begin{sphinxShadowBox}
\sphinxstyletopictitle{Label}

\sphinxAtStartPar
definition source
\end{sphinxShadowBox}

\begin{sphinxShadowBox}
\sphinxstyletopictitle{Definition}

\sphinxAtStartPar
Formal citation, e.g. identifier in external database to indicate / attribute source(s) for the definition. Free text indicate / attribute source(s) for the definition. EXAMPLE: Author Name, URI, MeSH Term C04, PUBMED ID, Wiki uri on 31.01.2007
\end{sphinxShadowBox}

\begin{sphinxShadowBox}
\sphinxstyletopictitle{Definition source}

\sphinxAtStartPar
Discussion on obo\sphinxhyphen{}discuss mailing\sphinxhyphen{}list, see \sphinxurl{http://bit.ly/hgm99w}

\sphinxAtStartPar
GROUP:OBI:\textless{}http://purl.obolibrary.org/obo/obi\textgreater{}
\end{sphinxShadowBox}

\begin{sphinxShadowBox}
\sphinxstyletopictitle{Term editor}

\sphinxAtStartPar
PERSON:Daniel Schober
\end{sphinxShadowBox}
\begin{quote}

\index{IAO\_0000232@\spxentry{IAO\_0000232}!curator note@\spxentry{curator note}}\index{curator note@\spxentry{curator note}!IAO\_0000232@\spxentry{IAO\_0000232}}\ignorespaces \end{quote}


\subsubsection{IAO\_0000232 \sphinxhyphen{} curator note}
\label{\detokenize{doc-IAO_0000232:iao-0000232-curator-note}}\label{\detokenize{doc-IAO_0000232:index-0}}\label{\detokenize{doc-IAO_0000232::doc}}
\begin{sphinxShadowBox}
\sphinxstyletopictitle{Label}

\sphinxAtStartPar
curator note
\end{sphinxShadowBox}

\begin{sphinxShadowBox}
\sphinxstyletopictitle{Definition}

\sphinxAtStartPar
An administrative note of use for a curator but of no use for a user
\end{sphinxShadowBox}

\begin{sphinxShadowBox}
\sphinxstyletopictitle{Term editor}

\sphinxAtStartPar
PERSON:Alan Ruttenberg
\end{sphinxShadowBox}
\begin{quote}

\index{IAO\_0000233@\spxentry{IAO\_0000233}!term tracker item@\spxentry{term tracker item}}\index{term tracker item@\spxentry{term tracker item}!IAO\_0000233@\spxentry{IAO\_0000233}}\ignorespaces \end{quote}


\subsubsection{IAO\_0000233 \sphinxhyphen{} term tracker item}
\label{\detokenize{doc-IAO_0000233:iao-0000233-term-tracker-item}}\label{\detokenize{doc-IAO_0000233:index-0}}\label{\detokenize{doc-IAO_0000233::doc}}
\begin{sphinxShadowBox}
\sphinxstyletopictitle{Label}

\sphinxAtStartPar
term tracker item
\end{sphinxShadowBox}

\begin{sphinxShadowBox}
\sphinxstyletopictitle{Definition}

\sphinxAtStartPar
An IRI or similar locator for a request or discussion of an ontology term.
\end{sphinxShadowBox}

\begin{sphinxShadowBox}
\sphinxstyletopictitle{Definition source}

\sphinxAtStartPar
Person: Jie Zheng, Chris Stoeckert, Alan Ruttenberg
\end{sphinxShadowBox}

\begin{sphinxShadowBox}
\sphinxstyletopictitle{Example}

\sphinxAtStartPar
the URI for an OBI Terms ticket at sourceforge, such as \sphinxurl{https://sourceforge.net/p/obi/obi-terms/772/}
\end{sphinxShadowBox}

\begin{sphinxShadowBox}
\sphinxstyletopictitle{Term editor}

\sphinxAtStartPar
Person: Jie Zheng, Chris Stoeckert, Alan Ruttenberg
\end{sphinxShadowBox}
\begin{quote}

\index{IAO\_0000412@\spxentry{IAO\_0000412}!imported from@\spxentry{imported from}}\index{imported from@\spxentry{imported from}!IAO\_0000412@\spxentry{IAO\_0000412}}\ignorespaces \end{quote}


\subsubsection{IAO\_0000412 \sphinxhyphen{} imported from}
\label{\detokenize{doc-IAO_0000412:iao-0000412-imported-from}}\label{\detokenize{doc-IAO_0000412:index-0}}\label{\detokenize{doc-IAO_0000412::doc}}
\begin{sphinxShadowBox}
\sphinxstyletopictitle{Label}

\sphinxAtStartPar
imported from
\end{sphinxShadowBox}

\begin{sphinxShadowBox}
\sphinxstyletopictitle{Definition}

\sphinxAtStartPar
For external terms/classes, the ontology from which the term was imported
\end{sphinxShadowBox}

\begin{sphinxShadowBox}
\sphinxstyletopictitle{Definition source}

\sphinxAtStartPar
GROUP:OBI:\textless{}http://purl.obolibrary.org/obo/obi\textgreater{}
\end{sphinxShadowBox}

\begin{sphinxShadowBox}
\sphinxstyletopictitle{Term editor}

\sphinxAtStartPar
PERSON:Melanie Courtot

\sphinxAtStartPar
PERSON:Alan Ruttenberg
\end{sphinxShadowBox}
\begin{quote}

\index{ORG\_1000001@\spxentry{ORG\_1000001}!vivo 1 ontology reference@\spxentry{vivo 1 ontology reference}}\index{vivo 1 ontology reference@\spxentry{vivo 1 ontology reference}!ORG\_1000001@\spxentry{ORG\_1000001}}\ignorespaces \end{quote}


\subsubsection{ORG\_1000001 \sphinxhyphen{} vivo 1 ontology reference}
\label{\detokenize{doc-ORG_1000001:org-1000001-vivo-1-ontology-reference}}\label{\detokenize{doc-ORG_1000001:index-0}}\label{\detokenize{doc-ORG_1000001::doc}}
\begin{sphinxShadowBox}
\sphinxstyletopictitle{Label}

\sphinxAtStartPar
vivo 1 ontology reference
\end{sphinxShadowBox}

\begin{sphinxShadowBox}
\sphinxstyletopictitle{Alternate name}

\sphinxAtStartPar
in vivo 1
\end{sphinxShadowBox}

\begin{sphinxShadowBox}
\sphinxstyletopictitle{Definition}

\sphinxAtStartPar
The term in the original VIVO ontology that is most like the annotated term.
\end{sphinxShadowBox}

\begin{sphinxShadowBox}
\sphinxstyletopictitle{Definition source}

\sphinxAtStartPar
Michael Conlon \sphinxurl{https://orcid.org/0000-0002-1304-8447}
\end{sphinxShadowBox}

\begin{sphinxShadowBox}
\sphinxstyletopictitle{Example}

\sphinxAtStartPar
The organization class in the original VIVO ontology is most like the organization class in the VIVO Organization Ontology.  They have different superclasses, so they are not the same.
\end{sphinxShadowBox}

\begin{sphinxShadowBox}
\sphinxstyletopictitle{Term editor}

\sphinxAtStartPar
Michael Conlon \sphinxurl{https://orcid.org/0000-0002-1304-8447}
\end{sphinxShadowBox}
\begin{quote}

\index{created@\spxentry{created}!No label@\spxentry{No label}}\index{No label@\spxentry{No label}!created@\spxentry{created}}\ignorespaces \end{quote}


\subsubsection{created \sphinxhyphen{} No label}
\label{\detokenize{doc-created:created-no-label}}\label{\detokenize{doc-created:index-0}}\label{\detokenize{doc-created::doc}}\begin{quote}

\index{creator@\spxentry{creator}!No label@\spxentry{No label}}\index{No label@\spxentry{No label}!creator@\spxentry{creator}}\ignorespaces \end{quote}


\subsubsection{creator \sphinxhyphen{} No label}
\label{\detokenize{doc-creator:creator-no-label}}\label{\detokenize{doc-creator:index-0}}\label{\detokenize{doc-creator::doc}}\begin{quote}

\index{description@\spxentry{description}!No label@\spxentry{No label}}\index{No label@\spxentry{No label}!description@\spxentry{description}}\ignorespaces \end{quote}


\subsubsection{description \sphinxhyphen{} No label}
\label{\detokenize{doc-description:description-no-label}}\label{\detokenize{doc-description:index-0}}\label{\detokenize{doc-description::doc}}\begin{quote}

\index{title@\spxentry{title}!No label@\spxentry{No label}}\index{No label@\spxentry{No label}!title@\spxentry{title}}\ignorespaces \end{quote}


\subsubsection{title \sphinxhyphen{} No label}
\label{\detokenize{doc-title:title-no-label}}\label{\detokenize{doc-title:index-0}}\label{\detokenize{doc-title::doc}}\begin{quote}

\index{license@\spxentry{license}!No label@\spxentry{No label}}\index{No label@\spxentry{No label}!license@\spxentry{license}}\ignorespaces \end{quote}


\subsubsection{license \sphinxhyphen{} No label}
\label{\detokenize{doc-license:license-no-label}}\label{\detokenize{doc-license:index-0}}\label{\detokenize{doc-license::doc}}\begin{quote}

\index{inverseOf@\spxentry{inverseOf}!No label@\spxentry{No label}}\index{No label@\spxentry{No label}!inverseOf@\spxentry{inverseOf}}\ignorespaces \end{quote}


\subsubsection{inverseOf \sphinxhyphen{} No label}
\label{\detokenize{doc-inverseOf:inverseof-no-label}}\label{\detokenize{doc-inverseOf:index-0}}\label{\detokenize{doc-inverseOf::doc}}

\chapter{Classes}
\label{\detokenize{classes:classes}}\label{\detokenize{classes::doc}}
\sphinxAtStartPar
\sphinxstylestrong{Classes} are collections of \sphinxstylestrong{individuals}.  A university, a nonprofit, building,
and a role in a project are
all individuals.  Classes of individuals are defined by specifying members
(extension) or more frequently
in scholarship, by specifying conditions (intension).

\sphinxAtStartPar
In ontologies, and using {\hyperref[\detokenize{glossary:glossary}]{\sphinxcrossref{\DUrole{std,std-ref}{OWL}}}}, individuals
instances of classes. Classes are arranged
in subsumption hierarchies, indicating that individuals in one class are members of some
higher class.  For example, a cat is a mammal.  Your cat is an instance of the
class cat.  All cats are mammals.  We say cat is a subclass of mammal.  We can infer
that all individuals that are
cats are mammals.  Your cat is a cat, so we can infer your cat is a mammal.


\section{All Classes}
\label{\detokenize{classes:all-classes}}
\sphinxAtStartPar
See {\hyperref[\detokenize{classes:table-5}]{\sphinxcrossref{Table 5}}}.


\begin{savenotes}\sphinxatlongtablestart\begin{longtable}[c]{|*{2}{\X{1}{2}|}}
\sphinxthelongtablecaptionisattop
\caption{Table 5 Classes\strut}\label{\detokenize{classes:id1}}\label{\detokenize{classes:table-5}}\\*[\sphinxlongtablecapskipadjust]
\hline
\sphinxstyletheadfamily 
\sphinxAtStartPar
Term ID \sphinxhyphen{} Label
&\sphinxstyletheadfamily 
\sphinxAtStartPar
Definition
\\
\hline
\endfirsthead

\multicolumn{2}{c}%
{\makebox[0pt]{\sphinxtablecontinued{\tablename\ \thetable{} \textendash{} continued from previous page}}}\\
\hline
\sphinxstyletheadfamily 
\sphinxAtStartPar
Term ID \sphinxhyphen{} Label
&\sphinxstyletheadfamily 
\sphinxAtStartPar
Definition
\\
\hline
\endhead

\hline
\multicolumn{2}{r}{\makebox[0pt][r]{\sphinxtablecontinued{continues on next page}}}\\
\endfoot

\endlastfoot

\sphinxAtStartPar
{\hyperref[\detokenize{doc-BFO_0000001::doc}]{\sphinxcrossref{\DUrole{doc}{BFO\_0000001 \sphinxhyphen{} entity}}}}
&
\sphinxAtStartPar
The fundamental thing that has existence. All

\sphinxAtStartPar
things that exist are entities.
\\
\hline
\sphinxAtStartPar
{\hyperref[\detokenize{doc-BFO_0000002::doc}]{\sphinxcrossref{\DUrole{doc}{BFO\_0000002 \sphinxhyphen{} continuant}}}}
&
\sphinxAtStartPar
An entity which has existence in time
\\
\hline
\sphinxAtStartPar
{\hyperref[\detokenize{doc-BFO_0000003::doc}]{\sphinxcrossref{\DUrole{doc}{BFO\_0000003 \sphinxhyphen{} occurrent}}}}
&
\sphinxAtStartPar
An entity which occurs in time
\\
\hline
\sphinxAtStartPar
{\hyperref[\detokenize{doc-BFO_0000004::doc}]{\sphinxcrossref{\DUrole{doc}{BFO\_0000004 \sphinxhyphen{} independent continuant}}}}
&
\sphinxAtStartPar
B is an independent continuant = Def. b is a

\sphinxAtStartPar
continuant which is such that there is no c and no

\sphinxAtStartPar
t such that b s\sphinxhyphen{}depends\_on c at t. (axiom label in

\sphinxAtStartPar
BFO2 Reference: {[}017\sphinxhyphen{}002{]})
\\
\hline
\sphinxAtStartPar
{\hyperref[\detokenize{doc-BFO_0000008::doc}]{\sphinxcrossref{\DUrole{doc}{BFO\_0000008 \sphinxhyphen{} temporal region}}}}
&
\sphinxAtStartPar
An occurent which is some part of time
\\
\hline
\sphinxAtStartPar
{\hyperref[\detokenize{doc-BFO_0000015::doc}]{\sphinxcrossref{\DUrole{doc}{BFO\_0000015 \sphinxhyphen{} process}}}}
&
\sphinxAtStartPar
P is a process = Def. p is an occurrent that has

\sphinxAtStartPar
temporal proper parts and for some time t, p

\sphinxAtStartPar
s\sphinxhyphen{}depends\_on some material entity at t. (axiom

\sphinxAtStartPar
label in BFO2 Reference: {[}083\sphinxhyphen{}003{]}) {[}has axiom

\sphinxAtStartPar
label:

\sphinxAtStartPar
\sphinxurl{http://purl.obolibrary.org/obo/bfo/axiom/083-003}{]}
\\
\hline
\sphinxAtStartPar
{\hyperref[\detokenize{doc-BFO_0000016::doc}]{\sphinxcrossref{\DUrole{doc}{BFO\_0000016 \sphinxhyphen{} disposition}}}}
&
\sphinxAtStartPar
A realizable entity that presents in a continuant
\\
\hline
\sphinxAtStartPar
{\hyperref[\detokenize{doc-BFO_0000017::doc}]{\sphinxcrossref{\DUrole{doc}{BFO\_0000017 \sphinxhyphen{} realizable entity}}}}
&
\sphinxAtStartPar
To say that b is a realizable entity is to say

\sphinxAtStartPar
that b is a specifically dependent continuant that

\sphinxAtStartPar
inheres in some independent continuant which is

\sphinxAtStartPar
not a spatial region and is of a type instances of

\sphinxAtStartPar
which are realized in processes of a correlated

\sphinxAtStartPar
type. (axiom label in BFO2 Reference: {[}058\sphinxhyphen{}002{]})
\\
\hline
\sphinxAtStartPar
{\hyperref[\detokenize{doc-BFO_0000019::doc}]{\sphinxcrossref{\DUrole{doc}{BFO\_0000019 \sphinxhyphen{} quality}}}}
&
\sphinxAtStartPar
A quality is a specifically dependent continuant

\sphinxAtStartPar
that, in contrast to roles and dispositions, does

\sphinxAtStartPar
not require any further process in order to be

\sphinxAtStartPar
realized.
\\
\hline
\sphinxAtStartPar
{\hyperref[\detokenize{doc-BFO_0000020::doc}]{\sphinxcrossref{\DUrole{doc}{BFO\_0000020 \sphinxhyphen{} specifically dependent continuant}}}}
&
\sphinxAtStartPar
B is a specifically dependent continuant = Def. b

\sphinxAtStartPar
is a continuant \& there is some independent

\sphinxAtStartPar
continuant c which is not a spatial region and

\sphinxAtStartPar
which is such that b s\sphinxhyphen{}depends\_on c at every time

\sphinxAtStartPar
t during the course of b’s existence. (axiom label

\sphinxAtStartPar
in BFO2 Reference: {[}050\sphinxhyphen{}003{]}) {[}has axiom label:

\sphinxAtStartPar
\sphinxurl{http://purl.obolibrary.org/obo/bfo/axiom/050-003}{]}
\\
\hline
\sphinxAtStartPar
{\hyperref[\detokenize{doc-BFO_0000023::doc}]{\sphinxcrossref{\DUrole{doc}{BFO\_0000023 \sphinxhyphen{} role}}}}
&
\sphinxAtStartPar
B is a role means: b is a realizable entity \& b

\sphinxAtStartPar
exists because there is some single bearer that is

\sphinxAtStartPar
in some special physical, social, or institutional

\sphinxAtStartPar
set of circumstances in which this bearer does not

\sphinxAtStartPar
have to be\& b is not such that, if it ceases to

\sphinxAtStartPar
exist, then the physical make\sphinxhyphen{}up of the bearer is

\sphinxAtStartPar
thereby changed.
\\
\hline
\sphinxAtStartPar
{\hyperref[\detokenize{doc-BFO_0000029::doc}]{\sphinxcrossref{\DUrole{doc}{BFO\_0000029 \sphinxhyphen{} site}}}}
&
\sphinxAtStartPar
B is a site means: b is a three\sphinxhyphen{}dimensional

\sphinxAtStartPar
immaterial entity that is (partially or wholly)

\sphinxAtStartPar
bounded by a material entity or it is a

\sphinxAtStartPar
three\sphinxhyphen{}dimensional immaterial part thereof. (axiom

\sphinxAtStartPar
label in BFO2 Reference: {[}034\sphinxhyphen{}002{]})
\\
\hline
\sphinxAtStartPar
{\hyperref[\detokenize{doc-BFO_0000031::doc}]{\sphinxcrossref{\DUrole{doc}{BFO\_0000031 \sphinxhyphen{} generically dependent continuant}}}}
&
\sphinxAtStartPar
B is a generically dependent continuant = Def. b

\sphinxAtStartPar
is a continuant that g\sphinxhyphen{}depends\_on one or more

\sphinxAtStartPar
other entities. (axiom label in BFO2 Reference:

\sphinxAtStartPar
{[}074\sphinxhyphen{}001{]}) {[}has axiom label:

\sphinxAtStartPar
\sphinxurl{http://purl.obolibrary.org/obo/bfo/axiom/074-001}{]}
\\
\hline
\sphinxAtStartPar
{\hyperref[\detokenize{doc-BFO_0000035::doc}]{\sphinxcrossref{\DUrole{doc}{BFO\_0000035 \sphinxhyphen{} process boundary}}}}
&
\sphinxAtStartPar
P is a process boundary =Def. p is a temporal part

\sphinxAtStartPar
of a process \& p has no proper temporal parts.

\sphinxAtStartPar
(axiom label in BFO2 Reference: {[}084\sphinxhyphen{}001{]}) {[}has

\sphinxAtStartPar
axiom label:

\sphinxAtStartPar
\sphinxurl{http://purl.obolibrary.org/obo/bfo/axiom/084-001}{]}
\\
\hline
\sphinxAtStartPar
{\hyperref[\detokenize{doc-BFO_0000038::doc}]{\sphinxcrossref{\DUrole{doc}{BFO\_0000038 \sphinxhyphen{} one\sphinxhyphen{}dimensional temporal region}}}}
&
\sphinxAtStartPar
A one\sphinxhyphen{}dimensional temporal region is a temporal

\sphinxAtStartPar
region that is extended. (axiom label in BFO2

\sphinxAtStartPar
Reference: {[}103\sphinxhyphen{}001{]})
\\
\hline
\sphinxAtStartPar
{\hyperref[\detokenize{doc-BFO_0000040::doc}]{\sphinxcrossref{\DUrole{doc}{BFO\_0000040 \sphinxhyphen{} material entity}}}}
&
\sphinxAtStartPar
A material entity is an independent continuant

\sphinxAtStartPar
that has some portion of matter as proper or

\sphinxAtStartPar
improper continuant part. (axiom label in BFO2

\sphinxAtStartPar
Reference: {[}019\sphinxhyphen{}002{]})
\\
\hline
\sphinxAtStartPar
{\hyperref[\detokenize{doc-BFO_0000141::doc}]{\sphinxcrossref{\DUrole{doc}{BFO\_0000141 \sphinxhyphen{} immaterial entity}}}}
&
\sphinxAtStartPar
An immaterial entity is the boundary or interior

\sphinxAtStartPar
of a material entity
\\
\hline
\sphinxAtStartPar
{\hyperref[\detokenize{doc-BFO_0000148::doc}]{\sphinxcrossref{\DUrole{doc}{BFO\_0000148 \sphinxhyphen{} zero\sphinxhyphen{}dimensional temporal region}}}}
&
\sphinxAtStartPar
A temporal region of no duration.
\\
\hline
\sphinxAtStartPar
{\hyperref[\detokenize{doc-IAO_0000030::doc}]{\sphinxcrossref{\DUrole{doc}{IAO\_0000030 \sphinxhyphen{} information content entity}}}}
&
\sphinxAtStartPar
A generically dependent continuant that is about

\sphinxAtStartPar
some thing.
\\
\hline
\sphinxAtStartPar
{\hyperref[\detokenize{doc-IAO_0000422::doc}]{\sphinxcrossref{\DUrole{doc}{IAO\_0000422 \sphinxhyphen{} postal address}}}}
&
\sphinxAtStartPar
A textual entity that is used as directive to

\sphinxAtStartPar
deliver something to a person, or organization
\\
\hline
\sphinxAtStartPar
{\hyperref[\detokenize{doc-IAO_0000429::doc}]{\sphinxcrossref{\DUrole{doc}{IAO\_0000429 \sphinxhyphen{} email address}}}}
&
\sphinxAtStartPar
A designation used to deliver email to a

\sphinxAtStartPar
recipient.  Typically has an associated email

\sphinxAtStartPar
representation
\\
\hline
\sphinxAtStartPar
{\hyperref[\detokenize{doc-IAO_0000578::doc}]{\sphinxcrossref{\DUrole{doc}{IAO\_0000578 \sphinxhyphen{} centrally registered identifier}}}}
&
\sphinxAtStartPar
An information content entity that consists of a

\sphinxAtStartPar
CRID symbol and additional information about the

\sphinxAtStartPar
CRID registry to which it belongs.
\\
\hline
\sphinxAtStartPar
{\hyperref[\detokenize{doc-IAO_0020000::doc}]{\sphinxcrossref{\DUrole{doc}{IAO\_0020000 \sphinxhyphen{} identifier}}}}
&
\sphinxAtStartPar
An identifier is an information content entity

\sphinxAtStartPar
that is the outcome of a dubbing process and is

\sphinxAtStartPar
used to refer to one instance of entity shared by

\sphinxAtStartPar
a group of people to refer to that individual

\sphinxAtStartPar
entity.
\\
\hline
\sphinxAtStartPar
{\hyperref[\detokenize{doc-IAO_0022003::doc}]{\sphinxcrossref{\DUrole{doc}{IAO\_0022003 \sphinxhyphen{} crossref funder identifier}}}}
&
\sphinxAtStartPar
An identifier assigned by CrossRef to an

\sphinxAtStartPar
organization which has funded a project resulting

\sphinxAtStartPar
in a published work
\\
\hline
\sphinxAtStartPar
{\hyperref[\detokenize{doc-IAO_0022006::doc}]{\sphinxcrossref{\DUrole{doc}{IAO\_0022006 \sphinxhyphen{} dbpedia identifier}}}}
&
\sphinxAtStartPar
A URL used by DBpedia to identify an entity
\\
\hline
\sphinxAtStartPar
{\hyperref[\detokenize{doc-IAO_0022010::doc}]{\sphinxcrossref{\DUrole{doc}{IAO\_0022010 \sphinxhyphen{} global research organization identifier}}}}
&
\sphinxAtStartPar
An identifier assigned and managed by Digital

\sphinxAtStartPar
Science for the purpose of denoting research

\sphinxAtStartPar
organizations
\\
\hline
\sphinxAtStartPar
{\hyperref[\detokenize{doc-IAO_0022014::doc}]{\sphinxcrossref{\DUrole{doc}{IAO\_0022014 \sphinxhyphen{} international standard name identifier}}}}
&
\sphinxAtStartPar
An identifier for persons and organizations which

\sphinxAtStartPar
may be assigned by matching algorithms based on

\sphinxAtStartPar
records provided by publishers
\\
\hline
\sphinxAtStartPar
{\hyperref[\detokenize{doc-IAO_0022022::doc}]{\sphinxcrossref{\DUrole{doc}{IAO\_0022022 \sphinxhyphen{} research organization registry identifier}}}}
&
\sphinxAtStartPar
An identifier assigned by ROR to research

\sphinxAtStartPar
organizations in the world
\\
\hline
\sphinxAtStartPar
{\hyperref[\detokenize{doc-IAO_0022027::doc}]{\sphinxcrossref{\DUrole{doc}{IAO\_0022027 \sphinxhyphen{} wikidata q number}}}}
&
\sphinxAtStartPar
QID (or Q number) is the unique identifier of a

\sphinxAtStartPar
data item on Wikidata, comprising the letter “Q”

\sphinxAtStartPar
followed by one or more digits.
\\
\hline
\sphinxAtStartPar
{\hyperref[\detokenize{doc-IAO_0022057::doc}]{\sphinxcrossref{\DUrole{doc}{IAO\_0022057 \sphinxhyphen{} ringgold identifier}}}}
&
\sphinxAtStartPar
The Ringgold Identifier is a unique numerical

\sphinxAtStartPar
identifier applied to organizations in the

\sphinxAtStartPar
scholarly supply chain
\\
\hline
\sphinxAtStartPar
{\hyperref[\detokenize{doc-NCBITaxon_9606::doc}]{\sphinxcrossref{\DUrole{doc}{NCBITaxon\_9606 \sphinxhyphen{} Homo sapiens}}}}
&
\sphinxAtStartPar
The species of bipedal primates to which moden

\sphinxAtStartPar
humans belong
\\
\hline
\sphinxAtStartPar
{\hyperref[\detokenize{doc-ORG_0000001::doc}]{\sphinxcrossref{\DUrole{doc}{ORG\_0000001 \sphinxhyphen{} organization}}}}
&
\sphinxAtStartPar
A group of people recognized as such by people

\sphinxAtStartPar
outside the group.
\\
\hline
\sphinxAtStartPar
{\hyperref[\detokenize{doc-ORG_0000002::doc}]{\sphinxcrossref{\DUrole{doc}{ORG\_0000002 \sphinxhyphen{} government organization}}}}
&
\sphinxAtStartPar
An organization which is the body of persons that

\sphinxAtStartPar
constitutes the governing authority of a political

\sphinxAtStartPar
unit
\\
\hline
\sphinxAtStartPar
{\hyperref[\detokenize{doc-ORG_0000003::doc}]{\sphinxcrossref{\DUrole{doc}{ORG\_0000003 \sphinxhyphen{} company}}}}
&
\sphinxAtStartPar
A legal entity of associated persons created for a

\sphinxAtStartPar
specific purpose, typically commercial, in which

\sphinxAtStartPar
excess revenue may be distributed to the company’s

\sphinxAtStartPar
owners.
\\
\hline
\sphinxAtStartPar
{\hyperref[\detokenize{doc-ORG_0000004::doc}]{\sphinxcrossref{\DUrole{doc}{ORG\_0000004 \sphinxhyphen{} nonprofit organization}}}}
&
\sphinxAtStartPar
A legal entity of associated persons created for a

\sphinxAtStartPar
specific purpose, typically a mission, in which

\sphinxAtStartPar
excess revenue is reinvested to serve the entity’s

\sphinxAtStartPar
mission
\\
\hline
\sphinxAtStartPar
{\hyperref[\detokenize{doc-ORG_0000005::doc}]{\sphinxcrossref{\DUrole{doc}{ORG\_0000005 \sphinxhyphen{} informal organization}}}}
&
\sphinxAtStartPar
A group of people recognized as such by people

\sphinxAtStartPar
outside the group. Without legal standing.
\\
\hline
\sphinxAtStartPar
{\hyperref[\detokenize{doc-ORG_0000006::doc}]{\sphinxcrossref{\DUrole{doc}{ORG\_0000006 \sphinxhyphen{} organization part}}}}
&
\sphinxAtStartPar
An organization which exists as part of another

\sphinxAtStartPar
organization.  Implies a part\_of relationship to

\sphinxAtStartPar
another organization
\\
\hline
\sphinxAtStartPar
{\hyperref[\detokenize{doc-ORG_0000007::doc}]{\sphinxcrossref{\DUrole{doc}{ORG\_0000007 \sphinxhyphen{} university disposition}}}}
&
\sphinxAtStartPar
A disposition to award academic degrees and

\sphinxAtStartPar
conduct research in a variety of academic

\sphinxAtStartPar
disciplines
\\
\hline
\sphinxAtStartPar
{\hyperref[\detokenize{doc-ORG_0000008::doc}]{\sphinxcrossref{\DUrole{doc}{ORG\_0000008 \sphinxhyphen{} association disposition}}}}
&
\sphinxAtStartPar
A disposition to organize organizations or

\sphinxAtStartPar
individuals along and industry or academic lines
\\
\hline
\sphinxAtStartPar
{\hyperref[\detokenize{doc-ORG_0000009::doc}]{\sphinxcrossref{\DUrole{doc}{ORG\_0000009 \sphinxhyphen{} consortium disposition}}}}
&
\sphinxAtStartPar
A disposition to organize organizations along

\sphinxAtStartPar
industry or academic lines
\\
\hline
\sphinxAtStartPar
{\hyperref[\detokenize{doc-ORG_0000010::doc}]{\sphinxcrossref{\DUrole{doc}{ORG\_0000010 \sphinxhyphen{} service provider disposition}}}}
&
\sphinxAtStartPar
A disposition to provide service with or without a

\sphinxAtStartPar
fee
\\
\hline
\sphinxAtStartPar
{\hyperref[\detokenize{doc-ORG_0000011::doc}]{\sphinxcrossref{\DUrole{doc}{ORG\_0000011 \sphinxhyphen{} laboratory service provider disposition}}}}
&
\sphinxAtStartPar
A disposition to provide laboratory services.  In

\sphinxAtStartPar
the US, organization parts in universities that

\sphinxAtStartPar
have a disposition to provide laboratory services

\sphinxAtStartPar
to others are called core laboratories
\\
\hline
\sphinxAtStartPar
{\hyperref[\detokenize{doc-ORG_0000012::doc}]{\sphinxcrossref{\DUrole{doc}{ORG\_0000012 \sphinxhyphen{} extension provider disposition}}}}
&
\sphinxAtStartPar
A disposition to provide extension services,

\sphinxAtStartPar
typically in agriculture.  Extension provides

\sphinxAtStartPar
access to university research findings and advice

\sphinxAtStartPar
to agriculturalists.
\\
\hline
\sphinxAtStartPar
{\hyperref[\detokenize{doc-ORG_0000013::doc}]{\sphinxcrossref{\DUrole{doc}{ORG\_0000013 \sphinxhyphen{} technology transfer disposition}}}}
&
\sphinxAtStartPar
A disposition to create licenses for intellectual

\sphinxAtStartPar
property for use by these beyond the creators
\\
\hline
\sphinxAtStartPar
{\hyperref[\detokenize{doc-ORG_0000014::doc}]{\sphinxcrossref{\DUrole{doc}{ORG\_0000014 \sphinxhyphen{} philanthropy disposition}}}}
&
\sphinxAtStartPar
A disposition to donate charitable causes,

\sphinxAtStartPar
sometimes in the form of grants involving

\sphinxAtStartPar
contracts regarding the use of the donated funds

\sphinxAtStartPar
or effort.
\\
\hline
\sphinxAtStartPar
{\hyperref[\detokenize{doc-ORG_0000015::doc}]{\sphinxcrossref{\DUrole{doc}{ORG\_0000015 \sphinxhyphen{} funding disposition}}}}
&
\sphinxAtStartPar
A disposition to fund proposals, often is response

\sphinxAtStartPar
to a call for proposals by the entity with the

\sphinxAtStartPar
funding disposition
\\
\hline
\sphinxAtStartPar
{\hyperref[\detokenize{doc-ORG_0000016::doc}]{\sphinxcrossref{\DUrole{doc}{ORG\_0000016 \sphinxhyphen{} health care service provider disposition}}}}
&
\sphinxAtStartPar
A disposition to provider health care to humans
\\
\hline
\sphinxAtStartPar
{\hyperref[\detokenize{doc-ORG_0000017::doc}]{\sphinxcrossref{\DUrole{doc}{ORG\_0000017 \sphinxhyphen{} hospital service provider disposition}}}}
&
\sphinxAtStartPar
A disposition to provide hospital\sphinxhyphen{}based health

\sphinxAtStartPar
care services to humans
\\
\hline
\sphinxAtStartPar
{\hyperref[\detokenize{doc-ORG_0000018::doc}]{\sphinxcrossref{\DUrole{doc}{ORG\_0000018 \sphinxhyphen{} archive disposition}}}}
&
\sphinxAtStartPar
A disposition to collect, store, and provide

\sphinxAtStartPar
access to inanimate material entities, and/or

\sphinxAtStartPar
information content entitites
\\
\hline
\sphinxAtStartPar
{\hyperref[\detokenize{doc-ORG_0000019::doc}]{\sphinxcrossref{\DUrole{doc}{ORG\_0000019 \sphinxhyphen{} museum disposition}}}}
&
\sphinxAtStartPar
A disposition to collect, store, and provide

\sphinxAtStartPar
access to inanimate material entities in a

\sphinxAtStartPar
facility
\\
\hline
\sphinxAtStartPar
{\hyperref[\detokenize{doc-ORG_0000020::doc}]{\sphinxcrossref{\DUrole{doc}{ORG\_0000020 \sphinxhyphen{} gallery disposition}}}}
&
\sphinxAtStartPar
A disposition to display collected works from an

\sphinxAtStartPar
archive
\\
\hline
\sphinxAtStartPar
{\hyperref[\detokenize{doc-ORG_0000021::doc}]{\sphinxcrossref{\DUrole{doc}{ORG\_0000021 \sphinxhyphen{} publishing disposition}}}}
&
\sphinxAtStartPar
A disposition to publish information content

\sphinxAtStartPar
entities
\\
\hline
\sphinxAtStartPar
{\hyperref[\detokenize{doc-ORG_0000022::doc}]{\sphinxcrossref{\DUrole{doc}{ORG\_0000022 \sphinxhyphen{} research disposition}}}}
&
\sphinxAtStartPar
A disposition to conduct research
\\
\hline
\sphinxAtStartPar
{\hyperref[\detokenize{doc-ORG_0000023::doc}]{\sphinxcrossref{\DUrole{doc}{ORG\_0000023 \sphinxhyphen{} education disposition}}}}
&
\sphinxAtStartPar
A disposition to teach, and provide experiential

\sphinxAtStartPar
opprtunities for students
\\
\hline
\sphinxAtStartPar
{\hyperref[\detokenize{doc-ORG_0000024::doc}]{\sphinxcrossref{\DUrole{doc}{ORG\_0000024 \sphinxhyphen{} training disposition}}}}
&
\sphinxAtStartPar
A disposition to train, and provide experiential

\sphinxAtStartPar
opportunities for trainees
\\
\hline
\sphinxAtStartPar
{\hyperref[\detokenize{doc-ORG_0000025::doc}]{\sphinxcrossref{\DUrole{doc}{ORG\_0000025 \sphinxhyphen{} research administration disposition}}}}
&
\sphinxAtStartPar
A disposition to provide resources and oversight

\sphinxAtStartPar
for those conducting research
\\
\hline
\sphinxAtStartPar
{\hyperref[\detokenize{doc-ORG_0000026::doc}]{\sphinxcrossref{\DUrole{doc}{ORG\_0000026 \sphinxhyphen{} library disposition}}}}
&
\sphinxAtStartPar
A disposition to provide library services
\\
\hline
\sphinxAtStartPar
{\hyperref[\detokenize{doc-ORG_0000027::doc}]{\sphinxcrossref{\DUrole{doc}{ORG\_0000027 \sphinxhyphen{} commerce disposition}}}}
&
\sphinxAtStartPar
A disposition to sell things
\\
\hline
\sphinxAtStartPar
{\hyperref[\detokenize{doc-ORG_0000028::doc}]{\sphinxcrossref{\DUrole{doc}{ORG\_0000028 \sphinxhyphen{} military disposition}}}}
&
\sphinxAtStartPar
A disposition to engage in warfare
\\
\hline
\sphinxAtStartPar
{\hyperref[\detokenize{doc-ORG_0000029::doc}]{\sphinxcrossref{\DUrole{doc}{ORG\_0000029 \sphinxhyphen{} religious disposition}}}}
&
\sphinxAtStartPar
A disposition to engage in matters of spirtuality

\sphinxAtStartPar
and faith
\\
\hline
\sphinxAtStartPar
{\hyperref[\detokenize{doc-ORG_0000030::doc}]{\sphinxcrossref{\DUrole{doc}{ORG\_0000030 \sphinxhyphen{} governing disposition}}}}
&
\sphinxAtStartPar
A disposition to provide governance
\\
\hline
\sphinxAtStartPar
{\hyperref[\detokenize{doc-ORG_0000031::doc}]{\sphinxcrossref{\DUrole{doc}{ORG\_0000031 \sphinxhyphen{} manufacturing disposition}}}}
&
\sphinxAtStartPar
A dispositon to construct material entities
\\
\hline
\sphinxAtStartPar
{\hyperref[\detokenize{doc-ORG_0000032::doc}]{\sphinxcrossref{\DUrole{doc}{ORG\_0000032 \sphinxhyphen{} project team disposition}}}}
&
\sphinxAtStartPar
A disposition to execute and finish a project.
\\
\hline
\sphinxAtStartPar
{\hyperref[\detokenize{doc-ORG_0000033::doc}]{\sphinxcrossref{\DUrole{doc}{ORG\_0000033 \sphinxhyphen{} sports disposition}}}}
&
\sphinxAtStartPar
A disposition to engage in sports activites,

\sphinxAtStartPar
typically competitive.
\\
\hline
\sphinxAtStartPar
{\hyperref[\detokenize{doc-ORG_0000034::doc}]{\sphinxcrossref{\DUrole{doc}{ORG\_0000034 \sphinxhyphen{} information address quality}}}}
&
\sphinxAtStartPar
A quality of an address to be used for information

\sphinxAtStartPar
inquiries
\\
\hline
\sphinxAtStartPar
{\hyperref[\detokenize{doc-ORG_0000035::doc}]{\sphinxcrossref{\DUrole{doc}{ORG\_0000035 \sphinxhyphen{} billing address quality}}}}
&
\sphinxAtStartPar
A quality of an address to be used to receive

\sphinxAtStartPar
bills
\\
\hline
\sphinxAtStartPar
{\hyperref[\detokenize{doc-ORG_0000036::doc}]{\sphinxcrossref{\DUrole{doc}{ORG\_0000036 \sphinxhyphen{} shipping address quality}}}}
&
\sphinxAtStartPar
A quality of an address to be used to receive

\sphinxAtStartPar
shipped goods
\\
\hline
\sphinxAtStartPar
{\hyperref[\detokenize{doc-ORG_0000037::doc}]{\sphinxcrossref{\DUrole{doc}{ORG\_0000037 \sphinxhyphen{} preferred address quality}}}}
&
\sphinxAtStartPar
A quality of an address to be displayed in most

\sphinxAtStartPar
settings
\\
\hline
\sphinxAtStartPar
{\hyperref[\detokenize{doc-ORG_0000038::doc}]{\sphinxcrossref{\DUrole{doc}{ORG\_0000038 \sphinxhyphen{} homepage quality}}}}
&
\sphinxAtStartPar
A quality to be the primary website for an entity.
\\
\hline
\sphinxAtStartPar
{\hyperref[\detokenize{doc-ORG_0000039::doc}]{\sphinxcrossref{\DUrole{doc}{ORG\_0000039 \sphinxhyphen{} wikipedia quality}}}}
&
\sphinxAtStartPar
A quality to be the webpage within WikiPedia

\sphinxAtStartPar
regarding the entity
\\
\hline
\sphinxAtStartPar
{\hyperref[\detokenize{doc-ORG_0000040::doc}]{\sphinxcrossref{\DUrole{doc}{ORG\_0000040 \sphinxhyphen{} architectural structure}}}}
&
\sphinxAtStartPar
A material entity that is a human made strcuture

\sphinxAtStartPar
with firm connection between its foundation and

\sphinxAtStartPar
the ground.
\\
\hline
\sphinxAtStartPar
{\hyperref[\detokenize{doc-ORG_0000041::doc}]{\sphinxcrossref{\DUrole{doc}{ORG\_0000041 \sphinxhyphen{} campus}}}}
&
\sphinxAtStartPar
The geographic location consisting of the  grounds

\sphinxAtStartPar
or property of a school, college, university,

\sphinxAtStartPar
business, church, or hospital, often understood to

\sphinxAtStartPar
include buildings and other structures.
\\
\hline
\sphinxAtStartPar
{\hyperref[\detokenize{doc-ORG_0000042::doc}]{\sphinxcrossref{\DUrole{doc}{ORG\_0000042 \sphinxhyphen{} facility}}}}
&
\sphinxAtStartPar
An architectural structure that bears some

\sphinxAtStartPar
function.
\\
\hline
\sphinxAtStartPar
{\hyperref[\detokenize{doc-ORG_0000043::doc}]{\sphinxcrossref{\DUrole{doc}{ORG\_0000043 \sphinxhyphen{} building}}}}
&
\sphinxAtStartPar
A permanent walled and roofed construction
\\
\hline
\sphinxAtStartPar
{\hyperref[\detokenize{doc-ORG_0000044::doc}]{\sphinxcrossref{\DUrole{doc}{ORG\_0000044 \sphinxhyphen{} room}}}}
&
\sphinxAtStartPar
A space contained by a partitioned part of the

\sphinxAtStartPar
inside of a building.  Often has an identifier.
\\
\hline
\sphinxAtStartPar
{\hyperref[\detokenize{doc-ORG_0000045::doc}]{\sphinxcrossref{\DUrole{doc}{ORG\_0000045 \sphinxhyphen{} geographic region}}}}
&
\sphinxAtStartPar
A place on the earth.  Not necessarily contiguous
\\
\hline
\sphinxAtStartPar
{\hyperref[\detokenize{doc-ORG_0000046::doc}]{\sphinxcrossref{\DUrole{doc}{ORG\_0000046 \sphinxhyphen{} geographic point}}}}
&
\sphinxAtStartPar
A point located on the earth
\\
\hline
\sphinxAtStartPar
{\hyperref[\detokenize{doc-ORG_0000047::doc}]{\sphinxcrossref{\DUrole{doc}{ORG\_0000047 \sphinxhyphen{} continent}}}}
&
\sphinxAtStartPar
One of the main landmasses of the globe, usually

\sphinxAtStartPar
reckoned as seven in number (Europe, Asia, Africa,

\sphinxAtStartPar
North America, South America, Australia, and

\sphinxAtStartPar
Antarctica).
\\
\hline
\sphinxAtStartPar
{\hyperref[\detokenize{doc-ORG_0000048::doc}]{\sphinxcrossref{\DUrole{doc}{ORG\_0000048 \sphinxhyphen{} country}}}}
&
\sphinxAtStartPar
The territory governed by a sovereign state.
\\
\hline
\sphinxAtStartPar
{\hyperref[\detokenize{doc-ORG_0000049::doc}]{\sphinxcrossref{\DUrole{doc}{ORG\_0000049 \sphinxhyphen{} region}}}}
&
\sphinxAtStartPar
A subdivision of the territory of a country
\\
\hline
\sphinxAtStartPar
{\hyperref[\detokenize{doc-ORG_0000050::doc}]{\sphinxcrossref{\DUrole{doc}{ORG\_0000050 \sphinxhyphen{} populated place}}}}
&
\sphinxAtStartPar
A named place on the earth occupied by people
\\
\hline
\sphinxAtStartPar
{\hyperref[\detokenize{doc-ORG_0000051::doc}]{\sphinxcrossref{\DUrole{doc}{ORG\_0000051 \sphinxhyphen{} founding process}}}}
&
\sphinxAtStartPar
The process by which the organization was founded
\\
\hline
\sphinxAtStartPar
{\hyperref[\detokenize{doc-ORG_0000052::doc}]{\sphinxcrossref{\DUrole{doc}{ORG\_0000052 \sphinxhyphen{} founding process boundary}}}}
&
\sphinxAtStartPar
The process boundary which defines the moment of

\sphinxAtStartPar
creation of an orgnization.  Before the moment

\sphinxAtStartPar
the organization does not exist.  After the

\sphinxAtStartPar
moment, the organization exists.
\\
\hline
\sphinxAtStartPar
{\hyperref[\detokenize{doc-ORG_0000053::doc}]{\sphinxcrossref{\DUrole{doc}{ORG\_0000053 \sphinxhyphen{} dissolution process}}}}
&
\sphinxAtStartPar
The process by which an organization no longer

\sphinxAtStartPar
exists.
\\
\hline
\sphinxAtStartPar
{\hyperref[\detokenize{doc-ORG_0000054::doc}]{\sphinxcrossref{\DUrole{doc}{ORG\_0000054 \sphinxhyphen{} dissolution process boundary}}}}
&
\sphinxAtStartPar
The process boundary which marks the moment at

\sphinxAtStartPar
which the organization no longer exists
\\
\hline
\sphinxAtStartPar
{\hyperref[\detokenize{doc-ORG_0000055::doc}]{\sphinxcrossref{\DUrole{doc}{ORG\_0000055 \sphinxhyphen{} succession process}}}}
&
\sphinxAtStartPar
The process by which one organization gores out of

\sphinxAtStartPar
existence and is succeeded by a new organization
\\
\hline
\sphinxAtStartPar
{\hyperref[\detokenize{doc-ORG_0000056::doc}]{\sphinxcrossref{\DUrole{doc}{ORG\_0000056 \sphinxhyphen{} succession process boundary}}}}
&
\sphinxAtStartPar
The process boundary within a succession process.

\sphinxAtStartPar
Complex succesion processes may have many

\sphinxAtStartPar
boundaries.
\\
\hline
\sphinxAtStartPar
{\hyperref[\detokenize{doc-ORG_0000057::doc}]{\sphinxcrossref{\DUrole{doc}{ORG\_0000057 \sphinxhyphen{} web site}}}}
&
\sphinxAtStartPar
The information content entity consisting of a

\sphinxAtStartPar
group of World Wide Web pages usually containing

\sphinxAtStartPar
hyperlinks to each other and made available online

\sphinxAtStartPar
by an individual, company, educational

\sphinxAtStartPar
institution, government, or organization
\\
\hline
\sphinxAtStartPar
{\hyperref[\detokenize{doc-ORG_0000058::doc}]{\sphinxcrossref{\DUrole{doc}{ORG\_0000058 \sphinxhyphen{} spin\sphinxhyphen{}off process}}}}
&
\sphinxAtStartPar
The process by which one organization spins off of

\sphinxAtStartPar
another
\\
\hline
\sphinxAtStartPar
{\hyperref[\detokenize{doc-ORG_0000059::doc}]{\sphinxcrossref{\DUrole{doc}{ORG\_0000059 \sphinxhyphen{} spin\sphinxhyphen{}off process boundary}}}}
&
\sphinxAtStartPar
The boundary of a spin\sphinxhyphen{}off process
\\
\hline
\sphinxAtStartPar
{\hyperref[\detokenize{doc-ORG_0000060::doc}]{\sphinxcrossref{\DUrole{doc}{ORG\_0000060 \sphinxhyphen{} organizational membership}}}}
&
\sphinxAtStartPar
The asymmetric relationship involving two

\sphinxAtStartPar
organizations in which one is a member of the

\sphinxAtStartPar
other
\\
\hline
\sphinxAtStartPar
{\hyperref[\detokenize{doc-ORG_0000061::doc}]{\sphinxcrossref{\DUrole{doc}{ORG\_0000061 \sphinxhyphen{} organizational member role}}}}
&
\sphinxAtStartPar
The role of an organization in being a member of

\sphinxAtStartPar
another
\\
\hline
\sphinxAtStartPar
{\hyperref[\detokenize{doc-ORG_0000062::doc}]{\sphinxcrossref{\DUrole{doc}{ORG\_0000062 \sphinxhyphen{} organizational member grantor role}}}}
&
\sphinxAtStartPar
The role of an organization in granting a member

\sphinxAtStartPar
role to another
\\
\hline
\sphinxAtStartPar
{\hyperref[\detokenize{doc-ORG_0000063::doc}]{\sphinxcrossref{\DUrole{doc}{ORG\_0000063 \sphinxhyphen{} student led organization quality}}}}
&
\sphinxAtStartPar
The quality of an organization that is led by a

\sphinxAtStartPar
student
\\
\hline
\sphinxAtStartPar
{\hyperref[\detokenize{doc-ORG_0000064::doc}]{\sphinxcrossref{\DUrole{doc}{ORG\_0000064 \sphinxhyphen{} woman led organization quality}}}}
&
\sphinxAtStartPar
The quality of an organization that is led by a

\sphinxAtStartPar
woman
\\
\hline
\sphinxAtStartPar
{\hyperref[\detokenize{doc-ORG_0000065::doc}]{\sphinxcrossref{\DUrole{doc}{ORG\_0000065 \sphinxhyphen{} minority led organization quality}}}}
&
\sphinxAtStartPar
The quality of an organiztion that is led by a

\sphinxAtStartPar
designated minority
\\
\hline
\sphinxAtStartPar
{\hyperref[\detokenize{doc-ORG_0000066::doc}]{\sphinxcrossref{\DUrole{doc}{ORG\_0000066 \sphinxhyphen{} registered address quality}}}}
&
\sphinxAtStartPar
The quality of a location that is the

\sphinxAtStartPar
legal/registered location for the organization
\\
\hline
\sphinxAtStartPar
{\hyperref[\detokenize{doc-ORG_0000067::doc}]{\sphinxcrossref{\DUrole{doc}{ORG\_0000067 \sphinxhyphen{} primary address quality}}}}
&
\sphinxAtStartPar
The quality of a location that is the

\sphinxAtStartPar
primary/preferred location for the organization
\\
\hline
\sphinxAtStartPar
{\hyperref[\detokenize{doc-ORG_0000068::doc}]{\sphinxcrossref{\DUrole{doc}{ORG\_0000068 \sphinxhyphen{} organizational position}}}}
&
\sphinxAtStartPar
The asymmetric relationship between an

\sphinxAtStartPar
organization and a person in which a person has a

\sphinxAtStartPar
role in the positon, as does the organization.
\\
\hline
\sphinxAtStartPar
{\hyperref[\detokenize{doc-ORG_0000069::doc}]{\sphinxcrossref{\DUrole{doc}{ORG\_0000069 \sphinxhyphen{} organizational employee role}}}}
&
\sphinxAtStartPar
The role of a person to be an employee of an

\sphinxAtStartPar
organization
\\
\hline
\sphinxAtStartPar
{\hyperref[\detokenize{doc-ORG_0000070::doc}]{\sphinxcrossref{\DUrole{doc}{ORG\_0000070 \sphinxhyphen{} organizational employor role}}}}
&
\sphinxAtStartPar
The role of an organization to employ a person
\\
\hline
\sphinxAtStartPar
{\hyperref[\detokenize{doc-ORG_0000071::doc}]{\sphinxcrossref{\DUrole{doc}{ORG\_0000071 \sphinxhyphen{} organizational associate role}}}}
&
\sphinxAtStartPar
The role of a person to be an associate of an

\sphinxAtStartPar
organization
\\
\hline
\sphinxAtStartPar
{\hyperref[\detokenize{doc-ORG_0000072::doc}]{\sphinxcrossref{\DUrole{doc}{ORG\_0000072 \sphinxhyphen{} organizational associate grantor role}}}}
&
\sphinxAtStartPar
The role of an organization to grant associate

\sphinxAtStartPar
status to a person
\\
\hline
\sphinxAtStartPar
{\hyperref[\detokenize{doc-ORG_0000073::doc}]{\sphinxcrossref{\DUrole{doc}{ORG\_0000073 \sphinxhyphen{} organizatonal head role}}}}
&
\sphinxAtStartPar
The role of a person to be the head of an

\sphinxAtStartPar
organization
\\
\hline
\sphinxAtStartPar
{\hyperref[\detokenize{doc-ORG_0000074::doc}]{\sphinxcrossref{\DUrole{doc}{ORG\_0000074 \sphinxhyphen{} organizational head grantor role}}}}
&
\sphinxAtStartPar
The role of an organization to grant head status

\sphinxAtStartPar
to a person
\\
\hline
\sphinxAtStartPar
{\hyperref[\detokenize{doc-ORG_0000075::doc}]{\sphinxcrossref{\DUrole{doc}{ORG\_0000075 \sphinxhyphen{} organizational appointee role}}}}
&
\sphinxAtStartPar
The role of a person to be an appointee of an

\sphinxAtStartPar
organization
\\
\hline
\sphinxAtStartPar
{\hyperref[\detokenize{doc-ORG_0000076::doc}]{\sphinxcrossref{\DUrole{doc}{ORG\_0000076 \sphinxhyphen{} organizational appointee grantor role}}}}
&
\sphinxAtStartPar
The role of an organization to grant appointee

\sphinxAtStartPar
status to a person
\\
\hline
\sphinxAtStartPar
{\hyperref[\detokenize{doc-ORG_0000077::doc}]{\sphinxcrossref{\DUrole{doc}{ORG\_0000077 \sphinxhyphen{} organizational volunteer role}}}}
&
\sphinxAtStartPar
The role of a person to be a volunteer of an

\sphinxAtStartPar
organization
\\
\hline
\sphinxAtStartPar
{\hyperref[\detokenize{doc-ORG_0000078::doc}]{\sphinxcrossref{\DUrole{doc}{ORG\_0000078 \sphinxhyphen{} organizational volunteer grantor role}}}}
&
\sphinxAtStartPar
The role of an organization to grant volunteer

\sphinxAtStartPar
status to a person
\\
\hline
\sphinxAtStartPar
{\hyperref[\detokenize{doc-ORG_0000079::doc}]{\sphinxcrossref{\DUrole{doc}{ORG\_0000079 \sphinxhyphen{} airline disposition}}}}
&
\sphinxAtStartPar
The disposition of an organization that operates

\sphinxAtStartPar
airplanes carrying frieght or passengers
\\
\hline
\sphinxAtStartPar
{\hyperref[\detokenize{doc-ORG_0000080::doc}]{\sphinxcrossref{\DUrole{doc}{ORG\_0000080 \sphinxhyphen{} media disposition}}}}
&
\sphinxAtStartPar
The disposition of an organization that creates,

\sphinxAtStartPar
transmits, and/or licenses live or recorded

\sphinxAtStartPar
material for viewing by others
\\
\hline
\sphinxAtStartPar
{\hyperref[\detokenize{doc-ORG_0000081::doc}]{\sphinxcrossref{\DUrole{doc}{ORG\_0000081 \sphinxhyphen{} performing disposition}}}}
&
\sphinxAtStartPar
The disposition of an organization to perform live

\sphinxAtStartPar
or recorded music, theatre, or dance
\\
\hline
\sphinxAtStartPar
{\hyperref[\detokenize{doc-ORG_0000082::doc}]{\sphinxcrossref{\DUrole{doc}{ORG\_0000082 \sphinxhyphen{} labor union disposition}}}}
&
\sphinxAtStartPar
The disposition of an organization to organize

\sphinxAtStartPar
workers for the purpose of negotiations with

\sphinxAtStartPar
employers of the workers
\\
\hline
\sphinxAtStartPar
{\hyperref[\detokenize{doc-ORG_0000083::doc}]{\sphinxcrossref{\DUrole{doc}{ORG\_0000083 \sphinxhyphen{} person membership}}}}
&
\sphinxAtStartPar
The asymmetric relationship representing a

\sphinxAtStartPar
person’s membership in an organization
\\
\hline
\sphinxAtStartPar
{\hyperref[\detokenize{doc-ORG_0000084::doc}]{\sphinxcrossref{\DUrole{doc}{ORG\_0000084 \sphinxhyphen{} person member role}}}}
&
\sphinxAtStartPar
The role of a person to be a member of an

\sphinxAtStartPar
organization
\\
\hline
\sphinxAtStartPar
{\hyperref[\detokenize{doc-ORG_0000085::doc}]{\sphinxcrossref{\DUrole{doc}{ORG\_0000085 \sphinxhyphen{} person member grantor role}}}}
&
\sphinxAtStartPar
The role of an organization to grant membership to

\sphinxAtStartPar
a person
\\
\hline
\sphinxAtStartPar
{\hyperref[\detokenize{doc-Concept::doc}]{\sphinxcrossref{\DUrole{doc}{Concept \sphinxhyphen{} concept (skos)}}}}
&
\sphinxAtStartPar
An idea or notion, a unit of thought
\\
\hline
\sphinxAtStartPar
{\hyperref[\detokenize{doc-Instant::doc}]{\sphinxcrossref{\DUrole{doc}{Instant \sphinxhyphen{} time instant}}}}
&
\sphinxAtStartPar
A zero\sphinxhyphen{}dimensional part of time.  Precision may

\sphinxAtStartPar
specify a range.  Represented by xsd datetime

\sphinxAtStartPar
string
\\
\hline
\sphinxAtStartPar
{\hyperref[\detokenize{doc-TemporalUnit::doc}]{\sphinxcrossref{\DUrole{doc}{TemporalUnit \sphinxhyphen{} temporal unit}}}}
&
\sphinxAtStartPar
A specification of a time duration.  Used to

\sphinxAtStartPar
specify precision of time instants
\\
\hline
\end{longtable}\sphinxatlongtableend\end{savenotes}
\begin{quote}

\index{BFO\_0000001@\spxentry{BFO\_0000001}!entity@\spxentry{entity}}\index{entity@\spxentry{entity}!BFO\_0000001@\spxentry{BFO\_0000001}}\ignorespaces \end{quote}


\subsection{BFO\_0000001 \sphinxhyphen{} entity}
\label{\detokenize{doc-BFO_0000001:bfo-0000001-entity}}\label{\detokenize{doc-BFO_0000001:index-0}}\label{\detokenize{doc-BFO_0000001::doc}}
\begin{sphinxShadowBox}
\sphinxstyletopictitle{Label}

\sphinxAtStartPar
entity
\end{sphinxShadowBox}

\begin{sphinxShadowBox}
\sphinxstyletopictitle{Definition}

\sphinxAtStartPar
The fundamental thing that has existence. All things that exist are entities.
\end{sphinxShadowBox}

\begin{sphinxShadowBox}
\sphinxstyletopictitle{Imported From}

\sphinxAtStartPar
\sphinxurl{http://purl.obolibrary.org/obo/bfo/2020/bfo.owl}
\end{sphinxShadowBox}
\begin{quote}

\index{BFO\_0000002@\spxentry{BFO\_0000002}!continuant@\spxentry{continuant}}\index{continuant@\spxentry{continuant}!BFO\_0000002@\spxentry{BFO\_0000002}}\ignorespaces \end{quote}


\subsection{BFO\_0000002 \sphinxhyphen{} continuant}
\label{\detokenize{doc-BFO_0000002:bfo-0000002-continuant}}\label{\detokenize{doc-BFO_0000002:index-0}}\label{\detokenize{doc-BFO_0000002::doc}}
\begin{sphinxShadowBox}
\sphinxstyletopictitle{Label}

\sphinxAtStartPar
continuant
\end{sphinxShadowBox}

\begin{sphinxShadowBox}
\sphinxstyletopictitle{Definition}

\sphinxAtStartPar
An entity which has existence in time
\end{sphinxShadowBox}

\begin{sphinxShadowBox}
\sphinxstyletopictitle{Imported From}

\sphinxAtStartPar
\sphinxurl{http://purl.obolibrary.org/obo/bfo/2020/bfo.owl}
\end{sphinxShadowBox}
\begin{quote}

\index{BFO\_0000003@\spxentry{BFO\_0000003}!occurrent@\spxentry{occurrent}}\index{occurrent@\spxentry{occurrent}!BFO\_0000003@\spxentry{BFO\_0000003}}\ignorespaces \end{quote}


\subsection{BFO\_0000003 \sphinxhyphen{} occurrent}
\label{\detokenize{doc-BFO_0000003:bfo-0000003-occurrent}}\label{\detokenize{doc-BFO_0000003:index-0}}\label{\detokenize{doc-BFO_0000003::doc}}
\begin{sphinxShadowBox}
\sphinxstyletopictitle{Label}

\sphinxAtStartPar
occurrent
\end{sphinxShadowBox}

\begin{sphinxShadowBox}
\sphinxstyletopictitle{Definition}

\sphinxAtStartPar
An entity which occurs in time
\end{sphinxShadowBox}

\begin{sphinxShadowBox}
\sphinxstyletopictitle{Imported From}

\sphinxAtStartPar
\sphinxurl{http://purl.obolibrary.org/obo/bfo/2020/bfo.owl}
\end{sphinxShadowBox}
\begin{quote}

\index{BFO\_0000004@\spxentry{BFO\_0000004}!independent continuant@\spxentry{independent continuant}}\index{independent continuant@\spxentry{independent continuant}!BFO\_0000004@\spxentry{BFO\_0000004}}\ignorespaces \end{quote}


\subsection{BFO\_0000004 \sphinxhyphen{} independent continuant}
\label{\detokenize{doc-BFO_0000004:bfo-0000004-independent-continuant}}\label{\detokenize{doc-BFO_0000004:index-0}}\label{\detokenize{doc-BFO_0000004::doc}}
\begin{sphinxShadowBox}
\sphinxstyletopictitle{Label}

\sphinxAtStartPar
independent continuant
\end{sphinxShadowBox}

\begin{sphinxShadowBox}
\sphinxstyletopictitle{Definition}

\sphinxAtStartPar
B is an independent continuant = Def. b is a continuant which is such that there is no c and no t such that b s\sphinxhyphen{}depends\_on c at t. (axiom label in BFO2 Reference: {[}017\sphinxhyphen{}002{]})
\end{sphinxShadowBox}

\begin{sphinxShadowBox}
\sphinxstyletopictitle{Example}

\sphinxAtStartPar
a molecule

\sphinxAtStartPar
a heart

\sphinxAtStartPar
an organism

\sphinxAtStartPar
a chair

\sphinxAtStartPar
a leg

\sphinxAtStartPar
the interior of your mouth

\sphinxAtStartPar
an orchestra.

\sphinxAtStartPar
an atom

\sphinxAtStartPar
the bottom right portion of a human torso

\sphinxAtStartPar
a spatial region
\end{sphinxShadowBox}

\begin{sphinxShadowBox}
\sphinxstyletopictitle{Imported From}

\sphinxAtStartPar
\sphinxurl{http://purl.obolibrary.org/obo/bfo/2019-08-26/bfo.owl}
\end{sphinxShadowBox}
\begin{quote}

\index{BFO\_0000008@\spxentry{BFO\_0000008}!temporal region@\spxentry{temporal region}}\index{temporal region@\spxentry{temporal region}!BFO\_0000008@\spxentry{BFO\_0000008}}\ignorespaces \end{quote}


\subsection{BFO\_0000008 \sphinxhyphen{} temporal region}
\label{\detokenize{doc-BFO_0000008:bfo-0000008-temporal-region}}\label{\detokenize{doc-BFO_0000008:index-0}}\label{\detokenize{doc-BFO_0000008::doc}}
\begin{sphinxShadowBox}
\sphinxstyletopictitle{Label}

\sphinxAtStartPar
temporal region
\end{sphinxShadowBox}

\begin{sphinxShadowBox}
\sphinxstyletopictitle{Definition}

\sphinxAtStartPar
An occurent which is some part of time
\end{sphinxShadowBox}

\begin{sphinxShadowBox}
\sphinxstyletopictitle{Imported From}

\sphinxAtStartPar
\sphinxurl{http://purl.obolibrary.org/obo/bfo/2020/bfo.owl}
\end{sphinxShadowBox}
\begin{quote}

\index{BFO\_0000015@\spxentry{BFO\_0000015}!process@\spxentry{process}}\index{process@\spxentry{process}!BFO\_0000015@\spxentry{BFO\_0000015}}\ignorespaces \end{quote}


\subsection{BFO\_0000015 \sphinxhyphen{} process}
\label{\detokenize{doc-BFO_0000015:bfo-0000015-process}}\label{\detokenize{doc-BFO_0000015:index-0}}\label{\detokenize{doc-BFO_0000015::doc}}
\begin{sphinxShadowBox}
\sphinxstyletopictitle{Label}

\sphinxAtStartPar
process
\end{sphinxShadowBox}

\begin{sphinxShadowBox}
\sphinxstyletopictitle{Definition}

\sphinxAtStartPar
P is a process = Def. p is an occurrent that has temporal proper parts and for some time t, p s\sphinxhyphen{}depends\_on some material entity at t. (axiom label in BFO2 Reference: {[}083\sphinxhyphen{}003{]}) {[}has axiom label: \sphinxurl{http://purl.obolibrary.org/obo/bfo/axiom/083-003}{]}
\end{sphinxShadowBox}

\begin{sphinxShadowBox}
\sphinxstyletopictitle{Imported From}

\sphinxAtStartPar
\sphinxurl{http://purl.obolibrary.org/obo/bfo/2020/bfo.owl}
\end{sphinxShadowBox}
\begin{quote}

\index{BFO\_0000016@\spxentry{BFO\_0000016}!disposition@\spxentry{disposition}}\index{disposition@\spxentry{disposition}!BFO\_0000016@\spxentry{BFO\_0000016}}\ignorespaces \end{quote}


\subsection{BFO\_0000016 \sphinxhyphen{} disposition}
\label{\detokenize{doc-BFO_0000016:bfo-0000016-disposition}}\label{\detokenize{doc-BFO_0000016:index-0}}\label{\detokenize{doc-BFO_0000016::doc}}
\begin{sphinxShadowBox}
\sphinxstyletopictitle{Label}

\sphinxAtStartPar
disposition
\end{sphinxShadowBox}

\begin{sphinxShadowBox}
\sphinxstyletopictitle{Definition}

\sphinxAtStartPar
A realizable entity that presents in a continuant
\end{sphinxShadowBox}

\begin{sphinxShadowBox}
\sphinxstyletopictitle{Imported From}

\sphinxAtStartPar
\sphinxurl{http://purl.obolibrary.org/obo/bfo/2020/bfo.owl}
\end{sphinxShadowBox}
\begin{quote}

\index{BFO\_0000017@\spxentry{BFO\_0000017}!realizable entity@\spxentry{realizable entity}}\index{realizable entity@\spxentry{realizable entity}!BFO\_0000017@\spxentry{BFO\_0000017}}\ignorespaces \end{quote}


\subsection{BFO\_0000017 \sphinxhyphen{} realizable entity}
\label{\detokenize{doc-BFO_0000017:bfo-0000017-realizable-entity}}\label{\detokenize{doc-BFO_0000017:index-0}}\label{\detokenize{doc-BFO_0000017::doc}}
\begin{sphinxShadowBox}
\sphinxstyletopictitle{Label}

\sphinxAtStartPar
realizable entity
\end{sphinxShadowBox}

\begin{sphinxShadowBox}
\sphinxstyletopictitle{Definition}

\sphinxAtStartPar
To say that b is a realizable entity is to say that b is a specifically dependent continuant that inheres in some independent continuant which is not a spatial region and is of a type instances of which are realized in processes of a correlated type. (axiom label in BFO2 Reference: {[}058\sphinxhyphen{}002{]})
\end{sphinxShadowBox}

\begin{sphinxShadowBox}
\sphinxstyletopictitle{Imported From}

\sphinxAtStartPar
\sphinxurl{http://purl.obolibrary.org/obo/bfo/2020/bfo.owl}
\end{sphinxShadowBox}
\begin{quote}

\index{BFO\_0000019@\spxentry{BFO\_0000019}!quality@\spxentry{quality}}\index{quality@\spxentry{quality}!BFO\_0000019@\spxentry{BFO\_0000019}}\ignorespaces \end{quote}


\subsection{BFO\_0000019 \sphinxhyphen{} quality}
\label{\detokenize{doc-BFO_0000019:bfo-0000019-quality}}\label{\detokenize{doc-BFO_0000019:index-0}}\label{\detokenize{doc-BFO_0000019::doc}}
\begin{sphinxShadowBox}
\sphinxstyletopictitle{Label}

\sphinxAtStartPar
quality
\end{sphinxShadowBox}

\begin{sphinxShadowBox}
\sphinxstyletopictitle{Definition}

\sphinxAtStartPar
A quality is a specifically dependent continuant that, in contrast to roles and dispositions, does not require any further process in order to be realized.
\end{sphinxShadowBox}

\begin{sphinxShadowBox}
\sphinxstyletopictitle{Example}

\sphinxAtStartPar
the shape of your nostril

\sphinxAtStartPar
the color of a tomato

\sphinxAtStartPar
the mass of this piece of gold.

\sphinxAtStartPar
the ambient temperature of this portion of air

\sphinxAtStartPar
the shape of your nose

\sphinxAtStartPar
the length of the circumference of your waist
\end{sphinxShadowBox}

\begin{sphinxShadowBox}
\sphinxstyletopictitle{Imported From}

\sphinxAtStartPar
\sphinxurl{http://purl.obolibrary.org/obo/bfo/2019-08-26/bfo.owl}
\end{sphinxShadowBox}
\begin{quote}

\index{BFO\_0000020@\spxentry{BFO\_0000020}!specifically dependent continuant@\spxentry{specifically dependent continuant}}\index{specifically dependent continuant@\spxentry{specifically dependent continuant}!BFO\_0000020@\spxentry{BFO\_0000020}}\ignorespaces \end{quote}


\subsection{BFO\_0000020 \sphinxhyphen{} specifically dependent continuant}
\label{\detokenize{doc-BFO_0000020:bfo-0000020-specifically-dependent-continuant}}\label{\detokenize{doc-BFO_0000020:index-0}}\label{\detokenize{doc-BFO_0000020::doc}}
\begin{sphinxShadowBox}
\sphinxstyletopictitle{Label}

\sphinxAtStartPar
specifically dependent continuant
\end{sphinxShadowBox}

\begin{sphinxShadowBox}
\sphinxstyletopictitle{Definition}

\sphinxAtStartPar
B is a specifically dependent continuant = Def. b is a continuant \& there is some independent continuant c which is not a spatial region and which is such that b s\sphinxhyphen{}depends\_on c at every time t during the course of b’s existence. (axiom label in BFO2 Reference: {[}050\sphinxhyphen{}003{]}) {[}has axiom label: \sphinxurl{http://purl.obolibrary.org/obo/bfo/axiom/050-003}{]}
\end{sphinxShadowBox}

\begin{sphinxShadowBox}
\sphinxstyletopictitle{Imported From}

\sphinxAtStartPar
\sphinxurl{http://purl.obolibrary.org/obo/bfo/2020/bfo.owl}
\end{sphinxShadowBox}
\begin{quote}

\index{BFO\_0000023@\spxentry{BFO\_0000023}!role@\spxentry{role}}\index{role@\spxentry{role}!BFO\_0000023@\spxentry{BFO\_0000023}}\ignorespaces \end{quote}


\subsection{BFO\_0000023 \sphinxhyphen{} role}
\label{\detokenize{doc-BFO_0000023:bfo-0000023-role}}\label{\detokenize{doc-BFO_0000023:index-0}}\label{\detokenize{doc-BFO_0000023::doc}}
\begin{sphinxShadowBox}
\sphinxstyletopictitle{Label}

\sphinxAtStartPar
role
\end{sphinxShadowBox}

\begin{sphinxShadowBox}
\sphinxstyletopictitle{Definition}

\sphinxAtStartPar
B is a role means: b is a realizable entity \& b exists because there is some single bearer that is in some special physical, social, or institutional set of circumstances in which this bearer does not have to be\& b is not such that, if it ceases to exist, then the physical make\sphinxhyphen{}up of the bearer is thereby changed.
\end{sphinxShadowBox}

\begin{sphinxShadowBox}
\sphinxstyletopictitle{Example}

\sphinxAtStartPar
John’s role of husband to Mary is dependent on Mary’s role of wife to John, and both are dependent on the object aggregate comprising John and Mary as member parts joined together through the relational quality of being married.

\sphinxAtStartPar
the role of a stone in marking a property boundary

\sphinxAtStartPar
the role of a building in serving as a military target

\sphinxAtStartPar
the role of subject in a clinical trial

\sphinxAtStartPar
the priest role

\sphinxAtStartPar
the role of a boundary to demarcate two neighboring administrative territories

\sphinxAtStartPar
the student role
\end{sphinxShadowBox}

\begin{sphinxShadowBox}
\sphinxstyletopictitle{Editor’s note}

\sphinxAtStartPar
BFO 2 Reference: One major family of examples of non\sphinxhyphen{}rigid universals involves roles, and ontologies developed for corresponding administrative purposes may consist entirely of representatives of entities of this sort. Thus ‘professor’, defined as follows,b instance\_of professor at t =Def. there is some c, c instance\_of professor role \& c inheres\_in b at t.denotes a non\sphinxhyphen{}rigid universal and so also do ‘nurse’, ‘student’, ‘colonel’, ‘taxpayer’, and so forth. (These terms are all, in the jargon of philosophy, phase sortals.) By using role terms in definitions, we can create a BFO conformant treatment of such entities drawing on the fact that, while an instance of professor may be simultaneously an instance of trade union member, no instance of the type professor role is also (at any time) an instance of the type trade union member role (any more than any instance of the type color is at any time an instance of the type length).If an ontology of employment positions should be defined in terms of roles following the above pattern, this enables the ontology to do justice to the fact that individuals instantiate the corresponding universals \textendash{}  professor, sergeant, nurse \textendash{} only during certain phases in their lives.
\end{sphinxShadowBox}

\begin{sphinxShadowBox}
\sphinxstyletopictitle{Imported From}

\sphinxAtStartPar
\sphinxurl{http://purl.obolibrary.org/obo/bfo/2019-08-26/bfo.owl}
\end{sphinxShadowBox}
\begin{quote}

\index{BFO\_0000029@\spxentry{BFO\_0000029}!site@\spxentry{site}}\index{site@\spxentry{site}!BFO\_0000029@\spxentry{BFO\_0000029}}\ignorespaces \end{quote}


\subsection{BFO\_0000029 \sphinxhyphen{} site}
\label{\detokenize{doc-BFO_0000029:bfo-0000029-site}}\label{\detokenize{doc-BFO_0000029:index-0}}\label{\detokenize{doc-BFO_0000029::doc}}
\begin{sphinxShadowBox}
\sphinxstyletopictitle{Label}

\sphinxAtStartPar
site
\end{sphinxShadowBox}

\begin{sphinxShadowBox}
\sphinxstyletopictitle{Definition}

\sphinxAtStartPar
B is a site means: b is a three\sphinxhyphen{}dimensional immaterial entity that is (partially or wholly) bounded by a material entity or it is a three\sphinxhyphen{}dimensional immaterial part thereof. (axiom label in BFO2 Reference: {[}034\sphinxhyphen{}002{]})
\end{sphinxShadowBox}

\begin{sphinxShadowBox}
\sphinxstyletopictitle{Example}

\sphinxAtStartPar
the interior of the trunk of your car

\sphinxAtStartPar
the interior of your refrigerator

\sphinxAtStartPar
a hole in the interior of a portion of cheese

\sphinxAtStartPar
the lumen of your gut

\sphinxAtStartPar
a rabbit hole

\sphinxAtStartPar
the Piazza San Marco

\sphinxAtStartPar
your left nostril (a fiat part \textendash{} the opening \textendash{} of your left nasal cavity)

\sphinxAtStartPar
the interior of your office

\sphinxAtStartPar
the interior of a kangaroo pouch

\sphinxAtStartPar
Manhattan Canyon)

\sphinxAtStartPar
the interior of your bedroom

\sphinxAtStartPar
an air traffic control region defined in the airspace above an airport

\sphinxAtStartPar
the Grand Canyon

\sphinxAtStartPar
the cockpit of an aircraft

\sphinxAtStartPar
the hold of a ship
\end{sphinxShadowBox}

\begin{sphinxShadowBox}
\sphinxstyletopictitle{Imported From}

\sphinxAtStartPar
\sphinxurl{http://purl.obolibrary.org/obo/bfo/2019-08-26/bfo.owl}
\end{sphinxShadowBox}
\begin{quote}

\index{BFO\_0000031@\spxentry{BFO\_0000031}!generically dependent continuant@\spxentry{generically dependent continuant}}\index{generically dependent continuant@\spxentry{generically dependent continuant}!BFO\_0000031@\spxentry{BFO\_0000031}}\ignorespaces \end{quote}


\subsection{BFO\_0000031 \sphinxhyphen{} generically dependent continuant}
\label{\detokenize{doc-BFO_0000031:bfo-0000031-generically-dependent-continuant}}\label{\detokenize{doc-BFO_0000031:index-0}}\label{\detokenize{doc-BFO_0000031::doc}}
\begin{sphinxShadowBox}
\sphinxstyletopictitle{Label}

\sphinxAtStartPar
generically dependent continuant
\end{sphinxShadowBox}

\begin{sphinxShadowBox}
\sphinxstyletopictitle{Definition}

\sphinxAtStartPar
B is a generically dependent continuant = Def. b is a continuant that g\sphinxhyphen{}depends\_on one or more other entities. (axiom label in BFO2 Reference: {[}074\sphinxhyphen{}001{]}) {[}has axiom label: \sphinxurl{http://purl.obolibrary.org/obo/bfo/axiom/074-001}{]}
\end{sphinxShadowBox}

\begin{sphinxShadowBox}
\sphinxstyletopictitle{Imported From}

\sphinxAtStartPar
\sphinxurl{http://purl.obolibrary.org/obo/bfo/2020/bfo.owl}
\end{sphinxShadowBox}
\begin{quote}

\index{BFO\_0000035@\spxentry{BFO\_0000035}!process boundary@\spxentry{process boundary}}\index{process boundary@\spxentry{process boundary}!BFO\_0000035@\spxentry{BFO\_0000035}}\ignorespaces \end{quote}


\subsection{BFO\_0000035 \sphinxhyphen{} process boundary}
\label{\detokenize{doc-BFO_0000035:bfo-0000035-process-boundary}}\label{\detokenize{doc-BFO_0000035:index-0}}\label{\detokenize{doc-BFO_0000035::doc}}
\begin{sphinxShadowBox}
\sphinxstyletopictitle{Label}

\sphinxAtStartPar
process boundary
\end{sphinxShadowBox}

\begin{sphinxShadowBox}
\sphinxstyletopictitle{Definition}

\sphinxAtStartPar
P is a process boundary =Def. p is a temporal part of a process \& p has no proper temporal parts. (axiom label in BFO2 Reference: {[}084\sphinxhyphen{}001{]}) {[}has axiom label: \sphinxurl{http://purl.obolibrary.org/obo/bfo/axiom/084-001}{]}
\end{sphinxShadowBox}

\begin{sphinxShadowBox}
\sphinxstyletopictitle{Imported From}

\sphinxAtStartPar
\sphinxurl{http://purl.obolibrary.org/obo/bfo/2020/bfo.owl}
\end{sphinxShadowBox}
\begin{quote}

\index{BFO\_0000038@\spxentry{BFO\_0000038}!one\sphinxhyphen{}dimensional temporal region@\spxentry{one\sphinxhyphen{}dimensional temporal region}}\index{one\sphinxhyphen{}dimensional temporal region@\spxentry{one\sphinxhyphen{}dimensional temporal region}!BFO\_0000038@\spxentry{BFO\_0000038}}\ignorespaces \end{quote}


\subsection{BFO\_0000038 \sphinxhyphen{} one\sphinxhyphen{}dimensional temporal region}
\label{\detokenize{doc-BFO_0000038:bfo-0000038-one-dimensional-temporal-region}}\label{\detokenize{doc-BFO_0000038:index-0}}\label{\detokenize{doc-BFO_0000038::doc}}
\begin{sphinxShadowBox}
\sphinxstyletopictitle{Label}

\sphinxAtStartPar
one\sphinxhyphen{}dimensional temporal region
\end{sphinxShadowBox}

\begin{sphinxShadowBox}
\sphinxstyletopictitle{Definition}

\sphinxAtStartPar
A one\sphinxhyphen{}dimensional temporal region is a temporal region that is extended. (axiom label in BFO2 Reference: {[}103\sphinxhyphen{}001{]})
\end{sphinxShadowBox}

\begin{sphinxShadowBox}
\sphinxstyletopictitle{Imported From}

\sphinxAtStartPar
\sphinxurl{http://purl.obolibrary.org/obo/bfo/2020/bfo.owl}
\end{sphinxShadowBox}
\begin{quote}

\index{BFO\_0000040@\spxentry{BFO\_0000040}!material entity@\spxentry{material entity}}\index{material entity@\spxentry{material entity}!BFO\_0000040@\spxentry{BFO\_0000040}}\ignorespaces \end{quote}


\subsection{BFO\_0000040 \sphinxhyphen{} material entity}
\label{\detokenize{doc-BFO_0000040:bfo-0000040-material-entity}}\label{\detokenize{doc-BFO_0000040:index-0}}\label{\detokenize{doc-BFO_0000040::doc}}
\begin{sphinxShadowBox}
\sphinxstyletopictitle{Label}

\sphinxAtStartPar
material entity
\end{sphinxShadowBox}

\begin{sphinxShadowBox}
\sphinxstyletopictitle{Definition}

\sphinxAtStartPar
A material entity is an independent continuant that has some portion of matter as proper or improper continuant part. (axiom label in BFO2 Reference: {[}019\sphinxhyphen{}002{]})
\end{sphinxShadowBox}

\begin{sphinxShadowBox}
\sphinxstyletopictitle{Example}

\sphinxAtStartPar
an energy wave

\sphinxAtStartPar
an aggregate of human beings.

\sphinxAtStartPar
a photon

\sphinxAtStartPar
a tornado

\sphinxAtStartPar
a flame

\sphinxAtStartPar
a human being

\sphinxAtStartPar
a hurricane

\sphinxAtStartPar
the undetached arm of a human being

\sphinxAtStartPar
an epidemic

\sphinxAtStartPar
a sea wave

\sphinxAtStartPar
a puff of smoke

\sphinxAtStartPar
a forest fire
\end{sphinxShadowBox}

\begin{sphinxShadowBox}
\sphinxstyletopictitle{Editor’s note}

\sphinxAtStartPar
BFO 2 Reference: ‘Matter’ is intended to encompass both mass and energy (we will address the ontological treatment of portions of energy in a later version of BFO). A portion of matter is anything that includes elementary particles among its proper or improper parts: quarks and leptons, including electrons, as the smallest particles thus far discovered; baryons (including protons and neutrons) at a higher level of granularity; atoms and molecules at still higher levels, forming the cells, organs, organisms and other material entities studied by biologists, the portions of rock studied by geologists, the fossils studied by paleontologists, and so on.Material entities are three\sphinxhyphen{}dimensional entities (entities extended in three spatial dimensions), as contrasted with the processes in which they participate, which are four\sphinxhyphen{}dimensional entities (entities extended also along the dimension of time).According to the FMA, material entities may have immaterial entities as parts \textendash{} including the entities identified below as sites; for example the interior (or ‘lumen’) of your small intestine is a part of your body. BFO 2.0 embodies a decision to follow the FMA here.

\sphinxAtStartPar
BFO 2 Reference: Material entities (continuants) can preserve their identity even while gaining and losing material parts. Continuants are contrasted with occurrents, which unfold themselves in successive temporal parts or phases {[}60

\sphinxAtStartPar
BFO 2 Reference: Object, Fiat Object Part and Object Aggregate are not intended to be exhaustive of Material Entity. Users are invited to propose new subcategories of Material Entity.
\end{sphinxShadowBox}

\begin{sphinxShadowBox}
\sphinxstyletopictitle{Imported From}

\sphinxAtStartPar
\sphinxurl{http://purl.obolibrary.org/obo/bfo/2019-08-26/bfo.owl}
\end{sphinxShadowBox}
\begin{quote}

\index{BFO\_0000141@\spxentry{BFO\_0000141}!immaterial entity@\spxentry{immaterial entity}}\index{immaterial entity@\spxentry{immaterial entity}!BFO\_0000141@\spxentry{BFO\_0000141}}\ignorespaces \end{quote}


\subsection{BFO\_0000141 \sphinxhyphen{} immaterial entity}
\label{\detokenize{doc-BFO_0000141:bfo-0000141-immaterial-entity}}\label{\detokenize{doc-BFO_0000141:index-0}}\label{\detokenize{doc-BFO_0000141::doc}}
\begin{sphinxShadowBox}
\sphinxstyletopictitle{Label}

\sphinxAtStartPar
immaterial entity
\end{sphinxShadowBox}

\begin{sphinxShadowBox}
\sphinxstyletopictitle{Definition}

\sphinxAtStartPar
An immaterial entity is the boundary or interior of a material entity
\end{sphinxShadowBox}

\begin{sphinxShadowBox}
\sphinxstyletopictitle{Editor’s note}

\sphinxAtStartPar
BFO 2 Reference: Immaterial entities are divided into two subgroups:boundaries and sites, which bound, or are demarcated in relation, to material entities, and which can thus change location, shape and size and as their material hosts move or change shape or size (for example: your nasal passage; the hold of a ship; the boundary of Wales (which moves with the rotation of the Earth) {[}38, 7, 10
\end{sphinxShadowBox}

\begin{sphinxShadowBox}
\sphinxstyletopictitle{Imported From}

\sphinxAtStartPar
\sphinxurl{http://purl.obolibrary.org/obo/bfo/2019-08-26/bfo.owl}
\end{sphinxShadowBox}
\begin{quote}

\index{BFO\_0000148@\spxentry{BFO\_0000148}!zero\sphinxhyphen{}dimensional temporal region@\spxentry{zero\sphinxhyphen{}dimensional temporal region}}\index{zero\sphinxhyphen{}dimensional temporal region@\spxentry{zero\sphinxhyphen{}dimensional temporal region}!BFO\_0000148@\spxentry{BFO\_0000148}}\ignorespaces \end{quote}


\subsection{BFO\_0000148 \sphinxhyphen{} zero\sphinxhyphen{}dimensional temporal region}
\label{\detokenize{doc-BFO_0000148:bfo-0000148-zero-dimensional-temporal-region}}\label{\detokenize{doc-BFO_0000148:index-0}}\label{\detokenize{doc-BFO_0000148::doc}}
\begin{sphinxShadowBox}
\sphinxstyletopictitle{Label}

\sphinxAtStartPar
zero\sphinxhyphen{}dimensional temporal region
\end{sphinxShadowBox}

\begin{sphinxShadowBox}
\sphinxstyletopictitle{Definition}

\sphinxAtStartPar
A temporal region of no duration.
\end{sphinxShadowBox}

\begin{sphinxShadowBox}
\sphinxstyletopictitle{Imported From}

\sphinxAtStartPar
\sphinxurl{http://purl.obolibrary.org/obo/bfo/2020/bfo.owl}
\end{sphinxShadowBox}
\begin{quote}

\index{IAO\_0000030@\spxentry{IAO\_0000030}!information content entity@\spxentry{information content entity}}\index{information content entity@\spxentry{information content entity}!IAO\_0000030@\spxentry{IAO\_0000030}}\ignorespaces \end{quote}


\subsection{IAO\_0000030 \sphinxhyphen{} information content entity}
\label{\detokenize{doc-IAO_0000030:iao-0000030-information-content-entity}}\label{\detokenize{doc-IAO_0000030:index-0}}\label{\detokenize{doc-IAO_0000030::doc}}
\begin{sphinxShadowBox}
\sphinxstyletopictitle{Label}

\sphinxAtStartPar
information content entity
\end{sphinxShadowBox}

\begin{sphinxShadowBox}
\sphinxstyletopictitle{Definition}

\sphinxAtStartPar
A generically dependent continuant that is about some thing.
\end{sphinxShadowBox}

\begin{sphinxShadowBox}
\sphinxstyletopictitle{Definition source}

\sphinxAtStartPar
OBI\_0000142
\end{sphinxShadowBox}

\begin{sphinxShadowBox}
\sphinxstyletopictitle{Example}

\sphinxAtStartPar
Examples of information content entites include journal articles, data, graphical layouts, and graphs.
\end{sphinxShadowBox}

\begin{sphinxShadowBox}
\sphinxstyletopictitle{Editor’s note}

\sphinxAtStartPar
2014\sphinxhyphen{}03\sphinxhyphen{}10: The use of “thing” is intended to be general enough to include universals and configurations (see \sphinxurl{https://groups.google.com/d/msg/information-ontology/GBxvYZCk1oc/-L6B5fSBBTQJ}).

\sphinxAtStartPar
information\_content\_entity ‘is\_encoded\_in’ some digital\_entity in obi before split (040907). information\_content\_entity ‘is\_encoded\_in’ some physical\_document in obi before split (040907).

\sphinxAtStartPar
Previous. An information content entity is a non\sphinxhyphen{}realizable information entity that ‘is encoded in’ some digital or physical entity.
\end{sphinxShadowBox}

\begin{sphinxShadowBox}
\sphinxstyletopictitle{Imported From}

\sphinxAtStartPar
\sphinxurl{http://purl.obolibrary.org/obo/iao/2020-12-09/iao.owl}
\end{sphinxShadowBox}

\begin{sphinxShadowBox}
\sphinxstyletopictitle{Term editor}

\sphinxAtStartPar
PERSON: Chris Stoeckert
\end{sphinxShadowBox}
\begin{quote}

\index{IAO\_0000422@\spxentry{IAO\_0000422}!postal address@\spxentry{postal address}}\index{postal address@\spxentry{postal address}!IAO\_0000422@\spxentry{IAO\_0000422}}\ignorespaces \end{quote}


\subsection{IAO\_0000422 \sphinxhyphen{} postal address}
\label{\detokenize{doc-IAO_0000422:iao-0000422-postal-address}}\label{\detokenize{doc-IAO_0000422:index-0}}\label{\detokenize{doc-IAO_0000422::doc}}
\begin{sphinxShadowBox}
\sphinxstyletopictitle{Label}

\sphinxAtStartPar
postal address
\end{sphinxShadowBox}

\begin{sphinxShadowBox}
\sphinxstyletopictitle{Definition}

\sphinxAtStartPar
A textual entity that is used as directive to deliver something to a person, or organization
\end{sphinxShadowBox}

\begin{sphinxShadowBox}
\sphinxstyletopictitle{Editor’s note}

\sphinxAtStartPar
2010\sphinxhyphen{}05\sphinxhyphen{}24 Alan Ruttenberg. Use label for the string representation. See issue \sphinxurl{https://github.com/information-artifact-ontology/IAO/issues/59}
\end{sphinxShadowBox}

\begin{sphinxShadowBox}
\sphinxstyletopictitle{Imported From}

\sphinxAtStartPar
\sphinxurl{http://purl.obolibrary.org/obo/iao/2017-03-24/iao.owl}
\end{sphinxShadowBox}
\begin{quote}

\index{IAO\_0000429@\spxentry{IAO\_0000429}!email address@\spxentry{email address}}\index{email address@\spxentry{email address}!IAO\_0000429@\spxentry{IAO\_0000429}}\ignorespaces \end{quote}


\subsection{IAO\_0000429 \sphinxhyphen{} email address}
\label{\detokenize{doc-IAO_0000429:iao-0000429-email-address}}\label{\detokenize{doc-IAO_0000429:index-0}}\label{\detokenize{doc-IAO_0000429::doc}}
\begin{sphinxShadowBox}
\sphinxstyletopictitle{Label}

\sphinxAtStartPar
email address
\end{sphinxShadowBox}

\begin{sphinxShadowBox}
\sphinxstyletopictitle{Definition}

\sphinxAtStartPar
A designation used to deliver email to a recipient.  Typically has an associated email representation
\end{sphinxShadowBox}

\begin{sphinxShadowBox}
\sphinxstyletopictitle{Editor’s note}

\sphinxAtStartPar
Alan Ruttenberg 1/3/2012 \sphinxhyphen{} Provisional id, see issue at \sphinxurl{https://github.com/information-artifact-ontology/IAO/issues/130\&thanks=130\&ts=1325636583}
\end{sphinxShadowBox}

\begin{sphinxShadowBox}
\sphinxstyletopictitle{Imported From}

\sphinxAtStartPar
\sphinxurl{http://purl.obolibrary.org/obo/iao/2017-03-24/iao.owl}
\end{sphinxShadowBox}

\begin{sphinxShadowBox}
\sphinxstyletopictitle{Term editor}

\sphinxAtStartPar
Person:Chris Stoeckart

\sphinxAtStartPar
Person:Alan Ruttenberg
\end{sphinxShadowBox}
\begin{quote}

\index{IAO\_0000578@\spxentry{IAO\_0000578}!centrally registered identifier@\spxentry{centrally registered identifier}}\index{centrally registered identifier@\spxentry{centrally registered identifier}!IAO\_0000578@\spxentry{IAO\_0000578}}\ignorespaces \end{quote}


\subsection{IAO\_0000578 \sphinxhyphen{} centrally registered identifier}
\label{\detokenize{doc-IAO_0000578:iao-0000578-centrally-registered-identifier}}\label{\detokenize{doc-IAO_0000578:index-0}}\label{\detokenize{doc-IAO_0000578::doc}}
\begin{sphinxShadowBox}
\sphinxstyletopictitle{Label}

\sphinxAtStartPar
centrally registered identifier
\end{sphinxShadowBox}

\begin{sphinxShadowBox}
\sphinxstyletopictitle{Alternate name}

\sphinxAtStartPar
CRID
\end{sphinxShadowBox}

\begin{sphinxShadowBox}
\sphinxstyletopictitle{Definition}

\sphinxAtStartPar
An information content entity that consists of a CRID symbol and additional information about the CRID registry to which it belongs.
\end{sphinxShadowBox}

\begin{sphinxShadowBox}
\sphinxstyletopictitle{Definition source}

\sphinxAtStartPar
Original proposal from Bjoern, discussions at IAO calls
\end{sphinxShadowBox}

\begin{sphinxShadowBox}
\sphinxstyletopictitle{Example}

\sphinxAtStartPar
The sentence “The article has Pubmed ID 12345.” contains a CRID that has two parts: one part is the CRID symbol, which is ‘12345’; the other part denotes the CRID registry, which is Pubmed.
\end{sphinxShadowBox}

\begin{sphinxShadowBox}
\sphinxstyletopictitle{Editor’s note}

\sphinxAtStartPar
Alan, IAO call 20101124: potentially the CRID denotes the instance it was associated with during creation.

\sphinxAtStartPar
Note, IAO call 20101124: URIs are not always CRID, as not centrally registered. We acknowledge that CRID is a subset of a larger identifier class, but this subset fulfills our current needs. OBI PURLs are CRID as they are registered with OCLC. UPCs (Universal Product Codes from AC Nielsen)are not CRID as they are not centrally registered.

\sphinxAtStartPar
2014\sphinxhyphen{}05\sphinxhyphen{}05: In defining this term we take no position on what the CRID denotes. In particular do not assume it denotes a \sphinxstyleemphasis{record} in the CRID registry (since the registry might not have ‘records’).
\end{sphinxShadowBox}

\begin{sphinxShadowBox}
\sphinxstyletopictitle{Imported From}

\sphinxAtStartPar
\sphinxurl{http://purl.obolibrary.org/obo/iao/ido/release/2021-02-19/ido.owl}
\end{sphinxShadowBox}

\begin{sphinxShadowBox}
\sphinxstyletopictitle{Term editor}

\sphinxAtStartPar
PERSON: Melanie Courtot

\sphinxAtStartPar
PERSON: Alan Ruttenberg

\sphinxAtStartPar
PERSON: Bjoern Peters

\sphinxAtStartPar
PERSON: Bill Hogan
\end{sphinxShadowBox}
\begin{quote}

\index{IAO\_0020000@\spxentry{IAO\_0020000}!identifier@\spxentry{identifier}}\index{identifier@\spxentry{identifier}!IAO\_0020000@\spxentry{IAO\_0020000}}\ignorespaces \end{quote}


\subsection{IAO\_0020000 \sphinxhyphen{} identifier}
\label{\detokenize{doc-IAO_0020000:iao-0020000-identifier}}\label{\detokenize{doc-IAO_0020000:index-0}}\label{\detokenize{doc-IAO_0020000::doc}}
\begin{sphinxShadowBox}
\sphinxstyletopictitle{Label}

\sphinxAtStartPar
identifier
\end{sphinxShadowBox}

\begin{sphinxShadowBox}
\sphinxstyletopictitle{Definition}

\sphinxAtStartPar
An identifier is an information content entity that is the outcome of a dubbing process and is used to refer to one instance of entity shared by a group of people to refer to that individual entity.
\end{sphinxShadowBox}

\begin{sphinxShadowBox}
\sphinxstyletopictitle{Imported From}

\sphinxAtStartPar
\sphinxurl{http://purl.obolibrary.org/obo/iao/ido/release/2021-02-19/ido.owl}
\end{sphinxShadowBox}
\begin{quote}

\index{IAO\_0022003@\spxentry{IAO\_0022003}!crossref funder identifier@\spxentry{crossref funder identifier}}\index{crossref funder identifier@\spxentry{crossref funder identifier}!IAO\_0022003@\spxentry{IAO\_0022003}}\ignorespaces \end{quote}


\subsection{IAO\_0022003 \sphinxhyphen{} crossref funder identifier}
\label{\detokenize{doc-IAO_0022003:iao-0022003-crossref-funder-identifier}}\label{\detokenize{doc-IAO_0022003:index-0}}\label{\detokenize{doc-IAO_0022003::doc}}
\begin{sphinxShadowBox}
\sphinxstyletopictitle{Label}

\sphinxAtStartPar
crossref funder identifier
\end{sphinxShadowBox}

\begin{sphinxShadowBox}
\sphinxstyletopictitle{Definition}

\sphinxAtStartPar
An identifier assigned by CrossRef to an organization which has funded a project resulting in a published work
\end{sphinxShadowBox}

\begin{sphinxShadowBox}
\sphinxstyletopictitle{Example}

\sphinxAtStartPar
The CrossRef ID for the University of Florida is 100007698.  When authors cite the university as a funder of their work, CrossRef uses this number to identify the university
\end{sphinxShadowBox}

\begin{sphinxShadowBox}
\sphinxstyletopictitle{Imported From}

\sphinxAtStartPar
\sphinxurl{http://purl.obolibrary.org/obo/iao/ido/release/2021-02-19/ido.owl}
\end{sphinxShadowBox}

\begin{sphinxShadowBox}
\sphinxstyletopictitle{Term editor}

\sphinxAtStartPar
PERSON: Michael Conlon
\end{sphinxShadowBox}
\begin{quote}

\index{IAO\_0022006@\spxentry{IAO\_0022006}!dbpedia identifier@\spxentry{dbpedia identifier}}\index{dbpedia identifier@\spxentry{dbpedia identifier}!IAO\_0022006@\spxentry{IAO\_0022006}}\ignorespaces \end{quote}


\subsection{IAO\_0022006 \sphinxhyphen{} dbpedia identifier}
\label{\detokenize{doc-IAO_0022006:iao-0022006-dbpedia-identifier}}\label{\detokenize{doc-IAO_0022006:index-0}}\label{\detokenize{doc-IAO_0022006::doc}}
\begin{sphinxShadowBox}
\sphinxstyletopictitle{Label}

\sphinxAtStartPar
dbpedia identifier
\end{sphinxShadowBox}

\begin{sphinxShadowBox}
\sphinxstyletopictitle{Definition}

\sphinxAtStartPar
A URL used by DBpedia to identify an entity
\end{sphinxShadowBox}

\begin{sphinxShadowBox}
\sphinxstyletopictitle{Definition source}

\sphinxAtStartPar
\sphinxurl{https://dbpedia.org}
\end{sphinxShadowBox}

\begin{sphinxShadowBox}
\sphinxstyletopictitle{Example}

\sphinxAtStartPar
The DBpedia URL for the football player Cristiano Ronaldo is \sphinxurl{http://dbpedia.org/resource/Cristiano\_Ronaldo}
\end{sphinxShadowBox}

\begin{sphinxShadowBox}
\sphinxstyletopictitle{Editor’s note}

\sphinxAtStartPar
clarify
\end{sphinxShadowBox}

\begin{sphinxShadowBox}
\sphinxstyletopictitle{Imported From}

\sphinxAtStartPar
\sphinxurl{http://purl.obolibrary.org/obo/iao/ido/release/2021-02-19/ido.owl}
\end{sphinxShadowBox}

\begin{sphinxShadowBox}
\sphinxstyletopictitle{Term editor}

\sphinxAtStartPar
PERSON: Michael Conlon
\end{sphinxShadowBox}
\begin{quote}

\index{IAO\_0022010@\spxentry{IAO\_0022010}!global research organization identifier@\spxentry{global research organization identifier}}\index{global research organization identifier@\spxentry{global research organization identifier}!IAO\_0022010@\spxentry{IAO\_0022010}}\ignorespaces \end{quote}


\subsection{IAO\_0022010 \sphinxhyphen{} global research organization identifier}
\label{\detokenize{doc-IAO_0022010:iao-0022010-global-research-organization-identifier}}\label{\detokenize{doc-IAO_0022010:index-0}}\label{\detokenize{doc-IAO_0022010::doc}}
\begin{sphinxShadowBox}
\sphinxstyletopictitle{Label}

\sphinxAtStartPar
global research organization identifier
\end{sphinxShadowBox}

\begin{sphinxShadowBox}
\sphinxstyletopictitle{Definition}

\sphinxAtStartPar
An identifier assigned and managed by Digital Science for the purpose of denoting research organizations
\end{sphinxShadowBox}

\begin{sphinxShadowBox}
\sphinxstyletopictitle{Definition source}

\sphinxAtStartPar
\sphinxurl{https://grid.ac}
\end{sphinxShadowBox}

\begin{sphinxShadowBox}
\sphinxstyletopictitle{Example}

\sphinxAtStartPar
The ISBN\sphinxhyphen{}13 for Native Son, by Richard Wright, Harper Perennial, Reissued 2005 is 978\sphinxhyphen{}0\sphinxhyphen{}06\sphinxhyphen{}083756\sphinxhyphen{}3
\end{sphinxShadowBox}

\begin{sphinxShadowBox}
\sphinxstyletopictitle{Imported From}

\sphinxAtStartPar
\sphinxurl{http://purl.obolibrary.org/obo/iao/ido/release/2021-02-19/ido.owl}
\end{sphinxShadowBox}

\begin{sphinxShadowBox}
\sphinxstyletopictitle{Term editor}

\sphinxAtStartPar
PERSON: Michael Conlon
\end{sphinxShadowBox}
\begin{quote}

\index{IAO\_0022014@\spxentry{IAO\_0022014}!international standard name identifier@\spxentry{international standard name identifier}}\index{international standard name identifier@\spxentry{international standard name identifier}!IAO\_0022014@\spxentry{IAO\_0022014}}\ignorespaces \end{quote}


\subsection{IAO\_0022014 \sphinxhyphen{} international standard name identifier}
\label{\detokenize{doc-IAO_0022014:iao-0022014-international-standard-name-identifier}}\label{\detokenize{doc-IAO_0022014:index-0}}\label{\detokenize{doc-IAO_0022014::doc}}
\begin{sphinxShadowBox}
\sphinxstyletopictitle{Label}

\sphinxAtStartPar
international standard name identifier
\end{sphinxShadowBox}

\begin{sphinxShadowBox}
\sphinxstyletopictitle{Definition}

\sphinxAtStartPar
An identifier for persons and organizations which may be assigned by matching algorithms based on records provided by publishers
\end{sphinxShadowBox}

\begin{sphinxShadowBox}
\sphinxstyletopictitle{Definition source}

\sphinxAtStartPar
\sphinxurl{https://isni.org/page/what-is-isni/}
\end{sphinxShadowBox}

\begin{sphinxShadowBox}
\sphinxstyletopictitle{Editor’s note}

\sphinxAtStartPar
spell out
\end{sphinxShadowBox}

\begin{sphinxShadowBox}
\sphinxstyletopictitle{Imported From}

\sphinxAtStartPar
\sphinxurl{http://purl.obolibrary.org/obo/iao/ido/release/2021-02-19/ido.owl}
\end{sphinxShadowBox}

\begin{sphinxShadowBox}
\sphinxstyletopictitle{Term editor}

\sphinxAtStartPar
PERSON: Michael Conlon
\end{sphinxShadowBox}
\begin{quote}

\index{IAO\_0022022@\spxentry{IAO\_0022022}!research organization registry identifier@\spxentry{research organization registry identifier}}\index{research organization registry identifier@\spxentry{research organization registry identifier}!IAO\_0022022@\spxentry{IAO\_0022022}}\ignorespaces \end{quote}


\subsection{IAO\_0022022 \sphinxhyphen{} research organization registry identifier}
\label{\detokenize{doc-IAO_0022022:iao-0022022-research-organization-registry-identifier}}\label{\detokenize{doc-IAO_0022022:index-0}}\label{\detokenize{doc-IAO_0022022::doc}}
\begin{sphinxShadowBox}
\sphinxstyletopictitle{Label}

\sphinxAtStartPar
research organization registry identifier
\end{sphinxShadowBox}

\begin{sphinxShadowBox}
\sphinxstyletopictitle{Definition}

\sphinxAtStartPar
An identifier assigned by ROR to research organizations in the world
\end{sphinxShadowBox}

\begin{sphinxShadowBox}
\sphinxstyletopictitle{Definition source}

\sphinxAtStartPar
\sphinxurl{http://ror.org}
\end{sphinxShadowBox}

\begin{sphinxShadowBox}
\sphinxstyletopictitle{Imported From}

\sphinxAtStartPar
\sphinxurl{http://purl.obolibrary.org/obo/iao/ido/release/2021-02-19/ido.owl}
\end{sphinxShadowBox}

\begin{sphinxShadowBox}
\sphinxstyletopictitle{Term editor}

\sphinxAtStartPar
PERSON: Michael Conlon
\end{sphinxShadowBox}
\begin{quote}

\index{IAO\_0022027@\spxentry{IAO\_0022027}!wikidata q number@\spxentry{wikidata q number}}\index{wikidata q number@\spxentry{wikidata q number}!IAO\_0022027@\spxentry{IAO\_0022027}}\ignorespaces \end{quote}


\subsection{IAO\_0022027 \sphinxhyphen{} wikidata q number}
\label{\detokenize{doc-IAO_0022027:iao-0022027-wikidata-q-number}}\label{\detokenize{doc-IAO_0022027:index-0}}\label{\detokenize{doc-IAO_0022027::doc}}
\begin{sphinxShadowBox}
\sphinxstyletopictitle{Label}

\sphinxAtStartPar
wikidata q number
\end{sphinxShadowBox}

\begin{sphinxShadowBox}
\sphinxstyletopictitle{Definition}

\sphinxAtStartPar
QID (or Q number) is the unique identifier of a data item on Wikidata, comprising the letter “Q” followed by one or more digits.
\end{sphinxShadowBox}

\begin{sphinxShadowBox}
\sphinxstyletopictitle{Definition source}

\sphinxAtStartPar
\sphinxurl{https://www.wikidata.org/wiki/Q43649390}
\end{sphinxShadowBox}

\begin{sphinxShadowBox}
\sphinxstyletopictitle{Imported From}

\sphinxAtStartPar
\sphinxurl{http://purl.obolibrary.org/obo/iao/ido/release/2021-02-19/ido.owl}
\end{sphinxShadowBox}

\begin{sphinxShadowBox}
\sphinxstyletopictitle{Term editor}

\sphinxAtStartPar
PERSON: Michael Conlon
\end{sphinxShadowBox}
\begin{quote}

\index{IAO\_0022057@\spxentry{IAO\_0022057}!ringgold identifier@\spxentry{ringgold identifier}}\index{ringgold identifier@\spxentry{ringgold identifier}!IAO\_0022057@\spxentry{IAO\_0022057}}\ignorespaces \end{quote}


\subsection{IAO\_0022057 \sphinxhyphen{} ringgold identifier}
\label{\detokenize{doc-IAO_0022057:iao-0022057-ringgold-identifier}}\label{\detokenize{doc-IAO_0022057:index-0}}\label{\detokenize{doc-IAO_0022057::doc}}
\begin{sphinxShadowBox}
\sphinxstyletopictitle{Label}

\sphinxAtStartPar
ringgold identifier
\end{sphinxShadowBox}

\begin{sphinxShadowBox}
\sphinxstyletopictitle{Definition}

\sphinxAtStartPar
The Ringgold Identifier is a unique numerical identifier applied to organizations in the scholarly supply chain
\end{sphinxShadowBox}

\begin{sphinxShadowBox}
\sphinxstyletopictitle{Definition source}

\sphinxAtStartPar
\sphinxurl{https://www.ringgold.com/ringgold-identifier/}
\end{sphinxShadowBox}

\begin{sphinxShadowBox}
\sphinxstyletopictitle{Imported From}

\sphinxAtStartPar
\sphinxurl{http://purl.obolibrary.org/obo/iao/ido/release/2021-02-19/ido.owl}
\end{sphinxShadowBox}

\begin{sphinxShadowBox}
\sphinxstyletopictitle{Term editor}

\sphinxAtStartPar
PERSON: Michael Conlon
\end{sphinxShadowBox}
\begin{quote}

\index{NCBITaxon\_9606@\spxentry{NCBITaxon\_9606}!Homo sapiens@\spxentry{Homo sapiens}}\index{Homo sapiens@\spxentry{Homo sapiens}!NCBITaxon\_9606@\spxentry{NCBITaxon\_9606}}\ignorespaces \end{quote}


\subsection{NCBITaxon\_9606 \sphinxhyphen{} Homo sapiens}
\label{\detokenize{doc-NCBITaxon_9606:ncbitaxon-9606-homo-sapiens}}\label{\detokenize{doc-NCBITaxon_9606:index-0}}\label{\detokenize{doc-NCBITaxon_9606::doc}}
\begin{sphinxShadowBox}
\sphinxstyletopictitle{Label}

\sphinxAtStartPar
Homo sapiens
\end{sphinxShadowBox}

\begin{sphinxShadowBox}
\sphinxstyletopictitle{Alternate name}

\sphinxAtStartPar
human being

\sphinxAtStartPar
human
\end{sphinxShadowBox}

\begin{sphinxShadowBox}
\sphinxstyletopictitle{Definition}

\sphinxAtStartPar
The species of bipedal primates to which moden humans belong
\end{sphinxShadowBox}

\begin{sphinxShadowBox}
\sphinxstyletopictitle{Imported From}

\sphinxAtStartPar
\sphinxurl{http://purl.obolibrary.org/obo/obi/2021-04-06/obi.owl}
\end{sphinxShadowBox}
\begin{quote}

\index{ORG\_0000001@\spxentry{ORG\_0000001}!organization@\spxentry{organization}}\index{organization@\spxentry{organization}!ORG\_0000001@\spxentry{ORG\_0000001}}\ignorespaces \end{quote}


\subsection{ORG\_0000001 \sphinxhyphen{} organization}
\label{\detokenize{doc-ORG_0000001:org-0000001-organization}}\label{\detokenize{doc-ORG_0000001:index-0}}\label{\detokenize{doc-ORG_0000001::doc}}
\begin{sphinxShadowBox}
\sphinxstyletopictitle{Label}

\sphinxAtStartPar
organization
\end{sphinxShadowBox}

\begin{sphinxShadowBox}
\sphinxstyletopictitle{Definition}

\sphinxAtStartPar
A group of people recognized as such by people outside the group.
\end{sphinxShadowBox}

\begin{sphinxShadowBox}
\sphinxstyletopictitle{Definition source}

\sphinxAtStartPar
Michael Conlon \sphinxurl{https://orcid.org/0000-0002-1304-8447}
\end{sphinxShadowBox}

\begin{sphinxShadowBox}
\sphinxstyletopictitle{Example}

\sphinxAtStartPar
A political party, a homeowners association, a football team, a publisher, a government agency, an organized religion
\end{sphinxShadowBox}

\begin{sphinxShadowBox}
\sphinxstyletopictitle{Similar term in VIVO 1 Ontology}

\sphinxAtStartPar
\sphinxurl{http://xmlns.com/foaf/0.1/Organization}
\end{sphinxShadowBox}

\begin{sphinxShadowBox}
\sphinxstyletopictitle{Term editor}

\sphinxAtStartPar
Michael Conlon \sphinxurl{https://orcid.org/0000-0002-1304-8447}
\end{sphinxShadowBox}
\begin{quote}

\index{ORG\_0000002@\spxentry{ORG\_0000002}!government organization@\spxentry{government organization}}\index{government organization@\spxentry{government organization}!ORG\_0000002@\spxentry{ORG\_0000002}}\ignorespaces \end{quote}


\subsection{ORG\_0000002 \sphinxhyphen{} government organization}
\label{\detokenize{doc-ORG_0000002:org-0000002-government-organization}}\label{\detokenize{doc-ORG_0000002:index-0}}\label{\detokenize{doc-ORG_0000002::doc}}
\begin{sphinxShadowBox}
\sphinxstyletopictitle{Label}

\sphinxAtStartPar
government organization
\end{sphinxShadowBox}

\begin{sphinxShadowBox}
\sphinxstyletopictitle{Alternate name}

\sphinxAtStartPar
government
\end{sphinxShadowBox}

\begin{sphinxShadowBox}
\sphinxstyletopictitle{Definition}

\sphinxAtStartPar
An organization which is the body of persons that constitutes the governing authority of a political unit
\end{sphinxShadowBox}

\begin{sphinxShadowBox}
\sphinxstyletopictitle{Definition source}

\sphinxAtStartPar
\sphinxurl{https://www.merriam-webster.com/dictionary/government}
\end{sphinxShadowBox}

\begin{sphinxShadowBox}
\sphinxstyletopictitle{Example}

\sphinxAtStartPar
The State of Florida is recognized as a government organization by the United States.  The government of the United States is recognized by other governments.  Note there is no “part of” relationship here.  Each is a government organization.
\end{sphinxShadowBox}

\begin{sphinxShadowBox}
\sphinxstyletopictitle{Similar term in VIVO 1 Ontology}

\sphinxAtStartPar
\sphinxurl{http://vivoweb.org/ontology/core\#GovernmentAgency}
\end{sphinxShadowBox}

\begin{sphinxShadowBox}
\sphinxstyletopictitle{Term editor}

\sphinxAtStartPar
Michael Conlon \sphinxurl{https://orcid.org/0000-0002-1304-8447}
\end{sphinxShadowBox}
\begin{quote}

\index{ORG\_0000003@\spxentry{ORG\_0000003}!company@\spxentry{company}}\index{company@\spxentry{company}!ORG\_0000003@\spxentry{ORG\_0000003}}\ignorespaces \end{quote}


\subsection{ORG\_0000003 \sphinxhyphen{} company}
\label{\detokenize{doc-ORG_0000003:org-0000003-company}}\label{\detokenize{doc-ORG_0000003:index-0}}\label{\detokenize{doc-ORG_0000003::doc}}
\begin{sphinxShadowBox}
\sphinxstyletopictitle{Label}

\sphinxAtStartPar
company
\end{sphinxShadowBox}

\begin{sphinxShadowBox}
\sphinxstyletopictitle{Definition}

\sphinxAtStartPar
A legal entity of associated persons created for a specific purpose, typically commercial, in which excess revenue may be distributed to the company’s owners.
\end{sphinxShadowBox}

\begin{sphinxShadowBox}
\sphinxstyletopictitle{Definition source}

\sphinxAtStartPar
See \sphinxurl{https://www.merriam-webster.com/dictionary/company}
\end{sphinxShadowBox}

\begin{sphinxShadowBox}
\sphinxstyletopictitle{Example}

\sphinxAtStartPar
BASF, The University of Phoenix, Facebook, Elsevier, Apple, Google, Amazon
\end{sphinxShadowBox}

\begin{sphinxShadowBox}
\sphinxstyletopictitle{Similar term in VIVO 1 Ontology}

\sphinxAtStartPar
\sphinxurl{http://vivoweb.org/ontology/core\#Company}
\end{sphinxShadowBox}

\begin{sphinxShadowBox}
\sphinxstyletopictitle{Term editor}

\sphinxAtStartPar
Michael Conlon \sphinxurl{https://orcid.org/0000-0002-1304-8447}
\end{sphinxShadowBox}
\begin{quote}

\index{ORG\_0000004@\spxentry{ORG\_0000004}!nonprofit organization@\spxentry{nonprofit organization}}\index{nonprofit organization@\spxentry{nonprofit organization}!ORG\_0000004@\spxentry{ORG\_0000004}}\ignorespaces \end{quote}


\subsection{ORG\_0000004 \sphinxhyphen{} nonprofit organization}
\label{\detokenize{doc-ORG_0000004:org-0000004-nonprofit-organization}}\label{\detokenize{doc-ORG_0000004:index-0}}\label{\detokenize{doc-ORG_0000004::doc}}
\begin{sphinxShadowBox}
\sphinxstyletopictitle{Label}

\sphinxAtStartPar
nonprofit organization
\end{sphinxShadowBox}

\begin{sphinxShadowBox}
\sphinxstyletopictitle{Alternate name}

\sphinxAtStartPar
not for profit
\end{sphinxShadowBox}

\begin{sphinxShadowBox}
\sphinxstyletopictitle{Definition}

\sphinxAtStartPar
A legal entity of associated persons created for a specific purpose, typically a mission, in which excess revenue is reinvested to serve the entity’s mission
\end{sphinxShadowBox}

\begin{sphinxShadowBox}
\sphinxstyletopictitle{Definition source}

\sphinxAtStartPar
Michael Conlon \sphinxurl{https://orcid.org/0000-0002-1304-8447}
\end{sphinxShadowBox}

\begin{sphinxShadowBox}
\sphinxstyletopictitle{Example}

\sphinxAtStartPar
Doctors Without Borders, Duke University, The American Medical Association
\end{sphinxShadowBox}

\begin{sphinxShadowBox}
\sphinxstyletopictitle{Term editor}

\sphinxAtStartPar
Michael Conlon \sphinxurl{https://orcid.org/0000-0002-1304-8447}
\end{sphinxShadowBox}
\begin{quote}

\index{ORG\_0000005@\spxentry{ORG\_0000005}!informal organization@\spxentry{informal organization}}\index{informal organization@\spxentry{informal organization}!ORG\_0000005@\spxentry{ORG\_0000005}}\ignorespaces \end{quote}


\subsection{ORG\_0000005 \sphinxhyphen{} informal organization}
\label{\detokenize{doc-ORG_0000005:org-0000005-informal-organization}}\label{\detokenize{doc-ORG_0000005:index-0}}\label{\detokenize{doc-ORG_0000005::doc}}
\begin{sphinxShadowBox}
\sphinxstyletopictitle{Label}

\sphinxAtStartPar
informal organization
\end{sphinxShadowBox}

\begin{sphinxShadowBox}
\sphinxstyletopictitle{Definition}

\sphinxAtStartPar
A group of people recognized as such by people outside the group. Without legal standing.
\end{sphinxShadowBox}

\begin{sphinxShadowBox}
\sphinxstyletopictitle{Definition source}

\sphinxAtStartPar
Michael Conlon \sphinxurl{https://orcid.org/0000-0002-1304-8447}
\end{sphinxShadowBox}

\begin{sphinxShadowBox}
\sphinxstyletopictitle{Example}

\sphinxAtStartPar
A student club, a university committee, the VIVO Ontology Interest Group
\end{sphinxShadowBox}

\begin{sphinxShadowBox}
\sphinxstyletopictitle{Editor’s note}

\sphinxAtStartPar
There are many names associated with informal organizations, such as club, group, party, team, cell, task force, work group, interest group, meeting, roundtable, .  So far dispositions and qualities that would distinguish the entities with various names have not been formulated.
\end{sphinxShadowBox}

\begin{sphinxShadowBox}
\sphinxstyletopictitle{Term editor}

\sphinxAtStartPar
Michael Conlon \sphinxurl{https://orcid.org/0000-0002-1304-8447}
\end{sphinxShadowBox}
\begin{quote}

\index{ORG\_0000006@\spxentry{ORG\_0000006}!organization part@\spxentry{organization part}}\index{organization part@\spxentry{organization part}!ORG\_0000006@\spxentry{ORG\_0000006}}\ignorespaces \end{quote}


\subsection{ORG\_0000006 \sphinxhyphen{} organization part}
\label{\detokenize{doc-ORG_0000006:org-0000006-organization-part}}\label{\detokenize{doc-ORG_0000006:index-0}}\label{\detokenize{doc-ORG_0000006::doc}}
\begin{sphinxShadowBox}
\sphinxstyletopictitle{Label}

\sphinxAtStartPar
organization part
\end{sphinxShadowBox}

\begin{sphinxShadowBox}
\sphinxstyletopictitle{Definition}

\sphinxAtStartPar
An organization which exists as part of another organization.  Implies a part\_of relationship to another organization
\end{sphinxShadowBox}

\begin{sphinxShadowBox}
\sphinxstyletopictitle{Definition source}

\sphinxAtStartPar
Michael Conlon \sphinxurl{https://orcid.org/0000-0002-1304-8447}
\end{sphinxShadowBox}

\begin{sphinxShadowBox}
\sphinxstyletopictitle{Example}

\sphinxAtStartPar
The US Navy is an organizational part of the US Department of Defense.  The History Department is an organizational part of its college
\end{sphinxShadowBox}

\begin{sphinxShadowBox}
\sphinxstyletopictitle{Term editor}

\sphinxAtStartPar
Michael Conlon \sphinxurl{https://orcid.org/0000-0002-1304-8447}
\end{sphinxShadowBox}
\begin{quote}

\index{ORG\_0000007@\spxentry{ORG\_0000007}!university disposition@\spxentry{university disposition}}\index{university disposition@\spxentry{university disposition}!ORG\_0000007@\spxentry{ORG\_0000007}}\ignorespaces \end{quote}


\subsection{ORG\_0000007 \sphinxhyphen{} university disposition}
\label{\detokenize{doc-ORG_0000007:org-0000007-university-disposition}}\label{\detokenize{doc-ORG_0000007:index-0}}\label{\detokenize{doc-ORG_0000007::doc}}
\begin{sphinxShadowBox}
\sphinxstyletopictitle{Label}

\sphinxAtStartPar
university disposition
\end{sphinxShadowBox}

\begin{sphinxShadowBox}
\sphinxstyletopictitle{Definition}

\sphinxAtStartPar
A disposition to award academic degrees and conduct research in a variety of academic disciplines
\end{sphinxShadowBox}

\begin{sphinxShadowBox}
\sphinxstyletopictitle{Definition source}

\sphinxAtStartPar
\sphinxurl{https://en.wikipedia.org/wiki/University}
\end{sphinxShadowBox}

\begin{sphinxShadowBox}
\sphinxstyletopictitle{Example}

\sphinxAtStartPar
The University of Bologna is an organization that has a university disposition
\end{sphinxShadowBox}

\begin{sphinxShadowBox}
\sphinxstyletopictitle{Editor’s note}

\sphinxAtStartPar
Some dispositions may well be functions \textendash{} the organization does not exist without it.
\end{sphinxShadowBox}

\begin{sphinxShadowBox}
\sphinxstyletopictitle{Term editor}

\sphinxAtStartPar
Michael Conlon \sphinxurl{https://orcid.org/0000-0002-1304-8447}
\end{sphinxShadowBox}
\begin{quote}

\index{ORG\_0000008@\spxentry{ORG\_0000008}!association disposition@\spxentry{association disposition}}\index{association disposition@\spxentry{association disposition}!ORG\_0000008@\spxentry{ORG\_0000008}}\ignorespaces \end{quote}


\subsection{ORG\_0000008 \sphinxhyphen{} association disposition}
\label{\detokenize{doc-ORG_0000008:org-0000008-association-disposition}}\label{\detokenize{doc-ORG_0000008:index-0}}\label{\detokenize{doc-ORG_0000008::doc}}
\begin{sphinxShadowBox}
\sphinxstyletopictitle{Label}

\sphinxAtStartPar
association disposition
\end{sphinxShadowBox}

\begin{sphinxShadowBox}
\sphinxstyletopictitle{Alternate name}

\sphinxAtStartPar
professional society
\end{sphinxShadowBox}

\begin{sphinxShadowBox}
\sphinxstyletopictitle{Definition}

\sphinxAtStartPar
A disposition to organize organizations or individuals along and industry or academic lines
\end{sphinxShadowBox}

\begin{sphinxShadowBox}
\sphinxstyletopictitle{Definition source}

\sphinxAtStartPar
Michael Conlon \sphinxurl{https://orcid.org/0000-0002-1304-8447}
\end{sphinxShadowBox}

\begin{sphinxShadowBox}
\sphinxstyletopictitle{Example}

\sphinxAtStartPar
The American Medical Association; The AFL\sphinxhyphen{}CIO; The Institution of Railway Signal Engineers
\end{sphinxShadowBox}

\begin{sphinxShadowBox}
\sphinxstyletopictitle{Term editor}

\sphinxAtStartPar
Michael Conlon \sphinxurl{https://orcid.org/0000-0002-1304-8447}
\end{sphinxShadowBox}
\begin{quote}

\index{ORG\_0000009@\spxentry{ORG\_0000009}!consortium disposition@\spxentry{consortium disposition}}\index{consortium disposition@\spxentry{consortium disposition}!ORG\_0000009@\spxentry{ORG\_0000009}}\ignorespaces \end{quote}


\subsection{ORG\_0000009 \sphinxhyphen{} consortium disposition}
\label{\detokenize{doc-ORG_0000009:org-0000009-consortium-disposition}}\label{\detokenize{doc-ORG_0000009:index-0}}\label{\detokenize{doc-ORG_0000009::doc}}
\begin{sphinxShadowBox}
\sphinxstyletopictitle{Label}

\sphinxAtStartPar
consortium disposition
\end{sphinxShadowBox}

\begin{sphinxShadowBox}
\sphinxstyletopictitle{Definition}

\sphinxAtStartPar
A disposition to organize organizations along industry or academic lines
\end{sphinxShadowBox}

\begin{sphinxShadowBox}
\sphinxstyletopictitle{Definition source}

\sphinxAtStartPar
Michael Conlon \sphinxurl{https://orcid.org/0000-0002-1304-8447}
\end{sphinxShadowBox}

\begin{sphinxShadowBox}
\sphinxstyletopictitle{Example}

\sphinxAtStartPar
The Association of Research Libraries; The NIH Common Fund Metabolomics Consortium
\end{sphinxShadowBox}

\begin{sphinxShadowBox}
\sphinxstyletopictitle{Editor’s note}

\sphinxAtStartPar
The word “association” in english may mean many diffierent things.  Often consortiums have the word “association” in their title.
\end{sphinxShadowBox}

\begin{sphinxShadowBox}
\sphinxstyletopictitle{Term editor}

\sphinxAtStartPar
Michael Conlon \sphinxurl{https://orcid.org/0000-0002-1304-8447}
\end{sphinxShadowBox}
\begin{quote}

\index{ORG\_0000010@\spxentry{ORG\_0000010}!service provider disposition@\spxentry{service provider disposition}}\index{service provider disposition@\spxentry{service provider disposition}!ORG\_0000010@\spxentry{ORG\_0000010}}\ignorespaces \end{quote}


\subsection{ORG\_0000010 \sphinxhyphen{} service provider disposition}
\label{\detokenize{doc-ORG_0000010:org-0000010-service-provider-disposition}}\label{\detokenize{doc-ORG_0000010:index-0}}\label{\detokenize{doc-ORG_0000010::doc}}
\begin{sphinxShadowBox}
\sphinxstyletopictitle{Label}

\sphinxAtStartPar
service provider disposition
\end{sphinxShadowBox}

\begin{sphinxShadowBox}
\sphinxstyletopictitle{Definition}

\sphinxAtStartPar
A disposition to provide service with or without a fee
\end{sphinxShadowBox}

\begin{sphinxShadowBox}
\sphinxstyletopictitle{Definition source}

\sphinxAtStartPar
Michael Conlon \sphinxurl{https://orcid.org/0000-0002-1304-8447}
\end{sphinxShadowBox}

\begin{sphinxShadowBox}
\sphinxstyletopictitle{Example}

\sphinxAtStartPar
A hospital has a service provider disposition to provide medical services to patients
\end{sphinxShadowBox}

\begin{sphinxShadowBox}
\sphinxstyletopictitle{Editor’s note}

\sphinxAtStartPar
We can expect to have many types of service provider dispositions
\end{sphinxShadowBox}

\begin{sphinxShadowBox}
\sphinxstyletopictitle{Term editor}

\sphinxAtStartPar
Michael Conlon \sphinxurl{https://orcid.org/0000-0002-1304-8447}
\end{sphinxShadowBox}
\begin{quote}

\index{ORG\_0000011@\spxentry{ORG\_0000011}!laboratory service provider disposition@\spxentry{laboratory service provider disposition}}\index{laboratory service provider disposition@\spxentry{laboratory service provider disposition}!ORG\_0000011@\spxentry{ORG\_0000011}}\ignorespaces \end{quote}


\subsection{ORG\_0000011 \sphinxhyphen{} laboratory service provider disposition}
\label{\detokenize{doc-ORG_0000011:org-0000011-laboratory-service-provider-disposition}}\label{\detokenize{doc-ORG_0000011:index-0}}\label{\detokenize{doc-ORG_0000011::doc}}
\begin{sphinxShadowBox}
\sphinxstyletopictitle{Label}

\sphinxAtStartPar
laboratory service provider disposition
\end{sphinxShadowBox}

\begin{sphinxShadowBox}
\sphinxstyletopictitle{Definition}

\sphinxAtStartPar
A disposition to provide laboratory services.  In the US, organization parts in universities that have a disposition to provide laboratory services to others are called core laboratories
\end{sphinxShadowBox}

\begin{sphinxShadowBox}
\sphinxstyletopictitle{Definition source}

\sphinxAtStartPar
Michael Conlon \sphinxurl{https://orcid.org/0000-0002-1304-8447}
\end{sphinxShadowBox}

\begin{sphinxShadowBox}
\sphinxstyletopictitle{Example}

\sphinxAtStartPar
The Interdisciplinary Center for Biotechnology Rsearch is a part of the university with a disposition to offer laboratory services to others
\end{sphinxShadowBox}

\begin{sphinxShadowBox}
\sphinxstyletopictitle{Editor’s note}

\sphinxAtStartPar
The word “core laboratory” may indicate that an organization has a disposition of laboratory services
\end{sphinxShadowBox}

\begin{sphinxShadowBox}
\sphinxstyletopictitle{Term editor}

\sphinxAtStartPar
Michael Conlon \sphinxurl{https://orcid.org/0000-0002-1304-8447}
\end{sphinxShadowBox}
\begin{quote}

\index{ORG\_0000012@\spxentry{ORG\_0000012}!extension provider disposition@\spxentry{extension provider disposition}}\index{extension provider disposition@\spxentry{extension provider disposition}!ORG\_0000012@\spxentry{ORG\_0000012}}\ignorespaces \end{quote}


\subsection{ORG\_0000012 \sphinxhyphen{} extension provider disposition}
\label{\detokenize{doc-ORG_0000012:org-0000012-extension-provider-disposition}}\label{\detokenize{doc-ORG_0000012:index-0}}\label{\detokenize{doc-ORG_0000012::doc}}
\begin{sphinxShadowBox}
\sphinxstyletopictitle{Label}

\sphinxAtStartPar
extension provider disposition
\end{sphinxShadowBox}

\begin{sphinxShadowBox}
\sphinxstyletopictitle{Alternate name}

\sphinxAtStartPar
agricultural extension
\end{sphinxShadowBox}

\begin{sphinxShadowBox}
\sphinxstyletopictitle{Definition}

\sphinxAtStartPar
A disposition to provide extension services, typically in agriculture.  Extension provides access to university research findings and advice to agriculturalists.
\end{sphinxShadowBox}

\begin{sphinxShadowBox}
\sphinxstyletopictitle{Definition source}

\sphinxAtStartPar
Michael Conlon \sphinxurl{https://orcid.org/0000-0002-1304-8447}
\end{sphinxShadowBox}

\begin{sphinxShadowBox}
\sphinxstyletopictitle{Example}

\sphinxAtStartPar
Texas A\&M Agrilife Extension is a part of the university with a disposition for providing agricultural exteension services
\end{sphinxShadowBox}

\begin{sphinxShadowBox}
\sphinxstyletopictitle{Editor’s note}

\sphinxAtStartPar
Extension is most commonly used to describe agircultural extension services.
\end{sphinxShadowBox}

\begin{sphinxShadowBox}
\sphinxstyletopictitle{Term editor}

\sphinxAtStartPar
Michael Conlon \sphinxurl{https://orcid.org/0000-0002-1304-8447}
\end{sphinxShadowBox}
\begin{quote}

\index{ORG\_0000013@\spxentry{ORG\_0000013}!technology transfer disposition@\spxentry{technology transfer disposition}}\index{technology transfer disposition@\spxentry{technology transfer disposition}!ORG\_0000013@\spxentry{ORG\_0000013}}\ignorespaces \end{quote}


\subsection{ORG\_0000013 \sphinxhyphen{} technology transfer disposition}
\label{\detokenize{doc-ORG_0000013:org-0000013-technology-transfer-disposition}}\label{\detokenize{doc-ORG_0000013:index-0}}\label{\detokenize{doc-ORG_0000013::doc}}
\begin{sphinxShadowBox}
\sphinxstyletopictitle{Label}

\sphinxAtStartPar
technology transfer disposition
\end{sphinxShadowBox}

\begin{sphinxShadowBox}
\sphinxstyletopictitle{Alternate name}

\sphinxAtStartPar
technology licensing

\sphinxAtStartPar
tech transfer
\end{sphinxShadowBox}

\begin{sphinxShadowBox}
\sphinxstyletopictitle{Definition}

\sphinxAtStartPar
A disposition to create licenses for intellectual property for use by these beyond the creators
\end{sphinxShadowBox}

\begin{sphinxShadowBox}
\sphinxstyletopictitle{Definition source}

\sphinxAtStartPar
Michael Conlon \sphinxurl{https://orcid.org/0000-0002-1304-8447}
\end{sphinxShadowBox}

\begin{sphinxShadowBox}
\sphinxstyletopictitle{Example}

\sphinxAtStartPar
The Duke Office of Licensing and Ventures has a technology transfer disposition
\end{sphinxShadowBox}

\begin{sphinxShadowBox}
\sphinxstyletopictitle{Term editor}

\sphinxAtStartPar
Michael Conlon \sphinxurl{https://orcid.org/0000-0002-1304-8447}
\end{sphinxShadowBox}
\begin{quote}

\index{ORG\_0000014@\spxentry{ORG\_0000014}!philanthropy disposition@\spxentry{philanthropy disposition}}\index{philanthropy disposition@\spxentry{philanthropy disposition}!ORG\_0000014@\spxentry{ORG\_0000014}}\ignorespaces \end{quote}


\subsection{ORG\_0000014 \sphinxhyphen{} philanthropy disposition}
\label{\detokenize{doc-ORG_0000014:org-0000014-philanthropy-disposition}}\label{\detokenize{doc-ORG_0000014:index-0}}\label{\detokenize{doc-ORG_0000014::doc}}
\begin{sphinxShadowBox}
\sphinxstyletopictitle{Label}

\sphinxAtStartPar
philanthropy disposition
\end{sphinxShadowBox}

\begin{sphinxShadowBox}
\sphinxstyletopictitle{Alternate name}

\sphinxAtStartPar
philanthropic
\end{sphinxShadowBox}

\begin{sphinxShadowBox}
\sphinxstyletopictitle{Definition}

\sphinxAtStartPar
A disposition to donate charitable causes, sometimes in the form of grants involving contracts regarding the use of the donated funds or effort.
\end{sphinxShadowBox}

\begin{sphinxShadowBox}
\sphinxstyletopictitle{Definition source}

\sphinxAtStartPar
\sphinxurl{https://www.dictionary.com/browse/philanthropy}
\end{sphinxShadowBox}

\begin{sphinxShadowBox}
\sphinxstyletopictitle{Example}

\sphinxAtStartPar
The Wellcome Trust, The Bill and Melinda Gates Foundation, The Sierra Club have dispositions of philanthropy
\end{sphinxShadowBox}

\begin{sphinxShadowBox}
\sphinxstyletopictitle{Term editor}

\sphinxAtStartPar
Michael Conlon \sphinxurl{https://orcid.org/0000-0002-1304-8447}
\end{sphinxShadowBox}
\begin{quote}

\index{ORG\_0000015@\spxentry{ORG\_0000015}!funding disposition@\spxentry{funding disposition}}\index{funding disposition@\spxentry{funding disposition}!ORG\_0000015@\spxentry{ORG\_0000015}}\ignorespaces \end{quote}


\subsection{ORG\_0000015 \sphinxhyphen{} funding disposition}
\label{\detokenize{doc-ORG_0000015:org-0000015-funding-disposition}}\label{\detokenize{doc-ORG_0000015:index-0}}\label{\detokenize{doc-ORG_0000015::doc}}
\begin{sphinxShadowBox}
\sphinxstyletopictitle{Label}

\sphinxAtStartPar
funding disposition
\end{sphinxShadowBox}

\begin{sphinxShadowBox}
\sphinxstyletopictitle{Alternate name}

\sphinxAtStartPar
grantor
\end{sphinxShadowBox}

\begin{sphinxShadowBox}
\sphinxstyletopictitle{Definition}

\sphinxAtStartPar
A disposition to fund proposals, often is response to a call for proposals by the entity with the funding disposition
\end{sphinxShadowBox}

\begin{sphinxShadowBox}
\sphinxstyletopictitle{Definition source}

\sphinxAtStartPar
Michael Conlon \sphinxurl{https://orcid.org/0000-0002-1304-8447}
\end{sphinxShadowBox}

\begin{sphinxShadowBox}
\sphinxstyletopictitle{Example}

\sphinxAtStartPar
The National Institutes of Health (NIH) and The National Science Foundation (NSF) have funder dispositions
\end{sphinxShadowBox}

\begin{sphinxShadowBox}
\sphinxstyletopictitle{Term editor}

\sphinxAtStartPar
Michael Conlon \sphinxurl{https://orcid.org/0000-0002-1304-8447}
\end{sphinxShadowBox}
\begin{quote}

\index{ORG\_0000016@\spxentry{ORG\_0000016}!health care service provider disposition@\spxentry{health care service provider disposition}}\index{health care service provider disposition@\spxentry{health care service provider disposition}!ORG\_0000016@\spxentry{ORG\_0000016}}\ignorespaces \end{quote}


\subsection{ORG\_0000016 \sphinxhyphen{} health care service provider disposition}
\label{\detokenize{doc-ORG_0000016:org-0000016-health-care-service-provider-disposition}}\label{\detokenize{doc-ORG_0000016:index-0}}\label{\detokenize{doc-ORG_0000016::doc}}
\begin{sphinxShadowBox}
\sphinxstyletopictitle{Label}

\sphinxAtStartPar
health care service provider disposition
\end{sphinxShadowBox}

\begin{sphinxShadowBox}
\sphinxstyletopictitle{Alternate name}

\sphinxAtStartPar
health care provider
\end{sphinxShadowBox}

\begin{sphinxShadowBox}
\sphinxstyletopictitle{Definition}

\sphinxAtStartPar
A disposition to provider health care to humans
\end{sphinxShadowBox}

\begin{sphinxShadowBox}
\sphinxstyletopictitle{Definition source}

\sphinxAtStartPar
Michael Conlon \sphinxurl{https://orcid.org/0000-0002-1304-8447}
\end{sphinxShadowBox}

\begin{sphinxShadowBox}
\sphinxstyletopictitle{Example}

\sphinxAtStartPar
A nurse has a disposition to provide health care servces
\end{sphinxShadowBox}

\begin{sphinxShadowBox}
\sphinxstyletopictitle{Term editor}

\sphinxAtStartPar
Michael Conlon \sphinxurl{https://orcid.org/0000-0002-1304-8447}
\end{sphinxShadowBox}
\begin{quote}

\index{ORG\_0000017@\spxentry{ORG\_0000017}!hospital service provider disposition@\spxentry{hospital service provider disposition}}\index{hospital service provider disposition@\spxentry{hospital service provider disposition}!ORG\_0000017@\spxentry{ORG\_0000017}}\ignorespaces \end{quote}


\subsection{ORG\_0000017 \sphinxhyphen{} hospital service provider disposition}
\label{\detokenize{doc-ORG_0000017:org-0000017-hospital-service-provider-disposition}}\label{\detokenize{doc-ORG_0000017:index-0}}\label{\detokenize{doc-ORG_0000017::doc}}
\begin{sphinxShadowBox}
\sphinxstyletopictitle{Label}

\sphinxAtStartPar
hospital service provider disposition
\end{sphinxShadowBox}

\begin{sphinxShadowBox}
\sphinxstyletopictitle{Alternate name}

\sphinxAtStartPar
hospital
\end{sphinxShadowBox}

\begin{sphinxShadowBox}
\sphinxstyletopictitle{Definition}

\sphinxAtStartPar
A disposition to provide hospital\sphinxhyphen{}based health care services to humans
\end{sphinxShadowBox}

\begin{sphinxShadowBox}
\sphinxstyletopictitle{Definition source}

\sphinxAtStartPar
Michael Conlon \sphinxurl{https://orcid.org/0000-0002-1304-8447}
\end{sphinxShadowBox}

\begin{sphinxShadowBox}
\sphinxstyletopictitle{Example}

\sphinxAtStartPar
Mt. Sinai Hospital in New York has a disposition to provide hospital services
\end{sphinxShadowBox}

\begin{sphinxShadowBox}
\sphinxstyletopictitle{Term editor}

\sphinxAtStartPar
Michael Conlon \sphinxurl{https://orcid.org/0000-0002-1304-8447}
\end{sphinxShadowBox}
\begin{quote}

\index{ORG\_0000018@\spxentry{ORG\_0000018}!archive disposition@\spxentry{archive disposition}}\index{archive disposition@\spxentry{archive disposition}!ORG\_0000018@\spxentry{ORG\_0000018}}\ignorespaces \end{quote}


\subsection{ORG\_0000018 \sphinxhyphen{} archive disposition}
\label{\detokenize{doc-ORG_0000018:org-0000018-archive-disposition}}\label{\detokenize{doc-ORG_0000018:index-0}}\label{\detokenize{doc-ORG_0000018::doc}}
\begin{sphinxShadowBox}
\sphinxstyletopictitle{Label}

\sphinxAtStartPar
archive disposition
\end{sphinxShadowBox}

\begin{sphinxShadowBox}
\sphinxstyletopictitle{Alternate name}

\sphinxAtStartPar
archives
\end{sphinxShadowBox}

\begin{sphinxShadowBox}
\sphinxstyletopictitle{Definition}

\sphinxAtStartPar
A disposition to collect, store, and provide access to inanimate material entities, and/or information content entitites
\end{sphinxShadowBox}

\begin{sphinxShadowBox}
\sphinxstyletopictitle{Definition source}

\sphinxAtStartPar
Michael Conlon \sphinxurl{https://orcid.org/0000-0002-1304-8447}
\end{sphinxShadowBox}

\begin{sphinxShadowBox}
\sphinxstyletopictitle{Example}

\sphinxAtStartPar
The British Museum has a disposition to archive, particularly works of the United Kingdom
\end{sphinxShadowBox}

\begin{sphinxShadowBox}
\sphinxstyletopictitle{Editor’s note}

\sphinxAtStartPar
We might expect to have refinements of this disposition as special cases
\end{sphinxShadowBox}

\begin{sphinxShadowBox}
\sphinxstyletopictitle{Term editor}

\sphinxAtStartPar
Michael Conlon \sphinxurl{https://orcid.org/0000-0002-1304-8447}
\end{sphinxShadowBox}
\begin{quote}

\index{ORG\_0000019@\spxentry{ORG\_0000019}!museum disposition@\spxentry{museum disposition}}\index{museum disposition@\spxentry{museum disposition}!ORG\_0000019@\spxentry{ORG\_0000019}}\ignorespaces \end{quote}


\subsection{ORG\_0000019 \sphinxhyphen{} museum disposition}
\label{\detokenize{doc-ORG_0000019:org-0000019-museum-disposition}}\label{\detokenize{doc-ORG_0000019:index-0}}\label{\detokenize{doc-ORG_0000019::doc}}
\begin{sphinxShadowBox}
\sphinxstyletopictitle{Label}

\sphinxAtStartPar
museum disposition
\end{sphinxShadowBox}

\begin{sphinxShadowBox}
\sphinxstyletopictitle{Definition}

\sphinxAtStartPar
A disposition to collect, store, and provide access to inanimate material entities in a facility
\end{sphinxShadowBox}

\begin{sphinxShadowBox}
\sphinxstyletopictitle{Definition source}

\sphinxAtStartPar
Michael Conlon \sphinxurl{https://orcid.org/0000-0002-1304-8447}
\end{sphinxShadowBox}

\begin{sphinxShadowBox}
\sphinxstyletopictitle{Example}

\sphinxAtStartPar
The National Portrait Gallery is an art museum
\end{sphinxShadowBox}

\begin{sphinxShadowBox}
\sphinxstyletopictitle{Term editor}

\sphinxAtStartPar
Michael Conlon \sphinxurl{https://orcid.org/0000-0002-1304-8447}
\end{sphinxShadowBox}
\begin{quote}

\index{ORG\_0000020@\spxentry{ORG\_0000020}!gallery disposition@\spxentry{gallery disposition}}\index{gallery disposition@\spxentry{gallery disposition}!ORG\_0000020@\spxentry{ORG\_0000020}}\ignorespaces \end{quote}


\subsection{ORG\_0000020 \sphinxhyphen{} gallery disposition}
\label{\detokenize{doc-ORG_0000020:org-0000020-gallery-disposition}}\label{\detokenize{doc-ORG_0000020:index-0}}\label{\detokenize{doc-ORG_0000020::doc}}
\begin{sphinxShadowBox}
\sphinxstyletopictitle{Label}

\sphinxAtStartPar
gallery disposition
\end{sphinxShadowBox}

\begin{sphinxShadowBox}
\sphinxstyletopictitle{Definition}

\sphinxAtStartPar
A disposition to display collected works from an archive
\end{sphinxShadowBox}

\begin{sphinxShadowBox}
\sphinxstyletopictitle{Definition source}

\sphinxAtStartPar
Michael Conlon \sphinxurl{https://orcid.org/0000-0002-1304-8447}
\end{sphinxShadowBox}

\begin{sphinxShadowBox}
\sphinxstyletopictitle{Example}

\sphinxAtStartPar
The Color Factory in New York City is an art gallery
\end{sphinxShadowBox}

\begin{sphinxShadowBox}
\sphinxstyletopictitle{Editor’s note}

\sphinxAtStartPar
The distinction between an art gallery and and art museum has to do with the temporary nature of displays in galleries, as well as the common practice in galleries to sell art on display
\end{sphinxShadowBox}

\begin{sphinxShadowBox}
\sphinxstyletopictitle{Term editor}

\sphinxAtStartPar
Michael Conlon \sphinxurl{https://orcid.org/0000-0002-1304-8447}
\end{sphinxShadowBox}
\begin{quote}

\index{ORG\_0000021@\spxentry{ORG\_0000021}!publishing disposition@\spxentry{publishing disposition}}\index{publishing disposition@\spxentry{publishing disposition}!ORG\_0000021@\spxentry{ORG\_0000021}}\ignorespaces \end{quote}


\subsection{ORG\_0000021 \sphinxhyphen{} publishing disposition}
\label{\detokenize{doc-ORG_0000021:org-0000021-publishing-disposition}}\label{\detokenize{doc-ORG_0000021:index-0}}\label{\detokenize{doc-ORG_0000021::doc}}
\begin{sphinxShadowBox}
\sphinxstyletopictitle{Label}

\sphinxAtStartPar
publishing disposition
\end{sphinxShadowBox}

\begin{sphinxShadowBox}
\sphinxstyletopictitle{Definition}

\sphinxAtStartPar
A disposition to publish information content entities
\end{sphinxShadowBox}

\begin{sphinxShadowBox}
\sphinxstyletopictitle{Definition source}

\sphinxAtStartPar
Michael Conlon \sphinxurl{https://orcid.org/0000-0002-1304-8447}
\end{sphinxShadowBox}

\begin{sphinxShadowBox}
\sphinxstyletopictitle{Example}

\sphinxAtStartPar
Random House, Taylor and Francis, The American Pyschological Association, The University of California Berkeley has a disposition to publish
\end{sphinxShadowBox}

\begin{sphinxShadowBox}
\sphinxstyletopictitle{Term editor}

\sphinxAtStartPar
Michael Conlon \sphinxurl{https://orcid.org/0000-0002-1304-8447}
\end{sphinxShadowBox}
\begin{quote}

\index{ORG\_0000022@\spxentry{ORG\_0000022}!research disposition@\spxentry{research disposition}}\index{research disposition@\spxentry{research disposition}!ORG\_0000022@\spxentry{ORG\_0000022}}\ignorespaces \end{quote}


\subsection{ORG\_0000022 \sphinxhyphen{} research disposition}
\label{\detokenize{doc-ORG_0000022:org-0000022-research-disposition}}\label{\detokenize{doc-ORG_0000022:index-0}}\label{\detokenize{doc-ORG_0000022::doc}}
\begin{sphinxShadowBox}
\sphinxstyletopictitle{Label}

\sphinxAtStartPar
research disposition
\end{sphinxShadowBox}

\begin{sphinxShadowBox}
\sphinxstyletopictitle{Definition}

\sphinxAtStartPar
A disposition to conduct research
\end{sphinxShadowBox}

\begin{sphinxShadowBox}
\sphinxstyletopictitle{Definition source}

\sphinxAtStartPar
Michael Conlon \sphinxurl{https://orcid.org/0000-0002-1304-8447}
\end{sphinxShadowBox}

\begin{sphinxShadowBox}
\sphinxstyletopictitle{Example}

\sphinxAtStartPar
The Max Plank Institute, Northwestern University, CERN (the European Organization for Nuclear Research) have have a disposition to conduct research
\end{sphinxShadowBox}

\begin{sphinxShadowBox}
\sphinxstyletopictitle{Term editor}

\sphinxAtStartPar
Michael Conlon \sphinxurl{https://orcid.org/0000-0002-1304-8447}
\end{sphinxShadowBox}
\begin{quote}

\index{ORG\_0000023@\spxentry{ORG\_0000023}!education disposition@\spxentry{education disposition}}\index{education disposition@\spxentry{education disposition}!ORG\_0000023@\spxentry{ORG\_0000023}}\ignorespaces \end{quote}


\subsection{ORG\_0000023 \sphinxhyphen{} education disposition}
\label{\detokenize{doc-ORG_0000023:org-0000023-education-disposition}}\label{\detokenize{doc-ORG_0000023:index-0}}\label{\detokenize{doc-ORG_0000023::doc}}
\begin{sphinxShadowBox}
\sphinxstyletopictitle{Label}

\sphinxAtStartPar
education disposition
\end{sphinxShadowBox}

\begin{sphinxShadowBox}
\sphinxstyletopictitle{Alternate name}

\sphinxAtStartPar
teaching
\end{sphinxShadowBox}

\begin{sphinxShadowBox}
\sphinxstyletopictitle{Definition}

\sphinxAtStartPar
A disposition to teach, and provide experiential opprtunities for students
\end{sphinxShadowBox}

\begin{sphinxShadowBox}
\sphinxstyletopictitle{Definition source}

\sphinxAtStartPar
Michael Conlon \sphinxurl{https://orcid.org/0000-0002-1304-8447}
\end{sphinxShadowBox}

\begin{sphinxShadowBox}
\sphinxstyletopictitle{Example}

\sphinxAtStartPar
A pre\sphinxhyphen{}school education program, a post\sphinxhyphen{}graduate education program, adult education programs all have education dispositions
\end{sphinxShadowBox}

\begin{sphinxShadowBox}
\sphinxstyletopictitle{Term editor}

\sphinxAtStartPar
Michael Conlon \sphinxurl{https://orcid.org/0000-0002-1304-8447}
\end{sphinxShadowBox}
\begin{quote}

\index{ORG\_0000024@\spxentry{ORG\_0000024}!training disposition@\spxentry{training disposition}}\index{training disposition@\spxentry{training disposition}!ORG\_0000024@\spxentry{ORG\_0000024}}\ignorespaces \end{quote}


\subsection{ORG\_0000024 \sphinxhyphen{} training disposition}
\label{\detokenize{doc-ORG_0000024:org-0000024-training-disposition}}\label{\detokenize{doc-ORG_0000024:index-0}}\label{\detokenize{doc-ORG_0000024::doc}}
\begin{sphinxShadowBox}
\sphinxstyletopictitle{Label}

\sphinxAtStartPar
training disposition
\end{sphinxShadowBox}

\begin{sphinxShadowBox}
\sphinxstyletopictitle{Alternate name}

\sphinxAtStartPar
trainer
\end{sphinxShadowBox}

\begin{sphinxShadowBox}
\sphinxstyletopictitle{Definition}

\sphinxAtStartPar
A disposition to train, and provide experiential opportunities for trainees
\end{sphinxShadowBox}

\begin{sphinxShadowBox}
\sphinxstyletopictitle{Definition source}

\sphinxAtStartPar
Michael Conlon \sphinxurl{https://orcid.org/0000-0002-1304-8447}
\end{sphinxShadowBox}

\begin{sphinxShadowBox}
\sphinxstyletopictitle{Example}

\sphinxAtStartPar
A military training program, a flight school training program, a CPR training program all have dispositions to train
\end{sphinxShadowBox}

\begin{sphinxShadowBox}
\sphinxstyletopictitle{Term editor}

\sphinxAtStartPar
Michael Conlon \sphinxurl{https://orcid.org/0000-0002-1304-8447}
\end{sphinxShadowBox}
\begin{quote}

\index{ORG\_0000025@\spxentry{ORG\_0000025}!research administration disposition@\spxentry{research administration disposition}}\index{research administration disposition@\spxentry{research administration disposition}!ORG\_0000025@\spxentry{ORG\_0000025}}\ignorespaces \end{quote}


\subsection{ORG\_0000025 \sphinxhyphen{} research administration disposition}
\label{\detokenize{doc-ORG_0000025:org-0000025-research-administration-disposition}}\label{\detokenize{doc-ORG_0000025:index-0}}\label{\detokenize{doc-ORG_0000025::doc}}
\begin{sphinxShadowBox}
\sphinxstyletopictitle{Label}

\sphinxAtStartPar
research administration disposition
\end{sphinxShadowBox}

\begin{sphinxShadowBox}
\sphinxstyletopictitle{Definition}

\sphinxAtStartPar
A disposition to provide resources and oversight for those conducting research
\end{sphinxShadowBox}

\begin{sphinxShadowBox}
\sphinxstyletopictitle{Definition source}

\sphinxAtStartPar
Michael Conlon \sphinxurl{https://orcid.org/0000-0002-1304-8447}
\end{sphinxShadowBox}

\begin{sphinxShadowBox}
\sphinxstyletopictitle{Example}

\sphinxAtStartPar
The Office of Research at a university typically does not conduct research, it has a disposition to administer research
\end{sphinxShadowBox}

\begin{sphinxShadowBox}
\sphinxstyletopictitle{Term editor}

\sphinxAtStartPar
Michael Conlon \sphinxurl{https://orcid.org/0000-0002-1304-8447}
\end{sphinxShadowBox}
\begin{quote}

\index{ORG\_0000026@\spxentry{ORG\_0000026}!library disposition@\spxentry{library disposition}}\index{library disposition@\spxentry{library disposition}!ORG\_0000026@\spxentry{ORG\_0000026}}\ignorespaces \end{quote}


\subsection{ORG\_0000026 \sphinxhyphen{} library disposition}
\label{\detokenize{doc-ORG_0000026:org-0000026-library-disposition}}\label{\detokenize{doc-ORG_0000026:index-0}}\label{\detokenize{doc-ORG_0000026::doc}}
\begin{sphinxShadowBox}
\sphinxstyletopictitle{Label}

\sphinxAtStartPar
library disposition
\end{sphinxShadowBox}

\begin{sphinxShadowBox}
\sphinxstyletopictitle{Definition}

\sphinxAtStartPar
A disposition to provide library services
\end{sphinxShadowBox}

\begin{sphinxShadowBox}
\sphinxstyletopictitle{Definition source}

\sphinxAtStartPar
Michael Conlon \sphinxurl{https://orcid.org/0000-0002-1304-8447}
\end{sphinxShadowBox}

\begin{sphinxShadowBox}
\sphinxstyletopictitle{Example}

\sphinxAtStartPar
The Library of Congress, the local public library, a university library allhave dispositions of library
\end{sphinxShadowBox}

\begin{sphinxShadowBox}
\sphinxstyletopictitle{Editor’s note}

\sphinxAtStartPar
Perhaps a function
\end{sphinxShadowBox}

\begin{sphinxShadowBox}
\sphinxstyletopictitle{Term editor}

\sphinxAtStartPar
Michael Conlon \sphinxurl{https://orcid.org/0000-0002-1304-8447}
\end{sphinxShadowBox}
\begin{quote}

\index{ORG\_0000027@\spxentry{ORG\_0000027}!commerce disposition@\spxentry{commerce disposition}}\index{commerce disposition@\spxentry{commerce disposition}!ORG\_0000027@\spxentry{ORG\_0000027}}\ignorespaces \end{quote}


\subsection{ORG\_0000027 \sphinxhyphen{} commerce disposition}
\label{\detokenize{doc-ORG_0000027:org-0000027-commerce-disposition}}\label{\detokenize{doc-ORG_0000027:index-0}}\label{\detokenize{doc-ORG_0000027::doc}}
\begin{sphinxShadowBox}
\sphinxstyletopictitle{Label}

\sphinxAtStartPar
commerce disposition
\end{sphinxShadowBox}

\begin{sphinxShadowBox}
\sphinxstyletopictitle{Alternate name}

\sphinxAtStartPar
business
\end{sphinxShadowBox}

\begin{sphinxShadowBox}
\sphinxstyletopictitle{Definition}

\sphinxAtStartPar
A disposition to sell things
\end{sphinxShadowBox}

\begin{sphinxShadowBox}
\sphinxstyletopictitle{Definition source}

\sphinxAtStartPar
Michael Conlon \sphinxurl{https://orcid.org/0000-0002-1304-8447}
\end{sphinxShadowBox}

\begin{sphinxShadowBox}
\sphinxstyletopictitle{Example}

\sphinxAtStartPar
Google, Tesla, General Electric, BASF, AstraZeneca all have a disposition to sell things
\end{sphinxShadowBox}

\begin{sphinxShadowBox}
\sphinxstyletopictitle{Term editor}

\sphinxAtStartPar
Michael Conlon \sphinxurl{https://orcid.org/0000-0002-1304-8447}
\end{sphinxShadowBox}
\begin{quote}

\index{ORG\_0000028@\spxentry{ORG\_0000028}!military disposition@\spxentry{military disposition}}\index{military disposition@\spxentry{military disposition}!ORG\_0000028@\spxentry{ORG\_0000028}}\ignorespaces \end{quote}


\subsection{ORG\_0000028 \sphinxhyphen{} military disposition}
\label{\detokenize{doc-ORG_0000028:org-0000028-military-disposition}}\label{\detokenize{doc-ORG_0000028:index-0}}\label{\detokenize{doc-ORG_0000028::doc}}
\begin{sphinxShadowBox}
\sphinxstyletopictitle{Label}

\sphinxAtStartPar
military disposition
\end{sphinxShadowBox}

\begin{sphinxShadowBox}
\sphinxstyletopictitle{Alternate name}

\sphinxAtStartPar
armed forces
\end{sphinxShadowBox}

\begin{sphinxShadowBox}
\sphinxstyletopictitle{Definition}

\sphinxAtStartPar
A disposition to engage in warfare
\end{sphinxShadowBox}

\begin{sphinxShadowBox}
\sphinxstyletopictitle{Definition source}

\sphinxAtStartPar
Michael Conlon \sphinxurl{https://orcid.org/0000-0002-1304-8447}
\end{sphinxShadowBox}

\begin{sphinxShadowBox}
\sphinxstyletopictitle{Example}

\sphinxAtStartPar
The US Space Force, the Chinese Navy, and the Bolivian Army are example of organizations with miltary disposition
\end{sphinxShadowBox}

\begin{sphinxShadowBox}
\sphinxstyletopictitle{Editor’s note}

\sphinxAtStartPar
Perhaps a function
\end{sphinxShadowBox}

\begin{sphinxShadowBox}
\sphinxstyletopictitle{Term editor}

\sphinxAtStartPar
Michael Conlon \sphinxurl{https://orcid.org/0000-0002-1304-8447}
\end{sphinxShadowBox}
\begin{quote}

\index{ORG\_0000029@\spxentry{ORG\_0000029}!religious disposition@\spxentry{religious disposition}}\index{religious disposition@\spxentry{religious disposition}!ORG\_0000029@\spxentry{ORG\_0000029}}\ignorespaces \end{quote}


\subsection{ORG\_0000029 \sphinxhyphen{} religious disposition}
\label{\detokenize{doc-ORG_0000029:org-0000029-religious-disposition}}\label{\detokenize{doc-ORG_0000029:index-0}}\label{\detokenize{doc-ORG_0000029::doc}}
\begin{sphinxShadowBox}
\sphinxstyletopictitle{Label}

\sphinxAtStartPar
religious disposition
\end{sphinxShadowBox}

\begin{sphinxShadowBox}
\sphinxstyletopictitle{Alternate name}

\sphinxAtStartPar
church
\end{sphinxShadowBox}

\begin{sphinxShadowBox}
\sphinxstyletopictitle{Definition}

\sphinxAtStartPar
A disposition to engage in matters of spirtuality and faith
\end{sphinxShadowBox}

\begin{sphinxShadowBox}
\sphinxstyletopictitle{Definition source}

\sphinxAtStartPar
Michael Conlon \sphinxurl{https://orcid.org/0000-0002-1304-8447}
\end{sphinxShadowBox}

\begin{sphinxShadowBox}
\sphinxstyletopictitle{Example}

\sphinxAtStartPar
The Catholic Church, a local synagogue, a mosque, a buudhist monestary have a relgisious disposition
\end{sphinxShadowBox}

\begin{sphinxShadowBox}
\sphinxstyletopictitle{Editor’s note}

\sphinxAtStartPar
Perhaps a function.
\end{sphinxShadowBox}

\begin{sphinxShadowBox}
\sphinxstyletopictitle{Term editor}

\sphinxAtStartPar
Michael Conlon \sphinxurl{https://orcid.org/0000-0002-1304-8447}
\end{sphinxShadowBox}
\begin{quote}

\index{ORG\_0000030@\spxentry{ORG\_0000030}!governing disposition@\spxentry{governing disposition}}\index{governing disposition@\spxentry{governing disposition}!ORG\_0000030@\spxentry{ORG\_0000030}}\ignorespaces \end{quote}


\subsection{ORG\_0000030 \sphinxhyphen{} governing disposition}
\label{\detokenize{doc-ORG_0000030:org-0000030-governing-disposition}}\label{\detokenize{doc-ORG_0000030:index-0}}\label{\detokenize{doc-ORG_0000030::doc}}
\begin{sphinxShadowBox}
\sphinxstyletopictitle{Label}

\sphinxAtStartPar
governing disposition
\end{sphinxShadowBox}

\begin{sphinxShadowBox}
\sphinxstyletopictitle{Definition}

\sphinxAtStartPar
A disposition to provide governance
\end{sphinxShadowBox}

\begin{sphinxShadowBox}
\sphinxstyletopictitle{Definition source}

\sphinxAtStartPar
Michael Conlon \sphinxurl{https://orcid.org/0000-0002-1304-8447}
\end{sphinxShadowBox}

\begin{sphinxShadowBox}
\sphinxstyletopictitle{Example}

\sphinxAtStartPar
A Board of Trustees, A Board of Directors, the Senate of the United States all have governing dispositions
\end{sphinxShadowBox}

\begin{sphinxShadowBox}
\sphinxstyletopictitle{Term editor}

\sphinxAtStartPar
Michael Conlon \sphinxurl{https://orcid.org/0000-0002-1304-8447}
\end{sphinxShadowBox}
\begin{quote}

\index{ORG\_0000031@\spxentry{ORG\_0000031}!manufacturing disposition@\spxentry{manufacturing disposition}}\index{manufacturing disposition@\spxentry{manufacturing disposition}!ORG\_0000031@\spxentry{ORG\_0000031}}\ignorespaces \end{quote}


\subsection{ORG\_0000031 \sphinxhyphen{} manufacturing disposition}
\label{\detokenize{doc-ORG_0000031:org-0000031-manufacturing-disposition}}\label{\detokenize{doc-ORG_0000031:index-0}}\label{\detokenize{doc-ORG_0000031::doc}}
\begin{sphinxShadowBox}
\sphinxstyletopictitle{Label}

\sphinxAtStartPar
manufacturing disposition
\end{sphinxShadowBox}

\begin{sphinxShadowBox}
\sphinxstyletopictitle{Alternate name}

\sphinxAtStartPar
manufacturer
\end{sphinxShadowBox}

\begin{sphinxShadowBox}
\sphinxstyletopictitle{Definition}

\sphinxAtStartPar
A dispositon to construct material entities
\end{sphinxShadowBox}

\begin{sphinxShadowBox}
\sphinxstyletopictitle{Definition source}

\sphinxAtStartPar
Michael Conlon \sphinxurl{https://orcid.org/0000-0002-1304-8447}
\end{sphinxShadowBox}

\begin{sphinxShadowBox}
\sphinxstyletopictitle{Example}

\sphinxAtStartPar
Volkswagon, Apple, Pfizer, Airbus, BASF, Nestle, and General Electric all have manufacturing dispositions
\end{sphinxShadowBox}

\begin{sphinxShadowBox}
\sphinxstyletopictitle{Similar term in VIVO 1 Ontology}

\sphinxAtStartPar
\sphinxurl{http://purl.obolibrary.org/obo/ERO\_0000034}
\end{sphinxShadowBox}

\begin{sphinxShadowBox}
\sphinxstyletopictitle{Term editor}

\sphinxAtStartPar
Michael Conlon \sphinxurl{https://orcid.org/0000-0002-1304-8447}
\end{sphinxShadowBox}

\begin{sphinxShadowBox}
\sphinxstyletopictitle{See also}

\sphinxAtStartPar
\sphinxurl{http://purl.obolibrary.org/obo/OBI\_0000835}
\end{sphinxShadowBox}
\begin{quote}

\index{ORG\_0000032@\spxentry{ORG\_0000032}!project team disposition@\spxentry{project team disposition}}\index{project team disposition@\spxentry{project team disposition}!ORG\_0000032@\spxentry{ORG\_0000032}}\ignorespaces \end{quote}


\subsection{ORG\_0000032 \sphinxhyphen{} project team disposition}
\label{\detokenize{doc-ORG_0000032:org-0000032-project-team-disposition}}\label{\detokenize{doc-ORG_0000032:index-0}}\label{\detokenize{doc-ORG_0000032::doc}}
\begin{sphinxShadowBox}
\sphinxstyletopictitle{Label}

\sphinxAtStartPar
project team disposition
\end{sphinxShadowBox}

\begin{sphinxShadowBox}
\sphinxstyletopictitle{Alternate name}

\sphinxAtStartPar
team
\end{sphinxShadowBox}

\begin{sphinxShadowBox}
\sphinxstyletopictitle{Definition}

\sphinxAtStartPar
A disposition to execute and finish a project.
\end{sphinxShadowBox}

\begin{sphinxShadowBox}
\sphinxstyletopictitle{Definition source}

\sphinxAtStartPar
Michael Conlon \sphinxurl{https://orcid.org/0000-0002-1304-8447}
\end{sphinxShadowBox}

\begin{sphinxShadowBox}
\sphinxstyletopictitle{Example}

\sphinxAtStartPar
A book writing team, a grant writing team, a time\sphinxhyphen{}limited work group, and a conference organizing committee all have a project team disposition
\end{sphinxShadowBox}

\begin{sphinxShadowBox}
\sphinxstyletopictitle{Editor’s note}

\sphinxAtStartPar
Organization may be formal or informal.
\end{sphinxShadowBox}

\begin{sphinxShadowBox}
\sphinxstyletopictitle{Term editor}

\sphinxAtStartPar
Michael Conlon \sphinxurl{https://orcid.org/0000-0002-1304-8447}
\end{sphinxShadowBox}
\begin{quote}

\index{ORG\_0000033@\spxentry{ORG\_0000033}!sports disposition@\spxentry{sports disposition}}\index{sports disposition@\spxentry{sports disposition}!ORG\_0000033@\spxentry{ORG\_0000033}}\ignorespaces \end{quote}


\subsection{ORG\_0000033 \sphinxhyphen{} sports disposition}
\label{\detokenize{doc-ORG_0000033:org-0000033-sports-disposition}}\label{\detokenize{doc-ORG_0000033:index-0}}\label{\detokenize{doc-ORG_0000033::doc}}
\begin{sphinxShadowBox}
\sphinxstyletopictitle{Label}

\sphinxAtStartPar
sports disposition
\end{sphinxShadowBox}

\begin{sphinxShadowBox}
\sphinxstyletopictitle{Definition}

\sphinxAtStartPar
A disposition to engage in sports activites, typically competitive.
\end{sphinxShadowBox}

\begin{sphinxShadowBox}
\sphinxstyletopictitle{Definition source}

\sphinxAtStartPar
Michael Conlon \sphinxurl{https://orcid.org/0000-0002-1304-8447}
\end{sphinxShadowBox}

\begin{sphinxShadowBox}
\sphinxstyletopictitle{Example}

\sphinxAtStartPar
The University of Alabama mens football team, Manchester United, the Mumbai Indians Cricket Team are all organizations with a disposition of sports
\end{sphinxShadowBox}

\begin{sphinxShadowBox}
\sphinxstyletopictitle{Editor’s note}

\sphinxAtStartPar
Perhaps a function
\end{sphinxShadowBox}

\begin{sphinxShadowBox}
\sphinxstyletopictitle{Term editor}

\sphinxAtStartPar
Michael Conlon \sphinxurl{https://orcid.org/0000-0002-1304-8447}
\end{sphinxShadowBox}
\begin{quote}

\index{ORG\_0000034@\spxentry{ORG\_0000034}!information address quality@\spxentry{information address quality}}\index{information address quality@\spxentry{information address quality}!ORG\_0000034@\spxentry{ORG\_0000034}}\ignorespaces \end{quote}


\subsection{ORG\_0000034 \sphinxhyphen{} information address quality}
\label{\detokenize{doc-ORG_0000034:org-0000034-information-address-quality}}\label{\detokenize{doc-ORG_0000034:index-0}}\label{\detokenize{doc-ORG_0000034::doc}}
\begin{sphinxShadowBox}
\sphinxstyletopictitle{Label}

\sphinxAtStartPar
information address quality
\end{sphinxShadowBox}

\begin{sphinxShadowBox}
\sphinxstyletopictitle{Definition}

\sphinxAtStartPar
A quality of an address to be used for information inquiries
\end{sphinxShadowBox}

\begin{sphinxShadowBox}
\sphinxstyletopictitle{Definition source}

\sphinxAtStartPar
Michael Conlon \sphinxurl{https://orcid.org/0000-0002-1304-8447}
\end{sphinxShadowBox}

\begin{sphinxShadowBox}
\sphinxstyletopictitle{Example}

\sphinxAtStartPar
Email addresses that begin help@ info@ contact@ often have information address address disposition
\end{sphinxShadowBox}

\begin{sphinxShadowBox}
\sphinxstyletopictitle{Term editor}

\sphinxAtStartPar
Michael Conlon \sphinxurl{https://orcid.org/0000-0002-1304-8447}
\end{sphinxShadowBox}
\begin{quote}

\index{ORG\_0000035@\spxentry{ORG\_0000035}!billing address quality@\spxentry{billing address quality}}\index{billing address quality@\spxentry{billing address quality}!ORG\_0000035@\spxentry{ORG\_0000035}}\ignorespaces \end{quote}


\subsection{ORG\_0000035 \sphinxhyphen{} billing address quality}
\label{\detokenize{doc-ORG_0000035:org-0000035-billing-address-quality}}\label{\detokenize{doc-ORG_0000035:index-0}}\label{\detokenize{doc-ORG_0000035::doc}}
\begin{sphinxShadowBox}
\sphinxstyletopictitle{Label}

\sphinxAtStartPar
billing address quality
\end{sphinxShadowBox}

\begin{sphinxShadowBox}
\sphinxstyletopictitle{Definition}

\sphinxAtStartPar
A quality of an address to be used to receive bills
\end{sphinxShadowBox}

\begin{sphinxShadowBox}
\sphinxstyletopictitle{Definition source}

\sphinxAtStartPar
Michael Conlon \sphinxurl{https://orcid.org/0000-0002-1304-8447}
\end{sphinxShadowBox}

\begin{sphinxShadowBox}
\sphinxstyletopictitle{Example}

\sphinxAtStartPar
Bills may be sent to Attn: Name at
\end{sphinxShadowBox}

\begin{sphinxShadowBox}
\sphinxstyletopictitle{Term editor}

\sphinxAtStartPar
Michael Conlon \sphinxurl{https://orcid.org/0000-0002-1304-8447}
\end{sphinxShadowBox}
\begin{quote}

\index{ORG\_0000036@\spxentry{ORG\_0000036}!shipping address quality@\spxentry{shipping address quality}}\index{shipping address quality@\spxentry{shipping address quality}!ORG\_0000036@\spxentry{ORG\_0000036}}\ignorespaces \end{quote}


\subsection{ORG\_0000036 \sphinxhyphen{} shipping address quality}
\label{\detokenize{doc-ORG_0000036:org-0000036-shipping-address-quality}}\label{\detokenize{doc-ORG_0000036:index-0}}\label{\detokenize{doc-ORG_0000036::doc}}
\begin{sphinxShadowBox}
\sphinxstyletopictitle{Label}

\sphinxAtStartPar
shipping address quality
\end{sphinxShadowBox}

\begin{sphinxShadowBox}
\sphinxstyletopictitle{Alternate name}

\sphinxAtStartPar
delivery address
\end{sphinxShadowBox}

\begin{sphinxShadowBox}
\sphinxstyletopictitle{Definition}

\sphinxAtStartPar
A quality of an address to be used to receive shipped goods
\end{sphinxShadowBox}

\begin{sphinxShadowBox}
\sphinxstyletopictitle{Definition source}

\sphinxAtStartPar
Michael Conlon \sphinxurl{https://orcid.org/0000-0002-1304-8447}
\end{sphinxShadowBox}

\begin{sphinxShadowBox}
\sphinxstyletopictitle{Example}

\sphinxAtStartPar
The loading dock address for central receiving of an organization
\end{sphinxShadowBox}

\begin{sphinxShadowBox}
\sphinxstyletopictitle{Term editor}

\sphinxAtStartPar
Michael Conlon \sphinxurl{https://orcid.org/0000-0002-1304-8447}
\end{sphinxShadowBox}
\begin{quote}

\index{ORG\_0000037@\spxentry{ORG\_0000037}!preferred address quality@\spxentry{preferred address quality}}\index{preferred address quality@\spxentry{preferred address quality}!ORG\_0000037@\spxentry{ORG\_0000037}}\ignorespaces \end{quote}


\subsection{ORG\_0000037 \sphinxhyphen{} preferred address quality}
\label{\detokenize{doc-ORG_0000037:org-0000037-preferred-address-quality}}\label{\detokenize{doc-ORG_0000037:index-0}}\label{\detokenize{doc-ORG_0000037::doc}}
\begin{sphinxShadowBox}
\sphinxstyletopictitle{Label}

\sphinxAtStartPar
preferred address quality
\end{sphinxShadowBox}

\begin{sphinxShadowBox}
\sphinxstyletopictitle{Definition}

\sphinxAtStartPar
A quality of an address to be displayed in most settings
\end{sphinxShadowBox}

\begin{sphinxShadowBox}
\sphinxstyletopictitle{Definition source}

\sphinxAtStartPar
Michael Conlon \sphinxurl{https://orcid.org/0000-0002-1304-8447}
\end{sphinxShadowBox}

\begin{sphinxShadowBox}
\sphinxstyletopictitle{Example}

\sphinxAtStartPar
The address the organization displays most prominently in promotional materials
\end{sphinxShadowBox}

\begin{sphinxShadowBox}
\sphinxstyletopictitle{Term editor}

\sphinxAtStartPar
Michael Conlon \sphinxurl{https://orcid.org/0000-0002-1304-8447}
\end{sphinxShadowBox}
\begin{quote}

\index{ORG\_0000038@\spxentry{ORG\_0000038}!homepage quality@\spxentry{homepage quality}}\index{homepage quality@\spxentry{homepage quality}!ORG\_0000038@\spxentry{ORG\_0000038}}\ignorespaces \end{quote}


\subsection{ORG\_0000038 \sphinxhyphen{} homepage quality}
\label{\detokenize{doc-ORG_0000038:org-0000038-homepage-quality}}\label{\detokenize{doc-ORG_0000038:index-0}}\label{\detokenize{doc-ORG_0000038::doc}}
\begin{sphinxShadowBox}
\sphinxstyletopictitle{Label}

\sphinxAtStartPar
homepage quality
\end{sphinxShadowBox}

\begin{sphinxShadowBox}
\sphinxstyletopictitle{Definition}

\sphinxAtStartPar
A quality to be the primary website for an entity.
\end{sphinxShadowBox}

\begin{sphinxShadowBox}
\sphinxstyletopictitle{Definition source}

\sphinxAtStartPar
Michael Conlon \sphinxurl{https://orcid.org/0000-0002-1304-8447}
\end{sphinxShadowBox}

\begin{sphinxShadowBox}
\sphinxstyletopictitle{Example}

\sphinxAtStartPar
The home page of Harvard is \sphinxurl{https://harvard.edu}
\end{sphinxShadowBox}

\begin{sphinxShadowBox}
\sphinxstyletopictitle{Term editor}

\sphinxAtStartPar
Michael Conlon \sphinxurl{https://orcid.org/0000-0002-1304-8447}
\end{sphinxShadowBox}
\begin{quote}

\index{ORG\_0000039@\spxentry{ORG\_0000039}!wikipedia quality@\spxentry{wikipedia quality}}\index{wikipedia quality@\spxentry{wikipedia quality}!ORG\_0000039@\spxentry{ORG\_0000039}}\ignorespaces \end{quote}


\subsection{ORG\_0000039 \sphinxhyphen{} wikipedia quality}
\label{\detokenize{doc-ORG_0000039:org-0000039-wikipedia-quality}}\label{\detokenize{doc-ORG_0000039:index-0}}\label{\detokenize{doc-ORG_0000039::doc}}
\begin{sphinxShadowBox}
\sphinxstyletopictitle{Label}

\sphinxAtStartPar
wikipedia quality
\end{sphinxShadowBox}

\begin{sphinxShadowBox}
\sphinxstyletopictitle{Definition}

\sphinxAtStartPar
A quality to be the webpage within WikiPedia regarding the entity
\end{sphinxShadowBox}

\begin{sphinxShadowBox}
\sphinxstyletopictitle{Definition source}

\sphinxAtStartPar
Michael Conlon \sphinxurl{https://orcid.org/0000-0002-1304-8447}
\end{sphinxShadowBox}

\begin{sphinxShadowBox}
\sphinxstyletopictitle{Example}

\sphinxAtStartPar
The Wikipedia page of Harvard is \sphinxurl{https://en.wikipedia.org/wiki/Harvard\_University}
\end{sphinxShadowBox}

\begin{sphinxShadowBox}
\sphinxstyletopictitle{Term editor}

\sphinxAtStartPar
Michael Conlon \sphinxurl{https://orcid.org/0000-0002-1304-8447}
\end{sphinxShadowBox}
\begin{quote}

\index{ORG\_0000040@\spxentry{ORG\_0000040}!architectural structure@\spxentry{architectural structure}}\index{architectural structure@\spxentry{architectural structure}!ORG\_0000040@\spxentry{ORG\_0000040}}\ignorespaces \end{quote}


\subsection{ORG\_0000040 \sphinxhyphen{} architectural structure}
\label{\detokenize{doc-ORG_0000040:org-0000040-architectural-structure}}\label{\detokenize{doc-ORG_0000040:index-0}}\label{\detokenize{doc-ORG_0000040::doc}}
\begin{sphinxShadowBox}
\sphinxstyletopictitle{Label}

\sphinxAtStartPar
architectural structure
\end{sphinxShadowBox}

\begin{sphinxShadowBox}
\sphinxstyletopictitle{Alternate name}

\sphinxAtStartPar
bauwerk
\end{sphinxShadowBox}

\begin{sphinxShadowBox}
\sphinxstyletopictitle{Definition}

\sphinxAtStartPar
A material entity that is a human made strcuture with firm connection between its foundation and the ground.
\end{sphinxShadowBox}

\begin{sphinxShadowBox}
\sphinxstyletopictitle{Definition source}

\sphinxAtStartPar
\sphinxurl{http://purl.obolibrary.org/obo/OMRSE\_00000061}
\end{sphinxShadowBox}

\begin{sphinxShadowBox}
\sphinxstyletopictitle{Example}

\sphinxAtStartPar
Yankee Stadium, Stonehenge, The Shard, The Large Hadron Collider. The Great Wall of China
\end{sphinxShadowBox}

\begin{sphinxShadowBox}
\sphinxstyletopictitle{Editor’s note}

\sphinxAtStartPar
Perhaps identical to the term in OMRSE
\end{sphinxShadowBox}

\begin{sphinxShadowBox}
\sphinxstyletopictitle{Term editor}

\sphinxAtStartPar
Michael Conlon \sphinxurl{https://orcid.org/0000-0002-1304-8447}
\end{sphinxShadowBox}

\begin{sphinxShadowBox}
\sphinxstyletopictitle{See also}

\sphinxAtStartPar
\sphinxurl{http://purl.obolibrary.org/obo/OMRSE\_00000061}
\end{sphinxShadowBox}
\begin{quote}

\index{ORG\_0000041@\spxentry{ORG\_0000041}!campus@\spxentry{campus}}\index{campus@\spxentry{campus}!ORG\_0000041@\spxentry{ORG\_0000041}}\ignorespaces \end{quote}


\subsection{ORG\_0000041 \sphinxhyphen{} campus}
\label{\detokenize{doc-ORG_0000041:org-0000041-campus}}\label{\detokenize{doc-ORG_0000041:index-0}}\label{\detokenize{doc-ORG_0000041::doc}}
\begin{sphinxShadowBox}
\sphinxstyletopictitle{Label}

\sphinxAtStartPar
campus
\end{sphinxShadowBox}

\begin{sphinxShadowBox}
\sphinxstyletopictitle{Definition}

\sphinxAtStartPar
The geographic location consisting of the  grounds or property of a school, college, university, business, church, or hospital, often understood to include buildings and other structures.
\end{sphinxShadowBox}

\begin{sphinxShadowBox}
\sphinxstyletopictitle{Definition source}

\sphinxAtStartPar
\sphinxurl{https://en.wiktionary.org/wiki/campus}
\end{sphinxShadowBox}

\begin{sphinxShadowBox}
\sphinxstyletopictitle{Example}

\sphinxAtStartPar
The campus of Oxford University, the campus of Microsoft in Redmond, Washington, the NIH campus in Bethesda, Maryland.
\end{sphinxShadowBox}

\begin{sphinxShadowBox}
\sphinxstyletopictitle{Similar term in VIVO 1 Ontology}

\sphinxAtStartPar
\sphinxurl{http://vivoweb.org/ontology/core\#Campus}
\end{sphinxShadowBox}

\begin{sphinxShadowBox}
\sphinxstyletopictitle{Term editor}

\sphinxAtStartPar
Michael Conlon \sphinxurl{https://orcid.org/0000-0002-1304-8447}
\end{sphinxShadowBox}
\begin{quote}

\index{ORG\_0000042@\spxentry{ORG\_0000042}!facility@\spxentry{facility}}\index{facility@\spxentry{facility}!ORG\_0000042@\spxentry{ORG\_0000042}}\ignorespaces \end{quote}


\subsection{ORG\_0000042 \sphinxhyphen{} facility}
\label{\detokenize{doc-ORG_0000042:org-0000042-facility}}\label{\detokenize{doc-ORG_0000042:index-0}}\label{\detokenize{doc-ORG_0000042::doc}}
\begin{sphinxShadowBox}
\sphinxstyletopictitle{Label}

\sphinxAtStartPar
facility
\end{sphinxShadowBox}

\begin{sphinxShadowBox}
\sphinxstyletopictitle{Definition}

\sphinxAtStartPar
An architectural structure that bears some function.
\end{sphinxShadowBox}

\begin{sphinxShadowBox}
\sphinxstyletopictitle{Definition source}

\sphinxAtStartPar
\sphinxurl{http://purl.obolibrary.org/obo/OMRSE\_00000062}
\end{sphinxShadowBox}

\begin{sphinxShadowBox}
\sphinxstyletopictitle{Example}

\sphinxAtStartPar
Walt Disney World, Kennedy Space Center, Golden Gate Bridge
\end{sphinxShadowBox}

\begin{sphinxShadowBox}
\sphinxstyletopictitle{Editor’s note}

\sphinxAtStartPar
Perhaps identical to the term in OMRSE
\end{sphinxShadowBox}

\begin{sphinxShadowBox}
\sphinxstyletopictitle{Term editor}

\sphinxAtStartPar
Michael Conlon \sphinxurl{https://orcid.org/0000-0002-1304-8447}
\end{sphinxShadowBox}

\begin{sphinxShadowBox}
\sphinxstyletopictitle{See also}

\sphinxAtStartPar
\sphinxurl{http://purl.obolibrary.org/obo/OMRSE\_00000062}
\end{sphinxShadowBox}
\begin{quote}

\index{ORG\_0000043@\spxentry{ORG\_0000043}!building@\spxentry{building}}\index{building@\spxentry{building}!ORG\_0000043@\spxentry{ORG\_0000043}}\ignorespaces \end{quote}


\subsection{ORG\_0000043 \sphinxhyphen{} building}
\label{\detokenize{doc-ORG_0000043:org-0000043-building}}\label{\detokenize{doc-ORG_0000043:index-0}}\label{\detokenize{doc-ORG_0000043::doc}}
\begin{sphinxShadowBox}
\sphinxstyletopictitle{Label}

\sphinxAtStartPar
building
\end{sphinxShadowBox}

\begin{sphinxShadowBox}
\sphinxstyletopictitle{Definition}

\sphinxAtStartPar
A permanent walled and roofed construction
\end{sphinxShadowBox}

\begin{sphinxShadowBox}
\sphinxstyletopictitle{Definition source}

\sphinxAtStartPar
\sphinxurl{https://en.wikipedia.org/wiki/Building}
\end{sphinxShadowBox}

\begin{sphinxShadowBox}
\sphinxstyletopictitle{Example}

\sphinxAtStartPar
The Pentagon, The Leaning Tower of Pisa, The Gherkin, The Coliseum
\end{sphinxShadowBox}

\begin{sphinxShadowBox}
\sphinxstyletopictitle{Editor’s note}

\sphinxAtStartPar
Similar to the term in ENVRO, but based on ‘architectural structure’ in OMRSE
\end{sphinxShadowBox}

\begin{sphinxShadowBox}
\sphinxstyletopictitle{Term editor}

\sphinxAtStartPar
Michael Conlon \sphinxurl{https://orcid.org/0000-0002-1304-8447}
\end{sphinxShadowBox}

\begin{sphinxShadowBox}
\sphinxstyletopictitle{See also}

\sphinxAtStartPar
\sphinxurl{http://purl.obolibrary.org/obo/ENVO\_00000073}
\end{sphinxShadowBox}
\begin{quote}

\index{ORG\_0000044@\spxentry{ORG\_0000044}!room@\spxentry{room}}\index{room@\spxentry{room}!ORG\_0000044@\spxentry{ORG\_0000044}}\ignorespaces \end{quote}


\subsection{ORG\_0000044 \sphinxhyphen{} room}
\label{\detokenize{doc-ORG_0000044:org-0000044-room}}\label{\detokenize{doc-ORG_0000044:index-0}}\label{\detokenize{doc-ORG_0000044::doc}}
\begin{sphinxShadowBox}
\sphinxstyletopictitle{Label}

\sphinxAtStartPar
room
\end{sphinxShadowBox}

\begin{sphinxShadowBox}
\sphinxstyletopictitle{Definition}

\sphinxAtStartPar
A space contained by a partitioned part of the inside of a building.  Often has an identifier.
\end{sphinxShadowBox}

\begin{sphinxShadowBox}
\sphinxstyletopictitle{Definition source}

\sphinxAtStartPar
\sphinxurl{https://www.merriam-webster.com/dictionary/room}
\end{sphinxShadowBox}

\begin{sphinxShadowBox}
\sphinxstyletopictitle{Example}

\sphinxAtStartPar
The Oval Office is a room in the White House.  A classroom, a locker room, a bedroom, a kitchen are all rooms.
\end{sphinxShadowBox}

\begin{sphinxShadowBox}
\sphinxstyletopictitle{Editor’s note}

\sphinxAtStartPar
A room is not a structure.  It is a space inside a structure.
\end{sphinxShadowBox}

\begin{sphinxShadowBox}
\sphinxstyletopictitle{Similar term in VIVO 1 Ontology}

\sphinxAtStartPar
\sphinxurl{http://vivoweb.org/ontology/core\#Room}
\end{sphinxShadowBox}

\begin{sphinxShadowBox}
\sphinxstyletopictitle{Term editor}

\sphinxAtStartPar
Michael Conlon \sphinxurl{https://orcid.org/0000-0002-1304-8447}
\end{sphinxShadowBox}
\begin{quote}

\index{ORG\_0000045@\spxentry{ORG\_0000045}!geographic region@\spxentry{geographic region}}\index{geographic region@\spxentry{geographic region}!ORG\_0000045@\spxentry{ORG\_0000045}}\ignorespaces \end{quote}


\subsection{ORG\_0000045 \sphinxhyphen{} geographic region}
\label{\detokenize{doc-ORG_0000045:org-0000045-geographic-region}}\label{\detokenize{doc-ORG_0000045:index-0}}\label{\detokenize{doc-ORG_0000045::doc}}
\begin{sphinxShadowBox}
\sphinxstyletopictitle{Label}

\sphinxAtStartPar
geographic region
\end{sphinxShadowBox}

\begin{sphinxShadowBox}
\sphinxstyletopictitle{Definition}

\sphinxAtStartPar
A place on the earth.  Not necessarily contiguous
\end{sphinxShadowBox}

\begin{sphinxShadowBox}
\sphinxstyletopictitle{Definition source}

\sphinxAtStartPar
Michael Conlon \sphinxurl{https://orcid.org/0000-0002-1304-8447}
\end{sphinxShadowBox}

\begin{sphinxShadowBox}
\sphinxstyletopictitle{Example}

\sphinxAtStartPar
The Indian Ocean, Africa, Metropolitan France, West 57th Street
\end{sphinxShadowBox}

\begin{sphinxShadowBox}
\sphinxstyletopictitle{Editor’s note}
\begin{enumerate}
\sphinxsetlistlabels{\arabic}{enumi}{enumii}{}{.}%
\item {} 
\sphinxAtStartPar
Geographical locations are often confused with the entities that are located in those places.  For example, France is a country with several discontiguous geographical loctions.  Metropolitan France is a geographical location in Europe.  2. Unlike the GAZ term, this term is not a reference to a place on the earth, rather it is a place on the earth.

\end{enumerate}
\end{sphinxShadowBox}

\begin{sphinxShadowBox}
\sphinxstyletopictitle{Similar term in VIVO 1 Ontology}

\sphinxAtStartPar
\sphinxurl{http://vivoweb.org/ontology/core\#GeographicRegion}
\end{sphinxShadowBox}

\begin{sphinxShadowBox}
\sphinxstyletopictitle{Term editor}

\sphinxAtStartPar
Michael Conlon \sphinxurl{https://orcid.org/0000-0002-1304-8447}
\end{sphinxShadowBox}

\begin{sphinxShadowBox}
\sphinxstyletopictitle{See also}

\sphinxAtStartPar
\sphinxurl{http://purl.obolibrary.org/obo/GAZ\_00000448}
\end{sphinxShadowBox}
\begin{quote}

\index{ORG\_0000046@\spxentry{ORG\_0000046}!geographic point@\spxentry{geographic point}}\index{geographic point@\spxentry{geographic point}!ORG\_0000046@\spxentry{ORG\_0000046}}\ignorespaces \end{quote}


\subsection{ORG\_0000046 \sphinxhyphen{} geographic point}
\label{\detokenize{doc-ORG_0000046:org-0000046-geographic-point}}\label{\detokenize{doc-ORG_0000046:index-0}}\label{\detokenize{doc-ORG_0000046::doc}}
\begin{sphinxShadowBox}
\sphinxstyletopictitle{Label}

\sphinxAtStartPar
geographic point
\end{sphinxShadowBox}

\begin{sphinxShadowBox}
\sphinxstyletopictitle{Alternate name}

\sphinxAtStartPar
latlong
\end{sphinxShadowBox}

\begin{sphinxShadowBox}
\sphinxstyletopictitle{Definition}

\sphinxAtStartPar
A point located on the earth
\end{sphinxShadowBox}

\begin{sphinxShadowBox}
\sphinxstyletopictitle{Definition source}

\sphinxAtStartPar
Michael Conlon \sphinxurl{https://orcid.org/0000-0002-1304-8447}
\end{sphinxShadowBox}

\begin{sphinxShadowBox}
\sphinxstyletopictitle{Example}

\sphinxAtStartPar
The point with latitude 27.9881199 and longitude 86.9161989
\end{sphinxShadowBox}

\begin{sphinxShadowBox}
\sphinxstyletopictitle{Editor’s note}
\begin{enumerate}
\sphinxsetlistlabels{\arabic}{enumi}{enumii}{}{.}%
\item {} 
\sphinxAtStartPar
Since a point is smaller than a building, we say the point is located in a building.  The building is the location of the point.  2. Typically use ORG\_0000004 to provide a lat/long value for the point.

\end{enumerate}
\end{sphinxShadowBox}

\begin{sphinxShadowBox}
\sphinxstyletopictitle{Term editor}

\sphinxAtStartPar
Michael Conlon \sphinxurl{https://orcid.org/0000-0002-1304-8447}
\end{sphinxShadowBox}
\begin{quote}

\index{ORG\_0000047@\spxentry{ORG\_0000047}!continent@\spxentry{continent}}\index{continent@\spxentry{continent}!ORG\_0000047@\spxentry{ORG\_0000047}}\ignorespaces \end{quote}


\subsection{ORG\_0000047 \sphinxhyphen{} continent}
\label{\detokenize{doc-ORG_0000047:org-0000047-continent}}\label{\detokenize{doc-ORG_0000047:index-0}}\label{\detokenize{doc-ORG_0000047::doc}}
\begin{sphinxShadowBox}
\sphinxstyletopictitle{Label}

\sphinxAtStartPar
continent
\end{sphinxShadowBox}

\begin{sphinxShadowBox}
\sphinxstyletopictitle{Definition}

\sphinxAtStartPar
One of the main landmasses of the globe, usually reckoned as seven in number (Europe, Asia, Africa, North America, South America, Australia, and Antarctica).
\end{sphinxShadowBox}

\begin{sphinxShadowBox}
\sphinxstyletopictitle{Definition source}

\sphinxAtStartPar
\sphinxurl{https://www.dictionary.com/browse/continent}
\end{sphinxShadowBox}

\begin{sphinxShadowBox}
\sphinxstyletopictitle{Example}

\sphinxAtStartPar
Asia, Europe, Antarctica, North America, South America, Africa, and Oceania are the continents
\end{sphinxShadowBox}

\begin{sphinxShadowBox}
\sphinxstyletopictitle{Editor’s note}

\sphinxAtStartPar
Should define 7 named individuals and then this class is explicitly defined
\end{sphinxShadowBox}

\begin{sphinxShadowBox}
\sphinxstyletopictitle{Similar term in VIVO 1 Ontology}

\sphinxAtStartPar
\sphinxurl{http://vivoweb.org/ontology/core\#Continent}
\end{sphinxShadowBox}

\begin{sphinxShadowBox}
\sphinxstyletopictitle{Term editor}

\sphinxAtStartPar
Michael Conlon \sphinxurl{https://orcid.org/0000-0002-1304-8447}
\end{sphinxShadowBox}
\begin{quote}

\index{ORG\_0000048@\spxentry{ORG\_0000048}!country@\spxentry{country}}\index{country@\spxentry{country}!ORG\_0000048@\spxentry{ORG\_0000048}}\ignorespaces \end{quote}


\subsection{ORG\_0000048 \sphinxhyphen{} country}
\label{\detokenize{doc-ORG_0000048:org-0000048-country}}\label{\detokenize{doc-ORG_0000048:index-0}}\label{\detokenize{doc-ORG_0000048::doc}}
\begin{sphinxShadowBox}
\sphinxstyletopictitle{Label}

\sphinxAtStartPar
country
\end{sphinxShadowBox}

\begin{sphinxShadowBox}
\sphinxstyletopictitle{Definition}

\sphinxAtStartPar
The territory governed by a sovereign state.
\end{sphinxShadowBox}

\begin{sphinxShadowBox}
\sphinxstyletopictitle{Definition source}

\sphinxAtStartPar
Michael Conlon \sphinxurl{https://orcid.org/0000-0002-1304-8447}
\end{sphinxShadowBox}

\begin{sphinxShadowBox}
\sphinxstyletopictitle{Example}

\sphinxAtStartPar
Canada, Ecuador, Slovakia, Namibia, Pakistan, New Zealand are all countries
\end{sphinxShadowBox}

\begin{sphinxShadowBox}
\sphinxstyletopictitle{Editor’s note}

\sphinxAtStartPar
This term is about the territory, not the geopolitical entity of the same name.
\end{sphinxShadowBox}

\begin{sphinxShadowBox}
\sphinxstyletopictitle{Term editor}

\sphinxAtStartPar
Michael Conlon \sphinxurl{https://orcid.org/0000-0002-1304-8447}
\end{sphinxShadowBox}
\begin{quote}

\index{ORG\_0000049@\spxentry{ORG\_0000049}!region@\spxentry{region}}\index{region@\spxentry{region}!ORG\_0000049@\spxentry{ORG\_0000049}}\ignorespaces \end{quote}


\subsection{ORG\_0000049 \sphinxhyphen{} region}
\label{\detokenize{doc-ORG_0000049:org-0000049-region}}\label{\detokenize{doc-ORG_0000049:index-0}}\label{\detokenize{doc-ORG_0000049::doc}}
\begin{sphinxShadowBox}
\sphinxstyletopictitle{Label}

\sphinxAtStartPar
region
\end{sphinxShadowBox}

\begin{sphinxShadowBox}
\sphinxstyletopictitle{Definition}

\sphinxAtStartPar
A subdivision of the territory of a country
\end{sphinxShadowBox}

\begin{sphinxShadowBox}
\sphinxstyletopictitle{Definition source}

\sphinxAtStartPar
Michael Conlon \sphinxurl{https://orcid.org/0000-0002-1304-8447}
\end{sphinxShadowBox}

\begin{sphinxShadowBox}
\sphinxstyletopictitle{Example}

\sphinxAtStartPar
US states, counties, Candian Provinces, US Zip codes, US SMSAs, US time zone territories are all examples of regions.
\end{sphinxShadowBox}

\begin{sphinxShadowBox}
\sphinxstyletopictitle{Editor’s note}

\sphinxAtStartPar
This term is about the territory, not its governance.  It could be subclassed to define verious types of regions \textendash{} census tracts, zip codes, and others.
\end{sphinxShadowBox}

\begin{sphinxShadowBox}
\sphinxstyletopictitle{Term editor}

\sphinxAtStartPar
Michael Conlon \sphinxurl{https://orcid.org/0000-0002-1304-8447}
\end{sphinxShadowBox}
\begin{quote}

\index{ORG\_0000050@\spxentry{ORG\_0000050}!populated place@\spxentry{populated place}}\index{populated place@\spxentry{populated place}!ORG\_0000050@\spxentry{ORG\_0000050}}\ignorespaces \end{quote}


\subsection{ORG\_0000050 \sphinxhyphen{} populated place}
\label{\detokenize{doc-ORG_0000050:org-0000050-populated-place}}\label{\detokenize{doc-ORG_0000050:index-0}}\label{\detokenize{doc-ORG_0000050::doc}}
\begin{sphinxShadowBox}
\sphinxstyletopictitle{Label}

\sphinxAtStartPar
populated place
\end{sphinxShadowBox}

\begin{sphinxShadowBox}
\sphinxstyletopictitle{Definition}

\sphinxAtStartPar
A named place on the earth occupied by people
\end{sphinxShadowBox}

\begin{sphinxShadowBox}
\sphinxstyletopictitle{Definition source}

\sphinxAtStartPar
Michael Conlon \sphinxurl{https://orcid.org/0000-0002-1304-8447}
\end{sphinxShadowBox}

\begin{sphinxShadowBox}
\sphinxstyletopictitle{Example}

\sphinxAtStartPar
Tokyo Japan, Eustis Florida, Podgorica Montenegro, and Stevenage United Kingdom are all populated places.
\end{sphinxShadowBox}

\begin{sphinxShadowBox}
\sphinxstyletopictitle{Editor’s note}

\sphinxAtStartPar
There is wide variation in the applicaton of this term.
\end{sphinxShadowBox}

\begin{sphinxShadowBox}
\sphinxstyletopictitle{Similar term in VIVO 1 Ontology}

\sphinxAtStartPar
\sphinxurl{http://vivoweb.org/ontology/core\#PopulatedPlace}
\end{sphinxShadowBox}

\begin{sphinxShadowBox}
\sphinxstyletopictitle{Term editor}

\sphinxAtStartPar
Michael Conlon \sphinxurl{https://orcid.org/0000-0002-1304-8447}
\end{sphinxShadowBox}
\begin{quote}

\index{ORG\_0000051@\spxentry{ORG\_0000051}!founding process@\spxentry{founding process}}\index{founding process@\spxentry{founding process}!ORG\_0000051@\spxentry{ORG\_0000051}}\ignorespaces \end{quote}


\subsection{ORG\_0000051 \sphinxhyphen{} founding process}
\label{\detokenize{doc-ORG_0000051:org-0000051-founding-process}}\label{\detokenize{doc-ORG_0000051:index-0}}\label{\detokenize{doc-ORG_0000051::doc}}
\begin{sphinxShadowBox}
\sphinxstyletopictitle{Label}

\sphinxAtStartPar
founding process
\end{sphinxShadowBox}

\begin{sphinxShadowBox}
\sphinxstyletopictitle{Definition}

\sphinxAtStartPar
The process by which the organization was founded
\end{sphinxShadowBox}

\begin{sphinxShadowBox}
\sphinxstyletopictitle{Definition source}

\sphinxAtStartPar
Michael Conlon \sphinxurl{https://orcid.org/0000-0002-1304-8447}
\end{sphinxShadowBox}

\begin{sphinxShadowBox}
\sphinxstyletopictitle{Example}

\sphinxAtStartPar
The United States government was founded by the constitutional convention process of 1787 resulting in a ratified constitution in 1790
\end{sphinxShadowBox}

\begin{sphinxShadowBox}
\sphinxstyletopictitle{Editor’s note}

\sphinxAtStartPar
The founding process might by a subproperty of a generic creation process if such a process ever appears at a higher level in the ontologies
\end{sphinxShadowBox}

\begin{sphinxShadowBox}
\sphinxstyletopictitle{Term editor}

\sphinxAtStartPar
Michael Conlon \sphinxurl{https://orcid.org/0000-0002-1304-8447}
\end{sphinxShadowBox}
\begin{quote}

\index{ORG\_0000052@\spxentry{ORG\_0000052}!founding process boundary@\spxentry{founding process boundary}}\index{founding process boundary@\spxentry{founding process boundary}!ORG\_0000052@\spxentry{ORG\_0000052}}\ignorespaces \end{quote}


\subsection{ORG\_0000052 \sphinxhyphen{} founding process boundary}
\label{\detokenize{doc-ORG_0000052:org-0000052-founding-process-boundary}}\label{\detokenize{doc-ORG_0000052:index-0}}\label{\detokenize{doc-ORG_0000052::doc}}
\begin{sphinxShadowBox}
\sphinxstyletopictitle{Label}

\sphinxAtStartPar
founding process boundary
\end{sphinxShadowBox}

\begin{sphinxShadowBox}
\sphinxstyletopictitle{Alternate name}

\sphinxAtStartPar
founding
\end{sphinxShadowBox}

\begin{sphinxShadowBox}
\sphinxstyletopictitle{Definition}

\sphinxAtStartPar
The process boundary which defines the moment of creation of an orgnization.  Before the moment  the organization does not exist.  After the moment, the organization exists.
\end{sphinxShadowBox}

\begin{sphinxShadowBox}
\sphinxstyletopictitle{Definition source}

\sphinxAtStartPar
Michael Conlon \sphinxurl{https://orcid.org/0000-0002-1304-8447}
\end{sphinxShadowBox}

\begin{sphinxShadowBox}
\sphinxstyletopictitle{Example}

\sphinxAtStartPar
The University of Bologna was founded in 1088.  Facebook was founded in 2004.
\end{sphinxShadowBox}

\begin{sphinxShadowBox}
\sphinxstyletopictitle{Term editor}

\sphinxAtStartPar
Michael Conlon \sphinxurl{https://orcid.org/0000-0002-1304-8447}
\end{sphinxShadowBox}
\begin{quote}

\index{ORG\_0000053@\spxentry{ORG\_0000053}!dissolution process@\spxentry{dissolution process}}\index{dissolution process@\spxentry{dissolution process}!ORG\_0000053@\spxentry{ORG\_0000053}}\ignorespaces \end{quote}


\subsection{ORG\_0000053 \sphinxhyphen{} dissolution process}
\label{\detokenize{doc-ORG_0000053:org-0000053-dissolution-process}}\label{\detokenize{doc-ORG_0000053:index-0}}\label{\detokenize{doc-ORG_0000053::doc}}
\begin{sphinxShadowBox}
\sphinxstyletopictitle{Label}

\sphinxAtStartPar
dissolution process
\end{sphinxShadowBox}

\begin{sphinxShadowBox}
\sphinxstyletopictitle{Definition}

\sphinxAtStartPar
The process by which an organization no longer exists.
\end{sphinxShadowBox}

\begin{sphinxShadowBox}
\sphinxstyletopictitle{Definition source}

\sphinxAtStartPar
Michael Conlon \sphinxurl{https://orcid.org/0000-0002-1304-8447}
\end{sphinxShadowBox}

\begin{sphinxShadowBox}
\sphinxstyletopictitle{Example}

\sphinxAtStartPar
The German Democratic Republic (aka East Germany) ended in 1990.
\end{sphinxShadowBox}

\begin{sphinxShadowBox}
\sphinxstyletopictitle{Editor’s note}

\sphinxAtStartPar
There are many dissolution processes (merger, acquisition, discontinuation).  These can be added at a later date if needed.
\end{sphinxShadowBox}

\begin{sphinxShadowBox}
\sphinxstyletopictitle{Term editor}

\sphinxAtStartPar
Michael Conlon \sphinxurl{https://orcid.org/0000-0002-1304-8447}
\end{sphinxShadowBox}
\begin{quote}

\index{ORG\_0000054@\spxentry{ORG\_0000054}!dissolution process boundary@\spxentry{dissolution process boundary}}\index{dissolution process boundary@\spxentry{dissolution process boundary}!ORG\_0000054@\spxentry{ORG\_0000054}}\ignorespaces \end{quote}


\subsection{ORG\_0000054 \sphinxhyphen{} dissolution process boundary}
\label{\detokenize{doc-ORG_0000054:org-0000054-dissolution-process-boundary}}\label{\detokenize{doc-ORG_0000054:index-0}}\label{\detokenize{doc-ORG_0000054::doc}}
\begin{sphinxShadowBox}
\sphinxstyletopictitle{Label}

\sphinxAtStartPar
dissolution process boundary
\end{sphinxShadowBox}

\begin{sphinxShadowBox}
\sphinxstyletopictitle{Alternate name}

\sphinxAtStartPar
dissolution
\end{sphinxShadowBox}

\begin{sphinxShadowBox}
\sphinxstyletopictitle{Definition}

\sphinxAtStartPar
The process boundary which marks the moment at which the organization no longer exists
\end{sphinxShadowBox}

\begin{sphinxShadowBox}
\sphinxstyletopictitle{Definition source}

\sphinxAtStartPar
Michael Conlon \sphinxurl{https://orcid.org/0000-0002-1304-8447}
\end{sphinxShadowBox}

\begin{sphinxShadowBox}
\sphinxstyletopictitle{Example}

\sphinxAtStartPar
Duraspace ceased to exist on July 1, 2019
\end{sphinxShadowBox}

\begin{sphinxShadowBox}
\sphinxstyletopictitle{Editor’s note}

\sphinxAtStartPar
The dissolution process may have many steps and boundaries, such as ceasing operations, transfer of funds, dissolution of board.  These can be added at a later date if needed.
\end{sphinxShadowBox}

\begin{sphinxShadowBox}
\sphinxstyletopictitle{Term editor}

\sphinxAtStartPar
Michael Conlon \sphinxurl{https://orcid.org/0000-0002-1304-8447}
\end{sphinxShadowBox}
\begin{quote}

\index{ORG\_0000055@\spxentry{ORG\_0000055}!succession process@\spxentry{succession process}}\index{succession process@\spxentry{succession process}!ORG\_0000055@\spxentry{ORG\_0000055}}\ignorespaces \end{quote}


\subsection{ORG\_0000055 \sphinxhyphen{} succession process}
\label{\detokenize{doc-ORG_0000055:org-0000055-succession-process}}\label{\detokenize{doc-ORG_0000055:index-0}}\label{\detokenize{doc-ORG_0000055::doc}}
\begin{sphinxShadowBox}
\sphinxstyletopictitle{Label}

\sphinxAtStartPar
succession process
\end{sphinxShadowBox}

\begin{sphinxShadowBox}
\sphinxstyletopictitle{Definition}

\sphinxAtStartPar
The process by which one organization gores out of existence and is succeeded by a new organization
\end{sphinxShadowBox}

\begin{sphinxShadowBox}
\sphinxstyletopictitle{Definition source}

\sphinxAtStartPar
Michael Conlon \sphinxurl{https://orcid.org/0000-0002-1304-8447}
\end{sphinxShadowBox}

\begin{sphinxShadowBox}
\sphinxstyletopictitle{Example}

\sphinxAtStartPar
The Central People’s Government of the People’s Republic of China was proclaimed by Mao Zedong on October 1, 1949 at 3PM replacing the government of the Republic of China.
\end{sphinxShadowBox}

\begin{sphinxShadowBox}
\sphinxstyletopictitle{Editor’s note}

\sphinxAtStartPar
The succession process typically has founding and dissolution processes as occurent parts
\end{sphinxShadowBox}

\begin{sphinxShadowBox}
\sphinxstyletopictitle{Term editor}

\sphinxAtStartPar
Michael Conlon \sphinxurl{https://orcid.org/0000-0002-1304-8447}
\end{sphinxShadowBox}
\begin{quote}

\index{ORG\_0000056@\spxentry{ORG\_0000056}!succession process boundary@\spxentry{succession process boundary}}\index{succession process boundary@\spxentry{succession process boundary}!ORG\_0000056@\spxentry{ORG\_0000056}}\ignorespaces \end{quote}


\subsection{ORG\_0000056 \sphinxhyphen{} succession process boundary}
\label{\detokenize{doc-ORG_0000056:org-0000056-succession-process-boundary}}\label{\detokenize{doc-ORG_0000056:index-0}}\label{\detokenize{doc-ORG_0000056::doc}}
\begin{sphinxShadowBox}
\sphinxstyletopictitle{Label}

\sphinxAtStartPar
succession process boundary
\end{sphinxShadowBox}

\begin{sphinxShadowBox}
\sphinxstyletopictitle{Definition}

\sphinxAtStartPar
The process boundary within a succession process.  Complex succesion processes may have many boundaries.
\end{sphinxShadowBox}

\begin{sphinxShadowBox}
\sphinxstyletopictitle{Definition source}

\sphinxAtStartPar
Michael Conlon \sphinxurl{https://orcid.org/0000-0002-1304-8447}
\end{sphinxShadowBox}

\begin{sphinxShadowBox}
\sphinxstyletopictitle{Example}

\sphinxAtStartPar
East Timor became a formally independent country on 20 May 2002
\end{sphinxShadowBox}

\begin{sphinxShadowBox}
\sphinxstyletopictitle{Term editor}

\sphinxAtStartPar
Michael Conlon \sphinxurl{https://orcid.org/0000-0002-1304-8447}
\end{sphinxShadowBox}
\begin{quote}

\index{ORG\_0000057@\spxentry{ORG\_0000057}!web site@\spxentry{web site}}\index{web site@\spxentry{web site}!ORG\_0000057@\spxentry{ORG\_0000057}}\ignorespaces \end{quote}


\subsection{ORG\_0000057 \sphinxhyphen{} web site}
\label{\detokenize{doc-ORG_0000057:org-0000057-web-site}}\label{\detokenize{doc-ORG_0000057:index-0}}\label{\detokenize{doc-ORG_0000057::doc}}
\begin{sphinxShadowBox}
\sphinxstyletopictitle{Label}

\sphinxAtStartPar
web site
\end{sphinxShadowBox}

\begin{sphinxShadowBox}
\sphinxstyletopictitle{Definition}

\sphinxAtStartPar
The information content entity consisting of a group of World Wide Web pages usually containing hyperlinks to each other and made available online by an individual, company, educational institution, government, or organization
\end{sphinxShadowBox}

\begin{sphinxShadowBox}
\sphinxstyletopictitle{Definition source}

\sphinxAtStartPar
\sphinxurl{https://www.merriam-webster.com/dictionary/website}
\end{sphinxShadowBox}

\begin{sphinxShadowBox}
\sphinxstyletopictitle{Example}

\sphinxAtStartPar
Facebook and Google have prominent web sites.
\end{sphinxShadowBox}

\begin{sphinxShadowBox}
\sphinxstyletopictitle{Similar term in VIVO 1 Ontology}

\sphinxAtStartPar
\sphinxurl{http://www.w3.org/2006/vcard/ns\#URL}
\end{sphinxShadowBox}

\begin{sphinxShadowBox}
\sphinxstyletopictitle{Term editor}

\sphinxAtStartPar
Michael Conlon \sphinxurl{https://orcid.org/0000-0002-1304-8447}
\end{sphinxShadowBox}
\begin{quote}

\index{ORG\_0000058@\spxentry{ORG\_0000058}!spin\sphinxhyphen{}off process@\spxentry{spin\sphinxhyphen{}off process}}\index{spin\sphinxhyphen{}off process@\spxentry{spin\sphinxhyphen{}off process}!ORG\_0000058@\spxentry{ORG\_0000058}}\ignorespaces \end{quote}


\subsection{ORG\_0000058 \sphinxhyphen{} spin\sphinxhyphen{}off process}
\label{\detokenize{doc-ORG_0000058:org-0000058-spin-off-process}}\label{\detokenize{doc-ORG_0000058:index-0}}\label{\detokenize{doc-ORG_0000058::doc}}
\begin{sphinxShadowBox}
\sphinxstyletopictitle{Label}

\sphinxAtStartPar
spin\sphinxhyphen{}off process
\end{sphinxShadowBox}

\begin{sphinxShadowBox}
\sphinxstyletopictitle{Definition}

\sphinxAtStartPar
The process by which one organization spins off of another
\end{sphinxShadowBox}

\begin{sphinxShadowBox}
\sphinxstyletopictitle{Definition source}

\sphinxAtStartPar
Michael Conlon \sphinxurl{https://orcid.org/0000-0002-1304-8447}
\end{sphinxShadowBox}

\begin{sphinxShadowBox}
\sphinxstyletopictitle{Example}

\sphinxAtStartPar
Google is the output of a spin\sphinxhyphen{}off process in which Stanford University was a participant
\end{sphinxShadowBox}

\begin{sphinxShadowBox}
\sphinxstyletopictitle{Term editor}

\sphinxAtStartPar
Michael Conlon \sphinxurl{https://orcid.org/0000-0002-1304-8447}
\end{sphinxShadowBox}
\begin{quote}

\index{ORG\_0000059@\spxentry{ORG\_0000059}!spin\sphinxhyphen{}off process boundary@\spxentry{spin\sphinxhyphen{}off process boundary}}\index{spin\sphinxhyphen{}off process boundary@\spxentry{spin\sphinxhyphen{}off process boundary}!ORG\_0000059@\spxentry{ORG\_0000059}}\ignorespaces \end{quote}


\subsection{ORG\_0000059 \sphinxhyphen{} spin\sphinxhyphen{}off process boundary}
\label{\detokenize{doc-ORG_0000059:org-0000059-spin-off-process-boundary}}\label{\detokenize{doc-ORG_0000059:index-0}}\label{\detokenize{doc-ORG_0000059::doc}}
\begin{sphinxShadowBox}
\sphinxstyletopictitle{Label}

\sphinxAtStartPar
spin\sphinxhyphen{}off process boundary
\end{sphinxShadowBox}

\begin{sphinxShadowBox}
\sphinxstyletopictitle{Definition}

\sphinxAtStartPar
The boundary of a spin\sphinxhyphen{}off process
\end{sphinxShadowBox}

\begin{sphinxShadowBox}
\sphinxstyletopictitle{Definition source}

\sphinxAtStartPar
Michael Conlon \sphinxurl{https://orcid.org/0000-0002-1304-8447}
\end{sphinxShadowBox}

\begin{sphinxShadowBox}
\sphinxstyletopictitle{Example}

\sphinxAtStartPar
Spin\sphinxhyphen{}off process boundaries include moments such as incorporation, board formation, funding ac quisition, and patent licensing
\end{sphinxShadowBox}

\begin{sphinxShadowBox}
\sphinxstyletopictitle{Term editor}

\sphinxAtStartPar
Michael Conlon \sphinxurl{https://orcid.org/0000-0002-1304-8447}
\end{sphinxShadowBox}
\begin{quote}

\index{ORG\_0000060@\spxentry{ORG\_0000060}!organizational membership@\spxentry{organizational membership}}\index{organizational membership@\spxentry{organizational membership}!ORG\_0000060@\spxentry{ORG\_0000060}}\ignorespaces \end{quote}


\subsection{ORG\_0000060 \sphinxhyphen{} organizational membership}
\label{\detokenize{doc-ORG_0000060:org-0000060-organizational-membership}}\label{\detokenize{doc-ORG_0000060:index-0}}\label{\detokenize{doc-ORG_0000060::doc}}
\begin{sphinxShadowBox}
\sphinxstyletopictitle{Label}

\sphinxAtStartPar
organizational membership
\end{sphinxShadowBox}

\begin{sphinxShadowBox}
\sphinxstyletopictitle{Definition}

\sphinxAtStartPar
The asymmetric relationship involving two organizations in which one is a member of the other
\end{sphinxShadowBox}

\begin{sphinxShadowBox}
\sphinxstyletopictitle{Definition source}

\sphinxAtStartPar
Michael Conlon \sphinxurl{https://orcid.org/0000-0002-1304-8447}
\end{sphinxShadowBox}

\begin{sphinxShadowBox}
\sphinxstyletopictitle{Example}

\sphinxAtStartPar
Texas A\&M University is an organizational member of the Association of Public and Land Grant Universities
\end{sphinxShadowBox}

\begin{sphinxShadowBox}
\sphinxstyletopictitle{Editor’s note}

\sphinxAtStartPar
Organizational Membership is a Membership, which is a Relationship, which is an Occurrent.  The subsumption is out of scope for the Organizationa Ontology. The existence of a membership implies the existence of a membership creation process and a membership creation process boundary.  These are currently out of scope for the Organization Ontology
\end{sphinxShadowBox}

\begin{sphinxShadowBox}
\sphinxstyletopictitle{Term editor}

\sphinxAtStartPar
Michael Conlon \sphinxurl{https://orcid.org/0000-0002-1304-8447}
\end{sphinxShadowBox}
\begin{quote}

\index{ORG\_0000061@\spxentry{ORG\_0000061}!organizational member role@\spxentry{organizational member role}}\index{organizational member role@\spxentry{organizational member role}!ORG\_0000061@\spxentry{ORG\_0000061}}\ignorespaces \end{quote}


\subsection{ORG\_0000061 \sphinxhyphen{} organizational member role}
\label{\detokenize{doc-ORG_0000061:org-0000061-organizational-member-role}}\label{\detokenize{doc-ORG_0000061:index-0}}\label{\detokenize{doc-ORG_0000061::doc}}
\begin{sphinxShadowBox}
\sphinxstyletopictitle{Label}

\sphinxAtStartPar
organizational member role
\end{sphinxShadowBox}

\begin{sphinxShadowBox}
\sphinxstyletopictitle{Alternate name}

\sphinxAtStartPar
member
\end{sphinxShadowBox}

\begin{sphinxShadowBox}
\sphinxstyletopictitle{Definition}

\sphinxAtStartPar
The role of an organization in being a member of another
\end{sphinxShadowBox}

\begin{sphinxShadowBox}
\sphinxstyletopictitle{Definition source}

\sphinxAtStartPar
Michael Conlon \sphinxurl{https://orcid.org/0000-0002-1304-8447}
\end{sphinxShadowBox}

\begin{sphinxShadowBox}
\sphinxstyletopictitle{Example}

\sphinxAtStartPar
The University of aToronto has organizational membership role which is realized in a membership granted by the American Association of Universities
\end{sphinxShadowBox}

\begin{sphinxShadowBox}
\sphinxstyletopictitle{Term editor}

\sphinxAtStartPar
Michael Conlon \sphinxurl{https://orcid.org/0000-0002-1304-8447}
\end{sphinxShadowBox}
\begin{quote}

\index{ORG\_0000062@\spxentry{ORG\_0000062}!organizational member grantor role@\spxentry{organizational member grantor role}}\index{organizational member grantor role@\spxentry{organizational member grantor role}!ORG\_0000062@\spxentry{ORG\_0000062}}\ignorespaces \end{quote}


\subsection{ORG\_0000062 \sphinxhyphen{} organizational member grantor role}
\label{\detokenize{doc-ORG_0000062:org-0000062-organizational-member-grantor-role}}\label{\detokenize{doc-ORG_0000062:index-0}}\label{\detokenize{doc-ORG_0000062::doc}}
\begin{sphinxShadowBox}
\sphinxstyletopictitle{Label}

\sphinxAtStartPar
organizational member grantor role
\end{sphinxShadowBox}

\begin{sphinxShadowBox}
\sphinxstyletopictitle{Definition}

\sphinxAtStartPar
The role of an organization in granting a member role to another
\end{sphinxShadowBox}

\begin{sphinxShadowBox}
\sphinxstyletopictitle{Definition source}

\sphinxAtStartPar
Michael Conlon \sphinxurl{https://orcid.org/0000-0002-1304-8447}
\end{sphinxShadowBox}

\begin{sphinxShadowBox}
\sphinxstyletopictitle{Example}

\sphinxAtStartPar
Most membership organizations grant theoir memberships to their members
\end{sphinxShadowBox}

\begin{sphinxShadowBox}
\sphinxstyletopictitle{Term editor}

\sphinxAtStartPar
Michael Conlon \sphinxurl{https://orcid.org/0000-0002-1304-8447}
\end{sphinxShadowBox}
\begin{quote}

\index{ORG\_0000063@\spxentry{ORG\_0000063}!student led organization quality@\spxentry{student led organization quality}}\index{student led organization quality@\spxentry{student led organization quality}!ORG\_0000063@\spxentry{ORG\_0000063}}\ignorespaces \end{quote}


\subsection{ORG\_0000063 \sphinxhyphen{} student led organization quality}
\label{\detokenize{doc-ORG_0000063:org-0000063-student-led-organization-quality}}\label{\detokenize{doc-ORG_0000063:index-0}}\label{\detokenize{doc-ORG_0000063::doc}}
\begin{sphinxShadowBox}
\sphinxstyletopictitle{Label}

\sphinxAtStartPar
student led organization quality
\end{sphinxShadowBox}

\begin{sphinxShadowBox}
\sphinxstyletopictitle{Definition}

\sphinxAtStartPar
The quality of an organization that is led by a student
\end{sphinxShadowBox}

\begin{sphinxShadowBox}
\sphinxstyletopictitle{Definition source}

\sphinxAtStartPar
Michael Conlon \sphinxurl{https://orcid.org/0000-0002-1304-8447}
\end{sphinxShadowBox}

\begin{sphinxShadowBox}
\sphinxstyletopictitle{Example}

\sphinxAtStartPar
The student chess club, the student data science club, the student newspaper may all be student\sphinxhyphen{}led organizations
\end{sphinxShadowBox}

\begin{sphinxShadowBox}
\sphinxstyletopictitle{Editor’s note}

\sphinxAtStartPar
Student\sphinxhyphen{}led organizations may be formal or informal
\end{sphinxShadowBox}

\begin{sphinxShadowBox}
\sphinxstyletopictitle{Similar term in VIVO 1 Ontology}

\sphinxAtStartPar
\sphinxurl{http://vivoweb.org/ontology/core\#StudentOrganization}
\end{sphinxShadowBox}

\begin{sphinxShadowBox}
\sphinxstyletopictitle{Term editor}

\sphinxAtStartPar
Michael Conlon \sphinxurl{https://orcid.org/0000-0002-1304-8447}
\end{sphinxShadowBox}
\begin{quote}

\index{ORG\_0000064@\spxentry{ORG\_0000064}!woman led organization quality@\spxentry{woman led organization quality}}\index{woman led organization quality@\spxentry{woman led organization quality}!ORG\_0000064@\spxentry{ORG\_0000064}}\ignorespaces \end{quote}


\subsection{ORG\_0000064 \sphinxhyphen{} woman led organization quality}
\label{\detokenize{doc-ORG_0000064:org-0000064-woman-led-organization-quality}}\label{\detokenize{doc-ORG_0000064:index-0}}\label{\detokenize{doc-ORG_0000064::doc}}
\begin{sphinxShadowBox}
\sphinxstyletopictitle{Label}

\sphinxAtStartPar
woman led organization quality
\end{sphinxShadowBox}

\begin{sphinxShadowBox}
\sphinxstyletopictitle{Definition}

\sphinxAtStartPar
The quality of an organization that is led by a woman
\end{sphinxShadowBox}

\begin{sphinxShadowBox}
\sphinxstyletopictitle{Definition source}

\sphinxAtStartPar
Michael Conlon \sphinxurl{https://orcid.org/0000-0002-1304-8447}
\end{sphinxShadowBox}

\begin{sphinxShadowBox}
\sphinxstyletopictitle{Example}

\sphinxAtStartPar
As of 2021, The Office of the Vice President of the United States, and the Federal Government of Germany are women\sphinxhyphen{}led organizations
\end{sphinxShadowBox}

\begin{sphinxShadowBox}
\sphinxstyletopictitle{Editor’s note}

\sphinxAtStartPar
In some jurisdictions, such as the US, counting women\sphinxhyphen{}led organizations is important
\end{sphinxShadowBox}

\begin{sphinxShadowBox}
\sphinxstyletopictitle{Term editor}

\sphinxAtStartPar
Michael Conlon \sphinxurl{https://orcid.org/0000-0002-1304-8447}
\end{sphinxShadowBox}
\begin{quote}

\index{ORG\_0000065@\spxentry{ORG\_0000065}!minority led organization quality@\spxentry{minority led organization quality}}\index{minority led organization quality@\spxentry{minority led organization quality}!ORG\_0000065@\spxentry{ORG\_0000065}}\ignorespaces \end{quote}


\subsection{ORG\_0000065 \sphinxhyphen{} minority led organization quality}
\label{\detokenize{doc-ORG_0000065:org-0000065-minority-led-organization-quality}}\label{\detokenize{doc-ORG_0000065:index-0}}\label{\detokenize{doc-ORG_0000065::doc}}
\begin{sphinxShadowBox}
\sphinxstyletopictitle{Label}

\sphinxAtStartPar
minority led organization quality
\end{sphinxShadowBox}

\begin{sphinxShadowBox}
\sphinxstyletopictitle{Definition}

\sphinxAtStartPar
The quality of an organiztion that is led by a designated minority
\end{sphinxShadowBox}

\begin{sphinxShadowBox}
\sphinxstyletopictitle{Definition source}

\sphinxAtStartPar
Michael Conlon \sphinxurl{https://orcid.org/0000-0002-1304-8447}
\end{sphinxShadowBox}

\begin{sphinxShadowBox}
\sphinxstyletopictitle{Example}

\sphinxAtStartPar
From 2008\sphinxhyphen{}2016, the Office of the President of the United States was a minority\sphinxhyphen{}led organization
\end{sphinxShadowBox}

\begin{sphinxShadowBox}
\sphinxstyletopictitle{Editor’s note}

\sphinxAtStartPar
In some jurisdictions, such as the US, counting minority\sphinxhyphen{}led organizations is important
\end{sphinxShadowBox}

\begin{sphinxShadowBox}
\sphinxstyletopictitle{Term editor}

\sphinxAtStartPar
Michael Conlon \sphinxurl{https://orcid.org/0000-0002-1304-8447}
\end{sphinxShadowBox}
\begin{quote}

\index{ORG\_0000066@\spxentry{ORG\_0000066}!registered address quality@\spxentry{registered address quality}}\index{registered address quality@\spxentry{registered address quality}!ORG\_0000066@\spxentry{ORG\_0000066}}\ignorespaces \end{quote}


\subsection{ORG\_0000066 \sphinxhyphen{} registered address quality}
\label{\detokenize{doc-ORG_0000066:org-0000066-registered-address-quality}}\label{\detokenize{doc-ORG_0000066:index-0}}\label{\detokenize{doc-ORG_0000066::doc}}
\begin{sphinxShadowBox}
\sphinxstyletopictitle{Label}

\sphinxAtStartPar
registered address quality
\end{sphinxShadowBox}

\begin{sphinxShadowBox}
\sphinxstyletopictitle{Definition}

\sphinxAtStartPar
The quality of a location that is the legal/registered location for the organization
\end{sphinxShadowBox}

\begin{sphinxShadowBox}
\sphinxstyletopictitle{Definition source}

\sphinxAtStartPar
Michael Conlon \sphinxurl{https://orcid.org/0000-0002-1304-8447}
\end{sphinxShadowBox}

\begin{sphinxShadowBox}
\sphinxstyletopictitle{Example}

\sphinxAtStartPar
The legal registered location for Google is Mountain View, California
\end{sphinxShadowBox}

\begin{sphinxShadowBox}
\sphinxstyletopictitle{Term editor}

\sphinxAtStartPar
Michael Conlon \sphinxurl{https://orcid.org/0000-0002-1304-8447}
\end{sphinxShadowBox}
\begin{quote}

\index{ORG\_0000067@\spxentry{ORG\_0000067}!primary address quality@\spxentry{primary address quality}}\index{primary address quality@\spxentry{primary address quality}!ORG\_0000067@\spxentry{ORG\_0000067}}\ignorespaces \end{quote}


\subsection{ORG\_0000067 \sphinxhyphen{} primary address quality}
\label{\detokenize{doc-ORG_0000067:org-0000067-primary-address-quality}}\label{\detokenize{doc-ORG_0000067:index-0}}\label{\detokenize{doc-ORG_0000067::doc}}
\begin{sphinxShadowBox}
\sphinxstyletopictitle{Label}

\sphinxAtStartPar
primary address quality
\end{sphinxShadowBox}

\begin{sphinxShadowBox}
\sphinxstyletopictitle{Definition}

\sphinxAtStartPar
The quality of a location that is the primary/preferred location for the organization
\end{sphinxShadowBox}

\begin{sphinxShadowBox}
\sphinxstyletopictitle{Definition source}

\sphinxAtStartPar
Michael Conlon \sphinxurl{https://orcid.org/0000-0002-1304-8447}
\end{sphinxShadowBox}

\begin{sphinxShadowBox}
\sphinxstyletopictitle{Example}

\sphinxAtStartPar
The primary site for Microsoft is Redmond, Washington
\end{sphinxShadowBox}

\begin{sphinxShadowBox}
\sphinxstyletopictitle{Term editor}

\sphinxAtStartPar
Michael Conlon \sphinxurl{https://orcid.org/0000-0002-1304-8447}
\end{sphinxShadowBox}
\begin{quote}

\index{ORG\_0000068@\spxentry{ORG\_0000068}!organizational position@\spxentry{organizational position}}\index{organizational position@\spxentry{organizational position}!ORG\_0000068@\spxentry{ORG\_0000068}}\ignorespaces \end{quote}


\subsection{ORG\_0000068 \sphinxhyphen{} organizational position}
\label{\detokenize{doc-ORG_0000068:org-0000068-organizational-position}}\label{\detokenize{doc-ORG_0000068:index-0}}\label{\detokenize{doc-ORG_0000068::doc}}
\begin{sphinxShadowBox}
\sphinxstyletopictitle{Label}

\sphinxAtStartPar
organizational position
\end{sphinxShadowBox}

\begin{sphinxShadowBox}
\sphinxstyletopictitle{Alternate name}

\sphinxAtStartPar
post
\end{sphinxShadowBox}

\begin{sphinxShadowBox}
\sphinxstyletopictitle{Definition}

\sphinxAtStartPar
The asymmetric relationship between an organization and a person in which a person has a role in the positon, as does the organization.
\end{sphinxShadowBox}

\begin{sphinxShadowBox}
\sphinxstyletopictitle{Definition source}

\sphinxAtStartPar
Michael Conlon \sphinxurl{https://orcid.org/0000-0002-1304-8447}
\end{sphinxShadowBox}

\begin{sphinxShadowBox}
\sphinxstyletopictitle{Example}

\sphinxAtStartPar
A university may have a position of chancellor, which is then filled by a person
\end{sphinxShadowBox}

\begin{sphinxShadowBox}
\sphinxstyletopictitle{Editor’s note}

\sphinxAtStartPar
The organization and the person in the position relationship each have roles with respect to the position relationship
\end{sphinxShadowBox}

\begin{sphinxShadowBox}
\sphinxstyletopictitle{Similar term in VIVO 1 Ontology}

\sphinxAtStartPar
\sphinxurl{http://vivoweb.org/ontology/core\#Position}
\end{sphinxShadowBox}

\begin{sphinxShadowBox}
\sphinxstyletopictitle{Term editor}

\sphinxAtStartPar
Michael Conlon \sphinxurl{https://orcid.org/0000-0002-1304-8447}
\end{sphinxShadowBox}
\begin{quote}

\index{ORG\_0000069@\spxentry{ORG\_0000069}!organizational employee role@\spxentry{organizational employee role}}\index{organizational employee role@\spxentry{organizational employee role}!ORG\_0000069@\spxentry{ORG\_0000069}}\ignorespaces \end{quote}


\subsection{ORG\_0000069 \sphinxhyphen{} organizational employee role}
\label{\detokenize{doc-ORG_0000069:org-0000069-organizational-employee-role}}\label{\detokenize{doc-ORG_0000069:index-0}}\label{\detokenize{doc-ORG_0000069::doc}}
\begin{sphinxShadowBox}
\sphinxstyletopictitle{Label}

\sphinxAtStartPar
organizational employee role
\end{sphinxShadowBox}

\begin{sphinxShadowBox}
\sphinxstyletopictitle{Definition}

\sphinxAtStartPar
The role of a person to be an employee of an organization
\end{sphinxShadowBox}

\begin{sphinxShadowBox}
\sphinxstyletopictitle{Definition source}

\sphinxAtStartPar
Michael Conlon \sphinxurl{https://orcid.org/0000-0002-1304-8447}
\end{sphinxShadowBox}

\begin{sphinxShadowBox}
\sphinxstyletopictitle{Example}

\sphinxAtStartPar
Walt Disney was an employee of the United States Post Office
\end{sphinxShadowBox}

\begin{sphinxShadowBox}
\sphinxstyletopictitle{Term editor}

\sphinxAtStartPar
Michael Conlon \sphinxurl{https://orcid.org/0000-0002-1304-8447}
\end{sphinxShadowBox}
\begin{quote}

\index{ORG\_0000070@\spxentry{ORG\_0000070}!organizational employor role@\spxentry{organizational employor role}}\index{organizational employor role@\spxentry{organizational employor role}!ORG\_0000070@\spxentry{ORG\_0000070}}\ignorespaces \end{quote}


\subsection{ORG\_0000070 \sphinxhyphen{} organizational employor role}
\label{\detokenize{doc-ORG_0000070:org-0000070-organizational-employor-role}}\label{\detokenize{doc-ORG_0000070:index-0}}\label{\detokenize{doc-ORG_0000070::doc}}
\begin{sphinxShadowBox}
\sphinxstyletopictitle{Label}

\sphinxAtStartPar
organizational employor role
\end{sphinxShadowBox}

\begin{sphinxShadowBox}
\sphinxstyletopictitle{Definition}

\sphinxAtStartPar
The role of an organization to employ a person
\end{sphinxShadowBox}

\begin{sphinxShadowBox}
\sphinxstyletopictitle{Definition source}

\sphinxAtStartPar
Michael Conlon \sphinxurl{https://orcid.org/0000-0002-1304-8447}
\end{sphinxShadowBox}

\begin{sphinxShadowBox}
\sphinxstyletopictitle{Example}

\sphinxAtStartPar
McDonalds once employed Jeff Bezos
\end{sphinxShadowBox}

\begin{sphinxShadowBox}
\sphinxstyletopictitle{Term editor}

\sphinxAtStartPar
Michael Conlon \sphinxurl{https://orcid.org/0000-0002-1304-8447}
\end{sphinxShadowBox}
\begin{quote}

\index{ORG\_0000071@\spxentry{ORG\_0000071}!organizational associate role@\spxentry{organizational associate role}}\index{organizational associate role@\spxentry{organizational associate role}!ORG\_0000071@\spxentry{ORG\_0000071}}\ignorespaces \end{quote}


\subsection{ORG\_0000071 \sphinxhyphen{} organizational associate role}
\label{\detokenize{doc-ORG_0000071:org-0000071-organizational-associate-role}}\label{\detokenize{doc-ORG_0000071:index-0}}\label{\detokenize{doc-ORG_0000071::doc}}
\begin{sphinxShadowBox}
\sphinxstyletopictitle{Label}

\sphinxAtStartPar
organizational associate role
\end{sphinxShadowBox}

\begin{sphinxShadowBox}
\sphinxstyletopictitle{Definition}

\sphinxAtStartPar
The role of a person to be an associate of an organization
\end{sphinxShadowBox}

\begin{sphinxShadowBox}
\sphinxstyletopictitle{Definition source}

\sphinxAtStartPar
Michael Conlon \sphinxurl{https://orcid.org/0000-0002-1304-8447}
\end{sphinxShadowBox}

\begin{sphinxShadowBox}
\sphinxstyletopictitle{Example}

\sphinxAtStartPar
Bill Gates is an associate of Microsoft, even though he is no longer employed there
\end{sphinxShadowBox}

\begin{sphinxShadowBox}
\sphinxstyletopictitle{Term editor}

\sphinxAtStartPar
Michael Conlon \sphinxurl{https://orcid.org/0000-0002-1304-8447}
\end{sphinxShadowBox}
\begin{quote}

\index{ORG\_0000072@\spxentry{ORG\_0000072}!organizational associate grantor role@\spxentry{organizational associate grantor role}}\index{organizational associate grantor role@\spxentry{organizational associate grantor role}!ORG\_0000072@\spxentry{ORG\_0000072}}\ignorespaces \end{quote}


\subsection{ORG\_0000072 \sphinxhyphen{} organizational associate grantor role}
\label{\detokenize{doc-ORG_0000072:org-0000072-organizational-associate-grantor-role}}\label{\detokenize{doc-ORG_0000072:index-0}}\label{\detokenize{doc-ORG_0000072::doc}}
\begin{sphinxShadowBox}
\sphinxstyletopictitle{Label}

\sphinxAtStartPar
organizational associate grantor role
\end{sphinxShadowBox}

\begin{sphinxShadowBox}
\sphinxstyletopictitle{Definition}

\sphinxAtStartPar
The role of an organization to grant associate status to a person
\end{sphinxShadowBox}

\begin{sphinxShadowBox}
\sphinxstyletopictitle{Definition source}

\sphinxAtStartPar
Michael Conlon \sphinxurl{https://orcid.org/0000-0002-1304-8447}
\end{sphinxShadowBox}

\begin{sphinxShadowBox}
\sphinxstyletopictitle{Example}

\sphinxAtStartPar
A university may grant various forms of associate status to people who are not formally associated with the university
\end{sphinxShadowBox}

\begin{sphinxShadowBox}
\sphinxstyletopictitle{Editor’s note}

\sphinxAtStartPar
This is a role of an organization, not a person within the organization
\end{sphinxShadowBox}

\begin{sphinxShadowBox}
\sphinxstyletopictitle{Term editor}

\sphinxAtStartPar
Michael Conlon \sphinxurl{https://orcid.org/0000-0002-1304-8447}
\end{sphinxShadowBox}
\begin{quote}

\index{ORG\_0000073@\spxentry{ORG\_0000073}!organizatonal head role@\spxentry{organizatonal head role}}\index{organizatonal head role@\spxentry{organizatonal head role}!ORG\_0000073@\spxentry{ORG\_0000073}}\ignorespaces \end{quote}


\subsection{ORG\_0000073 \sphinxhyphen{} organizatonal head role}
\label{\detokenize{doc-ORG_0000073:org-0000073-organizatonal-head-role}}\label{\detokenize{doc-ORG_0000073:index-0}}\label{\detokenize{doc-ORG_0000073::doc}}
\begin{sphinxShadowBox}
\sphinxstyletopictitle{Label}

\sphinxAtStartPar
organizatonal head role
\end{sphinxShadowBox}

\begin{sphinxShadowBox}
\sphinxstyletopictitle{Definition}

\sphinxAtStartPar
The role of a person to be the head of an organization
\end{sphinxShadowBox}

\begin{sphinxShadowBox}
\sphinxstyletopictitle{Definition source}

\sphinxAtStartPar
Michael Conlon \sphinxurl{https://orcid.org/0000-0002-1304-8447}
\end{sphinxShadowBox}

\begin{sphinxShadowBox}
\sphinxstyletopictitle{Example}

\sphinxAtStartPar
Joey Wat is the head of Yum China
\end{sphinxShadowBox}

\begin{sphinxShadowBox}
\sphinxstyletopictitle{Term editor}

\sphinxAtStartPar
Michael Conlon \sphinxurl{https://orcid.org/0000-0002-1304-8447}
\end{sphinxShadowBox}
\begin{quote}

\index{ORG\_0000074@\spxentry{ORG\_0000074}!organizational head grantor role@\spxentry{organizational head grantor role}}\index{organizational head grantor role@\spxentry{organizational head grantor role}!ORG\_0000074@\spxentry{ORG\_0000074}}\ignorespaces \end{quote}


\subsection{ORG\_0000074 \sphinxhyphen{} organizational head grantor role}
\label{\detokenize{doc-ORG_0000074:org-0000074-organizational-head-grantor-role}}\label{\detokenize{doc-ORG_0000074:index-0}}\label{\detokenize{doc-ORG_0000074::doc}}
\begin{sphinxShadowBox}
\sphinxstyletopictitle{Label}

\sphinxAtStartPar
organizational head grantor role
\end{sphinxShadowBox}

\begin{sphinxShadowBox}
\sphinxstyletopictitle{Definition}

\sphinxAtStartPar
The role of an organization to grant head status to a person
\end{sphinxShadowBox}

\begin{sphinxShadowBox}
\sphinxstyletopictitle{Definition source}

\sphinxAtStartPar
Michael Conlon \sphinxurl{https://orcid.org/0000-0002-1304-8447}
\end{sphinxShadowBox}

\begin{sphinxShadowBox}
\sphinxstyletopictitle{Example}

\sphinxAtStartPar
The board of trustees of a company often grants the head of role
\end{sphinxShadowBox}

\begin{sphinxShadowBox}
\sphinxstyletopictitle{Editor’s note}

\sphinxAtStartPar
This is a role of an organization, not a person within the organization
\end{sphinxShadowBox}

\begin{sphinxShadowBox}
\sphinxstyletopictitle{Term editor}

\sphinxAtStartPar
Michael Conlon \sphinxurl{https://orcid.org/0000-0002-1304-8447}
\end{sphinxShadowBox}
\begin{quote}

\index{ORG\_0000075@\spxentry{ORG\_0000075}!organizational appointee role@\spxentry{organizational appointee role}}\index{organizational appointee role@\spxentry{organizational appointee role}!ORG\_0000075@\spxentry{ORG\_0000075}}\ignorespaces \end{quote}


\subsection{ORG\_0000075 \sphinxhyphen{} organizational appointee role}
\label{\detokenize{doc-ORG_0000075:org-0000075-organizational-appointee-role}}\label{\detokenize{doc-ORG_0000075:index-0}}\label{\detokenize{doc-ORG_0000075::doc}}
\begin{sphinxShadowBox}
\sphinxstyletopictitle{Label}

\sphinxAtStartPar
organizational appointee role
\end{sphinxShadowBox}

\begin{sphinxShadowBox}
\sphinxstyletopictitle{Definition}

\sphinxAtStartPar
The role of a person to be an appointee of an organization
\end{sphinxShadowBox}

\begin{sphinxShadowBox}
\sphinxstyletopictitle{Definition source}

\sphinxAtStartPar
Michael Conlon \sphinxurl{https://orcid.org/0000-0002-1304-8447}
\end{sphinxShadowBox}

\begin{sphinxShadowBox}
\sphinxstyletopictitle{Example}

\sphinxAtStartPar
Jorge Mario Bergoglio was appointed to the position Pope by the papal conclave of the College of Cardinals of the Catholic Church in 2013
\end{sphinxShadowBox}

\begin{sphinxShadowBox}
\sphinxstyletopictitle{Term editor}

\sphinxAtStartPar
Michael Conlon \sphinxurl{https://orcid.org/0000-0002-1304-8447}
\end{sphinxShadowBox}
\begin{quote}

\index{ORG\_0000076@\spxentry{ORG\_0000076}!organizational appointee grantor role@\spxentry{organizational appointee grantor role}}\index{organizational appointee grantor role@\spxentry{organizational appointee grantor role}!ORG\_0000076@\spxentry{ORG\_0000076}}\ignorespaces \end{quote}


\subsection{ORG\_0000076 \sphinxhyphen{} organizational appointee grantor role}
\label{\detokenize{doc-ORG_0000076:org-0000076-organizational-appointee-grantor-role}}\label{\detokenize{doc-ORG_0000076:index-0}}\label{\detokenize{doc-ORG_0000076::doc}}
\begin{sphinxShadowBox}
\sphinxstyletopictitle{Label}

\sphinxAtStartPar
organizational appointee grantor role
\end{sphinxShadowBox}

\begin{sphinxShadowBox}
\sphinxstyletopictitle{Definition}

\sphinxAtStartPar
The role of an organization to grant appointee status to a person
\end{sphinxShadowBox}

\begin{sphinxShadowBox}
\sphinxstyletopictitle{Definition source}

\sphinxAtStartPar
Michael Conlon \sphinxurl{https://orcid.org/0000-0002-1304-8447}
\end{sphinxShadowBox}

\begin{sphinxShadowBox}
\sphinxstyletopictitle{Example}

\sphinxAtStartPar
The members of a chess club may elect one of their members to serve as secretary
\end{sphinxShadowBox}

\begin{sphinxShadowBox}
\sphinxstyletopictitle{Editor’s note}

\sphinxAtStartPar
This is a role of an organization, not a person within the organization
\end{sphinxShadowBox}

\begin{sphinxShadowBox}
\sphinxstyletopictitle{Term editor}

\sphinxAtStartPar
Michael Conlon \sphinxurl{https://orcid.org/0000-0002-1304-8447}
\end{sphinxShadowBox}
\begin{quote}

\index{ORG\_0000077@\spxentry{ORG\_0000077}!organizational volunteer role@\spxentry{organizational volunteer role}}\index{organizational volunteer role@\spxentry{organizational volunteer role}!ORG\_0000077@\spxentry{ORG\_0000077}}\ignorespaces \end{quote}


\subsection{ORG\_0000077 \sphinxhyphen{} organizational volunteer role}
\label{\detokenize{doc-ORG_0000077:org-0000077-organizational-volunteer-role}}\label{\detokenize{doc-ORG_0000077:index-0}}\label{\detokenize{doc-ORG_0000077::doc}}
\begin{sphinxShadowBox}
\sphinxstyletopictitle{Label}

\sphinxAtStartPar
organizational volunteer role
\end{sphinxShadowBox}

\begin{sphinxShadowBox}
\sphinxstyletopictitle{Definition}

\sphinxAtStartPar
The role of a person to be a volunteer of an organization
\end{sphinxShadowBox}

\begin{sphinxShadowBox}
\sphinxstyletopictitle{Definition source}

\sphinxAtStartPar
Michael Conlon \sphinxurl{https://orcid.org/0000-0002-1304-8447}
\end{sphinxShadowBox}

\begin{sphinxShadowBox}
\sphinxstyletopictitle{Example}

\sphinxAtStartPar
Lillian Carter, the mother of US president Jimmy Carter, served as a peace corps volunteer in India
\end{sphinxShadowBox}

\begin{sphinxShadowBox}
\sphinxstyletopictitle{Term editor}

\sphinxAtStartPar
Michael Conlon \sphinxurl{https://orcid.org/0000-0002-1304-8447}
\end{sphinxShadowBox}
\begin{quote}

\index{ORG\_0000078@\spxentry{ORG\_0000078}!organizational volunteer grantor role@\spxentry{organizational volunteer grantor role}}\index{organizational volunteer grantor role@\spxentry{organizational volunteer grantor role}!ORG\_0000078@\spxentry{ORG\_0000078}}\ignorespaces \end{quote}


\subsection{ORG\_0000078 \sphinxhyphen{} organizational volunteer grantor role}
\label{\detokenize{doc-ORG_0000078:org-0000078-organizational-volunteer-grantor-role}}\label{\detokenize{doc-ORG_0000078:index-0}}\label{\detokenize{doc-ORG_0000078::doc}}
\begin{sphinxShadowBox}
\sphinxstyletopictitle{Label}

\sphinxAtStartPar
organizational volunteer grantor role
\end{sphinxShadowBox}

\begin{sphinxShadowBox}
\sphinxstyletopictitle{Definition}

\sphinxAtStartPar
The role of an organization to grant volunteer status to a person
\end{sphinxShadowBox}

\begin{sphinxShadowBox}
\sphinxstyletopictitle{Definition source}

\sphinxAtStartPar
Michael Conlon \sphinxurl{https://orcid.org/0000-0002-1304-8447}
\end{sphinxShadowBox}

\begin{sphinxShadowBox}
\sphinxstyletopictitle{Example}

\sphinxAtStartPar
A hospital has a volunteer grantor role with respect to those persons who volunteer at the hospital
\end{sphinxShadowBox}

\begin{sphinxShadowBox}
\sphinxstyletopictitle{Editor’s note}

\sphinxAtStartPar
This is a role of an organization, not a person within the organization
\end{sphinxShadowBox}

\begin{sphinxShadowBox}
\sphinxstyletopictitle{Term editor}

\sphinxAtStartPar
Michael Conlon \sphinxurl{https://orcid.org/0000-0002-1304-8447}
\end{sphinxShadowBox}
\begin{quote}

\index{ORG\_0000079@\spxentry{ORG\_0000079}!airline disposition@\spxentry{airline disposition}}\index{airline disposition@\spxentry{airline disposition}!ORG\_0000079@\spxentry{ORG\_0000079}}\ignorespaces \end{quote}


\subsection{ORG\_0000079 \sphinxhyphen{} airline disposition}
\label{\detokenize{doc-ORG_0000079:org-0000079-airline-disposition}}\label{\detokenize{doc-ORG_0000079:index-0}}\label{\detokenize{doc-ORG_0000079::doc}}
\begin{sphinxShadowBox}
\sphinxstyletopictitle{Label}

\sphinxAtStartPar
airline disposition
\end{sphinxShadowBox}

\begin{sphinxShadowBox}
\sphinxstyletopictitle{Definition}

\sphinxAtStartPar
The disposition of an organization that operates airplanes carrying frieght or passengers
\end{sphinxShadowBox}

\begin{sphinxShadowBox}
\sphinxstyletopictitle{Definition source}

\sphinxAtStartPar
\sphinxurl{https://www.ahdictionary.com/word/search.html?q=airline}
\end{sphinxShadowBox}

\begin{sphinxShadowBox}
\sphinxstyletopictitle{Example}

\sphinxAtStartPar
Delta, Lufthanza, and Aeroflot all have airline disposition
\end{sphinxShadowBox}

\begin{sphinxShadowBox}
\sphinxstyletopictitle{Editor’s note}

\sphinxAtStartPar
Added for completeness with schema.org
\end{sphinxShadowBox}

\begin{sphinxShadowBox}
\sphinxstyletopictitle{Term editor}

\sphinxAtStartPar
Michael Conlon \sphinxurl{https://orcid.org/0000-0002-1304-8447}
\end{sphinxShadowBox}
\begin{quote}

\index{ORG\_0000080@\spxentry{ORG\_0000080}!media disposition@\spxentry{media disposition}}\index{media disposition@\spxentry{media disposition}!ORG\_0000080@\spxentry{ORG\_0000080}}\ignorespaces \end{quote}


\subsection{ORG\_0000080 \sphinxhyphen{} media disposition}
\label{\detokenize{doc-ORG_0000080:org-0000080-media-disposition}}\label{\detokenize{doc-ORG_0000080:index-0}}\label{\detokenize{doc-ORG_0000080::doc}}
\begin{sphinxShadowBox}
\sphinxstyletopictitle{Label}

\sphinxAtStartPar
media disposition
\end{sphinxShadowBox}

\begin{sphinxShadowBox}
\sphinxstyletopictitle{Definition}

\sphinxAtStartPar
The disposition of an organization that creates, transmits, and/or licenses live or recorded material for viewing by others
\end{sphinxShadowBox}

\begin{sphinxShadowBox}
\sphinxstyletopictitle{Definition source}

\sphinxAtStartPar
\sphinxurl{https://www.macmillandictionary.com/dictionary/british/media-organization}
\end{sphinxShadowBox}

\begin{sphinxShadowBox}
\sphinxstyletopictitle{Example}

\sphinxAtStartPar
NBC, BBC, CNN, EFE News Agency and the Japan Times all have media dispositions
\end{sphinxShadowBox}

\begin{sphinxShadowBox}
\sphinxstyletopictitle{Editor’s note}

\sphinxAtStartPar
Added for completeness with schema.org
\end{sphinxShadowBox}

\begin{sphinxShadowBox}
\sphinxstyletopictitle{Term editor}

\sphinxAtStartPar
Michael Conlon \sphinxurl{https://orcid.org/0000-0002-1304-8447}
\end{sphinxShadowBox}
\begin{quote}

\index{ORG\_0000081@\spxentry{ORG\_0000081}!performing disposition@\spxentry{performing disposition}}\index{performing disposition@\spxentry{performing disposition}!ORG\_0000081@\spxentry{ORG\_0000081}}\ignorespaces \end{quote}


\subsection{ORG\_0000081 \sphinxhyphen{} performing disposition}
\label{\detokenize{doc-ORG_0000081:org-0000081-performing-disposition}}\label{\detokenize{doc-ORG_0000081:index-0}}\label{\detokenize{doc-ORG_0000081::doc}}
\begin{sphinxShadowBox}
\sphinxstyletopictitle{Label}

\sphinxAtStartPar
performing disposition
\end{sphinxShadowBox}

\begin{sphinxShadowBox}
\sphinxstyletopictitle{Definition}

\sphinxAtStartPar
The disposition of an organization to perform live or recorded music, theatre, or dance
\end{sphinxShadowBox}

\begin{sphinxShadowBox}
\sphinxstyletopictitle{Definition source}

\sphinxAtStartPar
\sphinxurl{https://en.wikipedia.org/wiki/Performing\_arts}
\end{sphinxShadowBox}

\begin{sphinxShadowBox}
\sphinxstyletopictitle{Example}

\sphinxAtStartPar
The Bolshoi Ballet, the Royal Shakespeare Company, the local community theater, the Metropolitan Opera are all organizations with performing disposition
\end{sphinxShadowBox}

\begin{sphinxShadowBox}
\sphinxstyletopictitle{Editor’s note}

\sphinxAtStartPar
Added for completeness with schema.org
\end{sphinxShadowBox}

\begin{sphinxShadowBox}
\sphinxstyletopictitle{Term editor}

\sphinxAtStartPar
Michael Conlon \sphinxurl{https://orcid.org/0000-0002-1304-8447}
\end{sphinxShadowBox}
\begin{quote}

\index{ORG\_0000082@\spxentry{ORG\_0000082}!labor union disposition@\spxentry{labor union disposition}}\index{labor union disposition@\spxentry{labor union disposition}!ORG\_0000082@\spxentry{ORG\_0000082}}\ignorespaces \end{quote}


\subsection{ORG\_0000082 \sphinxhyphen{} labor union disposition}
\label{\detokenize{doc-ORG_0000082:org-0000082-labor-union-disposition}}\label{\detokenize{doc-ORG_0000082:index-0}}\label{\detokenize{doc-ORG_0000082::doc}}
\begin{sphinxShadowBox}
\sphinxstyletopictitle{Label}

\sphinxAtStartPar
labor union disposition
\end{sphinxShadowBox}

\begin{sphinxShadowBox}
\sphinxstyletopictitle{Alternate name}

\sphinxAtStartPar
workers union

\sphinxAtStartPar
trade union
\end{sphinxShadowBox}

\begin{sphinxShadowBox}
\sphinxstyletopictitle{Definition}

\sphinxAtStartPar
The disposition of an organization to organize workers for the purpose of negotiations with employers of the workers
\end{sphinxShadowBox}

\begin{sphinxShadowBox}
\sphinxstyletopictitle{Definition source}

\sphinxAtStartPar
\sphinxurl{https://duckduckgo.com/?t=ffab\&q=labor+union\&ia=web}
\end{sphinxShadowBox}

\begin{sphinxShadowBox}
\sphinxstyletopictitle{Example}

\sphinxAtStartPar
All\sphinxhyphen{}China Federation of Trade Unions (ACFTU), the Congress of South African Trade Unions (COSATU) and the General Confederation of Labor (CGT) in France are all organizations with disposition of labor union.
\end{sphinxShadowBox}

\begin{sphinxShadowBox}
\sphinxstyletopictitle{Editor’s note}

\sphinxAtStartPar
Added for completeness with schema.org
\end{sphinxShadowBox}

\begin{sphinxShadowBox}
\sphinxstyletopictitle{Term editor}

\sphinxAtStartPar
Michael Conlon \sphinxurl{https://orcid.org/0000-0002-1304-8447}
\end{sphinxShadowBox}
\begin{quote}

\index{ORG\_0000083@\spxentry{ORG\_0000083}!person membership@\spxentry{person membership}}\index{person membership@\spxentry{person membership}!ORG\_0000083@\spxentry{ORG\_0000083}}\ignorespaces \end{quote}


\subsection{ORG\_0000083 \sphinxhyphen{} person membership}
\label{\detokenize{doc-ORG_0000083:org-0000083-person-membership}}\label{\detokenize{doc-ORG_0000083:index-0}}\label{\detokenize{doc-ORG_0000083::doc}}
\begin{sphinxShadowBox}
\sphinxstyletopictitle{Label}

\sphinxAtStartPar
person membership
\end{sphinxShadowBox}

\begin{sphinxShadowBox}
\sphinxstyletopictitle{Definition}

\sphinxAtStartPar
The asymmetric relationship representing a person’s membership in an organization
\end{sphinxShadowBox}

\begin{sphinxShadowBox}
\sphinxstyletopictitle{Definition source}

\sphinxAtStartPar
Michael Conlon \sphinxurl{https://orcid.org/0000-0002-1304-8447}
\end{sphinxShadowBox}

\begin{sphinxShadowBox}
\sphinxstyletopictitle{Term editor}

\sphinxAtStartPar
Michael Conlon \sphinxurl{https://orcid.org/0000-0002-1304-8447}
\end{sphinxShadowBox}
\begin{quote}

\index{ORG\_0000084@\spxentry{ORG\_0000084}!person member role@\spxentry{person member role}}\index{person member role@\spxentry{person member role}!ORG\_0000084@\spxentry{ORG\_0000084}}\ignorespaces \end{quote}


\subsection{ORG\_0000084 \sphinxhyphen{} person member role}
\label{\detokenize{doc-ORG_0000084:org-0000084-person-member-role}}\label{\detokenize{doc-ORG_0000084:index-0}}\label{\detokenize{doc-ORG_0000084::doc}}
\begin{sphinxShadowBox}
\sphinxstyletopictitle{Label}

\sphinxAtStartPar
person member role
\end{sphinxShadowBox}

\begin{sphinxShadowBox}
\sphinxstyletopictitle{Definition}

\sphinxAtStartPar
The role of a person to be a member of an organization
\end{sphinxShadowBox}

\begin{sphinxShadowBox}
\sphinxstyletopictitle{Definition source}

\sphinxAtStartPar
Michael Conlon \sphinxurl{https://orcid.org/0000-0002-1304-8447}
\end{sphinxShadowBox}

\begin{sphinxShadowBox}
\sphinxstyletopictitle{Example}

\sphinxAtStartPar
George Harrison had a person member role with respect to The Beatles
\end{sphinxShadowBox}

\begin{sphinxShadowBox}
\sphinxstyletopictitle{Term editor}

\sphinxAtStartPar
Michael Conlon \sphinxurl{https://orcid.org/0000-0002-1304-8447}
\end{sphinxShadowBox}
\begin{quote}

\index{ORG\_0000085@\spxentry{ORG\_0000085}!person member grantor role@\spxentry{person member grantor role}}\index{person member grantor role@\spxentry{person member grantor role}!ORG\_0000085@\spxentry{ORG\_0000085}}\ignorespaces \end{quote}


\subsection{ORG\_0000085 \sphinxhyphen{} person member grantor role}
\label{\detokenize{doc-ORG_0000085:org-0000085-person-member-grantor-role}}\label{\detokenize{doc-ORG_0000085:index-0}}\label{\detokenize{doc-ORG_0000085::doc}}
\begin{sphinxShadowBox}
\sphinxstyletopictitle{Label}

\sphinxAtStartPar
person member grantor role
\end{sphinxShadowBox}

\begin{sphinxShadowBox}
\sphinxstyletopictitle{Definition}

\sphinxAtStartPar
The role of an organization to grant membership to a person
\end{sphinxShadowBox}

\begin{sphinxShadowBox}
\sphinxstyletopictitle{Definition source}

\sphinxAtStartPar
Michael Conlon \sphinxurl{https://orcid.org/0000-0002-1304-8447}
\end{sphinxShadowBox}

\begin{sphinxShadowBox}
\sphinxstyletopictitle{Example}

\sphinxAtStartPar
Professional socieities have a grantor role with respect to their members
\end{sphinxShadowBox}

\begin{sphinxShadowBox}
\sphinxstyletopictitle{Term editor}

\sphinxAtStartPar
Michael Conlon \sphinxurl{https://orcid.org/0000-0002-1304-8447}
\end{sphinxShadowBox}
\begin{quote}

\index{Concept@\spxentry{Concept}!concept (skos)@\spxentry{concept}\spxextra{skos}}\index{concept (skos)@\spxentry{concept}\spxextra{skos}!Concept@\spxentry{Concept}}\ignorespaces \end{quote}


\subsection{Concept \sphinxhyphen{} concept (skos)}
\label{\detokenize{doc-Concept:concept-concept-skos}}\label{\detokenize{doc-Concept:index-0}}\label{\detokenize{doc-Concept::doc}}
\begin{sphinxShadowBox}
\sphinxstyletopictitle{Label}

\sphinxAtStartPar
concept (skos)
\end{sphinxShadowBox}

\begin{sphinxShadowBox}
\sphinxstyletopictitle{Definition}

\sphinxAtStartPar
An idea or notion, a unit of thought
\end{sphinxShadowBox}

\begin{sphinxShadowBox}
\sphinxstyletopictitle{Editor’s note}

\sphinxAtStartPar
Minor hijack here.  We added the subClassOf assertion to tie in to the BFO subsumption hierarchy rather than adding a new term for ‘concept’
Concept as a generic dependent continuant seems right \textendash{} the concept is dependent on the collective of humans.  Thought is in human brains.
We also edited the label addding (skos) to insure unique labels as required by the OBO Principles. MC.
\end{sphinxShadowBox}
\begin{quote}

\index{Instant@\spxentry{Instant}!time instant@\spxentry{time instant}}\index{time instant@\spxentry{time instant}!Instant@\spxentry{Instant}}\ignorespaces \end{quote}


\subsection{Instant \sphinxhyphen{} time instant}
\label{\detokenize{doc-Instant:instant-time-instant}}\label{\detokenize{doc-Instant:index-0}}\label{\detokenize{doc-Instant::doc}}
\begin{sphinxShadowBox}
\sphinxstyletopictitle{Label}

\sphinxAtStartPar
time instant
\end{sphinxShadowBox}

\begin{sphinxShadowBox}
\sphinxstyletopictitle{Definition}

\sphinxAtStartPar
A zero\sphinxhyphen{}dimensional part of time.  Precision may specify a range.  Represented by xsd datetime string
\end{sphinxShadowBox}

\begin{sphinxShadowBox}
\sphinxstyletopictitle{Imported From}

\sphinxAtStartPar
\sphinxurl{http://www.w3.org/2006/time\#2016}
\end{sphinxShadowBox}
\begin{quote}

\index{TemporalUnit@\spxentry{TemporalUnit}!temporal unit@\spxentry{temporal unit}}\index{temporal unit@\spxentry{temporal unit}!TemporalUnit@\spxentry{TemporalUnit}}\ignorespaces \end{quote}


\subsection{TemporalUnit \sphinxhyphen{} temporal unit}
\label{\detokenize{doc-TemporalUnit:temporalunit-temporal-unit}}\label{\detokenize{doc-TemporalUnit:index-0}}\label{\detokenize{doc-TemporalUnit::doc}}
\begin{sphinxShadowBox}
\sphinxstyletopictitle{Label}

\sphinxAtStartPar
temporal unit
\end{sphinxShadowBox}

\begin{sphinxShadowBox}
\sphinxstyletopictitle{Definition}

\sphinxAtStartPar
A specification of a time duration.  Used to specify precision of time instants
\end{sphinxShadowBox}

\begin{sphinxShadowBox}
\sphinxstyletopictitle{Imported From}

\sphinxAtStartPar
\sphinxurl{http://www.w3.org/2006/time\#2016}
\end{sphinxShadowBox}


\chapter{Object Properties}
\label{\detokenize{object-properties:object-properties}}\label{\detokenize{object-properties::doc}}
\sphinxAtStartPar
Object properties relate one entity to another (not one class to another) \sphinxcite{object-properties:harmse2018}.
For example, an organization may be
“part of” another organization.  “part of” is an object property that describes the
relations between the two individual organizations.

\sphinxAtStartPar
Basic Formal Ontology (BFO) uses the
\sphinxhref{http://www.ontobee.org/ontology/RO}{Relation Ontology (RO)} to define object
properties.

\sphinxAtStartPar
Each object property can have a domain and a range.  When we say property p has domain D,
we mean that all triples of the form x P y, x is a D.  When we say property p has range R,
we mean that in all triples of the form x P y, y is an R.

\sphinxAtStartPar
For example, if we define an object property “author\_of”, we might define the domain to
be “Person” and the range to be “Information Content Entity”. If we write x author\_of y,
we know x is a Person and y is an Information Content Entity. %
\begin{footnote}[1]\sphinxAtStartFootnote
Are these the correct domain and range for such a property? Discuss.
%
\end{footnote}


\section{Common Object Properties}
\label{\detokenize{object-properties:common-object-properties}}
\sphinxAtStartPar
Some object properties are quite common in the representation of scholarship.  Many
representations involve the use of identifiers.  People, publications, organizations
and other entities may be “denoted\_by” an identifier.  We assert, for example,:

\begin{sphinxVerbatim}[commandchars=\\\{\}]
\PYG{n}{x} \PYG{n}{a} \PYG{n}{Person}
\PYG{n}{y} \PYG{n}{a} \PYG{n}{ORCID}
\PYG{n}{x} \PYG{n}{denoted\PYGZus{}by} \PYG{n}{y}
\end{sphinxVerbatim}

\sphinxAtStartPar
“denoted\_by” has an \sphinxstyleemphasis{inverse property} “denotes.”  If x is denoted\_by y, then y denotes x.
We could write the above as:

\begin{sphinxVerbatim}[commandchars=\\\{\}]
\PYG{n}{x} \PYG{n}{a} \PYG{n}{Person}
\PYG{n}{y} \PYG{n}{a} \PYG{n}{ORCID}
\PYG{n}{y} \PYG{n}{denotes} \PYG{n}{x}
\end{sphinxVerbatim}

\sphinxAtStartPar
See {\hyperref[\detokenize{locations:table-14}]{\sphinxcrossref{\DUrole{std,std-ref}{Table 14 Terms used to represent locations}}}}. The pattern \sphinxstyleemphasis{entity1 bearer\_of role; role realized\_in process;
process
has\_output entity2} is quite common and describes the role entity1 had through a process
in
the creation of entity2. Each of these properties has an inverse, so we could assert
equivalently, \sphinxstyleemphasis{entity2 output\_of process; process realizes role; role inheres in entity1}.


\begin{savenotes}\sphinxattablestart
\centering
\sphinxcapstartof{table}
\sphinxthecaptionisattop
\sphinxcaption{Table 15 Common Object Properties}\label{\detokenize{object-properties:id4}}\label{\detokenize{object-properties:table-15}}
\sphinxaftertopcaption
\begin{tabulary}{\linewidth}[t]{|T|T|T|}
\hline
\sphinxstyletheadfamily 
\sphinxAtStartPar
Property
&\sphinxstyletheadfamily 
\sphinxAtStartPar
Label
&\sphinxstyletheadfamily 
\sphinxAtStartPar
Notes
\\
\hline
\sphinxAtStartPar
\sphinxcode{\sphinxupquote{BFO\_0000050}}
&
\sphinxAtStartPar
part of
&
\sphinxAtStartPar
An entity is part of another entity
\\
\hline
\sphinxAtStartPar
\sphinxcode{\sphinxupquote{BFO\_0000051}}
&
\sphinxAtStartPar
has part
&
\sphinxAtStartPar
Inverse of part of
\\
\hline
\sphinxAtStartPar
\sphinxcode{\sphinxupquote{IAO\_0000219}}
&
\sphinxAtStartPar
denotes
&
\sphinxAtStartPar
The relation between an identifier and entity
\\
\hline
\sphinxAtStartPar
\sphinxcode{\sphinxupquote{IAO\_0000235}}
&
\sphinxAtStartPar
denoted by
&
\sphinxAtStartPar
Inverse of denotes
\\
\hline
\sphinxAtStartPar
\sphinxcode{\sphinxupquote{RO\_0000053}}
&
\sphinxAtStartPar
bearer of
&
\sphinxAtStartPar
relation between a dependent and its bearer
\\
\hline
\sphinxAtStartPar
\sphinxcode{\sphinxupquote{RO\_0000052}}
&
\sphinxAtStartPar
inheres in
&
\sphinxAtStartPar
The inverse of bearer of
\\
\hline
\sphinxAtStartPar
\sphinxcode{\sphinxupquote{BFO\_0000055}}
&
\sphinxAtStartPar
realizes
&
\sphinxAtStartPar
A process realizes a role
\\
\hline
\sphinxAtStartPar
\sphinxcode{\sphinxupquote{BFO\_0000054}}
&
\sphinxAtStartPar
realized in
&
\sphinxAtStartPar
A role is realized in a process
\\
\hline
\sphinxAtStartPar
\sphinxcode{\sphinxupquote{RO\_0002234}}
&
\sphinxAtStartPar
has output
&
\sphinxAtStartPar
An occurrent has a continuant as an output
\\
\hline
\sphinxAtStartPar
\sphinxcode{\sphinxupquote{RO\_0002353}}
&
\sphinxAtStartPar
output of
&
\sphinxAtStartPar
A continuant is the output of an occurrent
\\
\hline
\sphinxAtStartPar
\sphinxcode{\sphinxupquote{RO\_0000086}}
&
\sphinxAtStartPar
has quality
&
\sphinxAtStartPar
An entity has another entity as a quality
\\
\hline
\sphinxAtStartPar
\sphinxcode{\sphinxupquote{RO\_0000080}}
&
\sphinxAtStartPar
quality of
&
\sphinxAtStartPar
An entity is a quality of another entity
\\
\hline
\end{tabulary}
\par
\sphinxattableend\end{savenotes}


\section{All Object Properties}
\label{\detokenize{object-properties:all-object-properties}}
\sphinxAtStartPar
See {\hyperref[\detokenize{object-properties:table-7}]{\sphinxcrossref{Table 7}}}.


\begin{savenotes}\sphinxatlongtablestart\begin{longtable}[c]{|*{2}{\X{1}{2}|}}
\sphinxthelongtablecaptionisattop
\caption{Table 7 Object Properties\strut}\label{\detokenize{object-properties:id5}}\label{\detokenize{object-properties:table-7}}\\*[\sphinxlongtablecapskipadjust]
\hline
\sphinxstyletheadfamily 
\sphinxAtStartPar
Term ID \sphinxhyphen{} Label
&\sphinxstyletheadfamily 
\sphinxAtStartPar
Definition
\\
\hline
\endfirsthead

\multicolumn{2}{c}%
{\makebox[0pt]{\sphinxtablecontinued{\tablename\ \thetable{} \textendash{} continued from previous page}}}\\
\hline
\sphinxstyletheadfamily 
\sphinxAtStartPar
Term ID \sphinxhyphen{} Label
&\sphinxstyletheadfamily 
\sphinxAtStartPar
Definition
\\
\hline
\endhead

\hline
\multicolumn{2}{r}{\makebox[0pt][r]{\sphinxtablecontinued{continues on next page}}}\\
\endfoot

\endlastfoot

\sphinxAtStartPar
{\hyperref[\detokenize{doc-BFO_0000050::doc}]{\sphinxcrossref{\DUrole{doc}{BFO\_0000050 \sphinxhyphen{} part of}}}}
&
\sphinxAtStartPar
A core relation that holds between a part and its

\sphinxAtStartPar
whole
\\
\hline
\sphinxAtStartPar
{\hyperref[\detokenize{doc-BFO_0000051::doc}]{\sphinxcrossref{\DUrole{doc}{BFO\_0000051 \sphinxhyphen{} has part}}}}
&
\sphinxAtStartPar
A core relation that holds between a whole and its

\sphinxAtStartPar
part
\\
\hline
\sphinxAtStartPar
{\hyperref[\detokenize{doc-BFO_0000054::doc}]{\sphinxcrossref{\DUrole{doc}{BFO\_0000054 \sphinxhyphen{} realized in}}}}
&
\sphinxAtStartPar
A relation between a realizable entity and a

\sphinxAtStartPar
process, where there is some material entity that

\sphinxAtStartPar
is bearer of the realizable entity and

\sphinxAtStartPar
participates in the process, and the realizable

\sphinxAtStartPar
entity comes to be realized in the course of the

\sphinxAtStartPar
process
\\
\hline
\sphinxAtStartPar
{\hyperref[\detokenize{doc-BFO_0000055::doc}]{\sphinxcrossref{\DUrole{doc}{BFO\_0000055 \sphinxhyphen{} realizes}}}}
&
\sphinxAtStartPar
A relation between a process and a realizable

\sphinxAtStartPar
entity, where there is some material entity that

\sphinxAtStartPar
is bearer of the realizable entity and

\sphinxAtStartPar
participates in the process, and the realizable

\sphinxAtStartPar
entity comes to be realized in the course of the

\sphinxAtStartPar
process
\\
\hline
\sphinxAtStartPar
{\hyperref[\detokenize{doc-IAO_0000136::doc}]{\sphinxcrossref{\DUrole{doc}{IAO\_0000136 \sphinxhyphen{} is about}}}}
&
\sphinxAtStartPar
A (currently) primitive relation that relates an

\sphinxAtStartPar
information artifact to an entity.
\\
\hline
\sphinxAtStartPar
{\hyperref[\detokenize{doc-IAO_0000219::doc}]{\sphinxcrossref{\DUrole{doc}{IAO\_0000219 \sphinxhyphen{} denotes}}}}
&
\sphinxAtStartPar
A primitive, instance\sphinxhyphen{}level, relation obtaining

\sphinxAtStartPar
between an information content entity and some

\sphinxAtStartPar
portion of reality. Denotation is what happens

\sphinxAtStartPar
when someone creates an information content entity

\sphinxAtStartPar
E in order to specifically refer to something. The

\sphinxAtStartPar
only relation between E and the thing is that E

\sphinxAtStartPar
can be used to ‘pick out’ the thing. This relation

\sphinxAtStartPar
connects those two together. Freedictionary.com

\sphinxAtStartPar
sense 3: To signify directly; refer to

\sphinxAtStartPar
specifically
\\
\hline
\sphinxAtStartPar
{\hyperref[\detokenize{doc-IAO_0000235::doc}]{\sphinxcrossref{\DUrole{doc}{IAO\_0000235 \sphinxhyphen{} denoted by}}}}
&
\sphinxAtStartPar
Inverse of the relation ‘denotes’
\\
\hline
\sphinxAtStartPar
{\hyperref[\detokenize{doc-ORG_2000001::doc}]{\sphinxcrossref{\DUrole{doc}{ORG\_2000001 \sphinxhyphen{} occupies}}}}
&
\sphinxAtStartPar
The property that associates an organization

\sphinxAtStartPar
occupies a geographical location is it has one of

\sphinxAtStartPar
more people at the location, or has legal rights

\sphinxAtStartPar
to the location
\\
\hline
\sphinxAtStartPar
{\hyperref[\detokenize{doc-ORG_2000002::doc}]{\sphinxcrossref{\DUrole{doc}{ORG\_2000002 \sphinxhyphen{} has occurent part}}}}
&
\sphinxAtStartPar
The property that associates an occurent which has

\sphinxAtStartPar
an occurent part wholly contained within it
\\
\hline
\sphinxAtStartPar
{\hyperref[\detokenize{doc-ORG_2000003::doc}]{\sphinxcrossref{\DUrole{doc}{ORG\_2000003 \sphinxhyphen{} has time instant}}}}
&
\sphinxAtStartPar
The property that associates a process boundary

\sphinxAtStartPar
with the time instant at which the process

\sphinxAtStartPar
boundary occurs
\\
\hline
\sphinxAtStartPar
{\hyperref[\detokenize{doc-ORG_2000004::doc}]{\sphinxcrossref{\DUrole{doc}{ORG\_2000004 \sphinxhyphen{} has organizational interest}}}}
&
\sphinxAtStartPar
The property that associates an organization with

\sphinxAtStartPar
a concept that the organization manifests in its

\sphinxAtStartPar
work
\\
\hline
\sphinxAtStartPar
{\hyperref[\detokenize{doc-ORG_2000005::doc}]{\sphinxcrossref{\DUrole{doc}{ORG\_2000005 \sphinxhyphen{} has website}}}}
&
\sphinxAtStartPar
The property that associates an organization with

\sphinxAtStartPar
its website
\\
\hline
\sphinxAtStartPar
{\hyperref[\detokenize{doc-ORG_2000006::doc}]{\sphinxcrossref{\DUrole{doc}{ORG\_2000006 \sphinxhyphen{} website of}}}}
&
\sphinxAtStartPar
The property that associates a website with its

\sphinxAtStartPar
organization
\\
\hline
\sphinxAtStartPar
{\hyperref[\detokenize{doc-ORG_2000007::doc}]{\sphinxcrossref{\DUrole{doc}{ORG\_2000007 \sphinxhyphen{} has successor organization}}}}
&
\sphinxAtStartPar
The property that associates an organization with

\sphinxAtStartPar
the organization that succeeds it.  The preceeding

\sphinxAtStartPar
organization ceases to exist, the successor is

\sphinxAtStartPar
brought into existence
\\
\hline
\sphinxAtStartPar
{\hyperref[\detokenize{doc-ORG_2000008::doc}]{\sphinxcrossref{\DUrole{doc}{ORG\_2000008 \sphinxhyphen{} successor organization of}}}}
&
\sphinxAtStartPar
The property that associates an on organization

\sphinxAtStartPar
with the organization that preceeded it
\\
\hline
\sphinxAtStartPar
{\hyperref[\detokenize{doc-ORG_2000009::doc}]{\sphinxcrossref{\DUrole{doc}{ORG\_2000009 \sphinxhyphen{} has organizational part}}}}
&
\sphinxAtStartPar
The property that associates an organization with

\sphinxAtStartPar
one of its organizational parts
\\
\hline
\sphinxAtStartPar
{\hyperref[\detokenize{doc-ORG_2000010::doc}]{\sphinxcrossref{\DUrole{doc}{ORG\_2000010 \sphinxhyphen{} organizational part of}}}}
&
\sphinxAtStartPar
The property that associates an organization part

\sphinxAtStartPar
with the organization of which it is a part
\\
\hline
\sphinxAtStartPar
{\hyperref[\detokenize{doc-ORG_2000011::doc}]{\sphinxcrossref{\DUrole{doc}{ORG\_2000011 \sphinxhyphen{} affiliated with}}}}
&
\sphinxAtStartPar
The property that associates an organization with

\sphinxAtStartPar
another organization that is not an organizational

\sphinxAtStartPar
part associatiion
\\
\hline
\sphinxAtStartPar
{\hyperref[\detokenize{doc-ORG_2000012::doc}]{\sphinxcrossref{\DUrole{doc}{ORG\_2000012 \sphinxhyphen{} has spin\sphinxhyphen{}off organization}}}}
&
\sphinxAtStartPar
The property that associates an organization with

\sphinxAtStartPar
a part that is now independent
\\
\hline
\sphinxAtStartPar
{\hyperref[\detokenize{doc-ORG_2000013::doc}]{\sphinxcrossref{\DUrole{doc}{ORG\_2000013 \sphinxhyphen{} spin\sphinxhyphen{}off organization of}}}}
&
\sphinxAtStartPar
The property that associates an organization with

\sphinxAtStartPar
an organization of which it was formerly a part
\\
\hline
\sphinxAtStartPar
{\hyperref[\detokenize{doc-ORG_2000014::doc}]{\sphinxcrossref{\DUrole{doc}{ORG\_2000014 \sphinxhyphen{} has organizational member}}}}
&
\sphinxAtStartPar
The property that associates an organization with

\sphinxAtStartPar
one of its members
\\
\hline
\sphinxAtStartPar
{\hyperref[\detokenize{doc-ORG_2000015::doc}]{\sphinxcrossref{\DUrole{doc}{ORG\_2000015 \sphinxhyphen{} organizational member of}}}}
&
\sphinxAtStartPar
The property that associates a member organizaiton

\sphinxAtStartPar
with the organization of which it is a member
\\
\hline
\sphinxAtStartPar
{\hyperref[\detokenize{doc-ORG_2000016::doc}]{\sphinxcrossref{\DUrole{doc}{ORG\_2000016 \sphinxhyphen{} has organizational employee}}}}
&
\sphinxAtStartPar
The property that assocates an organization with

\sphinxAtStartPar
an employee of the organization
\\
\hline
\sphinxAtStartPar
{\hyperref[\detokenize{doc-ORG_2000017::doc}]{\sphinxcrossref{\DUrole{doc}{ORG\_2000017 \sphinxhyphen{} organizational employee of}}}}
&
\sphinxAtStartPar
The property that associates a person with the

\sphinxAtStartPar
organization of which that person is an employee
\\
\hline
\sphinxAtStartPar
{\hyperref[\detokenize{doc-ORG_2000018::doc}]{\sphinxcrossref{\DUrole{doc}{ORG\_2000018 \sphinxhyphen{} has organizational associate}}}}
&
\sphinxAtStartPar
The property that associates an organization with

\sphinxAtStartPar
a person who is associated with the organization
\\
\hline
\sphinxAtStartPar
{\hyperref[\detokenize{doc-ORG_2000019::doc}]{\sphinxcrossref{\DUrole{doc}{ORG\_2000019 \sphinxhyphen{} organizational associate of}}}}
&
\sphinxAtStartPar
The property that associates a person with an

\sphinxAtStartPar
organization with which they are associated
\\
\hline
\sphinxAtStartPar
{\hyperref[\detokenize{doc-ORG_2000020::doc}]{\sphinxcrossref{\DUrole{doc}{ORG\_2000020 \sphinxhyphen{} has organizational head}}}}
&
\sphinxAtStartPar
The property that associates an organziation with

\sphinxAtStartPar
the person who is the head of the organization
\\
\hline
\sphinxAtStartPar
{\hyperref[\detokenize{doc-ORG_2000021::doc}]{\sphinxcrossref{\DUrole{doc}{ORG\_2000021 \sphinxhyphen{} organizational head of}}}}
&
\sphinxAtStartPar
The property that associates a person with an

\sphinxAtStartPar
organization which they are they head of
\\
\hline
\sphinxAtStartPar
{\hyperref[\detokenize{doc-ORG_2000022::doc}]{\sphinxcrossref{\DUrole{doc}{ORG\_2000022 \sphinxhyphen{} has organizational volunteer}}}}
&
\sphinxAtStartPar
The property that associates an organization with

\sphinxAtStartPar
a person who volunteers for the organization
\\
\hline
\sphinxAtStartPar
{\hyperref[\detokenize{doc-ORG_2000023::doc}]{\sphinxcrossref{\DUrole{doc}{ORG\_2000023 \sphinxhyphen{} organizational volunteer of}}}}
&
\sphinxAtStartPar
The property that associates a person with an

\sphinxAtStartPar
organization for which they volunteer
\\
\hline
\sphinxAtStartPar
{\hyperref[\detokenize{doc-ORG_2000024::doc}]{\sphinxcrossref{\DUrole{doc}{ORG\_2000024 \sphinxhyphen{} has organizational appointee}}}}
&
\sphinxAtStartPar
The property that associates an organization with

\sphinxAtStartPar
a person who is appointed by the organization to

\sphinxAtStartPar
some post or position within the organization
\\
\hline
\sphinxAtStartPar
{\hyperref[\detokenize{doc-ORG_2000025::doc}]{\sphinxcrossref{\DUrole{doc}{ORG\_2000025 \sphinxhyphen{} organizational appointee of}}}}
&
\sphinxAtStartPar
The property that associates a person with the

\sphinxAtStartPar
organization which has appointed the person to

\sphinxAtStartPar
some post or position within the organization
\\
\hline
\sphinxAtStartPar
{\hyperref[\detokenize{doc-ORG_2000026::doc}]{\sphinxcrossref{\DUrole{doc}{ORG\_2000026 \sphinxhyphen{} has person member}}}}
&
\sphinxAtStartPar
The property that associates an organization with

\sphinxAtStartPar
a person who is  a member of the organization
\\
\hline
\sphinxAtStartPar
{\hyperref[\detokenize{doc-ORG_2000027::doc}]{\sphinxcrossref{\DUrole{doc}{ORG\_2000027 \sphinxhyphen{} person member of}}}}
&
\sphinxAtStartPar
The property that associates a person with an

\sphinxAtStartPar
organization of which the person is a member
\\
\hline
\sphinxAtStartPar
{\hyperref[\detokenize{doc-RO_0000052::doc}]{\sphinxcrossref{\DUrole{doc}{RO\_0000052 \sphinxhyphen{} inheres in}}}}
&
\sphinxAtStartPar
A relation between a specifically dependent

\sphinxAtStartPar
continuant (the dependent) and an independent

\sphinxAtStartPar
continuant (the bearer), in which the dependent

\sphinxAtStartPar
specifically depends on the bearer for its

\sphinxAtStartPar
existence
\\
\hline
\sphinxAtStartPar
{\hyperref[\detokenize{doc-RO_0000053::doc}]{\sphinxcrossref{\DUrole{doc}{RO\_0000053 \sphinxhyphen{} bearer of}}}}
&
\sphinxAtStartPar
A relation between an independent continuant (the

\sphinxAtStartPar
bearer) and a specifically dependent continuant

\sphinxAtStartPar
(the dependent), in which the dependent

\sphinxAtStartPar
specifically depends on the bearer for its

\sphinxAtStartPar
existence
\\
\hline
\sphinxAtStartPar
{\hyperref[\detokenize{doc-RO_0000056::doc}]{\sphinxcrossref{\DUrole{doc}{RO\_0000056 \sphinxhyphen{} participates in}}}}
&
\sphinxAtStartPar
A relation between a continuant and a process, in

\sphinxAtStartPar
which the continuant is somehow involved in the

\sphinxAtStartPar
process
\\
\hline
\sphinxAtStartPar
{\hyperref[\detokenize{doc-RO_0000057::doc}]{\sphinxcrossref{\DUrole{doc}{RO\_0000057 \sphinxhyphen{} has participant}}}}
&
\sphinxAtStartPar
A relation between a process and a continuant, in

\sphinxAtStartPar
which the continuant is somehow involved in the

\sphinxAtStartPar
process
\\
\hline
\sphinxAtStartPar
{\hyperref[\detokenize{doc-RO_0000080::doc}]{\sphinxcrossref{\DUrole{doc}{RO\_0000080 \sphinxhyphen{} quality of}}}}
&
\sphinxAtStartPar
A relation between a quality and an independent

\sphinxAtStartPar
continuant (the bearer), in which the quality

\sphinxAtStartPar
specifically depends on the bearer for its

\sphinxAtStartPar
existence
\\
\hline
\sphinxAtStartPar
{\hyperref[\detokenize{doc-RO_0000081::doc}]{\sphinxcrossref{\DUrole{doc}{RO\_0000081 \sphinxhyphen{} role of}}}}
&
\sphinxAtStartPar
A relation between a role and an independent

\sphinxAtStartPar
continuant (the bearer), in which the role

\sphinxAtStartPar
specifically depends on the bearer for its

\sphinxAtStartPar
existence
\\
\hline
\sphinxAtStartPar
{\hyperref[\detokenize{doc-RO_0000086::doc}]{\sphinxcrossref{\DUrole{doc}{RO\_0000086 \sphinxhyphen{} has quality}}}}
&
\sphinxAtStartPar
A relation between an independent continuant (the

\sphinxAtStartPar
bearer) and a quality, in which the quality

\sphinxAtStartPar
specifically depends on the bearer for its

\sphinxAtStartPar
existence
\\
\hline
\sphinxAtStartPar
{\hyperref[\detokenize{doc-RO_0000087::doc}]{\sphinxcrossref{\DUrole{doc}{RO\_0000087 \sphinxhyphen{} has role}}}}
&
\sphinxAtStartPar
A relation between an independent continuant (the

\sphinxAtStartPar
bearer) and a role, in which the role specifically

\sphinxAtStartPar
depends on the bearer for its existence
\\
\hline
\sphinxAtStartPar
{\hyperref[\detokenize{doc-RO_0000091::doc}]{\sphinxcrossref{\DUrole{doc}{RO\_0000091 \sphinxhyphen{} has disposition}}}}
&
\sphinxAtStartPar
A relation between an independent continuant (the

\sphinxAtStartPar
bearer) and a disposition, in which the

\sphinxAtStartPar
disposition specifically depends on the bearer for

\sphinxAtStartPar
its existence
\\
\hline
\sphinxAtStartPar
{\hyperref[\detokenize{doc-RO_0000092::doc}]{\sphinxcrossref{\DUrole{doc}{RO\_0000092 \sphinxhyphen{} disposition of}}}}
&
\sphinxAtStartPar
Inverse of has disposition
\\
\hline
\sphinxAtStartPar
{\hyperref[\detokenize{doc-RO_0001015::doc}]{\sphinxcrossref{\DUrole{doc}{RO\_0001015 \sphinxhyphen{} location of}}}}
&
\sphinxAtStartPar
A relation between two independent continuants,

\sphinxAtStartPar
the location and the target, in which the target

\sphinxAtStartPar
is entirely within the location
\\
\hline
\sphinxAtStartPar
{\hyperref[\detokenize{doc-RO_0001025::doc}]{\sphinxcrossref{\DUrole{doc}{RO\_0001025 \sphinxhyphen{} located in}}}}
&
\sphinxAtStartPar
A relation between two independent continuants,

\sphinxAtStartPar
the target and the location, in which the target

\sphinxAtStartPar
is entirely within the location
\\
\hline
\sphinxAtStartPar
{\hyperref[\detokenize{doc-RO_0002012::doc}]{\sphinxcrossref{\DUrole{doc}{RO\_0002012 \sphinxhyphen{} occurent part of}}}}
&
\sphinxAtStartPar
A part of relation that applies only between

\sphinxAtStartPar
occurents.
\\
\hline
\sphinxAtStartPar
{\hyperref[\detokenize{doc-RO_0002131::doc}]{\sphinxcrossref{\DUrole{doc}{RO\_0002131 \sphinxhyphen{} overlaps}}}}
&
\sphinxAtStartPar
X overlaps y if and only if there exists some z

\sphinxAtStartPar
such that x has part z and z part of y
\\
\hline
\sphinxAtStartPar
{\hyperref[\detokenize{doc-RO_0002234::doc}]{\sphinxcrossref{\DUrole{doc}{RO\_0002234 \sphinxhyphen{} has output}}}}
&
\sphinxAtStartPar
P has output c iff c is a participant in p, c is

\sphinxAtStartPar
present at the end of p, and c is not present at

\sphinxAtStartPar
the beginning of p.
\\
\hline
\sphinxAtStartPar
{\hyperref[\detokenize{doc-RO_0002323::doc}]{\sphinxcrossref{\DUrole{doc}{RO\_0002323 \sphinxhyphen{} mereotopologically related to}}}}
&
\sphinxAtStartPar
A mereological relationship or a topological

\sphinxAtStartPar
relationship
\\
\hline
\sphinxAtStartPar
{\hyperref[\detokenize{doc-RO_0002353::doc}]{\sphinxcrossref{\DUrole{doc}{RO\_0002353 \sphinxhyphen{} output of}}}}
&
\sphinxAtStartPar
Inverse of has output
\\
\hline
\sphinxAtStartPar
{\hyperref[\detokenize{doc-unitType::doc}]{\sphinxcrossref{\DUrole{doc}{unitType \sphinxhyphen{} temporal unit type}}}}
&
\sphinxAtStartPar
An indicator of the temporal precision of a time

\sphinxAtStartPar
instant
\\
\hline
\end{longtable}\sphinxatlongtableend\end{savenotes}
\begin{quote}

\index{BFO\_0000050@\spxentry{BFO\_0000050}!part of@\spxentry{part of}}\index{part of@\spxentry{part of}!BFO\_0000050@\spxentry{BFO\_0000050}}\ignorespaces \end{quote}


\subsection{BFO\_0000050 \sphinxhyphen{} part of}
\label{\detokenize{doc-BFO_0000050:bfo-0000050-part-of}}\label{\detokenize{doc-BFO_0000050:index-0}}\label{\detokenize{doc-BFO_0000050::doc}}
\begin{sphinxShadowBox}
\sphinxstyletopictitle{Label}

\sphinxAtStartPar
part of
\end{sphinxShadowBox}

\begin{sphinxShadowBox}
\sphinxstyletopictitle{Definition}

\sphinxAtStartPar
A core relation that holds between a part and its whole
\end{sphinxShadowBox}

\begin{sphinxShadowBox}
\sphinxstyletopictitle{Example}

\sphinxAtStartPar
my brain is part of my body (continuant parthood, two material entities)

\sphinxAtStartPar
this day is part of this year (occurrent parthood)

\sphinxAtStartPar
my stomach cavity is part of my stomach (continuant parthood, immaterial entity is part of material entity)
\end{sphinxShadowBox}

\begin{sphinxShadowBox}
\sphinxstyletopictitle{Editor’s note}

\sphinxAtStartPar
Parthood requires the part and the whole to have compatible classes: only an occurrent can be part of an occurrent; only a process can be part of a process; only a continuant can be part of a continuant; only an independent continuant can be part of an independent continuant; only an immaterial entity can be part of an immaterial entity; only a specifically dependent continuant can be part of a specifically dependent continuant; only a generically dependent continuant can be part of a generically dependent continuant. (This list is not exhaustive.)

\sphinxAtStartPar
A continuant cannot be part of an occurrent: use ‘participates in’. An occurrent cannot be part of a continuant: use ‘has participant’. A material entity cannot be part of an immaterial entity: use ‘has location’. A specifically dependent continuant cannot be part of an independent continuant: use ‘inheres in’. An independent continuant cannot be part of a specifically dependent continuant: use ‘bearer of’.

\sphinxAtStartPar
Occurrents are not subject to change and so parthood between occurrents holds for all the times that the part exists. Many continuants are subject to change, so parthood between continuants will only hold at certain times, but this is difficult to specify in OWL. See \sphinxurl{https://code.google.com/p/obo-relations/wiki/ROAndTime}

\sphinxAtStartPar
Everything is part of itself. Any part of any part of a thing is itself part of that thing. Two distinct things cannot be part of each other.
\end{sphinxShadowBox}

\begin{sphinxShadowBox}
\sphinxstyletopictitle{Imported From}

\sphinxAtStartPar
\sphinxurl{http://purl.obolibrary.org/obo/ro/releases/2020-12-18/ro.owl}
\end{sphinxShadowBox}

\begin{sphinxShadowBox}
\sphinxstyletopictitle{See also}

\sphinxAtStartPar
\sphinxurl{http://ontologydesignpatterns.org/wiki/Submissions:PartOf}

\sphinxAtStartPar
\sphinxurl{http://www.obofoundry.org/ro/\#OBO\_REL:part\_of}

\sphinxAtStartPar
\sphinxurl{http://ontologydesignpatterns.org/wiki/Community:Parts\_and\_Collections}
\end{sphinxShadowBox}
\begin{quote}

\index{BFO\_0000051@\spxentry{BFO\_0000051}!has part@\spxentry{has part}}\index{has part@\spxentry{has part}!BFO\_0000051@\spxentry{BFO\_0000051}}\ignorespaces \end{quote}


\subsection{BFO\_0000051 \sphinxhyphen{} has part}
\label{\detokenize{doc-BFO_0000051:bfo-0000051-has-part}}\label{\detokenize{doc-BFO_0000051:index-0}}\label{\detokenize{doc-BFO_0000051::doc}}
\begin{sphinxShadowBox}
\sphinxstyletopictitle{Label}

\sphinxAtStartPar
has part
\end{sphinxShadowBox}

\begin{sphinxShadowBox}
\sphinxstyletopictitle{Definition}

\sphinxAtStartPar
A core relation that holds between a whole and its part
\end{sphinxShadowBox}

\begin{sphinxShadowBox}
\sphinxstyletopictitle{Example}

\sphinxAtStartPar
my body has part my brain (continuant parthood, two material entities)

\sphinxAtStartPar
my stomach has part my stomach cavity (continuant parthood, material entity has part immaterial entity)

\sphinxAtStartPar
this year has part this day (occurrent parthood)
\end{sphinxShadowBox}

\begin{sphinxShadowBox}
\sphinxstyletopictitle{Editor’s note}

\sphinxAtStartPar
Parthood requires the part and the whole to have compatible classes: only an occurrent have an occurrent as part; only a process can have a process as part; only a continuant can have a continuant as part; only an independent continuant can have an independent continuant as part; only a specifically dependent continuant can have a specifically dependent continuant as part; only a generically dependent continuant can have a generically dependent continuant as part. (This list is not exhaustive.)

\sphinxAtStartPar
A continuant cannot have an occurrent as part: use ‘participates in’. An occurrent cannot have a continuant as part: use ‘has participant’. An immaterial entity cannot have a material entity as part: use ‘location of’. An independent continuant cannot have a specifically dependent continuant as part: use ‘bearer of’. A specifically dependent continuant cannot have an independent continuant as part: use ‘inheres in’.

\sphinxAtStartPar
Everything has itself as a part. Any part of any part of a thing is itself part of that thing. Two distinct things cannot have each other as a part.

\sphinxAtStartPar
Occurrents are not subject to change and so parthood between occurrents holds for all the times that the part exists. Many continuants are subject to change, so parthood between continuants will only hold at certain times, but this is difficult to specify in OWL. See \sphinxurl{https://code.google.com/p/obo-relations/wiki/ROAndTime}
\end{sphinxShadowBox}

\begin{sphinxShadowBox}
\sphinxstyletopictitle{Imported From}

\sphinxAtStartPar
\sphinxurl{http://purl.obolibrary.org/obo/ro/releases/2020-12-18/ro.owl}
\end{sphinxShadowBox}
\begin{quote}

\index{BFO\_0000054@\spxentry{BFO\_0000054}!realized in@\spxentry{realized in}}\index{realized in@\spxentry{realized in}!BFO\_0000054@\spxentry{BFO\_0000054}}\ignorespaces \end{quote}


\subsection{BFO\_0000054 \sphinxhyphen{} realized in}
\label{\detokenize{doc-BFO_0000054:bfo-0000054-realized-in}}\label{\detokenize{doc-BFO_0000054:index-0}}\label{\detokenize{doc-BFO_0000054::doc}}
\begin{sphinxShadowBox}
\sphinxstyletopictitle{Label}

\sphinxAtStartPar
realized in
\end{sphinxShadowBox}

\begin{sphinxShadowBox}
\sphinxstyletopictitle{Alternate name}

\sphinxAtStartPar
is realized by

\sphinxAtStartPar
realized\_in
\end{sphinxShadowBox}

\begin{sphinxShadowBox}
\sphinxstyletopictitle{Definition}

\sphinxAtStartPar
A relation between a realizable entity and a process, where there is some material entity that is bearer of the realizable entity and participates in the process, and the realizable entity comes to be realized in the course of the process
\end{sphinxShadowBox}

\begin{sphinxShadowBox}
\sphinxstyletopictitle{Example}

\sphinxAtStartPar
this investigator role is realized in this investigation

\sphinxAtStartPar
this fragility is realized in this shattering

\sphinxAtStartPar
this disease is realized in this disease course
\end{sphinxShadowBox}

\begin{sphinxShadowBox}
\sphinxstyletopictitle{Imported From}

\sphinxAtStartPar
\sphinxurl{http://purl.obolibrary.org/obo/ro/releases/2021-03-08/ro.owl}
\end{sphinxShadowBox}
\begin{quote}

\index{BFO\_0000055@\spxentry{BFO\_0000055}!realizes@\spxentry{realizes}}\index{realizes@\spxentry{realizes}!BFO\_0000055@\spxentry{BFO\_0000055}}\ignorespaces \end{quote}


\subsection{BFO\_0000055 \sphinxhyphen{} realizes}
\label{\detokenize{doc-BFO_0000055:bfo-0000055-realizes}}\label{\detokenize{doc-BFO_0000055:index-0}}\label{\detokenize{doc-BFO_0000055::doc}}
\begin{sphinxShadowBox}
\sphinxstyletopictitle{Label}

\sphinxAtStartPar
realizes
\end{sphinxShadowBox}

\begin{sphinxShadowBox}
\sphinxstyletopictitle{Definition}

\sphinxAtStartPar
A relation between a process and a realizable entity, where there is some material entity that is bearer of the realizable entity and participates in the process, and the realizable entity comes to be realized in the course of the process
\end{sphinxShadowBox}

\begin{sphinxShadowBox}
\sphinxstyletopictitle{Example}

\sphinxAtStartPar
this investigation realizes this investigator role

\sphinxAtStartPar
this shattering realizes this fragility

\sphinxAtStartPar
this disease course realizes this disease
\end{sphinxShadowBox}

\begin{sphinxShadowBox}
\sphinxstyletopictitle{Imported From}

\sphinxAtStartPar
\sphinxurl{http://purl.obolibrary.org/obo/ro/releases/2021-03-08/ro.owl}
\end{sphinxShadowBox}
\begin{quote}

\index{IAO\_0000136@\spxentry{IAO\_0000136}!is about@\spxentry{is about}}\index{is about@\spxentry{is about}!IAO\_0000136@\spxentry{IAO\_0000136}}\ignorespaces \end{quote}


\subsection{IAO\_0000136 \sphinxhyphen{} is about}
\label{\detokenize{doc-IAO_0000136:iao-0000136-is-about}}\label{\detokenize{doc-IAO_0000136:index-0}}\label{\detokenize{doc-IAO_0000136::doc}}
\begin{sphinxShadowBox}
\sphinxstyletopictitle{Label}

\sphinxAtStartPar
is about
\end{sphinxShadowBox}

\begin{sphinxShadowBox}
\sphinxstyletopictitle{Definition}

\sphinxAtStartPar
A (currently) primitive relation that relates an information artifact to an entity.
\end{sphinxShadowBox}

\begin{sphinxShadowBox}
\sphinxstyletopictitle{Definition source}

\sphinxAtStartPar
Smith, Ceusters, Ruttenberg, 2000 years of philosophy
\end{sphinxShadowBox}

\begin{sphinxShadowBox}
\sphinxstyletopictitle{Example}

\sphinxAtStartPar
This document is about information artifacts and their representations
\end{sphinxShadowBox}

\begin{sphinxShadowBox}
\sphinxstyletopictitle{Editor’s note}

\sphinxAtStartPar
7/6/2009 Alan Ruttenberg. Following discussion with Jonathan Rees, and introduction of “mentions” relation. Weaken the is\_about relationship to be primitive.

\sphinxAtStartPar
We will try to build it back up by elaborating the various subproperties that are more precisely defined.

\sphinxAtStartPar
Some currently missing phenomena that should be considered “about” are predications \sphinxhyphen{} “The only person who knows the answer is sitting beside me” , Allegory, Satire, and other literary forms that can be topical without explicitly mentioning the topic.
\end{sphinxShadowBox}

\begin{sphinxShadowBox}
\sphinxstyletopictitle{Imported From}

\sphinxAtStartPar
\sphinxurl{http://purl.obolibrary.org/obo/iao/2020-12-09/iao.owl}
\end{sphinxShadowBox}

\begin{sphinxShadowBox}
\sphinxstyletopictitle{Term editor}

\sphinxAtStartPar
person:Alan Ruttenberg
\end{sphinxShadowBox}
\begin{quote}

\index{IAO\_0000219@\spxentry{IAO\_0000219}!denotes@\spxentry{denotes}}\index{denotes@\spxentry{denotes}!IAO\_0000219@\spxentry{IAO\_0000219}}\ignorespaces \end{quote}


\subsection{IAO\_0000219 \sphinxhyphen{} denotes}
\label{\detokenize{doc-IAO_0000219:iao-0000219-denotes}}\label{\detokenize{doc-IAO_0000219:index-0}}\label{\detokenize{doc-IAO_0000219::doc}}
\begin{sphinxShadowBox}
\sphinxstyletopictitle{Label}

\sphinxAtStartPar
denotes
\end{sphinxShadowBox}

\begin{sphinxShadowBox}
\sphinxstyletopictitle{Definition}

\sphinxAtStartPar
A primitive, instance\sphinxhyphen{}level, relation obtaining between an information content entity and some portion of reality. Denotation is what happens when someone creates an information content entity E in order to specifically refer to something. The only relation between E and the thing is that E can be used to ‘pick out’ the thing. This relation connects those two together. Freedictionary.com sense 3: To signify directly; refer to specifically
\end{sphinxShadowBox}

\begin{sphinxShadowBox}
\sphinxstyletopictitle{Definition source}

\sphinxAtStartPar
Conversations with Barry Smith, Werner Ceusters, Bjoern Peters, Michel Dumontier, Melanie Courtot, James Malone, Bill Hogan
\end{sphinxShadowBox}

\begin{sphinxShadowBox}
\sphinxstyletopictitle{Example}

\sphinxAtStartPar
A person’s name denotes the person. A variable name in a computer program denotes some piece of memory. Lexically equivalent strings can denote different things, for instance “Alan” can denote different people. In each case of use, there is a case of the denotation relation obtaining, between “Alan” and the person that is being named.
\end{sphinxShadowBox}

\begin{sphinxShadowBox}
\sphinxstyletopictitle{Editor’s note}

\sphinxAtStartPar
2009\sphinxhyphen{}11\sphinxhyphen{}10 Alan Ruttenberg. Old definition said the following to emphasize the generic nature of this relation. We no longer have ‘specifically denotes’, which would have been primitive, so make this relation primitive.
g denotes r =def
r is a portion of reality
there is some c that is a concretization of g
every c that is a concretization of g specifically denotes r
\end{sphinxShadowBox}

\begin{sphinxShadowBox}
\sphinxstyletopictitle{Imported From}

\sphinxAtStartPar
\sphinxurl{http://purl.obolibrary.org/obo/iao/2020-12-09/iao.owl}
\end{sphinxShadowBox}

\begin{sphinxShadowBox}
\sphinxstyletopictitle{Term editor}

\sphinxAtStartPar
person:Alan Ruttenberg
\end{sphinxShadowBox}
\begin{quote}

\index{IAO\_0000235@\spxentry{IAO\_0000235}!denoted by@\spxentry{denoted by}}\index{denoted by@\spxentry{denoted by}!IAO\_0000235@\spxentry{IAO\_0000235}}\ignorespaces \end{quote}


\subsection{IAO\_0000235 \sphinxhyphen{} denoted by}
\label{\detokenize{doc-IAO_0000235:iao-0000235-denoted-by}}\label{\detokenize{doc-IAO_0000235:index-0}}\label{\detokenize{doc-IAO_0000235::doc}}
\begin{sphinxShadowBox}
\sphinxstyletopictitle{Label}

\sphinxAtStartPar
denoted by
\end{sphinxShadowBox}

\begin{sphinxShadowBox}
\sphinxstyletopictitle{Definition}

\sphinxAtStartPar
Inverse of the relation ‘denotes’
\end{sphinxShadowBox}

\begin{sphinxShadowBox}
\sphinxstyletopictitle{Imported From}

\sphinxAtStartPar
\sphinxurl{http://purl.obolibrary.org/obo/iao/2020-12-09/iao.owl}
\end{sphinxShadowBox}

\begin{sphinxShadowBox}
\sphinxstyletopictitle{Term editor}

\sphinxAtStartPar
Person: Jie Zheng, Chris Stoeckert, Mike Conlon
\end{sphinxShadowBox}
\begin{quote}

\index{ORG\_2000001@\spxentry{ORG\_2000001}!occupies@\spxentry{occupies}}\index{occupies@\spxentry{occupies}!ORG\_2000001@\spxentry{ORG\_2000001}}\ignorespaces \end{quote}


\subsection{ORG\_2000001 \sphinxhyphen{} occupies}
\label{\detokenize{doc-ORG_2000001:org-2000001-occupies}}\label{\detokenize{doc-ORG_2000001:index-0}}\label{\detokenize{doc-ORG_2000001::doc}}
\begin{sphinxShadowBox}
\sphinxstyletopictitle{Label}

\sphinxAtStartPar
occupies
\end{sphinxShadowBox}

\begin{sphinxShadowBox}
\sphinxstyletopictitle{Definition}

\sphinxAtStartPar
The property that associates an organization occupies a geographical location is it has one of more people at the location, or has legal rights to the location
\end{sphinxShadowBox}

\begin{sphinxShadowBox}
\sphinxstyletopictitle{Definition source}

\sphinxAtStartPar
Michael Conlon \sphinxurl{https://orcid.org/0000-0002-1304-8447}
\end{sphinxShadowBox}

\begin{sphinxShadowBox}
\sphinxstyletopictitle{Example}

\sphinxAtStartPar
The Navaho Nation occupies the Navaho Nation reservation.  Microsoft occupies a campus in Redmond Washington.  The University of Pittsburg occupies the
\end{sphinxShadowBox}

\begin{sphinxShadowBox}
\sphinxstyletopictitle{Term editor}

\sphinxAtStartPar
Michael Conlon \sphinxurl{https://orcid.org/0000-0002-1304-8447}
\end{sphinxShadowBox}
\begin{quote}

\index{ORG\_2000002@\spxentry{ORG\_2000002}!has occurent part@\spxentry{has occurent part}}\index{has occurent part@\spxentry{has occurent part}!ORG\_2000002@\spxentry{ORG\_2000002}}\ignorespaces \end{quote}


\subsection{ORG\_2000002 \sphinxhyphen{} has occurent part}
\label{\detokenize{doc-ORG_2000002:org-2000002-has-occurent-part}}\label{\detokenize{doc-ORG_2000002:index-0}}\label{\detokenize{doc-ORG_2000002::doc}}
\begin{sphinxShadowBox}
\sphinxstyletopictitle{Label}

\sphinxAtStartPar
has occurent part
\end{sphinxShadowBox}

\begin{sphinxShadowBox}
\sphinxstyletopictitle{Definition}

\sphinxAtStartPar
The property that associates an occurent which has an occurent part wholly contained within it
\end{sphinxShadowBox}

\begin{sphinxShadowBox}
\sphinxstyletopictitle{Definition source}

\sphinxAtStartPar
Michael Conlon \sphinxurl{https://orcid.org/0000-0002-1304-8447}
\end{sphinxShadowBox}

\begin{sphinxShadowBox}
\sphinxstyletopictitle{Example}

\sphinxAtStartPar
The project to put a man on the man achieved success on July 20, 1969
\end{sphinxShadowBox}

\begin{sphinxShadowBox}
\sphinxstyletopictitle{Editor’s note}

\sphinxAtStartPar
The inverse of RO\_2012 and should be defined in RO.  Defined here for convenience since it is missing from RO.
\end{sphinxShadowBox}

\begin{sphinxShadowBox}
\sphinxstyletopictitle{Term editor}

\sphinxAtStartPar
Michael Conlon \sphinxurl{https://orcid.org/0000-0002-1304-8447}
\end{sphinxShadowBox}
\begin{quote}

\index{ORG\_2000003@\spxentry{ORG\_2000003}!has time instant@\spxentry{has time instant}}\index{has time instant@\spxentry{has time instant}!ORG\_2000003@\spxentry{ORG\_2000003}}\ignorespaces \end{quote}


\subsection{ORG\_2000003 \sphinxhyphen{} has time instant}
\label{\detokenize{doc-ORG_2000003:org-2000003-has-time-instant}}\label{\detokenize{doc-ORG_2000003:index-0}}\label{\detokenize{doc-ORG_2000003::doc}}
\begin{sphinxShadowBox}
\sphinxstyletopictitle{Label}

\sphinxAtStartPar
has time instant
\end{sphinxShadowBox}

\begin{sphinxShadowBox}
\sphinxstyletopictitle{Definition}

\sphinxAtStartPar
The property that associates a process boundary with the time instant at which the process boundary occurs
\end{sphinxShadowBox}

\begin{sphinxShadowBox}
\sphinxstyletopictitle{Definition source}

\sphinxAtStartPar
Michael Conlon \sphinxurl{https://orcid.org/0000-0002-1304-8448}
\end{sphinxShadowBox}

\begin{sphinxShadowBox}
\sphinxstyletopictitle{Example}

\sphinxAtStartPar
Duraspace was dissolved (a process boundary) in 2019 (a time instant with an XSDdatetimestamp and a year precision)
\end{sphinxShadowBox}

\begin{sphinxShadowBox}
\sphinxstyletopictitle{Editor’s note}

\sphinxAtStartPar
This provides a bidge between BFO ontologies and the W3C time ontology
\end{sphinxShadowBox}

\begin{sphinxShadowBox}
\sphinxstyletopictitle{Similar term in VIVO 1 Ontology}

\sphinxAtStartPar
\sphinxurl{http://vivoweb.org/ontology/core\#dateTimeValue}
\end{sphinxShadowBox}

\begin{sphinxShadowBox}
\sphinxstyletopictitle{Term editor}

\sphinxAtStartPar
Michael Conlon \sphinxurl{https://orcid.org/0000-0002-1304-8448}
\end{sphinxShadowBox}
\begin{quote}

\index{ORG\_2000004@\spxentry{ORG\_2000004}!has organizational interest@\spxentry{has organizational interest}}\index{has organizational interest@\spxentry{has organizational interest}!ORG\_2000004@\spxentry{ORG\_2000004}}\ignorespaces \end{quote}


\subsection{ORG\_2000004 \sphinxhyphen{} has organizational interest}
\label{\detokenize{doc-ORG_2000004:org-2000004-has-organizational-interest}}\label{\detokenize{doc-ORG_2000004:index-0}}\label{\detokenize{doc-ORG_2000004::doc}}
\begin{sphinxShadowBox}
\sphinxstyletopictitle{Label}

\sphinxAtStartPar
has organizational interest
\end{sphinxShadowBox}

\begin{sphinxShadowBox}
\sphinxstyletopictitle{Alternate name}

\sphinxAtStartPar
interested in
\end{sphinxShadowBox}

\begin{sphinxShadowBox}
\sphinxstyletopictitle{Definition}

\sphinxAtStartPar
The property that associates an organization with a concept that the organization manifests in its work
\end{sphinxShadowBox}

\begin{sphinxShadowBox}
\sphinxstyletopictitle{Definition source}

\sphinxAtStartPar
Michael Conlon \sphinxurl{https://orcid.org/0000-0002-1304-8448}
\end{sphinxShadowBox}

\begin{sphinxShadowBox}
\sphinxstyletopictitle{Example}

\sphinxAtStartPar
LYRASIS has organizational interest open source software.  The New York Yankees Baseball Club has organizational interest baseball.  The Gettysburg Foundation has organizational interest The Battle of Gettysburg.
\end{sphinxShadowBox}

\begin{sphinxShadowBox}
\sphinxstyletopictitle{Editor’s note}

\sphinxAtStartPar
Additional subproperties could provide insight regarding the nature of the interest such as research interest, commercial interest, and so on
\end{sphinxShadowBox}

\begin{sphinxShadowBox}
\sphinxstyletopictitle{Similar term in VIVO 1 Ontology}

\sphinxAtStartPar
\sphinxurl{http://vivoweb.org/ontology/core\#hasResearchArea}
\end{sphinxShadowBox}

\begin{sphinxShadowBox}
\sphinxstyletopictitle{Term editor}

\sphinxAtStartPar
Michael Conlon \sphinxurl{https://orcid.org/0000-0002-1304-8448}
\end{sphinxShadowBox}
\begin{quote}

\index{ORG\_2000005@\spxentry{ORG\_2000005}!has website@\spxentry{has website}}\index{has website@\spxentry{has website}!ORG\_2000005@\spxentry{ORG\_2000005}}\ignorespaces \end{quote}


\subsection{ORG\_2000005 \sphinxhyphen{} has website}
\label{\detokenize{doc-ORG_2000005:org-2000005-has-website}}\label{\detokenize{doc-ORG_2000005:index-0}}\label{\detokenize{doc-ORG_2000005::doc}}
\begin{sphinxShadowBox}
\sphinxstyletopictitle{Label}

\sphinxAtStartPar
has website
\end{sphinxShadowBox}

\begin{sphinxShadowBox}
\sphinxstyletopictitle{Definition}

\sphinxAtStartPar
The property that associates an organization with its website
\end{sphinxShadowBox}

\begin{sphinxShadowBox}
\sphinxstyletopictitle{Definition source}

\sphinxAtStartPar
Michael Conlon \sphinxurl{https://orcid.org/0000-0002-1304-8448}
\end{sphinxShadowBox}

\begin{sphinxShadowBox}
\sphinxstyletopictitle{Example}

\sphinxAtStartPar
LYRASIS has website \sphinxurl{https://lyrasis.org}.  Moscow State University has website \sphinxurl{https://msu.ru}
\end{sphinxShadowBox}

\begin{sphinxShadowBox}
\sphinxstyletopictitle{Term editor}

\sphinxAtStartPar
Michael Conlon \sphinxurl{https://orcid.org/0000-0002-1304-8448}
\end{sphinxShadowBox}
\begin{quote}

\index{ORG\_2000006@\spxentry{ORG\_2000006}!website of@\spxentry{website of}}\index{website of@\spxentry{website of}!ORG\_2000006@\spxentry{ORG\_2000006}}\ignorespaces \end{quote}


\subsection{ORG\_2000006 \sphinxhyphen{} website of}
\label{\detokenize{doc-ORG_2000006:org-2000006-website-of}}\label{\detokenize{doc-ORG_2000006:index-0}}\label{\detokenize{doc-ORG_2000006::doc}}
\begin{sphinxShadowBox}
\sphinxstyletopictitle{Label}

\sphinxAtStartPar
website of
\end{sphinxShadowBox}

\begin{sphinxShadowBox}
\sphinxstyletopictitle{Definition}

\sphinxAtStartPar
The property that associates a website with its organization
\end{sphinxShadowBox}

\begin{sphinxShadowBox}
\sphinxstyletopictitle{Definition source}

\sphinxAtStartPar
Michael Conlon \sphinxurl{https://orcid.org/0000-0002-1304-8448}
\end{sphinxShadowBox}

\begin{sphinxShadowBox}
\sphinxstyletopictitle{Example}

\sphinxAtStartPar
The website \sphinxurl{https://vivoweb.org} website of The VIVO Project
\end{sphinxShadowBox}

\begin{sphinxShadowBox}
\sphinxstyletopictitle{Term editor}

\sphinxAtStartPar
Michael Conlon \sphinxurl{https://orcid.org/0000-0002-1304-8448}
\end{sphinxShadowBox}
\begin{quote}

\index{ORG\_2000007@\spxentry{ORG\_2000007}!has successor organization@\spxentry{has successor organization}}\index{has successor organization@\spxentry{has successor organization}!ORG\_2000007@\spxentry{ORG\_2000007}}\ignorespaces \end{quote}


\subsection{ORG\_2000007 \sphinxhyphen{} has successor organization}
\label{\detokenize{doc-ORG_2000007:org-2000007-has-successor-organization}}\label{\detokenize{doc-ORG_2000007:index-0}}\label{\detokenize{doc-ORG_2000007::doc}}
\begin{sphinxShadowBox}
\sphinxstyletopictitle{Label}

\sphinxAtStartPar
has successor organization
\end{sphinxShadowBox}

\begin{sphinxShadowBox}
\sphinxstyletopictitle{Definition}

\sphinxAtStartPar
The property that associates an organization with the organization that succeeds it.  The preceeding organization ceases to exist, the successor is brought into existence
\end{sphinxShadowBox}

\begin{sphinxShadowBox}
\sphinxstyletopictitle{Definition source}

\sphinxAtStartPar
Michael Conlon \sphinxurl{https://orcid.org/0000-0002-1304-8448}
\end{sphinxShadowBox}

\begin{sphinxShadowBox}
\sphinxstyletopictitle{Example}

\sphinxAtStartPar
The United Colonies has successor organization The United States of America
\end{sphinxShadowBox}

\begin{sphinxShadowBox}
\sphinxstyletopictitle{Term editor}

\sphinxAtStartPar
Michael Conlon \sphinxurl{https://orcid.org/0000-0002-1304-8448}
\end{sphinxShadowBox}
\begin{quote}

\index{ORG\_2000008@\spxentry{ORG\_2000008}!successor organization of@\spxentry{successor organization of}}\index{successor organization of@\spxentry{successor organization of}!ORG\_2000008@\spxentry{ORG\_2000008}}\ignorespaces \end{quote}


\subsection{ORG\_2000008 \sphinxhyphen{} successor organization of}
\label{\detokenize{doc-ORG_2000008:org-2000008-successor-organization-of}}\label{\detokenize{doc-ORG_2000008:index-0}}\label{\detokenize{doc-ORG_2000008::doc}}
\begin{sphinxShadowBox}
\sphinxstyletopictitle{Label}

\sphinxAtStartPar
successor organization of
\end{sphinxShadowBox}

\begin{sphinxShadowBox}
\sphinxstyletopictitle{Definition}

\sphinxAtStartPar
The property that associates an on organization with the organization that preceeded it
\end{sphinxShadowBox}

\begin{sphinxShadowBox}
\sphinxstyletopictitle{Definition source}

\sphinxAtStartPar
Michael Conlon \sphinxurl{https://orcid.org/0000-0002-1304-8448}
\end{sphinxShadowBox}

\begin{sphinxShadowBox}
\sphinxstyletopictitle{Example}

\sphinxAtStartPar
The United Nations is the successor of the The League of Nations.
\end{sphinxShadowBox}

\begin{sphinxShadowBox}
\sphinxstyletopictitle{Term editor}

\sphinxAtStartPar
Michael Conlon \sphinxurl{https://orcid.org/0000-0002-1304-8448}
\end{sphinxShadowBox}
\begin{quote}

\index{ORG\_2000009@\spxentry{ORG\_2000009}!has organizational part@\spxentry{has organizational part}}\index{has organizational part@\spxentry{has organizational part}!ORG\_2000009@\spxentry{ORG\_2000009}}\ignorespaces \end{quote}


\subsection{ORG\_2000009 \sphinxhyphen{} has organizational part}
\label{\detokenize{doc-ORG_2000009:org-2000009-has-organizational-part}}\label{\detokenize{doc-ORG_2000009:index-0}}\label{\detokenize{doc-ORG_2000009::doc}}
\begin{sphinxShadowBox}
\sphinxstyletopictitle{Label}

\sphinxAtStartPar
has organizational part
\end{sphinxShadowBox}

\begin{sphinxShadowBox}
\sphinxstyletopictitle{Alternate name}

\sphinxAtStartPar
parent of
\end{sphinxShadowBox}

\begin{sphinxShadowBox}
\sphinxstyletopictitle{Definition}

\sphinxAtStartPar
The property that associates an organization with one of its organizational parts
\end{sphinxShadowBox}

\begin{sphinxShadowBox}
\sphinxstyletopictitle{Definition source}

\sphinxAtStartPar
Michael Conlon \sphinxurl{https://orcid.org/0000-0002-1304-8448}
\end{sphinxShadowBox}

\begin{sphinxShadowBox}
\sphinxstyletopictitle{Example}

\sphinxAtStartPar
The University of Florida has organizational part College of Medicine; The College of Medicine has organizational part Department of Anesthesiology
\end{sphinxShadowBox}

\begin{sphinxShadowBox}
\sphinxstyletopictitle{Term editor}

\sphinxAtStartPar
Michael Conlon \sphinxurl{https://orcid.org/0000-0002-1304-8448}
\end{sphinxShadowBox}
\begin{quote}

\index{ORG\_2000010@\spxentry{ORG\_2000010}!organizational part of@\spxentry{organizational part of}}\index{organizational part of@\spxentry{organizational part of}!ORG\_2000010@\spxentry{ORG\_2000010}}\ignorespaces \end{quote}


\subsection{ORG\_2000010 \sphinxhyphen{} organizational part of}
\label{\detokenize{doc-ORG_2000010:org-2000010-organizational-part-of}}\label{\detokenize{doc-ORG_2000010:index-0}}\label{\detokenize{doc-ORG_2000010::doc}}
\begin{sphinxShadowBox}
\sphinxstyletopictitle{Label}

\sphinxAtStartPar
organizational part of
\end{sphinxShadowBox}

\begin{sphinxShadowBox}
\sphinxstyletopictitle{Alternate name}

\sphinxAtStartPar
child of
\end{sphinxShadowBox}

\begin{sphinxShadowBox}
\sphinxstyletopictitle{Definition}

\sphinxAtStartPar
The property that associates an organization part with the organization of which it is a part
\end{sphinxShadowBox}

\begin{sphinxShadowBox}
\sphinxstyletopictitle{Definition source}

\sphinxAtStartPar
Michael Conlon \sphinxurl{https://orcid.org/0000-0002-1304-8448}
\end{sphinxShadowBox}

\begin{sphinxShadowBox}
\sphinxstyletopictitle{Example}

\sphinxAtStartPar
The College of Medicine at UF organizational part of UF; Department of Anethesiology at UF organizational part of College of Medicine at UF
\end{sphinxShadowBox}

\begin{sphinxShadowBox}
\sphinxstyletopictitle{Term editor}

\sphinxAtStartPar
Michael Conlon \sphinxurl{https://orcid.org/0000-0002-1304-8448}
\end{sphinxShadowBox}
\begin{quote}

\index{ORG\_2000011@\spxentry{ORG\_2000011}!affiliated with@\spxentry{affiliated with}}\index{affiliated with@\spxentry{affiliated with}!ORG\_2000011@\spxentry{ORG\_2000011}}\ignorespaces \end{quote}


\subsection{ORG\_2000011 \sphinxhyphen{} affiliated with}
\label{\detokenize{doc-ORG_2000011:org-2000011-affiliated-with}}\label{\detokenize{doc-ORG_2000011:index-0}}\label{\detokenize{doc-ORG_2000011::doc}}
\begin{sphinxShadowBox}
\sphinxstyletopictitle{Label}

\sphinxAtStartPar
affiliated with
\end{sphinxShadowBox}

\begin{sphinxShadowBox}
\sphinxstyletopictitle{Alternate name}

\sphinxAtStartPar
linked to
\end{sphinxShadowBox}

\begin{sphinxShadowBox}
\sphinxstyletopictitle{Definition}

\sphinxAtStartPar
The property that associates an organization with another organization that is not an organizational part associatiion
\end{sphinxShadowBox}

\begin{sphinxShadowBox}
\sphinxstyletopictitle{Definition source}

\sphinxAtStartPar
Michael Conlon \sphinxurl{https://orcid.org/0000-0002-1304-8448}
\end{sphinxShadowBox}

\begin{sphinxShadowBox}
\sphinxstyletopictitle{Example}

\sphinxAtStartPar
The Theatre Library Association is an affiliate of the American Library Association
\end{sphinxShadowBox}

\begin{sphinxShadowBox}
\sphinxstyletopictitle{Editor’s note}

\sphinxAtStartPar
Affiliate relationships exist in many different types of configurations across all sorts of industries.  This term has very broad semantics.
\end{sphinxShadowBox}

\begin{sphinxShadowBox}
\sphinxstyletopictitle{Similar term in VIVO 1 Ontology}

\sphinxAtStartPar
\sphinxurl{http://vivoweb.org/ontology/core\#affiliatedOrganization}
\end{sphinxShadowBox}

\begin{sphinxShadowBox}
\sphinxstyletopictitle{Term editor}

\sphinxAtStartPar
Michael Conlon \sphinxurl{https://orcid.org/0000-0002-1304-8448}
\end{sphinxShadowBox}
\begin{quote}

\index{ORG\_2000012@\spxentry{ORG\_2000012}!has spin\sphinxhyphen{}off organization@\spxentry{has spin\sphinxhyphen{}off organization}}\index{has spin\sphinxhyphen{}off organization@\spxentry{has spin\sphinxhyphen{}off organization}!ORG\_2000012@\spxentry{ORG\_2000012}}\ignorespaces \end{quote}


\subsection{ORG\_2000012 \sphinxhyphen{} has spin\sphinxhyphen{}off organization}
\label{\detokenize{doc-ORG_2000012:org-2000012-has-spin-off-organization}}\label{\detokenize{doc-ORG_2000012:index-0}}\label{\detokenize{doc-ORG_2000012::doc}}
\begin{sphinxShadowBox}
\sphinxstyletopictitle{Label}

\sphinxAtStartPar
has spin\sphinxhyphen{}off organization
\end{sphinxShadowBox}

\begin{sphinxShadowBox}
\sphinxstyletopictitle{Alternate name}

\sphinxAtStartPar
has spinout
\end{sphinxShadowBox}

\begin{sphinxShadowBox}
\sphinxstyletopictitle{Definition}

\sphinxAtStartPar
The property that associates an organization with a part that is now independent
\end{sphinxShadowBox}

\begin{sphinxShadowBox}
\sphinxstyletopictitle{Definition source}

\sphinxAtStartPar
Michael Conlon \sphinxurl{https://orcid.org/0000-0002-1304-8448}
\end{sphinxShadowBox}

\begin{sphinxShadowBox}
\sphinxstyletopictitle{Example}

\sphinxAtStartPar
Google is a spin\sphinxhyphen{}off company of Stanford University; AGTC is a spin\sphinxhyphen{}off company of the University of Florida
\end{sphinxShadowBox}

\begin{sphinxShadowBox}
\sphinxstyletopictitle{Editor’s note}

\sphinxAtStartPar
To say that a is a spin\sphinxhyphen{}off of b implies there is a spin\sphinxhyphen{}off process and that a is the output of that spin\sphinxhyphen{}off process
\end{sphinxShadowBox}

\begin{sphinxShadowBox}
\sphinxstyletopictitle{Term editor}

\sphinxAtStartPar
Michael Conlon \sphinxurl{https://orcid.org/0000-0002-1304-8448}
\end{sphinxShadowBox}
\begin{quote}

\index{ORG\_2000013@\spxentry{ORG\_2000013}!spin\sphinxhyphen{}off organization of@\spxentry{spin\sphinxhyphen{}off organization of}}\index{spin\sphinxhyphen{}off organization of@\spxentry{spin\sphinxhyphen{}off organization of}!ORG\_2000013@\spxentry{ORG\_2000013}}\ignorespaces \end{quote}


\subsection{ORG\_2000013 \sphinxhyphen{} spin\sphinxhyphen{}off organization of}
\label{\detokenize{doc-ORG_2000013:org-2000013-spin-off-organization-of}}\label{\detokenize{doc-ORG_2000013:index-0}}\label{\detokenize{doc-ORG_2000013::doc}}
\begin{sphinxShadowBox}
\sphinxstyletopictitle{Label}

\sphinxAtStartPar
spin\sphinxhyphen{}off organization of
\end{sphinxShadowBox}

\begin{sphinxShadowBox}
\sphinxstyletopictitle{Alternate name}

\sphinxAtStartPar
spinout of
\end{sphinxShadowBox}

\begin{sphinxShadowBox}
\sphinxstyletopictitle{Definition}

\sphinxAtStartPar
The property that associates an organization with an organization of which it was formerly a part
\end{sphinxShadowBox}

\begin{sphinxShadowBox}
\sphinxstyletopictitle{Definition source}

\sphinxAtStartPar
Michael Conlon \sphinxurl{https://orcid.org/0000-0002-1304-8448}
\end{sphinxShadowBox}

\begin{sphinxShadowBox}
\sphinxstyletopictitle{Example}

\sphinxAtStartPar
Open Clinical is a spin\sphinxhyphen{}off of Oxford University. ToposNomos Ltd. is a spin\sphinxhyphen{}off company of the University of Magdeburg
\end{sphinxShadowBox}

\begin{sphinxShadowBox}
\sphinxstyletopictitle{Term editor}

\sphinxAtStartPar
Michael Conlon \sphinxurl{https://orcid.org/0000-0002-1304-8448}
\end{sphinxShadowBox}
\begin{quote}

\index{ORG\_2000014@\spxentry{ORG\_2000014}!has organizational member@\spxentry{has organizational member}}\index{has organizational member@\spxentry{has organizational member}!ORG\_2000014@\spxentry{ORG\_2000014}}\ignorespaces \end{quote}


\subsection{ORG\_2000014 \sphinxhyphen{} has organizational member}
\label{\detokenize{doc-ORG_2000014:org-2000014-has-organizational-member}}\label{\detokenize{doc-ORG_2000014:index-0}}\label{\detokenize{doc-ORG_2000014::doc}}
\begin{sphinxShadowBox}
\sphinxstyletopictitle{Label}

\sphinxAtStartPar
has organizational member
\end{sphinxShadowBox}

\begin{sphinxShadowBox}
\sphinxstyletopictitle{Definition}

\sphinxAtStartPar
The property that associates an organization with one of its members
\end{sphinxShadowBox}

\begin{sphinxShadowBox}
\sphinxstyletopictitle{Definition source}

\sphinxAtStartPar
Michael Conlon \sphinxurl{https://orcid.org/0000-0002-1304-8448}
\end{sphinxShadowBox}

\begin{sphinxShadowBox}
\sphinxstyletopictitle{Example}

\sphinxAtStartPar
The EU has member Republic of Ireland
\end{sphinxShadowBox}

\begin{sphinxShadowBox}
\sphinxstyletopictitle{Editor’s note}

\sphinxAtStartPar
To say that on organizaton is a member of another implies there is a membership and that the two organizations each have  roles that
\end{sphinxShadowBox}

\begin{sphinxShadowBox}
\sphinxstyletopictitle{Term editor}

\sphinxAtStartPar
Michael Conlon \sphinxurl{https://orcid.org/0000-0002-1304-8448}
\end{sphinxShadowBox}
\begin{quote}

\index{ORG\_2000015@\spxentry{ORG\_2000015}!organizational member of@\spxentry{organizational member of}}\index{organizational member of@\spxentry{organizational member of}!ORG\_2000015@\spxentry{ORG\_2000015}}\ignorespaces \end{quote}


\subsection{ORG\_2000015 \sphinxhyphen{} organizational member of}
\label{\detokenize{doc-ORG_2000015:org-2000015-organizational-member-of}}\label{\detokenize{doc-ORG_2000015:index-0}}\label{\detokenize{doc-ORG_2000015::doc}}
\begin{sphinxShadowBox}
\sphinxstyletopictitle{Label}

\sphinxAtStartPar
organizational member of
\end{sphinxShadowBox}

\begin{sphinxShadowBox}
\sphinxstyletopictitle{Definition}

\sphinxAtStartPar
The property that associates a member organizaiton with the organization of which it is a member
\end{sphinxShadowBox}

\begin{sphinxShadowBox}
\sphinxstyletopictitle{Definition source}

\sphinxAtStartPar
Michael Conlon \sphinxurl{https://orcid.org/0000-0002-1304-8448}
\end{sphinxShadowBox}

\begin{sphinxShadowBox}
\sphinxstyletopictitle{Example}

\sphinxAtStartPar
The University of North Carolina is a member of the Southeast University Research Association
\end{sphinxShadowBox}

\begin{sphinxShadowBox}
\sphinxstyletopictitle{Term editor}

\sphinxAtStartPar
Michael Conlon \sphinxurl{https://orcid.org/0000-0002-1304-8448}
\end{sphinxShadowBox}
\begin{quote}

\index{ORG\_2000016@\spxentry{ORG\_2000016}!has organizational employee@\spxentry{has organizational employee}}\index{has organizational employee@\spxentry{has organizational employee}!ORG\_2000016@\spxentry{ORG\_2000016}}\ignorespaces \end{quote}


\subsection{ORG\_2000016 \sphinxhyphen{} has organizational employee}
\label{\detokenize{doc-ORG_2000016:org-2000016-has-organizational-employee}}\label{\detokenize{doc-ORG_2000016:index-0}}\label{\detokenize{doc-ORG_2000016::doc}}
\begin{sphinxShadowBox}
\sphinxstyletopictitle{Label}

\sphinxAtStartPar
has organizational employee
\end{sphinxShadowBox}

\begin{sphinxShadowBox}
\sphinxstyletopictitle{Definition}

\sphinxAtStartPar
The property that assocates an organization with an employee of the organization
\end{sphinxShadowBox}

\begin{sphinxShadowBox}
\sphinxstyletopictitle{Definition source}

\sphinxAtStartPar
Michael Conlon \sphinxurl{https://orcid.org/0000-0002-1304-8448}
\end{sphinxShadowBox}

\begin{sphinxShadowBox}
\sphinxstyletopictitle{Example}

\sphinxAtStartPar
For the season ending in 2021, the Los Angeles Lakers have employee LeBron James.
\end{sphinxShadowBox}

\begin{sphinxShadowBox}
\sphinxstyletopictitle{Similar term in VIVO 1 Ontology}

\sphinxAtStartPar
\sphinxurl{http://purl.obolibrary.org/obo/ERO\_0000787}
\end{sphinxShadowBox}

\begin{sphinxShadowBox}
\sphinxstyletopictitle{Term editor}

\sphinxAtStartPar
Michael Conlon \sphinxurl{https://orcid.org/0000-0002-1304-8448}
\end{sphinxShadowBox}
\begin{quote}

\index{ORG\_2000017@\spxentry{ORG\_2000017}!organizational employee of@\spxentry{organizational employee of}}\index{organizational employee of@\spxentry{organizational employee of}!ORG\_2000017@\spxentry{ORG\_2000017}}\ignorespaces \end{quote}


\subsection{ORG\_2000017 \sphinxhyphen{} organizational employee of}
\label{\detokenize{doc-ORG_2000017:org-2000017-organizational-employee-of}}\label{\detokenize{doc-ORG_2000017:index-0}}\label{\detokenize{doc-ORG_2000017::doc}}
\begin{sphinxShadowBox}
\sphinxstyletopictitle{Label}

\sphinxAtStartPar
organizational employee of
\end{sphinxShadowBox}

\begin{sphinxShadowBox}
\sphinxstyletopictitle{Definition}

\sphinxAtStartPar
The property that associates a person with the organization of which that person is an employee
\end{sphinxShadowBox}

\begin{sphinxShadowBox}
\sphinxstyletopictitle{Definition source}

\sphinxAtStartPar
Michael Conlon \sphinxurl{https://orcid.org/0000-0002-1304-8448}
\end{sphinxShadowBox}

\begin{sphinxShadowBox}
\sphinxstyletopictitle{Example}

\sphinxAtStartPar
The president of a university is typically an employee of the university
\end{sphinxShadowBox}

\begin{sphinxShadowBox}
\sphinxstyletopictitle{Similar term in VIVO 1 Ontology}

\sphinxAtStartPar
\sphinxurl{http://purl.obolibrary.org/obo/ERO\_0000787}
\end{sphinxShadowBox}

\begin{sphinxShadowBox}
\sphinxstyletopictitle{Term editor}

\sphinxAtStartPar
Michael Conlon \sphinxurl{https://orcid.org/0000-0002-1304-8448}
\end{sphinxShadowBox}
\begin{quote}

\index{ORG\_2000018@\spxentry{ORG\_2000018}!has organizational associate@\spxentry{has organizational associate}}\index{has organizational associate@\spxentry{has organizational associate}!ORG\_2000018@\spxentry{ORG\_2000018}}\ignorespaces \end{quote}


\subsection{ORG\_2000018 \sphinxhyphen{} has organizational associate}
\label{\detokenize{doc-ORG_2000018:org-2000018-has-organizational-associate}}\label{\detokenize{doc-ORG_2000018:index-0}}\label{\detokenize{doc-ORG_2000018::doc}}
\begin{sphinxShadowBox}
\sphinxstyletopictitle{Label}

\sphinxAtStartPar
has organizational associate
\end{sphinxShadowBox}

\begin{sphinxShadowBox}
\sphinxstyletopictitle{Definition}

\sphinxAtStartPar
The property that associates an organization with a person who is associated with the organization
\end{sphinxShadowBox}

\begin{sphinxShadowBox}
\sphinxstyletopictitle{Definition source}

\sphinxAtStartPar
Michael Conlon \sphinxurl{https://orcid.org/0000-0002-1304-8448}
\end{sphinxShadowBox}

\begin{sphinxShadowBox}
\sphinxstyletopictitle{Example}

\sphinxAtStartPar
One might say that an organization has associates who are their former employees
\end{sphinxShadowBox}

\begin{sphinxShadowBox}
\sphinxstyletopictitle{Editor’s note}

\sphinxAtStartPar
Associate is intentionally loose.  A person can be associated with an organization in many ways.  Subsequent additional properties may be needed to clarify.
\end{sphinxShadowBox}

\begin{sphinxShadowBox}
\sphinxstyletopictitle{Term editor}

\sphinxAtStartPar
Michael Conlon \sphinxurl{https://orcid.org/0000-0002-1304-8448}
\end{sphinxShadowBox}
\begin{quote}

\index{ORG\_2000019@\spxentry{ORG\_2000019}!organizational associate of@\spxentry{organizational associate of}}\index{organizational associate of@\spxentry{organizational associate of}!ORG\_2000019@\spxentry{ORG\_2000019}}\ignorespaces \end{quote}


\subsection{ORG\_2000019 \sphinxhyphen{} organizational associate of}
\label{\detokenize{doc-ORG_2000019:org-2000019-organizational-associate-of}}\label{\detokenize{doc-ORG_2000019:index-0}}\label{\detokenize{doc-ORG_2000019::doc}}
\begin{sphinxShadowBox}
\sphinxstyletopictitle{Label}

\sphinxAtStartPar
organizational associate of
\end{sphinxShadowBox}

\begin{sphinxShadowBox}
\sphinxstyletopictitle{Definition}

\sphinxAtStartPar
The property that associates a person with an organization with which they are associated
\end{sphinxShadowBox}

\begin{sphinxShadowBox}
\sphinxstyletopictitle{Definition source}

\sphinxAtStartPar
Michael Conlon \sphinxurl{https://orcid.org/0000-0002-1304-8448}
\end{sphinxShadowBox}

\begin{sphinxShadowBox}
\sphinxstyletopictitle{Example}

\sphinxAtStartPar
One might say that people who have reviewed articles for a journal are associateds of that journal
\end{sphinxShadowBox}

\begin{sphinxShadowBox}
\sphinxstyletopictitle{Term editor}

\sphinxAtStartPar
Michael Conlon \sphinxurl{https://orcid.org/0000-0002-1304-8448}
\end{sphinxShadowBox}
\begin{quote}

\index{ORG\_2000020@\spxentry{ORG\_2000020}!has organizational head@\spxentry{has organizational head}}\index{has organizational head@\spxentry{has organizational head}!ORG\_2000020@\spxentry{ORG\_2000020}}\ignorespaces \end{quote}


\subsection{ORG\_2000020 \sphinxhyphen{} has organizational head}
\label{\detokenize{doc-ORG_2000020:org-2000020-has-organizational-head}}\label{\detokenize{doc-ORG_2000020:index-0}}\label{\detokenize{doc-ORG_2000020::doc}}
\begin{sphinxShadowBox}
\sphinxstyletopictitle{Label}

\sphinxAtStartPar
has organizational head
\end{sphinxShadowBox}

\begin{sphinxShadowBox}
\sphinxstyletopictitle{Definition}

\sphinxAtStartPar
The property that associates an organziation with the person who is the head of the organization
\end{sphinxShadowBox}

\begin{sphinxShadowBox}
\sphinxstyletopictitle{Definition source}

\sphinxAtStartPar
Michael Conlon \sphinxurl{https://orcid.org/0000-0002-1304-8448}
\end{sphinxShadowBox}

\begin{sphinxShadowBox}
\sphinxstyletopictitle{Example}

\sphinxAtStartPar
As of this writing, Google has head Sundar Pichai
\end{sphinxShadowBox}

\begin{sphinxShadowBox}
\sphinxstyletopictitle{Similar term in VIVO 1 Ontology}

\sphinxAtStartPar
\sphinxurl{http://vivoweb.org/ontology/core\#LeaderRole}
\end{sphinxShadowBox}

\begin{sphinxShadowBox}
\sphinxstyletopictitle{Term editor}

\sphinxAtStartPar
Michael Conlon \sphinxurl{https://orcid.org/0000-0002-1304-8448}
\end{sphinxShadowBox}
\begin{quote}

\index{ORG\_2000021@\spxentry{ORG\_2000021}!organizational head of@\spxentry{organizational head of}}\index{organizational head of@\spxentry{organizational head of}!ORG\_2000021@\spxentry{ORG\_2000021}}\ignorespaces \end{quote}


\subsection{ORG\_2000021 \sphinxhyphen{} organizational head of}
\label{\detokenize{doc-ORG_2000021:org-2000021-organizational-head-of}}\label{\detokenize{doc-ORG_2000021:index-0}}\label{\detokenize{doc-ORG_2000021::doc}}
\begin{sphinxShadowBox}
\sphinxstyletopictitle{Label}

\sphinxAtStartPar
organizational head of
\end{sphinxShadowBox}

\begin{sphinxShadowBox}
\sphinxstyletopictitle{Definition}

\sphinxAtStartPar
The property that associates a person with an organization which they are they head of
\end{sphinxShadowBox}

\begin{sphinxShadowBox}
\sphinxstyletopictitle{Definition source}

\sphinxAtStartPar
Michael Conlon \sphinxurl{https://orcid.org/0000-0002-1304-8448}
\end{sphinxShadowBox}

\begin{sphinxShadowBox}
\sphinxstyletopictitle{Example}

\sphinxAtStartPar
As of this writing, Elon Musk is head of Tesla
\end{sphinxShadowBox}

\begin{sphinxShadowBox}
\sphinxstyletopictitle{Term editor}

\sphinxAtStartPar
Michael Conlon \sphinxurl{https://orcid.org/0000-0002-1304-8448}
\end{sphinxShadowBox}
\begin{quote}

\index{ORG\_2000022@\spxentry{ORG\_2000022}!has organizational volunteer@\spxentry{has organizational volunteer}}\index{has organizational volunteer@\spxentry{has organizational volunteer}!ORG\_2000022@\spxentry{ORG\_2000022}}\ignorespaces \end{quote}


\subsection{ORG\_2000022 \sphinxhyphen{} has organizational volunteer}
\label{\detokenize{doc-ORG_2000022:org-2000022-has-organizational-volunteer}}\label{\detokenize{doc-ORG_2000022:index-0}}\label{\detokenize{doc-ORG_2000022::doc}}
\begin{sphinxShadowBox}
\sphinxstyletopictitle{Label}

\sphinxAtStartPar
has organizational volunteer
\end{sphinxShadowBox}

\begin{sphinxShadowBox}
\sphinxstyletopictitle{Definition}

\sphinxAtStartPar
The property that associates an organization with a person who volunteers for the organization
\end{sphinxShadowBox}

\begin{sphinxShadowBox}
\sphinxstyletopictitle{Definition source}

\sphinxAtStartPar
Michael Conlon \sphinxurl{https://orcid.org/0000-0002-1304-8448}
\end{sphinxShadowBox}

\begin{sphinxShadowBox}
\sphinxstyletopictitle{Example}

\sphinxAtStartPar
Volunteer fire departments have firefighters who are volunteers of the fire department
\end{sphinxShadowBox}

\begin{sphinxShadowBox}
\sphinxstyletopictitle{Term editor}

\sphinxAtStartPar
Michael Conlon \sphinxurl{https://orcid.org/0000-0002-1304-8448}
\end{sphinxShadowBox}
\begin{quote}

\index{ORG\_2000023@\spxentry{ORG\_2000023}!organizational volunteer of@\spxentry{organizational volunteer of}}\index{organizational volunteer of@\spxentry{organizational volunteer of}!ORG\_2000023@\spxentry{ORG\_2000023}}\ignorespaces \end{quote}


\subsection{ORG\_2000023 \sphinxhyphen{} organizational volunteer of}
\label{\detokenize{doc-ORG_2000023:org-2000023-organizational-volunteer-of}}\label{\detokenize{doc-ORG_2000023:index-0}}\label{\detokenize{doc-ORG_2000023::doc}}
\begin{sphinxShadowBox}
\sphinxstyletopictitle{Label}

\sphinxAtStartPar
organizational volunteer of
\end{sphinxShadowBox}

\begin{sphinxShadowBox}
\sphinxstyletopictitle{Definition}

\sphinxAtStartPar
The property that associates a person with an organization for which they volunteer
\end{sphinxShadowBox}

\begin{sphinxShadowBox}
\sphinxstyletopictitle{Definition source}

\sphinxAtStartPar
Michael Conlon \sphinxurl{https://orcid.org/0000-0002-1304-8448}
\end{sphinxShadowBox}

\begin{sphinxShadowBox}
\sphinxstyletopictitle{Example}

\sphinxAtStartPar
Many Americans have served as volunteers of the Peace Corps
\end{sphinxShadowBox}

\begin{sphinxShadowBox}
\sphinxstyletopictitle{Term editor}

\sphinxAtStartPar
Michael Conlon \sphinxurl{https://orcid.org/0000-0002-1304-8448}
\end{sphinxShadowBox}
\begin{quote}

\index{ORG\_2000024@\spxentry{ORG\_2000024}!has organizational appointee@\spxentry{has organizational appointee}}\index{has organizational appointee@\spxentry{has organizational appointee}!ORG\_2000024@\spxentry{ORG\_2000024}}\ignorespaces \end{quote}


\subsection{ORG\_2000024 \sphinxhyphen{} has organizational appointee}
\label{\detokenize{doc-ORG_2000024:org-2000024-has-organizational-appointee}}\label{\detokenize{doc-ORG_2000024:index-0}}\label{\detokenize{doc-ORG_2000024::doc}}
\begin{sphinxShadowBox}
\sphinxstyletopictitle{Label}

\sphinxAtStartPar
has organizational appointee
\end{sphinxShadowBox}

\begin{sphinxShadowBox}
\sphinxstyletopictitle{Definition}

\sphinxAtStartPar
The property that associates an organization with a person who is appointed by the organization to some post or position within the organization
\end{sphinxShadowBox}

\begin{sphinxShadowBox}
\sphinxstyletopictitle{Definition source}

\sphinxAtStartPar
Michael Conlon \sphinxurl{https://orcid.org/0000-0002-1304-8448}
\end{sphinxShadowBox}

\begin{sphinxShadowBox}
\sphinxstyletopictitle{Example}

\sphinxAtStartPar
As of this writing, the government of the UK has appointed Amanda Milling Minister without Portfolio (unpaid)
\end{sphinxShadowBox}

\begin{sphinxShadowBox}
\sphinxstyletopictitle{Editor’s note}

\sphinxAtStartPar
Appointee typically includes a title, may or may not include a salary.
\end{sphinxShadowBox}

\begin{sphinxShadowBox}
\sphinxstyletopictitle{Term editor}

\sphinxAtStartPar
Michael Conlon \sphinxurl{https://orcid.org/0000-0002-1304-8448}
\end{sphinxShadowBox}
\begin{quote}

\index{ORG\_2000025@\spxentry{ORG\_2000025}!organizational appointee of@\spxentry{organizational appointee of}}\index{organizational appointee of@\spxentry{organizational appointee of}!ORG\_2000025@\spxentry{ORG\_2000025}}\ignorespaces \end{quote}


\subsection{ORG\_2000025 \sphinxhyphen{} organizational appointee of}
\label{\detokenize{doc-ORG_2000025:org-2000025-organizational-appointee-of}}\label{\detokenize{doc-ORG_2000025:index-0}}\label{\detokenize{doc-ORG_2000025::doc}}
\begin{sphinxShadowBox}
\sphinxstyletopictitle{Label}

\sphinxAtStartPar
organizational appointee of
\end{sphinxShadowBox}

\begin{sphinxShadowBox}
\sphinxstyletopictitle{Definition}

\sphinxAtStartPar
The property that associates a person with the organization which has appointed the person to some post or position within the organization
\end{sphinxShadowBox}

\begin{sphinxShadowBox}
\sphinxstyletopictitle{Definition source}

\sphinxAtStartPar
Michael Conlon \sphinxurl{https://orcid.org/0000-0002-1304-8448}
\end{sphinxShadowBox}

\begin{sphinxShadowBox}
\sphinxstyletopictitle{Example}

\sphinxAtStartPar
The board of trustees of a university may appoint the university president
\end{sphinxShadowBox}

\begin{sphinxShadowBox}
\sphinxstyletopictitle{Term editor}

\sphinxAtStartPar
Michael Conlon \sphinxurl{https://orcid.org/0000-0002-1304-8448}
\end{sphinxShadowBox}
\begin{quote}

\index{ORG\_2000026@\spxentry{ORG\_2000026}!has person member@\spxentry{has person member}}\index{has person member@\spxentry{has person member}!ORG\_2000026@\spxentry{ORG\_2000026}}\ignorespaces \end{quote}


\subsection{ORG\_2000026 \sphinxhyphen{} has person member}
\label{\detokenize{doc-ORG_2000026:org-2000026-has-person-member}}\label{\detokenize{doc-ORG_2000026:index-0}}\label{\detokenize{doc-ORG_2000026::doc}}
\begin{sphinxShadowBox}
\sphinxstyletopictitle{Label}

\sphinxAtStartPar
has person member
\end{sphinxShadowBox}

\begin{sphinxShadowBox}
\sphinxstyletopictitle{Definition}

\sphinxAtStartPar
The property that associates an organization with a person who is  a member of the organization
\end{sphinxShadowBox}

\begin{sphinxShadowBox}
\sphinxstyletopictitle{Definition source}

\sphinxAtStartPar
Michael Conlon \sphinxurl{https://orcid.org/0000-0002-1304-8448}
\end{sphinxShadowBox}

\begin{sphinxShadowBox}
\sphinxstyletopictitle{Example}

\sphinxAtStartPar
As of 2021, The Conservative Party of the UK has person member Boris Johnson
\end{sphinxShadowBox}

\begin{sphinxShadowBox}
\sphinxstyletopictitle{Term editor}

\sphinxAtStartPar
Michael Conlon \sphinxurl{https://orcid.org/0000-0002-1304-8448}
\end{sphinxShadowBox}
\begin{quote}

\index{ORG\_2000027@\spxentry{ORG\_2000027}!person member of@\spxentry{person member of}}\index{person member of@\spxentry{person member of}!ORG\_2000027@\spxentry{ORG\_2000027}}\ignorespaces \end{quote}


\subsection{ORG\_2000027 \sphinxhyphen{} person member of}
\label{\detokenize{doc-ORG_2000027:org-2000027-person-member-of}}\label{\detokenize{doc-ORG_2000027:index-0}}\label{\detokenize{doc-ORG_2000027::doc}}
\begin{sphinxShadowBox}
\sphinxstyletopictitle{Label}

\sphinxAtStartPar
person member of
\end{sphinxShadowBox}

\begin{sphinxShadowBox}
\sphinxstyletopictitle{Definition}

\sphinxAtStartPar
The property that associates a person with an organization of which the person is a member
\end{sphinxShadowBox}

\begin{sphinxShadowBox}
\sphinxstyletopictitle{Definition source}

\sphinxAtStartPar
Michael Conlon \sphinxurl{https://orcid.org/0000-0002-1304-8448}
\end{sphinxShadowBox}

\begin{sphinxShadowBox}
\sphinxstyletopictitle{Example}

\sphinxAtStartPar
Florence Nightingale was a person member of the American Statistical Association
\end{sphinxShadowBox}

\begin{sphinxShadowBox}
\sphinxstyletopictitle{Term editor}

\sphinxAtStartPar
Michael Conlon \sphinxurl{https://orcid.org/0000-0002-1304-8448}
\end{sphinxShadowBox}
\begin{quote}

\index{RO\_0000052@\spxentry{RO\_0000052}!inheres in@\spxentry{inheres in}}\index{inheres in@\spxentry{inheres in}!RO\_0000052@\spxentry{RO\_0000052}}\ignorespaces \end{quote}


\subsection{RO\_0000052 \sphinxhyphen{} inheres in}
\label{\detokenize{doc-RO_0000052:ro-0000052-inheres-in}}\label{\detokenize{doc-RO_0000052:index-0}}\label{\detokenize{doc-RO_0000052::doc}}
\begin{sphinxShadowBox}
\sphinxstyletopictitle{Label}

\sphinxAtStartPar
inheres in
\end{sphinxShadowBox}

\begin{sphinxShadowBox}
\sphinxstyletopictitle{Alternate name}

\sphinxAtStartPar
inheres\_in
\end{sphinxShadowBox}

\begin{sphinxShadowBox}
\sphinxstyletopictitle{Definition}

\sphinxAtStartPar
A relation between a specifically dependent continuant (the dependent) and an independent continuant (the bearer), in which the dependent specifically depends on the bearer for its existence
\end{sphinxShadowBox}

\begin{sphinxShadowBox}
\sphinxstyletopictitle{Example}

\sphinxAtStartPar
this red color inheres in this apple

\sphinxAtStartPar
this fragility inheres in this vase
\end{sphinxShadowBox}

\begin{sphinxShadowBox}
\sphinxstyletopictitle{Editor’s note}

\sphinxAtStartPar
A dependent inheres in its bearer at all times for which the dependent exists.
\end{sphinxShadowBox}

\begin{sphinxShadowBox}
\sphinxstyletopictitle{Imported From}

\sphinxAtStartPar
\sphinxurl{http://purl.obolibrary.org/obo/ro/releases/2021-03-08/ro.owl}
\end{sphinxShadowBox}
\begin{quote}

\index{RO\_0000053@\spxentry{RO\_0000053}!bearer of@\spxentry{bearer of}}\index{bearer of@\spxentry{bearer of}!RO\_0000053@\spxentry{RO\_0000053}}\ignorespaces \end{quote}


\subsection{RO\_0000053 \sphinxhyphen{} bearer of}
\label{\detokenize{doc-RO_0000053:ro-0000053-bearer-of}}\label{\detokenize{doc-RO_0000053:index-0}}\label{\detokenize{doc-RO_0000053::doc}}
\begin{sphinxShadowBox}
\sphinxstyletopictitle{Label}

\sphinxAtStartPar
bearer of
\end{sphinxShadowBox}

\begin{sphinxShadowBox}
\sphinxstyletopictitle{Definition}

\sphinxAtStartPar
A relation between an independent continuant (the bearer) and a specifically dependent continuant (the dependent), in which the dependent specifically depends on the bearer for its existence
\end{sphinxShadowBox}

\begin{sphinxShadowBox}
\sphinxstyletopictitle{Example}

\sphinxAtStartPar
this apple is bearer of this red color

\sphinxAtStartPar
this vase is bearer of this fragility
\end{sphinxShadowBox}

\begin{sphinxShadowBox}
\sphinxstyletopictitle{Editor’s note}

\sphinxAtStartPar
A bearer can have many dependents, and its dependents can exist for different periods of time, but none of its dependents can exist when the bearer does not exist.
\end{sphinxShadowBox}

\begin{sphinxShadowBox}
\sphinxstyletopictitle{Imported From}

\sphinxAtStartPar
\sphinxurl{http://purl.obolibrary.org/obo/ro/releases/2020-12-18/ro.owl}
\end{sphinxShadowBox}
\begin{quote}

\index{RO\_0000056@\spxentry{RO\_0000056}!participates in@\spxentry{participates in}}\index{participates in@\spxentry{participates in}!RO\_0000056@\spxentry{RO\_0000056}}\ignorespaces \end{quote}


\subsection{RO\_0000056 \sphinxhyphen{} participates in}
\label{\detokenize{doc-RO_0000056:ro-0000056-participates-in}}\label{\detokenize{doc-RO_0000056:index-0}}\label{\detokenize{doc-RO_0000056::doc}}
\begin{sphinxShadowBox}
\sphinxstyletopictitle{Label}

\sphinxAtStartPar
participates in
\end{sphinxShadowBox}

\begin{sphinxShadowBox}
\sphinxstyletopictitle{Definition}

\sphinxAtStartPar
A relation between a continuant and a process, in which the continuant is somehow involved in the process
\end{sphinxShadowBox}

\begin{sphinxShadowBox}
\sphinxstyletopictitle{Example}

\sphinxAtStartPar
this input material (or this output material) participates in this process

\sphinxAtStartPar
this blood clot participates in this blood coagulation

\sphinxAtStartPar
this investigator participates in this investigation
\end{sphinxShadowBox}

\begin{sphinxShadowBox}
\sphinxstyletopictitle{Imported From}

\sphinxAtStartPar
\sphinxurl{http://purl.obolibrary.org/obo/ro/releases/2020-12-18/ro.owl}
\end{sphinxShadowBox}
\begin{quote}

\index{RO\_0000057@\spxentry{RO\_0000057}!has participant@\spxentry{has participant}}\index{has participant@\spxentry{has participant}!RO\_0000057@\spxentry{RO\_0000057}}\ignorespaces \end{quote}


\subsection{RO\_0000057 \sphinxhyphen{} has participant}
\label{\detokenize{doc-RO_0000057:ro-0000057-has-participant}}\label{\detokenize{doc-RO_0000057:index-0}}\label{\detokenize{doc-RO_0000057::doc}}
\begin{sphinxShadowBox}
\sphinxstyletopictitle{Label}

\sphinxAtStartPar
has participant
\end{sphinxShadowBox}

\begin{sphinxShadowBox}
\sphinxstyletopictitle{Definition}

\sphinxAtStartPar
A relation between a process and a continuant, in which the continuant is somehow involved in the process
\end{sphinxShadowBox}

\begin{sphinxShadowBox}
\sphinxstyletopictitle{Example}

\sphinxAtStartPar
this investigation has participant this investigator

\sphinxAtStartPar
this blood coagulation has participant this blood clot

\sphinxAtStartPar
this process has participant this input material (or this output material)
\end{sphinxShadowBox}

\begin{sphinxShadowBox}
\sphinxstyletopictitle{Editor’s note}

\sphinxAtStartPar
Has\_participant is a primitive instance\sphinxhyphen{}level relation between a process, a continuant, and a time at which the continuant participates in some way in the process. The relation obtains, for example, when this particular process of oxygen exchange across this particular alveolar membrane has\_participant this particular sample of hemoglobin at this particular time.
\end{sphinxShadowBox}

\begin{sphinxShadowBox}
\sphinxstyletopictitle{Imported From}

\sphinxAtStartPar
\sphinxurl{http://purl.obolibrary.org/obo/ro/releases/2020-12-18/ro.owl}
\end{sphinxShadowBox}
\begin{quote}

\index{RO\_0000080@\spxentry{RO\_0000080}!quality of@\spxentry{quality of}}\index{quality of@\spxentry{quality of}!RO\_0000080@\spxentry{RO\_0000080}}\ignorespaces \end{quote}


\subsection{RO\_0000080 \sphinxhyphen{} quality of}
\label{\detokenize{doc-RO_0000080:ro-0000080-quality-of}}\label{\detokenize{doc-RO_0000080:index-0}}\label{\detokenize{doc-RO_0000080::doc}}
\begin{sphinxShadowBox}
\sphinxstyletopictitle{Label}

\sphinxAtStartPar
quality of
\end{sphinxShadowBox}

\begin{sphinxShadowBox}
\sphinxstyletopictitle{Alternate name}

\sphinxAtStartPar
quality\_of

\sphinxAtStartPar
is quality of
\end{sphinxShadowBox}

\begin{sphinxShadowBox}
\sphinxstyletopictitle{Definition}

\sphinxAtStartPar
A relation between a quality and an independent continuant (the bearer), in which the quality specifically depends on the bearer for its existence
\end{sphinxShadowBox}

\begin{sphinxShadowBox}
\sphinxstyletopictitle{Example}

\sphinxAtStartPar
this red color is a quality of this apple
\end{sphinxShadowBox}

\begin{sphinxShadowBox}
\sphinxstyletopictitle{Editor’s note}

\sphinxAtStartPar
A quality inheres in its bearer at all times for which the quality exists.
\end{sphinxShadowBox}

\begin{sphinxShadowBox}
\sphinxstyletopictitle{Imported From}

\sphinxAtStartPar
\sphinxurl{http://purl.obolibrary.org/obo/ro/releases/2021-03-08/ro.owl}
\end{sphinxShadowBox}
\begin{quote}

\index{RO\_0000081@\spxentry{RO\_0000081}!role of@\spxentry{role of}}\index{role of@\spxentry{role of}!RO\_0000081@\spxentry{RO\_0000081}}\ignorespaces \end{quote}


\subsection{RO\_0000081 \sphinxhyphen{} role of}
\label{\detokenize{doc-RO_0000081:ro-0000081-role-of}}\label{\detokenize{doc-RO_0000081:index-0}}\label{\detokenize{doc-RO_0000081::doc}}
\begin{sphinxShadowBox}
\sphinxstyletopictitle{Label}

\sphinxAtStartPar
role of
\end{sphinxShadowBox}

\begin{sphinxShadowBox}
\sphinxstyletopictitle{Alternate name}

\sphinxAtStartPar
role\_of

\sphinxAtStartPar
is role of
\end{sphinxShadowBox}

\begin{sphinxShadowBox}
\sphinxstyletopictitle{Definition}

\sphinxAtStartPar
A relation between a role and an independent continuant (the bearer), in which the role specifically depends on the bearer for its existence
\end{sphinxShadowBox}

\begin{sphinxShadowBox}
\sphinxstyletopictitle{Example}

\sphinxAtStartPar
This investigator role is a role of this person
\end{sphinxShadowBox}

\begin{sphinxShadowBox}
\sphinxstyletopictitle{Editor’s note}

\sphinxAtStartPar
A role inheres in its bearer at all times for which the role exists, however the role need not be realized at all the times that the role exists.
\end{sphinxShadowBox}

\begin{sphinxShadowBox}
\sphinxstyletopictitle{Imported From}

\sphinxAtStartPar
\sphinxurl{http://purl.obolibrary.org/obo/ro/releases/2021-03-08/ro.owl}
\end{sphinxShadowBox}
\begin{quote}

\index{RO\_0000086@\spxentry{RO\_0000086}!has quality@\spxentry{has quality}}\index{has quality@\spxentry{has quality}!RO\_0000086@\spxentry{RO\_0000086}}\ignorespaces \end{quote}


\subsection{RO\_0000086 \sphinxhyphen{} has quality}
\label{\detokenize{doc-RO_0000086:ro-0000086-has-quality}}\label{\detokenize{doc-RO_0000086:index-0}}\label{\detokenize{doc-RO_0000086::doc}}
\begin{sphinxShadowBox}
\sphinxstyletopictitle{Label}

\sphinxAtStartPar
has quality
\end{sphinxShadowBox}

\begin{sphinxShadowBox}
\sphinxstyletopictitle{Alternate name}

\sphinxAtStartPar
has\_quality
\end{sphinxShadowBox}

\begin{sphinxShadowBox}
\sphinxstyletopictitle{Definition}

\sphinxAtStartPar
A relation between an independent continuant (the bearer) and a quality, in which the quality specifically depends on the bearer for its existence
\end{sphinxShadowBox}

\begin{sphinxShadowBox}
\sphinxstyletopictitle{Example}

\sphinxAtStartPar
this apple has quality this red color
\end{sphinxShadowBox}

\begin{sphinxShadowBox}
\sphinxstyletopictitle{Editor’s note}

\sphinxAtStartPar
A bearer can have many qualities, and its qualities can exist for different periods of time, but none of its qualities can exist when the bearer does not exist.
\end{sphinxShadowBox}

\begin{sphinxShadowBox}
\sphinxstyletopictitle{Imported From}

\sphinxAtStartPar
\sphinxurl{http://purl.obolibrary.org/obo/ro/releases/2021-03-08/ro.owl}
\end{sphinxShadowBox}
\begin{quote}

\index{RO\_0000087@\spxentry{RO\_0000087}!has role@\spxentry{has role}}\index{has role@\spxentry{has role}!RO\_0000087@\spxentry{RO\_0000087}}\ignorespaces \end{quote}


\subsection{RO\_0000087 \sphinxhyphen{} has role}
\label{\detokenize{doc-RO_0000087:ro-0000087-has-role}}\label{\detokenize{doc-RO_0000087:index-0}}\label{\detokenize{doc-RO_0000087::doc}}
\begin{sphinxShadowBox}
\sphinxstyletopictitle{Label}

\sphinxAtStartPar
has role
\end{sphinxShadowBox}

\begin{sphinxShadowBox}
\sphinxstyletopictitle{Alternate name}

\sphinxAtStartPar
has\_role
\end{sphinxShadowBox}

\begin{sphinxShadowBox}
\sphinxstyletopictitle{Definition}

\sphinxAtStartPar
A relation between an independent continuant (the bearer) and a role, in which the role specifically depends on the bearer for its existence
\end{sphinxShadowBox}

\begin{sphinxShadowBox}
\sphinxstyletopictitle{Example}

\sphinxAtStartPar
This person has role this investigator role (more colloquially: this person has this role of investigator)
\end{sphinxShadowBox}

\begin{sphinxShadowBox}
\sphinxstyletopictitle{Editor’s note}

\sphinxAtStartPar
A bearer can have many roles, and its roles can exist for different periods of time, but none of its roles can exist when the bearer does not exist. A role need not be realized at all the times that the role exists.
\end{sphinxShadowBox}

\begin{sphinxShadowBox}
\sphinxstyletopictitle{Imported From}

\sphinxAtStartPar
\sphinxurl{http://purl.obolibrary.org/obo/ro/releases/2021-03-08/ro.owl}
\end{sphinxShadowBox}
\begin{quote}

\index{RO\_0000091@\spxentry{RO\_0000091}!has disposition@\spxentry{has disposition}}\index{has disposition@\spxentry{has disposition}!RO\_0000091@\spxentry{RO\_0000091}}\ignorespaces \end{quote}


\subsection{RO\_0000091 \sphinxhyphen{} has disposition}
\label{\detokenize{doc-RO_0000091:ro-0000091-has-disposition}}\label{\detokenize{doc-RO_0000091:index-0}}\label{\detokenize{doc-RO_0000091::doc}}
\begin{sphinxShadowBox}
\sphinxstyletopictitle{Label}

\sphinxAtStartPar
has disposition
\end{sphinxShadowBox}

\begin{sphinxShadowBox}
\sphinxstyletopictitle{Definition}

\sphinxAtStartPar
A relation between an independent continuant (the bearer) and a disposition, in which the disposition specifically depends on the bearer for its existence
\end{sphinxShadowBox}

\begin{sphinxShadowBox}
\sphinxstyletopictitle{Imported From}

\sphinxAtStartPar
\sphinxurl{http://purl.obolibrary.org/obo/ro/releases/2020-12-18/ro.owl}
\end{sphinxShadowBox}
\begin{quote}

\index{RO\_0000092@\spxentry{RO\_0000092}!disposition of@\spxentry{disposition of}}\index{disposition of@\spxentry{disposition of}!RO\_0000092@\spxentry{RO\_0000092}}\ignorespaces \end{quote}


\subsection{RO\_0000092 \sphinxhyphen{} disposition of}
\label{\detokenize{doc-RO_0000092:ro-0000092-disposition-of}}\label{\detokenize{doc-RO_0000092:index-0}}\label{\detokenize{doc-RO_0000092::doc}}
\begin{sphinxShadowBox}
\sphinxstyletopictitle{Label}

\sphinxAtStartPar
disposition of
\end{sphinxShadowBox}

\begin{sphinxShadowBox}
\sphinxstyletopictitle{Definition}

\sphinxAtStartPar
Inverse of has disposition
\end{sphinxShadowBox}

\begin{sphinxShadowBox}
\sphinxstyletopictitle{Imported From}

\sphinxAtStartPar
\sphinxurl{http://purl.obolibrary.org/obo/ro/releases/2021-03-08/ro.owl}
\end{sphinxShadowBox}
\begin{quote}

\index{RO\_0001015@\spxentry{RO\_0001015}!location of@\spxentry{location of}}\index{location of@\spxentry{location of}!RO\_0001015@\spxentry{RO\_0001015}}\ignorespaces \end{quote}


\subsection{RO\_0001015 \sphinxhyphen{} location of}
\label{\detokenize{doc-RO_0001015:ro-0001015-location-of}}\label{\detokenize{doc-RO_0001015:index-0}}\label{\detokenize{doc-RO_0001015::doc}}
\begin{sphinxShadowBox}
\sphinxstyletopictitle{Label}

\sphinxAtStartPar
location of
\end{sphinxShadowBox}

\begin{sphinxShadowBox}
\sphinxstyletopictitle{Definition}

\sphinxAtStartPar
A relation between two independent continuants, the location and the target, in which the target is entirely within the location
\end{sphinxShadowBox}

\begin{sphinxShadowBox}
\sphinxstyletopictitle{Example}

\sphinxAtStartPar
this cage is the location of this rat

\sphinxAtStartPar
my head is the location of my brain
\end{sphinxShadowBox}

\begin{sphinxShadowBox}
\sphinxstyletopictitle{Editor’s note}

\sphinxAtStartPar
Most location relations will only hold at certain times, but this is difficult to specify in OWL. See \sphinxurl{https://code.google.com/p/obo-relations/wiki/ROAndTime}
\end{sphinxShadowBox}

\begin{sphinxShadowBox}
\sphinxstyletopictitle{Imported From}

\sphinxAtStartPar
\sphinxurl{http://purl.obolibrary.org/obo/ro/releases/2021-03-08/ro.owl}
\end{sphinxShadowBox}
\begin{quote}

\index{RO\_0001025@\spxentry{RO\_0001025}!located in@\spxentry{located in}}\index{located in@\spxentry{located in}!RO\_0001025@\spxentry{RO\_0001025}}\ignorespaces \end{quote}


\subsection{RO\_0001025 \sphinxhyphen{} located in}
\label{\detokenize{doc-RO_0001025:ro-0001025-located-in}}\label{\detokenize{doc-RO_0001025:index-0}}\label{\detokenize{doc-RO_0001025::doc}}
\begin{sphinxShadowBox}
\sphinxstyletopictitle{Label}

\sphinxAtStartPar
located in
\end{sphinxShadowBox}

\begin{sphinxShadowBox}
\sphinxstyletopictitle{Definition}

\sphinxAtStartPar
A relation between two independent continuants, the target and the location, in which the target is entirely within the location
\end{sphinxShadowBox}

\begin{sphinxShadowBox}
\sphinxstyletopictitle{Example}

\sphinxAtStartPar
this rat is located in this cage

\sphinxAtStartPar
my brain is located in my head
\end{sphinxShadowBox}

\begin{sphinxShadowBox}
\sphinxstyletopictitle{Editor’s note}

\sphinxAtStartPar
Location as a relation between instances: The primitive instance\sphinxhyphen{}level relation c located\_in r at t reflects the fact that each continuant is at any given time associated with exactly one spatial region, namely its exact location. Following we can use this relation to define a further instance\sphinxhyphen{}level location relation \sphinxhyphen{} not between a continuant and the region which it exactly occupies, but rather between one continuant and another. c is located in c1, in this sense, whenever the spatial region occupied by c is part\_of the spatial region occupied by c1.    Note that this relation comprehends both the relation of exact location between one continuant and another which obtains when r and r1 are identical (for example, when a portion of fluid exactly fills a cavity), as well as those sorts of inexact location relations which obtain, for example, between brain and head or between ovum and uterus

\sphinxAtStartPar
Most location relations will only hold at certain times, but this is difficult to specify in OWL. See \sphinxurl{https://code.google.com/p/obo-relations/wiki/ROAndTime}
\end{sphinxShadowBox}

\begin{sphinxShadowBox}
\sphinxstyletopictitle{Imported From}

\sphinxAtStartPar
\sphinxurl{http://purl.obolibrary.org/obo/ro/releases/2021-03-08/ro.owl}
\end{sphinxShadowBox}
\begin{quote}

\index{RO\_0002012@\spxentry{RO\_0002012}!occurent part of@\spxentry{occurent part of}}\index{occurent part of@\spxentry{occurent part of}!RO\_0002012@\spxentry{RO\_0002012}}\ignorespaces \end{quote}


\subsection{RO\_0002012 \sphinxhyphen{} occurent part of}
\label{\detokenize{doc-RO_0002012:ro-0002012-occurent-part-of}}\label{\detokenize{doc-RO_0002012:index-0}}\label{\detokenize{doc-RO_0002012::doc}}
\begin{sphinxShadowBox}
\sphinxstyletopictitle{Label}

\sphinxAtStartPar
occurent part of
\end{sphinxShadowBox}

\begin{sphinxShadowBox}
\sphinxstyletopictitle{Definition}

\sphinxAtStartPar
A part of relation that applies only between occurents.
\end{sphinxShadowBox}

\begin{sphinxShadowBox}
\sphinxstyletopictitle{Imported From}

\sphinxAtStartPar
\sphinxurl{http://purl.obolibrary.org/obo/ro/releases/2020-12-18/ro.owl}

\sphinxAtStartPar
\sphinxurl{http://purl.obolibrary.org/obo/ro/releases/2021-03-08/ro.owl}
\end{sphinxShadowBox}
\begin{quote}

\index{RO\_0002131@\spxentry{RO\_0002131}!overlaps@\spxentry{overlaps}}\index{overlaps@\spxentry{overlaps}!RO\_0002131@\spxentry{RO\_0002131}}\ignorespaces \end{quote}


\subsection{RO\_0002131 \sphinxhyphen{} overlaps}
\label{\detokenize{doc-RO_0002131:ro-0002131-overlaps}}\label{\detokenize{doc-RO_0002131:index-0}}\label{\detokenize{doc-RO_0002131::doc}}
\begin{sphinxShadowBox}
\sphinxstyletopictitle{Label}

\sphinxAtStartPar
overlaps
\end{sphinxShadowBox}

\begin{sphinxShadowBox}
\sphinxstyletopictitle{Definition}

\sphinxAtStartPar
X overlaps y if and only if there exists some z such that x has part z and z part of y
\end{sphinxShadowBox}

\begin{sphinxShadowBox}
\sphinxstyletopictitle{Imported From}

\sphinxAtStartPar
\sphinxurl{http://purl.obolibrary.org/obo/ro/releases/2020-12-18/ro.owl}
\end{sphinxShadowBox}
\begin{quote}

\index{RO\_0002234@\spxentry{RO\_0002234}!has output@\spxentry{has output}}\index{has output@\spxentry{has output}!RO\_0002234@\spxentry{RO\_0002234}}\ignorespaces \end{quote}


\subsection{RO\_0002234 \sphinxhyphen{} has output}
\label{\detokenize{doc-RO_0002234:ro-0002234-has-output}}\label{\detokenize{doc-RO_0002234:index-0}}\label{\detokenize{doc-RO_0002234::doc}}
\begin{sphinxShadowBox}
\sphinxstyletopictitle{Label}

\sphinxAtStartPar
has output
\end{sphinxShadowBox}

\begin{sphinxShadowBox}
\sphinxstyletopictitle{Definition}

\sphinxAtStartPar
P has output c iff c is a participant in p, c is present at the end of p, and c is not present at the beginning of p.
\end{sphinxShadowBox}

\begin{sphinxShadowBox}
\sphinxstyletopictitle{Imported From}

\sphinxAtStartPar
\sphinxurl{http://purl.obolibrary.org/obo/ro/releases/2021-03-08/ro.owl}
\end{sphinxShadowBox}

\begin{sphinxShadowBox}
\sphinxstyletopictitle{Term editor}

\sphinxAtStartPar
Chris Mungall
\end{sphinxShadowBox}
\begin{quote}

\index{RO\_0002323@\spxentry{RO\_0002323}!mereotopologically related to@\spxentry{mereotopologically related to}}\index{mereotopologically related to@\spxentry{mereotopologically related to}!RO\_0002323@\spxentry{RO\_0002323}}\ignorespaces \end{quote}


\subsection{RO\_0002323 \sphinxhyphen{} mereotopologically related to}
\label{\detokenize{doc-RO_0002323:ro-0002323-mereotopologically-related-to}}\label{\detokenize{doc-RO_0002323:index-0}}\label{\detokenize{doc-RO_0002323::doc}}
\begin{sphinxShadowBox}
\sphinxstyletopictitle{Label}

\sphinxAtStartPar
mereotopologically related to
\end{sphinxShadowBox}

\begin{sphinxShadowBox}
\sphinxstyletopictitle{Definition}

\sphinxAtStartPar
A mereological relationship or a topological relationship
\end{sphinxShadowBox}

\begin{sphinxShadowBox}
\sphinxstyletopictitle{Imported From}

\sphinxAtStartPar
\sphinxurl{http://purl.obolibrary.org/obo/ro/releases/2020-12-18/ro.owl}
\end{sphinxShadowBox}

\begin{sphinxShadowBox}
\sphinxstyletopictitle{Term editor}

\sphinxAtStartPar
Chris Mungall
\end{sphinxShadowBox}
\begin{quote}

\index{RO\_0002353@\spxentry{RO\_0002353}!output of@\spxentry{output of}}\index{output of@\spxentry{output of}!RO\_0002353@\spxentry{RO\_0002353}}\ignorespaces \end{quote}


\subsection{RO\_0002353 \sphinxhyphen{} output of}
\label{\detokenize{doc-RO_0002353:ro-0002353-output-of}}\label{\detokenize{doc-RO_0002353:index-0}}\label{\detokenize{doc-RO_0002353::doc}}
\begin{sphinxShadowBox}
\sphinxstyletopictitle{Label}

\sphinxAtStartPar
output of
\end{sphinxShadowBox}

\begin{sphinxShadowBox}
\sphinxstyletopictitle{Definition}

\sphinxAtStartPar
Inverse of has output
\end{sphinxShadowBox}

\begin{sphinxShadowBox}
\sphinxstyletopictitle{Imported From}

\sphinxAtStartPar
\sphinxurl{http://purl.obolibrary.org/obo/ro/releases/2021-03-08/ro.owl}
\end{sphinxShadowBox}

\begin{sphinxShadowBox}
\sphinxstyletopictitle{Term editor}

\sphinxAtStartPar
Chris Mungall
\end{sphinxShadowBox}
\begin{quote}

\index{unitType@\spxentry{unitType}!temporal unit type@\spxentry{temporal unit type}}\index{temporal unit type@\spxentry{temporal unit type}!unitType@\spxentry{unitType}}\ignorespaces \end{quote}


\subsection{unitType \sphinxhyphen{} temporal unit type}
\label{\detokenize{doc-unitType:unittype-temporal-unit-type}}\label{\detokenize{doc-unitType:index-0}}\label{\detokenize{doc-unitType::doc}}
\begin{sphinxShadowBox}
\sphinxstyletopictitle{Label}

\sphinxAtStartPar
temporal unit type
\end{sphinxShadowBox}

\begin{sphinxShadowBox}
\sphinxstyletopictitle{Definition}

\sphinxAtStartPar
An indicator of the temporal precision of a time instant
\end{sphinxShadowBox}

\begin{sphinxShadowBox}
\sphinxstyletopictitle{Imported From}

\sphinxAtStartPar
\sphinxurl{http://www.w3.org/2006/time\#2016}
\end{sphinxShadowBox}


\chapter{Datatype Properties}
\label{\detokenize{datatype-properties:datatype-properties}}\label{\detokenize{datatype-properties::doc}}
\sphinxAtStartPar
See {\hyperref[\detokenize{datatype-properties:table-8}]{\sphinxcrossref{Table 8}}}.


\begin{savenotes}\sphinxattablestart
\centering
\sphinxcapstartof{table}
\sphinxthecaptionisattop
\sphinxcaption{Table 8 Datatype Properties}\label{\detokenize{datatype-properties:id1}}\label{\detokenize{datatype-properties:table-8}}
\sphinxaftertopcaption
\begin{tabulary}{\linewidth}[t]{|T|T|}
\hline
\sphinxstyletheadfamily 
\sphinxAtStartPar
Term ID \sphinxhyphen{} Label
&\sphinxstyletheadfamily 
\sphinxAtStartPar
Definition
\\
\hline
\sphinxAtStartPar
{\hyperref[\detokenize{doc-OBI_0002815::doc}]{\sphinxcrossref{\DUrole{doc}{OBI\_0002815 \sphinxhyphen{} has representation}}}}
&
\sphinxAtStartPar
Define ‘has representation’ is a data property

\sphinxAtStartPar
that attaches between an information content

\sphinxAtStartPar
entity and a value that contains linguistically or

\sphinxAtStartPar
computationally coded text.
\\
\hline
\sphinxAtStartPar
{\hyperref[\detokenize{doc-ORG_3000001::doc}]{\sphinxcrossref{\DUrole{doc}{ORG\_3000001 \sphinxhyphen{} number of employees}}}}
&
\sphinxAtStartPar
Specifies the number of people (headcount) who

\sphinxAtStartPar
receive paychecks from the organization for work

\sphinxAtStartPar
performed
\\
\hline
\sphinxAtStartPar
{\hyperref[\detokenize{doc-ORG_3000002::doc}]{\sphinxcrossref{\DUrole{doc}{ORG\_3000002 \sphinxhyphen{} has email representation}}}}
&
\sphinxAtStartPar
Specifies the email address string for an email

\sphinxAtStartPar
address
\\
\hline
\sphinxAtStartPar
{\hyperref[\detokenize{doc-ORG_3000003::doc}]{\sphinxcrossref{\DUrole{doc}{ORG\_3000003 \sphinxhyphen{} has postal address representation}}}}
&
\sphinxAtStartPar
Specifies the postal address string for a postal

\sphinxAtStartPar
address
\\
\hline
\sphinxAtStartPar
{\hyperref[\detokenize{doc-ORG_3000004::doc}]{\sphinxcrossref{\DUrole{doc}{ORG\_3000004 \sphinxhyphen{} has geolocation representation}}}}
&
\sphinxAtStartPar
Specifies the geolocation string for a geolocation
\\
\hline
\sphinxAtStartPar
{\hyperref[\detokenize{doc-ORG_3000005::doc}]{\sphinxcrossref{\DUrole{doc}{ORG\_3000005 \sphinxhyphen{} has URL representation}}}}
&
\sphinxAtStartPar
Specifies the value of a URL that represents the

\sphinxAtStartPar
address of a web site on the world wide web
\\
\hline
\sphinxAtStartPar
{\hyperref[\detokenize{doc-ORG_3000006::doc}]{\sphinxcrossref{\DUrole{doc}{ORG\_3000006 \sphinxhyphen{} has official organization name}}}}
&
\sphinxAtStartPar
Specifies the value of a name of the organization

\sphinxAtStartPar
which appears in the current documents authorizing

\sphinxAtStartPar
its existence
\\
\hline
\sphinxAtStartPar
{\hyperref[\detokenize{doc-ORG_3000007::doc}]{\sphinxcrossref{\DUrole{doc}{ORG\_3000007 \sphinxhyphen{} has organization name}}}}
&
\sphinxAtStartPar
Specifies the value of a name of an organization
\\
\hline
\sphinxAtStartPar
{\hyperref[\detokenize{doc-inXSDDateTimeStamp::doc}]{\sphinxcrossref{\DUrole{doc}{inXSDDateTimeStamp \sphinxhyphen{} in XSD Date\sphinxhyphen{}Time\sphinxhyphen{}Stamp}}}}
&
\sphinxAtStartPar
A datatype property to contain the representation

\sphinxAtStartPar
of a time instant as an xsd datetimestamp
\\
\hline
\end{tabulary}
\par
\sphinxattableend\end{savenotes}
\begin{quote}

\index{OBI\_0002815@\spxentry{OBI\_0002815}!has representation@\spxentry{has representation}}\index{has representation@\spxentry{has representation}!OBI\_0002815@\spxentry{OBI\_0002815}}\ignorespaces \end{quote}


\section{OBI\_0002815 \sphinxhyphen{} has representation}
\label{\detokenize{doc-OBI_0002815:obi-0002815-has-representation}}\label{\detokenize{doc-OBI_0002815:index-0}}\label{\detokenize{doc-OBI_0002815::doc}}
\begin{sphinxShadowBox}
\sphinxstyletopictitle{Label}

\sphinxAtStartPar
has representation
\end{sphinxShadowBox}

\begin{sphinxShadowBox}
\sphinxstyletopictitle{Definition}

\sphinxAtStartPar
Define ‘has representation’ is a data property that attaches between an information content entity and a value that contains linguistically or computationally coded text.
\end{sphinxShadowBox}

\begin{sphinxShadowBox}
\sphinxstyletopictitle{Example}

\sphinxAtStartPar
I feel sick to my stomach every Tuesday

\sphinxAtStartPar
12th arrondissement of Paris

\sphinxAtStartPar
20g
\end{sphinxShadowBox}

\begin{sphinxShadowBox}
\sphinxstyletopictitle{Editor’s note}

\sphinxAtStartPar
Further processing may enable the value to be represented in a component data structure such as an OBI value specification.
\end{sphinxShadowBox}

\begin{sphinxShadowBox}
\sphinxstyletopictitle{Imported From}

\sphinxAtStartPar
\sphinxurl{http://purl.obolibrary.org/obo/obi/2020-08-24/obi.owl}
\end{sphinxShadowBox}

\begin{sphinxShadowBox}
\sphinxstyletopictitle{Term editor}

\sphinxAtStartPar
Damion Dooley

\sphinxAtStartPar
Mark Miller
\end{sphinxShadowBox}
\begin{quote}

\index{ORG\_3000001@\spxentry{ORG\_3000001}!number of employees@\spxentry{number of employees}}\index{number of employees@\spxentry{number of employees}!ORG\_3000001@\spxentry{ORG\_3000001}}\ignorespaces \end{quote}


\section{ORG\_3000001 \sphinxhyphen{} number of employees}
\label{\detokenize{doc-ORG_3000001:org-3000001-number-of-employees}}\label{\detokenize{doc-ORG_3000001:index-0}}\label{\detokenize{doc-ORG_3000001::doc}}
\begin{sphinxShadowBox}
\sphinxstyletopictitle{Label}

\sphinxAtStartPar
number of employees
\end{sphinxShadowBox}

\begin{sphinxShadowBox}
\sphinxstyletopictitle{Alternate name}

\sphinxAtStartPar
\# of employees
\end{sphinxShadowBox}

\begin{sphinxShadowBox}
\sphinxstyletopictitle{Definition}

\sphinxAtStartPar
Specifies the number of people (headcount) who receive paychecks from the organization for work performed
\end{sphinxShadowBox}

\begin{sphinxShadowBox}
\sphinxstyletopictitle{Definition source}

\sphinxAtStartPar
Michael Conlon \sphinxurl{https://orcid.org/0000-0002-1304-8447}
\end{sphinxShadowBox}

\begin{sphinxShadowBox}
\sphinxstyletopictitle{Example}

\sphinxAtStartPar
Amazon.com number\_of\_employees 1300000
\end{sphinxShadowBox}

\begin{sphinxShadowBox}
\sphinxstyletopictitle{Editor’s note}

\sphinxAtStartPar
Informal organizations do not have employees.  Create a class restriction as a domain.
\end{sphinxShadowBox}

\begin{sphinxShadowBox}
\sphinxstyletopictitle{Term editor}

\sphinxAtStartPar
Michael Conlon \sphinxurl{https://orcid.org/0000-0002-1304-8447}
\end{sphinxShadowBox}
\begin{quote}

\index{ORG\_3000002@\spxentry{ORG\_3000002}!has email representation@\spxentry{has email representation}}\index{has email representation@\spxentry{has email representation}!ORG\_3000002@\spxentry{ORG\_3000002}}\ignorespaces \end{quote}


\section{ORG\_3000002 \sphinxhyphen{} has email representation}
\label{\detokenize{doc-ORG_3000002:org-3000002-has-email-representation}}\label{\detokenize{doc-ORG_3000002:index-0}}\label{\detokenize{doc-ORG_3000002::doc}}
\begin{sphinxShadowBox}
\sphinxstyletopictitle{Label}

\sphinxAtStartPar
has email representation
\end{sphinxShadowBox}

\begin{sphinxShadowBox}
\sphinxstyletopictitle{Alternate name}

\sphinxAtStartPar
has email value
\end{sphinxShadowBox}

\begin{sphinxShadowBox}
\sphinxstyletopictitle{Definition}

\sphinxAtStartPar
Specifies the email address string for an email address
\end{sphinxShadowBox}

\begin{sphinxShadowBox}
\sphinxstyletopictitle{Definition source}

\sphinxAtStartPar
Michael Conlon \sphinxurl{https://orcid.org/0000-0002-1304-8447}
\end{sphinxShadowBox}

\begin{sphinxShadowBox}
\sphinxstyletopictitle{Example}

\sphinxAtStartPar
\sphinxhref{mailto:info@metabolomics.info}{info@metabolomics.info}
\end{sphinxShadowBox}

\begin{sphinxShadowBox}
\sphinxstyletopictitle{Editor’s note}

\sphinxAtStartPar
Regex pattern restriction for RFC ??? Email addresses
\end{sphinxShadowBox}

\begin{sphinxShadowBox}
\sphinxstyletopictitle{Term editor}

\sphinxAtStartPar
Michael Conlon \sphinxurl{https://orcid.org/0000-0002-1304-8447}
\end{sphinxShadowBox}
\begin{quote}

\index{ORG\_3000003@\spxentry{ORG\_3000003}!has postal address representation@\spxentry{has postal address representation}}\index{has postal address representation@\spxentry{has postal address representation}!ORG\_3000003@\spxentry{ORG\_3000003}}\ignorespaces \end{quote}


\section{ORG\_3000003 \sphinxhyphen{} has postal address representation}
\label{\detokenize{doc-ORG_3000003:org-3000003-has-postal-address-representation}}\label{\detokenize{doc-ORG_3000003:index-0}}\label{\detokenize{doc-ORG_3000003::doc}}
\begin{sphinxShadowBox}
\sphinxstyletopictitle{Label}

\sphinxAtStartPar
has postal address representation
\end{sphinxShadowBox}

\begin{sphinxShadowBox}
\sphinxstyletopictitle{Alternate name}

\sphinxAtStartPar
has address value
\end{sphinxShadowBox}

\begin{sphinxShadowBox}
\sphinxstyletopictitle{Definition}

\sphinxAtStartPar
Specifies the postal address string for a postal address
\end{sphinxShadowBox}

\begin{sphinxShadowBox}
\sphinxstyletopictitle{Definition source}

\sphinxAtStartPar
Michael Conlon \sphinxurl{https://orcid.org/0000-0002-1304-8447}
\end{sphinxShadowBox}

\begin{sphinxShadowBox}
\sphinxstyletopictitle{Example}

\sphinxAtStartPar
1210 NW 14th Ave; ;Gainesville; Fl; USA; 32601
\end{sphinxShadowBox}

\begin{sphinxShadowBox}
\sphinxstyletopictitle{Editor’s note}

\sphinxAtStartPar
Regex pattern restriction for UPC email addresses
\end{sphinxShadowBox}

\begin{sphinxShadowBox}
\sphinxstyletopictitle{Term editor}

\sphinxAtStartPar
Michael Conlon \sphinxurl{https://orcid.org/0000-0002-1304-8447}
\end{sphinxShadowBox}
\begin{quote}

\index{ORG\_3000004@\spxentry{ORG\_3000004}!has geolocation representation@\spxentry{has geolocation representation}}\index{has geolocation representation@\spxentry{has geolocation representation}!ORG\_3000004@\spxentry{ORG\_3000004}}\ignorespaces \end{quote}


\section{ORG\_3000004 \sphinxhyphen{} has geolocation representation}
\label{\detokenize{doc-ORG_3000004:org-3000004-has-geolocation-representation}}\label{\detokenize{doc-ORG_3000004:index-0}}\label{\detokenize{doc-ORG_3000004::doc}}
\begin{sphinxShadowBox}
\sphinxstyletopictitle{Label}

\sphinxAtStartPar
has geolocation representation
\end{sphinxShadowBox}

\begin{sphinxShadowBox}
\sphinxstyletopictitle{Alternate name}

\sphinxAtStartPar
has geolocation
\end{sphinxShadowBox}

\begin{sphinxShadowBox}
\sphinxstyletopictitle{Definition}

\sphinxAtStartPar
Specifies the geolocation string for a geolocation
\end{sphinxShadowBox}

\begin{sphinxShadowBox}
\sphinxstyletopictitle{Definition source}

\sphinxAtStartPar
Michael Conlon \sphinxurl{https://orcid.org/0000-0002-1304-8447}
\end{sphinxShadowBox}

\begin{sphinxShadowBox}
\sphinxstyletopictitle{Example}

\sphinxAtStartPar
29.6651224,\sphinxhyphen{}82.3396949
\end{sphinxShadowBox}

\begin{sphinxShadowBox}
\sphinxstyletopictitle{Editor’s note}

\sphinxAtStartPar
Regex pattern restriction to +/\sphinxhyphen{}lat;+/\sphinxhyphen{}long
\end{sphinxShadowBox}

\begin{sphinxShadowBox}
\sphinxstyletopictitle{Term editor}

\sphinxAtStartPar
Michael Conlon \sphinxurl{https://orcid.org/0000-0002-1304-8447}
\end{sphinxShadowBox}
\begin{quote}

\index{ORG\_3000005@\spxentry{ORG\_3000005}!has URL representation@\spxentry{has URL representation}}\index{has URL representation@\spxentry{has URL representation}!ORG\_3000005@\spxentry{ORG\_3000005}}\ignorespaces \end{quote}


\section{ORG\_3000005 \sphinxhyphen{} has URL representation}
\label{\detokenize{doc-ORG_3000005:org-3000005-has-url-representation}}\label{\detokenize{doc-ORG_3000005:index-0}}\label{\detokenize{doc-ORG_3000005::doc}}
\begin{sphinxShadowBox}
\sphinxstyletopictitle{Label}

\sphinxAtStartPar
has URL representation
\end{sphinxShadowBox}

\begin{sphinxShadowBox}
\sphinxstyletopictitle{Alternate name}

\sphinxAtStartPar
has uniform resource locator
\end{sphinxShadowBox}

\begin{sphinxShadowBox}
\sphinxstyletopictitle{Definition}

\sphinxAtStartPar
Specifies the value of a URL that represents the address of a web site on the world wide web
\end{sphinxShadowBox}

\begin{sphinxShadowBox}
\sphinxstyletopictitle{Definition source}

\sphinxAtStartPar
Michael Conlon \sphinxurl{https://orcid.org/0000-0002-1304-8447}
\end{sphinxShadowBox}

\begin{sphinxShadowBox}
\sphinxstyletopictitle{Example}

\sphinxAtStartPar
The URL of the United Nations home page is \sphinxurl{http://un.org}
\end{sphinxShadowBox}

\begin{sphinxShadowBox}
\sphinxstyletopictitle{Editor’s note}

\sphinxAtStartPar
Replace with terms from IDO?
\end{sphinxShadowBox}

\begin{sphinxShadowBox}
\sphinxstyletopictitle{Term editor}

\sphinxAtStartPar
Michael Conlon \sphinxurl{https://orcid.org/0000-0002-1304-8447}
\end{sphinxShadowBox}
\begin{quote}

\index{ORG\_3000006@\spxentry{ORG\_3000006}!has official organization name@\spxentry{has official organization name}}\index{has official organization name@\spxentry{has official organization name}!ORG\_3000006@\spxentry{ORG\_3000006}}\ignorespaces \end{quote}


\section{ORG\_3000006 \sphinxhyphen{} has official organization name}
\label{\detokenize{doc-ORG_3000006:org-3000006-has-official-organization-name}}\label{\detokenize{doc-ORG_3000006:index-0}}\label{\detokenize{doc-ORG_3000006::doc}}
\begin{sphinxShadowBox}
\sphinxstyletopictitle{Label}

\sphinxAtStartPar
has official organization name
\end{sphinxShadowBox}

\begin{sphinxShadowBox}
\sphinxstyletopictitle{Definition}

\sphinxAtStartPar
Specifies the value of a name of the organization which appears in the current documents authorizing its existence
\end{sphinxShadowBox}

\begin{sphinxShadowBox}
\sphinxstyletopictitle{Definition source}

\sphinxAtStartPar
Michael Conlon \sphinxurl{https://orcid.org/0000-0002-1304-8447}
\end{sphinxShadowBox}

\begin{sphinxShadowBox}
\sphinxstyletopictitle{Example}

\sphinxAtStartPar
The official name of the united states is The United States of America.
\end{sphinxShadowBox}

\begin{sphinxShadowBox}
\sphinxstyletopictitle{Term editor}

\sphinxAtStartPar
Michael Conlon \sphinxurl{https://orcid.org/0000-0002-1304-8447}
\end{sphinxShadowBox}
\begin{quote}

\index{ORG\_3000007@\spxentry{ORG\_3000007}!has organization name@\spxentry{has organization name}}\index{has organization name@\spxentry{has organization name}!ORG\_3000007@\spxentry{ORG\_3000007}}\ignorespaces \end{quote}


\section{ORG\_3000007 \sphinxhyphen{} has organization name}
\label{\detokenize{doc-ORG_3000007:org-3000007-has-organization-name}}\label{\detokenize{doc-ORG_3000007:index-0}}\label{\detokenize{doc-ORG_3000007::doc}}
\begin{sphinxShadowBox}
\sphinxstyletopictitle{Label}

\sphinxAtStartPar
has organization name
\end{sphinxShadowBox}

\begin{sphinxShadowBox}
\sphinxstyletopictitle{Definition}

\sphinxAtStartPar
Specifies the value of a name of an organization
\end{sphinxShadowBox}

\begin{sphinxShadowBox}
\sphinxstyletopictitle{Definition source}

\sphinxAtStartPar
Michael Conlon \sphinxurl{https://orcid.org/0000-0002-1304-8447}
\end{sphinxShadowBox}

\begin{sphinxShadowBox}
\sphinxstyletopictitle{Example}

\sphinxAtStartPar
Names of The United States of America incude: USA, US, United States, America.
\end{sphinxShadowBox}

\begin{sphinxShadowBox}
\sphinxstyletopictitle{Editor’s note}

\sphinxAtStartPar
Abbreviations and acronyms are also names of organizations
\end{sphinxShadowBox}

\begin{sphinxShadowBox}
\sphinxstyletopictitle{Term editor}

\sphinxAtStartPar
Michael Conlon \sphinxurl{https://orcid.org/0000-0002-1304-8447}
\end{sphinxShadowBox}
\begin{quote}

\index{inXSDDateTimeStamp@\spxentry{inXSDDateTimeStamp}!in XSD Date\sphinxhyphen{}Time\sphinxhyphen{}Stamp@\spxentry{in XSD Date\sphinxhyphen{}Time\sphinxhyphen{}Stamp}}\index{in XSD Date\sphinxhyphen{}Time\sphinxhyphen{}Stamp@\spxentry{in XSD Date\sphinxhyphen{}Time\sphinxhyphen{}Stamp}!inXSDDateTimeStamp@\spxentry{inXSDDateTimeStamp}}\ignorespaces \end{quote}


\section{inXSDDateTimeStamp \sphinxhyphen{} in XSD Date\sphinxhyphen{}Time\sphinxhyphen{}Stamp}
\label{\detokenize{doc-inXSDDateTimeStamp:inxsddatetimestamp-in-xsd-date-time-stamp}}\label{\detokenize{doc-inXSDDateTimeStamp:index-0}}\label{\detokenize{doc-inXSDDateTimeStamp::doc}}
\begin{sphinxShadowBox}
\sphinxstyletopictitle{Label}

\sphinxAtStartPar
in XSD Date\sphinxhyphen{}Time\sphinxhyphen{}Stamp
\end{sphinxShadowBox}

\begin{sphinxShadowBox}
\sphinxstyletopictitle{Definition}

\sphinxAtStartPar
A datatype property to contain the representation of a time instant as an xsd datetimestamp
\end{sphinxShadowBox}

\begin{sphinxShadowBox}
\sphinxstyletopictitle{Imported From}

\sphinxAtStartPar
\sphinxurl{http://www.w3.org/2006/time\#2016}
\end{sphinxShadowBox}


\chapter{Named Individuals}
\label{\detokenize{named-individuals:named-individuals}}\label{\detokenize{named-individuals::doc}}
\sphinxAtStartPar
\sphinxstylestrong{Named individuals} are entities which are instances of classes.  A university, a
continent, a building, a date, and a role in a project are all named individuals.


\section{All Named Individuals}
\label{\detokenize{named-individuals:all-named-individuals}}
\sphinxAtStartPar
See {\hyperref[\detokenize{named-individuals:table-9}]{\sphinxcrossref{Table 9}}}.


\begin{savenotes}\sphinxattablestart
\centering
\sphinxcapstartof{table}
\sphinxthecaptionisattop
\sphinxcaption{Table 9 Named Individuals}\label{\detokenize{named-individuals:id1}}\label{\detokenize{named-individuals:table-9}}
\sphinxaftertopcaption
\begin{tabulary}{\linewidth}[t]{|T|T|}
\hline
\sphinxstyletheadfamily 
\sphinxAtStartPar
Term ID \sphinxhyphen{} Label
&\sphinxstyletheadfamily 
\sphinxAtStartPar
Definition
\\
\hline
\sphinxAtStartPar
{\hyperref[\detokenize{doc-unitDay::doc}]{\sphinxcrossref{\DUrole{doc}{unitDay \sphinxhyphen{} day (unit of temporal duration)}}}}
&
\sphinxAtStartPar
None
\\
\hline
\sphinxAtStartPar
{\hyperref[\detokenize{doc-unitHour::doc}]{\sphinxcrossref{\DUrole{doc}{unitHour \sphinxhyphen{} hour (unit of temporal duration)}}}}
&
\sphinxAtStartPar
None
\\
\hline
\sphinxAtStartPar
{\hyperref[\detokenize{doc-unitMinute::doc}]{\sphinxcrossref{\DUrole{doc}{unitMinute \sphinxhyphen{} minute (unit of temporal duration)}}}}
&
\sphinxAtStartPar
None
\\
\hline
\sphinxAtStartPar
{\hyperref[\detokenize{doc-unitMonth::doc}]{\sphinxcrossref{\DUrole{doc}{unitMonth \sphinxhyphen{} month (unit of temporal duration)}}}}
&
\sphinxAtStartPar
None
\\
\hline
\sphinxAtStartPar
{\hyperref[\detokenize{doc-unitSecond::doc}]{\sphinxcrossref{\DUrole{doc}{unitSecond \sphinxhyphen{} second (unit of temporal duration)}}}}
&
\sphinxAtStartPar
None
\\
\hline
\sphinxAtStartPar
{\hyperref[\detokenize{doc-unitWeek::doc}]{\sphinxcrossref{\DUrole{doc}{unitWeek \sphinxhyphen{} week (unit of temporal duration)}}}}
&
\sphinxAtStartPar
None
\\
\hline
\sphinxAtStartPar
{\hyperref[\detokenize{doc-unitYear::doc}]{\sphinxcrossref{\DUrole{doc}{unitYear \sphinxhyphen{} year (unit of temporal duration)}}}}
&
\sphinxAtStartPar
None
\\
\hline
\end{tabulary}
\par
\sphinxattableend\end{savenotes}
\begin{quote}

\index{unitDay@\spxentry{unitDay}!day (unit of temporal duration)@\spxentry{day}\spxextra{unit of temporal duration}}\index{day (unit of temporal duration)@\spxentry{day}\spxextra{unit of temporal duration}!unitDay@\spxentry{unitDay}}\ignorespaces \end{quote}


\subsection{unitDay \sphinxhyphen{} day (unit of temporal duration)}
\label{\detokenize{doc-unitDay:unitday-day-unit-of-temporal-duration}}\label{\detokenize{doc-unitDay:index-0}}\label{\detokenize{doc-unitDay::doc}}
\begin{sphinxShadowBox}
\sphinxstyletopictitle{Label}

\sphinxAtStartPar
day (unit of temporal duration)
\end{sphinxShadowBox}
\begin{quote}

\index{unitHour@\spxentry{unitHour}!hour (unit of temporal duration)@\spxentry{hour}\spxextra{unit of temporal duration}}\index{hour (unit of temporal duration)@\spxentry{hour}\spxextra{unit of temporal duration}!unitHour@\spxentry{unitHour}}\ignorespaces \end{quote}


\subsection{unitHour \sphinxhyphen{} hour (unit of temporal duration)}
\label{\detokenize{doc-unitHour:unithour-hour-unit-of-temporal-duration}}\label{\detokenize{doc-unitHour:index-0}}\label{\detokenize{doc-unitHour::doc}}
\begin{sphinxShadowBox}
\sphinxstyletopictitle{Label}

\sphinxAtStartPar
hour (unit of temporal duration)
\end{sphinxShadowBox}
\begin{quote}

\index{unitMinute@\spxentry{unitMinute}!minute (unit of temporal duration)@\spxentry{minute}\spxextra{unit of temporal duration}}\index{minute (unit of temporal duration)@\spxentry{minute}\spxextra{unit of temporal duration}!unitMinute@\spxentry{unitMinute}}\ignorespaces \end{quote}


\subsection{unitMinute \sphinxhyphen{} minute (unit of temporal duration)}
\label{\detokenize{doc-unitMinute:unitminute-minute-unit-of-temporal-duration}}\label{\detokenize{doc-unitMinute:index-0}}\label{\detokenize{doc-unitMinute::doc}}
\begin{sphinxShadowBox}
\sphinxstyletopictitle{Label}

\sphinxAtStartPar
minute (unit of temporal duration)
\end{sphinxShadowBox}
\begin{quote}

\index{unitMonth@\spxentry{unitMonth}!month (unit of temporal duration)@\spxentry{month}\spxextra{unit of temporal duration}}\index{month (unit of temporal duration)@\spxentry{month}\spxextra{unit of temporal duration}!unitMonth@\spxentry{unitMonth}}\ignorespaces \end{quote}


\subsection{unitMonth \sphinxhyphen{} month (unit of temporal duration)}
\label{\detokenize{doc-unitMonth:unitmonth-month-unit-of-temporal-duration}}\label{\detokenize{doc-unitMonth:index-0}}\label{\detokenize{doc-unitMonth::doc}}
\begin{sphinxShadowBox}
\sphinxstyletopictitle{Label}

\sphinxAtStartPar
month (unit of temporal duration)
\end{sphinxShadowBox}
\begin{quote}

\index{unitSecond@\spxentry{unitSecond}!second (unit of temporal duration)@\spxentry{second}\spxextra{unit of temporal duration}}\index{second (unit of temporal duration)@\spxentry{second}\spxextra{unit of temporal duration}!unitSecond@\spxentry{unitSecond}}\ignorespaces \end{quote}


\subsection{unitSecond \sphinxhyphen{} second (unit of temporal duration)}
\label{\detokenize{doc-unitSecond:unitsecond-second-unit-of-temporal-duration}}\label{\detokenize{doc-unitSecond:index-0}}\label{\detokenize{doc-unitSecond::doc}}
\begin{sphinxShadowBox}
\sphinxstyletopictitle{Label}

\sphinxAtStartPar
second (unit of temporal duration)
\end{sphinxShadowBox}
\begin{quote}

\index{unitWeek@\spxentry{unitWeek}!week (unit of temporal duration)@\spxentry{week}\spxextra{unit of temporal duration}}\index{week (unit of temporal duration)@\spxentry{week}\spxextra{unit of temporal duration}!unitWeek@\spxentry{unitWeek}}\ignorespaces \end{quote}


\subsection{unitWeek \sphinxhyphen{} week (unit of temporal duration)}
\label{\detokenize{doc-unitWeek:unitweek-week-unit-of-temporal-duration}}\label{\detokenize{doc-unitWeek:index-0}}\label{\detokenize{doc-unitWeek::doc}}
\begin{sphinxShadowBox}
\sphinxstyletopictitle{Label}

\sphinxAtStartPar
week (unit of temporal duration)
\end{sphinxShadowBox}
\begin{quote}

\index{unitYear@\spxentry{unitYear}!year (unit of temporal duration)@\spxentry{year}\spxextra{unit of temporal duration}}\index{year (unit of temporal duration)@\spxentry{year}\spxextra{unit of temporal duration}!unitYear@\spxentry{unitYear}}\ignorespaces \end{quote}


\subsection{unitYear \sphinxhyphen{} year (unit of temporal duration)}
\label{\detokenize{doc-unitYear:unityear-year-unit-of-temporal-duration}}\label{\detokenize{doc-unitYear:index-0}}\label{\detokenize{doc-unitYear::doc}}
\begin{sphinxShadowBox}
\sphinxstyletopictitle{Label}

\sphinxAtStartPar
year (unit of temporal duration)
\end{sphinxShadowBox}


\chapter{Out of Scope Terms}
\label{\detokenize{out-of-scope:out-of-scope-terms}}\label{\detokenize{out-of-scope::doc}}
\sphinxAtStartPar
In designing and building ontologies, one seeks to find a coherent domain for the
ontology \textendash{} a set of terms that are useful in representing the entities of the domain,
include and reuse terms from other ontologies as appropriate, while not including terms
that belong elsewhere.  These choices are somewhat arbitrary, as long as we have a
cler view of the domain we are attempting to represent, and we are willing to “give away”
terms that we included, but can be reused.

\sphinxAtStartPar
For the Organization Ontology, we adhered as best we could to several guiding principles
regarding terms, domains, inclusion and exclusion.

\sphinxAtStartPar
And, of course, we may have a change of heart regarding any term or set of terms.


\section{Out of Scope but Defined Here}
\label{\detokenize{out-of-scope:out-of-scope-but-defined-here}}
\begin{sphinxShadowBox}
\sphinxstyletopictitle{Locations}

\sphinxAtStartPar
It is important for organizations and their facilities to be located on the surface of
the earth.  We found the existing OBI ontologies \sphinxhref{https://sites.google.com/site/environmentontology/}{ENVO} and \sphinxhref{http://environmentontology.github.io/gaz/}{GAZ} to have
inconsistencies
and/or complexities that prohibited their reuse.  We created a simple  set of terms
within
the Organization Ontology to define a nested set of locations from continents down
to rooms that can have geographical representations (latitude and longitude) attached
to them.

\sphinxAtStartPar
We would be happy to use terms from another ontology that defines location terms we
could use.
\end{sphinxShadowBox}

\begin{sphinxShadowBox}
\sphinxstyletopictitle{Facilities}

\sphinxAtStartPar
The Organization Ontology has a need to make assertions regarding occupancy of
structures
\end{sphinxShadowBox}


\section{Out of Scope and Included Here}
\label{\detokenize{out-of-scope:out-of-scope-and-included-here}}
\sphinxAtStartPar
An organization ontology should reuse terms it needs from other ontologies.

\begin{sphinxShadowBox}
\sphinxstyletopictitle{Upper Level Ontology and Annotation Properties}

\sphinxAtStartPar
\sphinxhref{http://www.ontobee.org/ontology/BFO}{Basic Formal Ontology (BFO)} is used for an upper level ontology.  We use the \sphinxhref{http://www.ontobee.org/ontology/IAO}{Information Artifact Ontology (IAO)} annotation properties to
annotate terms.  We use Dublin Core and OWL annotation properties annotate the
ontology.
\end{sphinxShadowBox}

\begin{sphinxShadowBox}
\sphinxstyletopictitle{Identifiers}

\sphinxAtStartPar
\sphinxhref{https://github.com/mconlon17/identifier-ontology}{The Identifier Ontology} %
\begin{footnote}[1]\sphinxAtStartFootnote
The Identifier Ontology is underdevelopment as a planned expansion of \sphinxhref{http://www.ontobee.org/ontology/IAO}{Information Artifact Ontology (IAO)}
%
\end{footnote}
defines identifiers and semantics for using
identifiers to identify organizations, people, and scholarly works.
\end{sphinxShadowBox}

\begin{sphinxShadowBox}
\sphinxstyletopictitle{Information Artifacts}

\sphinxAtStartPar
The \sphinxhref{http://www.ontobee.org/ontology/IAO}{Information Artifact Ontology (IAO)} defines information artifacts needed here.
\end{sphinxShadowBox}

\begin{sphinxShadowBox}
\sphinxstyletopictitle{Time}

\sphinxAtStartPar
The \sphinxhref{https://www.w3.org/TR/owl-time/}{W3C Time Ontology (Time)} is used to define time:Instant and associated properties for using
time:Instant.  We have asserted a superclass for time:Instant to align it with
BFO.
\end{sphinxShadowBox}

\begin{sphinxShadowBox}
\sphinxstyletopictitle{Concept}

\sphinxAtStartPar
The \sphinxhref{https://www.w3.org/2004/02/skos/}{Simple Knowledge Organization System (SKOS)} issued to define skos:Concept.  We have asserted a superclass for
skos:Concept
to align it with BFO.
\end{sphinxShadowBox}


\section{Out of Scope and Not Included Here}
\label{\detokenize{out-of-scope:out-of-scope-and-not-included-here}}
\begin{sphinxShadowBox}
\sphinxstyletopictitle{Reports to / has report}

\sphinxAtStartPar
reports to / has report are properties in the \sphinxhref{https://www.w3.org/TR/vocab-org/}{W3C Organization Ontology} for
asserting that individual people report to other individual people in an organization.

\sphinxAtStartPar
We believe these are out of scope for an organization ontology, and are best left to
an
administrative ontology.
\end{sphinxShadowBox}

\begin{sphinxShadowBox}
\sphinxstyletopictitle{Additional detail regarding locations}

\sphinxAtStartPar
We have tried to include enough, but not too much.  This is not a locations
ontology. %
\begin{footnote}[2]\sphinxAtStartFootnote
We follow in the footsteps of {\hyperref[\detokenize{glossary:glossary}]{\sphinxcrossref{\DUrole{std,std-ref}{VIVO 1}}}}, including terms that have
shown their value over a decade of use.
%
\end{footnote}
\end{sphinxShadowBox}

\begin{sphinxShadowBox}
\sphinxstyletopictitle{Additional detail regarding structures}

\sphinxAtStartPar
We have tried to include enough, but not too much.  This is not a structures
ontology. %
\begin{footnote}[3]\sphinxAtStartFootnote
Same as the comment on locations.
%
\end{footnote}.
\end{sphinxShadowBox}

\begin{sphinxShadowBox}
\sphinxstyletopictitle{Properties related to Academic Events}

\sphinxAtStartPar
We have not included properties related to organizations must host, sponsor or otherwise
participate in.  See \sphinxhref{https://github.com/tibonto/aeon}{The Academic Event Ontology} for terms associating
organizations and academic events.
\end{sphinxShadowBox}


\chapter{Translating from VIVO to the Organization Ontology}
\label{\detokenize{vivo-to-org:translating-from-vivo-to-the-organization-ontology}}\label{\detokenize{vivo-to-org::doc}}
\sphinxAtStartPar
For those familiar with representing organizations using the VIVO Ontology,
we provide a guide for translating assertions in the VIVO Ontology to assertions
in the Organization Ontology %
\begin{footnote}[1]\sphinxAtStartFootnote
We in tend to provide SPARQL CONSTRUCT queries for automatica translation of
VIVO Ontology organization assertions to Organization Ontology assertions
in the future.  Consider this guide as advice to the adventurous, or to collaborators
who would like to draft, test, and contribute such queries.
%
\end{footnote}.


\section{Translating Types}
\label{\detokenize{vivo-to-org:translating-types}}
\sphinxAtStartPar
The VIVO Ontology organization types are presented in {\hyperref[\detokenize{vivo-to-org:table-16}]{\sphinxcrossref{Table 16}}} with instructions
for translating each.  The Organization Ontology separates the concept of
what the organization “is” (company, organization part, etc) from what the
organization “does” (hospital, library, etc).  In many cases, the VIVO Ontology
combined these and information about one or the other (“is”, “does”) is missing.

\sphinxAtStartPar
For example, consider vivo:Museum.  This assertion of type is actually an
assertion of purpose.  The type of organization (organization part, nonprofit) is
missing.  We can assert the museum is an organization, and has a disposition of
museum.  We may be able to bring additional information to bear and assert a
a specific type %
\begin{footnote}[2]\sphinxAtStartFootnote
Note that additional information is needed.  The Metropolitan Museum of Art
in New York City is a nonprofit organization.  The Florida Museum of Natural
History in Gainesville, Florida, is an organizational part of the University of
Florida.  In the VIVO
Ontology, both would be asserted to be type vivo:Museum.  In the Organization
Ontology, the first would be asserted to be nonprofit, the second organization
part.  Both would be asserted to have disposition museum.
%
\end{footnote}.

\sphinxAtStartPar
{\hyperref[\detokenize{vivo-to-org:table-16}]{\sphinxcrossref{Table 16}}} provides a guide for translating Organizational types to assertions
in the Organization Ontology.


\begin{savenotes}\sphinxattablestart
\centering
\sphinxcapstartof{table}
\sphinxthecaptionisattop
\sphinxcaption{Table 16 Translating VIVO types to Organizational Ontology assertions}\label{\detokenize{vivo-to-org:id9}}\label{\detokenize{vivo-to-org:table-16}}
\sphinxaftertopcaption
\begin{tabulary}{\linewidth}[t]{|T|T|}
\hline
\sphinxstyletheadfamily 
\sphinxAtStartPar
VIVO Type
&\sphinxstyletheadfamily 
\sphinxAtStartPar
Organization Ontology Assertions
\\
\hline
\sphinxAtStartPar
Association
&
\sphinxAtStartPar
Unknown type.  Assert Organization only.

\sphinxAtStartPar
Type is often nonprofit.

\sphinxAtStartPar
Disposition is association.
\\
\hline
\sphinxAtStartPar
Center
&
\sphinxAtStartPar
Unknown type.  Assert organization only.

\sphinxAtStartPar
Type is often an organization part.

\sphinxAtStartPar
Unknown dispositions.  Often research.
\\
\hline
\sphinxAtStartPar
College
&
\sphinxAtStartPar
Unknown type.  Assert organization only.

\sphinxAtStartPar
At a US university, an organizational part.

\sphinxAtStartPar
At a US university, dispositions of education, service,
research
\\
\hline
\sphinxAtStartPar
Company
&
\sphinxAtStartPar
Type is Company

\sphinxAtStartPar
Disposition is often commerce
\\
\hline
\sphinxAtStartPar
Consortium
&
\sphinxAtStartPar
Unknown type.  Assert Organization only.

\sphinxAtStartPar
Type is often nonprofit

\sphinxAtStartPar
Disposition is often association
\\
\hline
\sphinxAtStartPar
CoreLaboratory
&
\sphinxAtStartPar
Type is organization part

\sphinxAtStartPar
Dispositions are laboratory and service provider
\\
\hline
\sphinxAtStartPar
Department
&
\sphinxAtStartPar
Type is organizational part

\sphinxAtStartPar
Unknown dispositions
\\
\hline
\sphinxAtStartPar
Division
&
\sphinxAtStartPar
Type is organizational part

\sphinxAtStartPar
Unknown dispositions
\\
\hline
\sphinxAtStartPar
ExtensionUnit
&
\sphinxAtStartPar
Type is organizational part

\sphinxAtStartPar
Disposition is agricultural extension
\\
\hline
\sphinxAtStartPar
ERO\_00000565
&
\sphinxAtStartPar
Type is organizational part

\sphinxAtStartPar
Disposition is technology transfer
\\
\hline
\sphinxAtStartPar
Foundation
&
\sphinxAtStartPar
Type may be nonprofit

\sphinxAtStartPar
Type may be organizational part

\sphinxAtStartPar
May be affiliated with another organization

\sphinxAtStartPar
Disposition is philanthropy

\sphinxAtStartPar
Disposition may be funder
\\
\hline
\sphinxAtStartPar
FundingOrganization
&
\sphinxAtStartPar
Unknown type.  Assert organization only.

\sphinxAtStartPar
Disposition is funder
\\
\hline
\sphinxAtStartPar
GovernmentAgency
&
\sphinxAtStartPar
Type is government organization or organizational part

\sphinxAtStartPar
Disposition is unknown
\\
\hline
\sphinxAtStartPar
Hospital
&
\sphinxAtStartPar
Unknown type.  Assert organization only.

\sphinxAtStartPar
Disposition is hospital.
\\
\hline
\sphinxAtStartPar
Institute
&
\sphinxAtStartPar
Unknown type.  Assert organization only.

\sphinxAtStartPar
Disposition is unknown.  Often research.
\\
\hline
\sphinxAtStartPar
Laboratory
&
\sphinxAtStartPar
Unknown type.  Assert organization only.

\sphinxAtStartPar
Disposition is laboratory.
\\
\hline
\sphinxAtStartPar
Library
&
\sphinxAtStartPar
Unknown type.  Assert organization only.

\sphinxAtStartPar
Disposition is library.
\\
\hline
\sphinxAtStartPar
Museum
&
\sphinxAtStartPar
Unknown type.  Assert organization only.

\sphinxAtStartPar
Disposition is library.
\\
\hline
\sphinxAtStartPar
Program
&
\sphinxAtStartPar
Type is organizational part.

\sphinxAtStartPar
Disposition is unknown.
\\
\hline
\sphinxAtStartPar
Publisher
&
\sphinxAtStartPar
Type is unknown.  Assert organization only.

\sphinxAtStartPar
Often type is company.  But all others possible.

\sphinxAtStartPar
Disposition is publisher.
\\
\hline
\sphinxAtStartPar
ResearchOrganization
&
\sphinxAtStartPar
Unknown type.  Assert organization only.

\sphinxAtStartPar
Disposition is research.
\\
\hline
\sphinxAtStartPar
School
&
\sphinxAtStartPar
Type is unknown.  Assert organization only.

\sphinxAtStartPar
At US university, an organizational part.

\sphinxAtStartPar
Disposition is often education.
\\
\hline
\sphinxAtStartPar
ServiceProvidingLaboratory
&
\sphinxAtStartPar
Type is unknown.  Assert organization only.

\sphinxAtStartPar
Dispositions are laboratory and service provider.
\\
\hline
\sphinxAtStartPar
StudentOrganization
&
\sphinxAtStartPar
Type is organizational part

\sphinxAtStartPar
Disposition is unknown

\sphinxAtStartPar
Quality is student led %
\begin{footnote}[3]\sphinxAtStartFootnote
To be determined.
%
\end{footnote}.
\\
\hline
\sphinxAtStartPar
Team
&
\sphinxAtStartPar
Type is unknown.  Assert organization only.

\sphinxAtStartPar
Disposition is unknown %
\begin{footnote}[4]\sphinxAtStartFootnote
Team may mean “sports team” or “project team” or other.  A sports team may
be formal, such as Manchester United Football Club, or informal as in “my weekly
bowling team.”  A project team may be considered part of an organization, or
informally organized to move work forward.
%
\end{footnote}
\\
\hline
\sphinxAtStartPar
University
&
\sphinxAtStartPar
Type is unknown.  Assert organization only.

\sphinxAtStartPar
Disposition is university

\sphinxAtStartPar
Dispositions are typically education, research, service
\\
\hline
\end{tabulary}
\par
\sphinxattableend\end{savenotes}


\chapter{Translating from ROR to the Organization Ontology}
\label{\detokenize{ror-to-org:translating-from-ror-to-the-organization-ontology}}\label{\detokenize{ror-to-org::doc}}
\sphinxAtStartPar
\sphinxhref{http://ror.org}{Research Organization Registry (ROR)} provides data on over 95,000 research organizations in the world.  ROR data
is available CC0, curated, and via an open API.


\section{Translating Types}
\label{\detokenize{ror-to-org:translating-types}}
\sphinxAtStartPar
The ROR Organization types are listed in {\hyperref[\detokenize{ror-to-org:table-17}]{\sphinxcrossref{Table 17}}}  ROR types are high\sphinxhyphen{}level and can
be multi\sphinxhyphen{}valued, much as Organization Ontology dispositions are multi\sphinxhyphen{}valued.

\sphinxAtStartPar
Organizations without research disposition are out of scope for ROR.  All organizations in
ROR can be asserted to have research disposition.

\sphinxAtStartPar
{\hyperref[\detokenize{ror-to-org:table-17}]{\sphinxcrossref{Table 17}}} provides a guide for translating ROR organizational types to assertions
in the VIVO Organization Ontology.


\begin{savenotes}\sphinxattablestart
\centering
\sphinxcapstartof{table}
\sphinxthecaptionisattop
\sphinxcaption{Table 17 Translating ROR types to VIVO Organizational assertions}\label{\detokenize{ror-to-org:id1}}\label{\detokenize{ror-to-org:table-17}}
\sphinxaftertopcaption
\begin{tabulary}{\linewidth}[t]{|T|T|}
\hline
\sphinxstyletheadfamily 
\sphinxAtStartPar
ROR Type
&\sphinxstyletheadfamily 
\sphinxAtStartPar
VIVO Organization Ontology Assertions
\\
\hline
\sphinxAtStartPar
Education
&
\sphinxAtStartPar
Unknown type.  Assert Organization only.

\sphinxAtStartPar
Disposition is education, research
\\
\hline
\sphinxAtStartPar
Healthcare
&
\sphinxAtStartPar
Unknown type.  Assert organization only.

\sphinxAtStartPar
Disposition is healthcare, research.
\\
\hline
\sphinxAtStartPar
Company
&
\sphinxAtStartPar
Type is company.

\sphinxAtStartPar
Disposition is research.
\\
\hline
\sphinxAtStartPar
Archive
&
\sphinxAtStartPar
Type is unknown.  Assert Organization only.

\sphinxAtStartPar
Disposition is archive, research.
\\
\hline
\sphinxAtStartPar
Nonprofit
&
\sphinxAtStartPar
Type is nonprofit

\sphinxAtStartPar
Disposition is research.
\\
\hline
\sphinxAtStartPar
Government
&
\sphinxAtStartPar
Type is government organization

\sphinxAtStartPar
Disposition is research.
\\
\hline
\sphinxAtStartPar
Facility
&
\sphinxAtStartPar
Type is unknown.  Organization only.

\sphinxAtStartPar
Disposition is research.
\\
\hline
\sphinxAtStartPar
Other
&
\sphinxAtStartPar
Type is unknown.

\sphinxAtStartPar
Disposition is research.
\\
\hline
\end{tabulary}
\par
\sphinxattableend\end{savenotes}


\chapter{Translating from schema.org to the Organization Ontology}
\label{\detokenize{schema-to-org:translating-from-schema-org-to-the-organization-ontology}}\label{\detokenize{schema-to-org::doc}}
\sphinxAtStartPar
\sphinxhref{https:/schema.org}{schema.org} is an important folksonomy and JSON model
for representing common entities on the web.  The organization model of
schema.org has organization types and properties which can be represented using the
Organization Ontology.

\sphinxAtStartPar
Full interoperability with schema.org is not currently a goal of the Organization
Ontology work \sphinxstepexplicit %
\begin{footnote}[1]\phantomsection\label{\thesphinxscope.1}%
\sphinxAtStartFootnote
Full interoperability between schema.org and the Organization Ontology could be
future goal.  It appears that all the ontological structure is in place to add
additional
properties and entities from schema.org to the Organization Ontology.
%
\end{footnote}


\section{Translating Types}
\label{\detokenize{schema-to-org:translating-types}}
\sphinxAtStartPar
The schema.org organization types are listed in {\hyperref[\detokenize{schema-to-org:table-18}]{\sphinxcrossref{Table 18}}}  schema.org types are
high\sphinxhyphen{}level and can
be multi\sphinxhyphen{}valued, much as Organization Ontology dispositions are multi\sphinxhyphen{}valued.

\sphinxAtStartPar
{\hyperref[\detokenize{schema-to-org:table-18}]{\sphinxcrossref{Table 18}}} provides a guide for translating schema.org organizational types to assertions
in the VIVO Organization Ontology.


\begin{savenotes}\sphinxattablestart
\centering
\sphinxcapstartof{table}
\sphinxthecaptionisattop
\sphinxcaption{Table 18 Translating schema.org types to VIVO Organizational assertions}\label{\detokenize{schema-to-org:id7}}\label{\detokenize{schema-to-org:table-18}}
\sphinxaftertopcaption
\begin{tabulary}{\linewidth}[t]{|T|T|}
\hline
\sphinxstyletheadfamily 
\sphinxAtStartPar
schema.org Type
&\sphinxstyletheadfamily 
\sphinxAtStartPar
Organization Ontology Assertions
\\
\hline
\sphinxAtStartPar
Airline
&
\sphinxAtStartPar
Unknown type.  Typically company.

\sphinxAtStartPar
Disposition is airline.
\\
\hline
\sphinxAtStartPar
Consortium
&
\sphinxAtStartPar
Unknown type.  Assert Organization only.

\sphinxAtStartPar
Type is often nonprofit

\sphinxAtStartPar
Disposition is often association
\\
\hline
\sphinxAtStartPar
Corporation
&
\sphinxAtStartPar
Type is company.

\sphinxAtStartPar
Disposition is often commerce.
\\
\hline
\sphinxAtStartPar
EducationalOrganization
&
\sphinxAtStartPar
Type is unknown.  Assert Organization only.

\sphinxAtStartPar
Disposition is education.
\\
\hline
\sphinxAtStartPar
FundingScheme
&
\sphinxAtStartPar
Type is unknown.  Assert organization only. \sphinxstepexplicit %
\begin{footnote}[2]\phantomsection\label{\thesphinxscope.2}%
\sphinxAtStartFootnote
Unclear if a funding scheme is an organization.  It might be an informal
organization of those participating in the “scheme” or it may be an
organizational part of an organization with funding disposition.
%
\end{footnote}

\sphinxAtStartPar
Disposition is funding.
\\
\hline
\sphinxAtStartPar
GovernmentOrganization
&
\sphinxAtStartPar
Type is government organization

\sphinxAtStartPar
Disposition is unknown.
\\
\hline
\sphinxAtStartPar
LibrarySystem
&
\sphinxAtStartPar
Type is unknown.  Organization only.

\sphinxAtStartPar
Disposition is library.
\\
\hline
\sphinxAtStartPar
LocalBusiness
&
\sphinxAtStartPar
Type is company.

\sphinxAtStartPar
Disposition is often commerce.
\\
\hline
\sphinxAtStartPar
MedicalOrganization
&
\sphinxAtStartPar
Unknown type.  Assert Organization only.

\sphinxAtStartPar
Disposition is healthcare provider
\\
\hline
\sphinxAtStartPar
NGO
&
\sphinxAtStartPar
Type is nonprofit

\sphinxAtStartPar
Disposition is unknown.
\\
\hline
\sphinxAtStartPar
NewsMediaOrganization
&
\sphinxAtStartPar
Type is unknown.  Assert Organization only.

\sphinxAtStartPar
Disposition is media.
\\
\hline
\sphinxAtStartPar
PerformingGroup
&
\sphinxAtStartPar
Type is unknown.  Assert Organization only.

\sphinxAtStartPar
Disposition is performing
\\
\hline
\sphinxAtStartPar
Project
&
\sphinxAtStartPar
Type is unknown.  Perhaps informal. \sphinxstepexplicit %
\begin{footnote}[3]\phantomsection\label{\thesphinxscope.3}%
\sphinxAtStartFootnote
In Basic Formal Ontology (BFO) ontologies, the word “project” is used to describe a particular
type of process, that is, an occurent.  A project is not an organization.  A
project may “have” an
organization, an organization may conduct a project.  In english, when people
refer to a “project,” they may be referring to an organization that was created
for the purpose of executing a defined piece of work.  Such an organization
may be formal or informal, it may be an organizational part, or an organization
of its own.
%
\end{footnote}

\sphinxAtStartPar
Disposition is project.
\\
\hline
\sphinxAtStartPar
SportsOrganization
&
\sphinxAtStartPar
Type is unknown.  Assert organization only.

\sphinxAtStartPar
Disposition is sports.
\\
\hline
\sphinxAtStartPar
WorkersUnion
&
\sphinxAtStartPar
Type is unknown.  Often nonprofit.

\sphinxAtStartPar
Disposition is labor union.
\\
\hline
\end{tabulary}
\par
\sphinxattableend\end{savenotes}
\phantomsection\label{\detokenize{data-dates:datetimes}}
\index{Dates@\spxentry{Dates}}\index{Data@\spxentry{Data}}\index{Times@\spxentry{Times}}\index{Datetimes@\spxentry{Datetimes}}\ignorespaces 

\chapter{Dates Data}
\label{\detokenize{data-dates:dates-data}}\label{\detokenize{data-dates:index-0}}\label{\detokenize{data-dates::doc}}
\sphinxAtStartPar
The Organization Ontology includes a template (\sphinxcode{\sphinxupquote{templates\textbackslash{}dates.tsv}}) and
data (\sphinxcode{\sphinxupquote{templates\textbackslash{}dates.ttl}}) built from the template for the dates
from 1800\sphinxhyphen{}2050 in year precision, one individual per year.  Most organizations have been
established, or changed in this time
period, and year precision is often “good enough” for specifying these events.

\sphinxAtStartPar
The data have been created with standard URLs of the form

\begin{sphinxVerbatim}[commandchars=\\\{\}]
\PYG{n}{http}\PYG{p}{:}\PYG{o}{/}\PYG{o}{/}\PYG{n}{vivoweb}\PYG{o}{.}\PYG{n}{org}\PYG{o}{/}\PYG{n}{data}\PYG{o}{/}\PYG{n}{date}\PYG{o}{/}\PYG{n}{xxxx}
\end{sphinxVerbatim}

\sphinxAtStartPar
A sample date is given below:

\begin{sphinxVerbatim}[commandchars=\\\{\}]
\PYG{c+c1}{\PYGZsh{}\PYGZsh{}\PYGZsh{}  http://vivoweb.org/data/year/2021}
\PYG{o}{\PYGZlt{}}\PYG{n}{http}\PYG{p}{:}\PYG{o}{/}\PYG{o}{/}\PYG{n}{vivoweb}\PYG{o}{.}\PYG{n}{org}\PYG{o}{/}\PYG{n}{data}\PYG{o}{/}\PYG{n}{year}\PYG{o}{/}\PYG{l+m+mi}{2021}\PYG{o}{\PYGZgt{}}
    \PYG{n}{rdf}\PYG{p}{:}\PYG{n+nb}{type} \PYG{n}{owl}\PYG{p}{:}\PYG{n}{NamedIndividual} \PYG{p}{,}
                 \PYG{o}{\PYGZlt{}}\PYG{n}{http}\PYG{p}{:}\PYG{o}{/}\PYG{o}{/}\PYG{n}{www}\PYG{o}{.}\PYG{n}{w3}\PYG{o}{.}\PYG{n}{org}\PYG{o}{/}\PYG{l+m+mi}{2006}\PYG{o}{/}\PYG{n}{time}\PYG{c+c1}{\PYGZsh{}Instant\PYGZgt{} ;}
        \PYG{o}{\PYGZlt{}}\PYG{n}{http}\PYG{p}{:}\PYG{o}{/}\PYG{o}{/}\PYG{n}{www}\PYG{o}{.}\PYG{n}{w3}\PYG{o}{.}\PYG{n}{org}\PYG{o}{/}\PYG{l+m+mi}{2006}\PYG{o}{/}\PYG{n}{time}\PYG{c+c1}{\PYGZsh{}unitType\PYGZgt{} \PYGZlt{}http://www.w3.org/2006/time\PYGZsh{}unitYear\PYGZgt{} ;}
        \PYG{o}{\PYGZlt{}}\PYG{n}{http}\PYG{p}{:}\PYG{o}{/}\PYG{o}{/}\PYG{n}{www}\PYG{o}{.}\PYG{n}{w3}\PYG{o}{.}\PYG{n}{org}\PYG{o}{/}\PYG{l+m+mi}{2006}\PYG{o}{/}\PYG{n}{time}\PYG{c+c1}{\PYGZsh{}inXSDDateTimeStamp\PYGZgt{} \PYGZdq{}2021\PYGZhy{}01\PYGZhy{}01T00:00:00Z\PYGZdq{}\PYGZca{}\PYGZca{}xsd:dateTimeStamp .}
\end{sphinxVerbatim}

\sphinxAtStartPar
Including the file \sphinxcode{\sphinxupquote{data/dates.ttl}} in your graph should provide you with all the
dates in year precision from 1800\sphinxhyphen{}2050.  You can then use these dates in
assertions about years.  For example, to assert organization x was
established in 1853 (see {\hyperref[\detokenize{datetimes::doc}]{\sphinxcrossref{\DUrole{doc}{Dates and Time}}}}) you can say:

\begin{sphinxVerbatim}[commandchars=\\\{\}]
\PYG{n}{x} \PYG{n}{output\PYGZus{}of} \PYG{n}{y}
\PYG{n}{y} \PYG{n}{a} \PYG{n}{founding\PYGZus{}process}
\PYG{n}{y} \PYG{n}{has\PYGZus{}occurent\PYGZus{}part} \PYG{n}{z}
\PYG{n}{z} \PYG{n}{a} \PYG{n}{founding\PYGZus{}process\PYGZus{}boundary}
\PYG{n}{z} \PYG{n}{has\PYGZus{}instant} \PYG{o}{\PYGZlt{}}\PYG{n}{http}\PYG{p}{:}\PYG{o}{/}\PYG{o}{/}\PYG{n}{vivoweb}\PYG{o}{.}\PYG{n}{org}\PYG{o}{/}\PYG{n}{data}\PYG{o}{/}\PYG{n}{year}\PYG{o}{/}\PYG{l+m+mi}{1853}\PYG{o}{\PYGZgt{}}
\end{sphinxVerbatim}


\chapter{For Ontology Authors}
\label{\detokenize{ontology-authors:for-ontology-authors}}\label{\detokenize{ontology-authors::doc}}
\sphinxAtStartPar
The Organization Ontology has been developed using \sphinxhref{http://www.ontobee.org/ontology/BFO}{Basic Formal Ontology (BFO)} as an upper level ontology,
and in an attempt to follow the Open Biomedical Ontologies (OBO) Principles.  Many ontologies have been developed using
this approach.  Many of these ontologies can be found on \sphinxhref{http://ontobee.org}{Ontobee}


\section{Domain}
\label{\detokenize{ontology-authors:domain}}
\sphinxAtStartPar
In designing the Organization Ontology, we first conceive of the domain of organizations,
that is we develop a use case for the ontology.  This \sphinxstyleemphasis{domain definition} guides us
throughout design and implementation, indicating terms that should be included and those
that should be defined elsewhere.  From time to time, terms must be defined to
express important assertions regarding organizations, but have not been defined
elsewhere in a forma that can be used here.  Such terms have a curation status
indicating that we we would prefer if these terms are defined elsewhere.


\section{Reusing terms}
\label{\detokenize{ontology-authors:reusing-terms}}
\sphinxAtStartPar
We are generally cautious to reuse terms from other ontologies.  To reuse terms, we
require the ontology in which they are defined to:
\begin{enumerate}
\sphinxsetlistlabels{\arabic}{enumi}{enumii}{}{.}%
\item {} 
\sphinxAtStartPar
Use BFO has an upper level ontology.  We have made just a few exceptions and
in each case we have provided a superclass for the term in BFO to create a
consistent and complete subsumption hierarchy.

\item {} 
\sphinxAtStartPar
Conform to OBO principles. Again, we have made just a few exceptions.

\item {} 
\sphinxAtStartPar
An appropriate license for the ontology whose terms we will reuse.  When a license
for an ontology is not clear, we cannot use its terms.

\item {} 
\sphinxAtStartPar
Active maintenance.  Ontologies can be slow to add terms and to fix things that
need fixing.  If an ontology is not actively maintained, we cannot use its terms.

\item {} 
\sphinxAtStartPar
Use of MIREOT, a plug\sphinxhyphen{}in for protege.  When terms from other ontologies
are needed, we use protege to edit \sphinxtitleref{org\sphinxhyphen{}header.ttl} and add the the terms using
the MIREOT plug\sphinxhyphen{}in.  This provides a consistent means for adding terms.

\item {} 
\sphinxAtStartPar
We trim out annotation properties of included terms that are not of interest.

\end{enumerate}


\section{Use of templates}
\label{\detokenize{ontology-authors:use-of-templates}}
\sphinxAtStartPar
All terms defined in the Organization Ontology are created using templates.  There are
templates for classes, annotation properties, datatype properties, object properties,
and named individuals.  In each case, the columns correspond to annotations, class
expressions and other declarations used to create each term.


\section{Consistent build}
\label{\detokenize{ontology-authors:consistent-build}}
\sphinxAtStartPar
The ontology file \sphinxcode{\sphinxupquote{org.ttl}} is built using a simple script, \sphinxcode{\sphinxupquote{build.sh}}  The script
performs four operations:
\begin{enumerate}
\sphinxsetlistlabels{\arabic}{enumi}{enumii}{}{.}%
\item {} 
\sphinxAtStartPar
Makes data useful for ontology users.

\item {} 
\sphinxAtStartPar
Makes ontological assertions from the templates

\item {} 
\sphinxAtStartPar
Merges org\sphinxhyphen{}header and the template assertions into \sphinxcode{\sphinxupquote{org.ttl}}

\item {} 
\sphinxAtStartPar
Validates \sphinxcode{\sphinxupquote{org.ttl}}

\end{enumerate}


\section{Validation}
\label{\detokenize{ontology-authors:validation}}
\sphinxAtStartPar
The ontology is validated on each build using \sphinxstyleemphasis{robot validate}


\section{Documentation}
\label{\detokenize{ontology-authors:documentation}}
\sphinxAtStartPar
We document the ontology as it is written.  See \sphinxtitleref{For Documentation Authors
\textless{}documentation\sphinxhyphen{}authors\textgreater{}\_}  Documenting as we write the ontology helps with
consistency, accuracy, and completeness.


\chapter{For Documentation Authors}
\label{\detokenize{documentation-authors:for-documentation-authors}}\label{\detokenize{documentation-authors::doc}}
\sphinxAtStartPar
We are learning about Sphinx and ReadTheDocs, and considering their use
for creating documentation for the VIVO Ontology and related ontologies.

\sphinxAtStartPar
Documentation is produced using a hybrid of manual text production, manual
figure production, and automated table of contents, table, term page, and index
production.


\section{Manual Text Production}
\label{\detokenize{documentation-authors:manual-text-production}}
\sphinxAtStartPar
The top page is \sphinxcode{\sphinxupquote{index.rst}} which contains a preface, a table of contents, listy
of tables, list of figures, and reference to the index, which is automatically
generated by Sphinx.

\sphinxAtStartPar
Text is stored in pages with \sphinxcode{\sphinxupquote{.rst}} file tpes.  Pages are written using a text editor
such as BBEdit, vim, or Sublime.

\sphinxAtStartPar
Documentation in a \sphinxcode{\sphinxupquote{docs}} folder of the ontology GitHub repository, keeping ontology and
documentation together.


\section{Manual Figure Production}
\label{\detokenize{documentation-authors:manual-figure-production}}
\sphinxAtStartPar
We use \sphinxtitleref{draw.io \textless{}http://draw.io\textgreater{}}, also known as diagrams.net.  This free to use,
open source software is available for use through a browser or as a downloaded
app.


\section{Automated Text Production}
\label{\detokenize{documentation-authors:automated-text-production}}
\sphinxAtStartPar
One need is to provide search and index capability at the term level. A
user should be able to find the documentation for \sphinxtitleref{date} or \sphinxtitleref{person} or
\sphinxtitleref{disposition} without difficulty.

\sphinxAtStartPar
A simple python script \sphinxcode{\sphinxupquote{ontology\sphinxhyphen{}docs.py}} is included with the Organization
Ontology.  Given \sphinxstyleemphasis{any} ontology, the script can:
\begin{enumerate}
\sphinxsetlistlabels{\arabic}{enumi}{enumii}{}{.}%
\item {} 
\sphinxAtStartPar
Create pages for each term using python scripts \textendash{} scripts would use annotation
property values to automatically write pages of documentation from the ontology.

\item {} 
\sphinxAtStartPar
Create tables of terms using queries of the ontology.

\item {} 
\sphinxAtStartPar
Update lists of properties and classes as subsidiary tables of contents.

\end{enumerate}


\section{Automated GitHub pages}
\label{\detokenize{documentation-authors:automated-github-pages}}
\sphinxAtStartPar
The documentation is automatically built and deployed on GitHub Pages using GitHub
Actions.  The scripts to do this were written by Michael Altfield and
documented here: \sphinxhref{https://tech.michaelaltfield.net/2020/07/18/sphinx-rtd-github-pages-1/}{Continuous Documentation: Hosting Read the Docs on GitHub Pages
(1/2)}


\section{Resulting Documentation Features}
\label{\detokenize{documentation-authors:resulting-documentation-features}}\begin{enumerate}
\sphinxsetlistlabels{\arabic}{enumi}{enumii}{}{.}%
\item {} 
\sphinxAtStartPar
No need for formatting examples \textendash{} use “View Page Source” on any page to see how it
was written

\item {} 
\sphinxAtStartPar
No need to write about the tools.  Each tool has outstanding documentation.

\item {} 
\sphinxAtStartPar
See
\sphinxhref{https://docs.readthedocs.io/en/stable/intro/getting-started-with-sphinx.html}{Sphinx}
to get
started with the documentation.

\item {} 
\sphinxAtStartPar
Use \sphinxhref{https://docutils.sourceforge.io/docs/ref/rst/restructuredtext.html}{RestructuredText} to
write the documentation.  RestructuredText is a mark\sphinxhyphen{}up language originally developed
to document python.

\item {} 
\sphinxAtStartPar
Use GitHub for collaboration, issue tracking, version control, and release
management for
the documentation.  GitHub renders RestructuredText pages (pages with .rst
file types) automatically for those who wish to check oor read pages directly from
GitHub.

\item {} 
\sphinxAtStartPar
Use Makefiles included with \sphinxhref{https://docs.readthedocs.io/en/stable/index.html}{ReadTheDocs} for rendering
the documentation via HTML, PDF, or ePub.

\item {} 
\sphinxAtStartPar
Automated generation of HTML, PDF, and ePub documentation formats, and hosting of
the documentation in the Organization Ontology GitHub repository, using GitHub
actions.

\end{enumerate}


\chapter{Notes and Sources}
\label{\detokenize{notes-and-sources:notes-and-sources}}\label{\detokenize{notes-and-sources::doc}}
\sphinxAtStartPar
For early work on the Organization Ontology and thoughts behind what might be
needed and how things might be addressed, we relied on “Early Thoughts on
Representing Organizations in VIVO” by the VIVO Ontology Interest Group
\sphinxcite{notes-and-sources:voig2019a}.  While not everything there has been implemented here, and not
everything here is implemented as described there, the general outline of
representing organizations using \sphinxhref{http://www.ontobee.org/ontology/BFO}{Basic Formal Ontology (BFO)} according to \sphinxhref{http://www.obofoundry.org/principles/fp-000-summary.html}{Open Biomedical Ontologies (OBO) Principles} was first described
there. A good reference for BFO is \sphinxcite{notes-and-sources:arp2015}.  The \sphinxhref{https://wiki.lyrasis.org/display/VIVO/Ontology+Interest+Group}{VIVO Ontology Interest
Group} has been
considering BFO/OBO ontologies for scholarship and related domains for some
time. The first white paper \sphinxcite{notes-and-sources:voig2019} led to papers on subsumption \sphinxcite{notes-and-sources:voig2019b},
domains \sphinxcite{notes-and-sources:voig2019c}, and use of other ontologies \sphinxcite{notes-and-sources:voig2019d}.  Ideas from each of
these papers is reflected in the Organization Ontology.

\sphinxAtStartPar
We use \sphinxcite{notes-and-sources:wikipedia}, \sphinxcite{notes-and-sources:wiktionary}, and \sphinxcite{notes-and-sources:wikidata} often.  Term definitions,
references, fact\sphinxhyphen{}checking, and identifiers may come from these sources.

\sphinxAtStartPar
We use Ontobee \sphinxcite{notes-and-sources:ong2017} for looking up terms in OBO Foundry ontologies.

\sphinxAtStartPar
We use protege \sphinxcite{notes-and-sources:musen2015} for modeling \sphinxcode{\sphinxupquote{org\sphinxhyphen{}header.ttl}} and the MIREOT plug\sphinxhyphen{}in for
protege \sphinxcite{notes-and-sources:hannah2012} for
adding terms from other ontologies to \sphinxcode{\sphinxupquote{org\sphinxhyphen{}header.ttl}}  We use robot \sphinxcite{notes-and-sources:jackson2019} for
processing templates of properties, merging them and \sphinxcode{\sphinxupquote{org\sphinxhyphen{}header.ttl}} together to
produce \sphinxcode{\sphinxupquote{org.ttl}} and then to run reports against \sphinxcode{\sphinxupquote{org.ttl}} for validation.

\sphinxAtStartPar
We have tried to represent organizations in a manner that is inclusive of ideas
regarding organizations that have been represented elsewhere.  The VIVO
Ontology \sphinxcite{notes-and-sources:vivo2013}
provides organizational representation, but is not BFO or OBO conformant.  We
hope we have represented here what is represented in the VIVO Ontology.  The W3C
Organization Ontology \sphinxcite{notes-and-sources:reynolds2014} has been a second source for terms and
concepts that might be included in a BFO/OBO conformant ontology.

\sphinxAtStartPar
We have used the \sphinxcite{notes-and-sources:grid}, \sphinxcite{notes-and-sources:ror21}, and \sphinxcite{notes-and-sources:schema-org} data models as sources of
concepts and properties that may need to be represented in the Organization
Ontology.  See {\hyperref[\detokenize{vivo-to-org::doc}]{\sphinxcrossref{\DUrole{doc}{Translating from VIVO to the Organization Ontology}}}}, {\hyperref[\detokenize{ror-to-org::doc}]{\sphinxcrossref{\DUrole{doc}{Translating from ROR to the Organization Ontology}}}}, and {\hyperref[\detokenize{schema-to-org::doc}]{\sphinxcrossref{\DUrole{doc}{Translating from schema.org to the Organization Ontology}}}} for
details of how
types and other properties are mapped from these sources to the Organization Ontology.


\section{Regarding the W3C Organization Ontology}
\label{\detokenize{notes-and-sources:regarding-the-w3c-organization-ontology}}
\sphinxAtStartPar
The W3C Organization Ontology (W3CO) provides a set of useful terms for representing
organizations.  Many terms there are represented in this work.  Our work
uses BFO as an upper level ontology \textendash{} everything in the Organization Ontology
fits in the BFO subsumption hierarchy.  cross\sphinxhyphen{}walking the W3C Organization
Ontology and the VIVO Organization Ontology (VORG) is straightforward.  Below are
comments related to mapping.
\begin{itemize}
\item {} 
\sphinxAtStartPar
Purpose in W3CO is open\sphinxhyphen{}ended text.  In VORG, purpose is represented by dispositions

\item {} 
\sphinxAtStartPar
Classification in W3CO are interests in VORG.

\item {} 
\sphinxAtStartPar
Identifiers in VORG are handled using IDO

\item {} 
\sphinxAtStartPar
Linked to in W3CO is replaced by semantic object properties indicating the
relationship between
organizations

\item {} 
\sphinxAtStartPar
Formal Organization in W3CO is any organization that is not an Informal Organization
in VORG.

\item {} 
\sphinxAtStartPar
OrganizationUnit in W3CO is Organization Part in VORG.

\item {} 
\sphinxAtStartPar
Membership in VORG is modeled using standard BFO roles and occurent part representation

\item {} 
\sphinxAtStartPar
Posts in W3CO are modeled as positions in VORG in a manner analogous to memberships
(same conceptual model, different roles and entities)

\item {} 
\sphinxAtStartPar
Reports to in W3CO is deconstructed.  Personnel relationships are distinct from org
relationships in VORG.  Person to person relationships are out of scope for VORG.

\item {} 
\sphinxAtStartPar
Locations in VORG are modeled as BFO sites.  See \sphinxtitleref{Locations \textless{}locations\textgreater{}}

\item {} 
\sphinxAtStartPar
Addresses in VORG are modeled as IAO entities.  See \sphinxtitleref{Addresses \textless{}addresses\textgreater{}}

\item {} 
\sphinxAtStartPar
\sphinxstyleemphasis{based at} is a property of a person and is out of scope for VORG.

\item {} 
\sphinxAtStartPar
OrganizationCollaboration is a project and is modeled using standard BFO constructs.
Organizations have \sphinxstyleemphasis{participant in} projects

\item {} 
\sphinxAtStartPar
Change event is a BFO process boundary

\end{itemize}


\section{References}
\label{\detokenize{notes-and-sources:references}}

\chapter{Glossary}
\label{\detokenize{glossary:glossary}}\label{\detokenize{glossary:id1}}\label{\detokenize{glossary::doc}}\begin{description}
\item[{Basic Formal Ontology (BFO)\index{BFO@\spxentry{BFO}|spxpagem}\phantomsection\label{\detokenize{glossary:term-BFO}}}] \leavevmode
\sphinxAtStartPar
Basic Formal Ontology.  An upper level ontology used to represent things that exist.

\item[{CC0\index{CC0@\spxentry{CC0}|spxpagem}\phantomsection\label{\detokenize{glossary:term-CC0}}}] \leavevmode
\sphinxAtStartPar
\sphinxhref{https://creativecommons.org/share-your-work/public-domain/cc0/}{Creative Commons 0 license},
a license asserting no rights by anyone, that is “no rights reserved”.  CC0 material
can be freely used in any manner without attribution.

\item[{Domain\index{Domain@\spxentry{Domain}|spxpagem}\phantomsection\label{\detokenize{glossary:term-Domain}}}] \leavevmode
\sphinxAtStartPar
A part of the world consisting of related entities.

\item[{Dubbing Process\index{Dubbing Process@\spxentry{Dubbing Process}|spxpagem}\phantomsection\label{\detokenize{glossary:term-Dubbing-Process}}}] \leavevmode
\sphinxAtStartPar
A process by which an identifier is assigned to an entity.

\item[{Entity\index{Entity@\spxentry{Entity}|spxpagem}\phantomsection\label{\detokenize{glossary:term-Entity}}}] \leavevmode
\sphinxAtStartPar
A thing, as defined in an ontology.

\item[{Generically dependent continuant\index{Generically dependent continuant@\spxentry{Generically dependent continuant}|spxpagem}\phantomsection\label{\detokenize{glossary:term-Generically-dependent-continuant}}}] \leavevmode
\sphinxAtStartPar
In BFO, an entity whose existence depends generically on the existence of other
entities.  Examples include information artifacts (which depend on representations,
and “memory”) and organizatins which depend on the people and purpose which define
the organization.

\item[{IAO\index{IAO@\spxentry{IAO}|spxpagem}\phantomsection\label{\detokenize{glossary:term-IAO}}}] \leavevmode
\sphinxAtStartPar
Information Artifacts Ontology.  A BFO\sphinxhyphen{}based, OBO\sphinxhyphen{}compliant ontology for
representing information artifacts

\item[{Information artifacts\index{Information artifacts@\spxentry{Information artifacts}|spxpagem}\phantomsection\label{\detokenize{glossary:term-Information-artifacts}}}] \leavevmode
\sphinxAtStartPar
Things that contain or represent information.  Examples include documents, software,
databases, data elements, and photographs.

\item[{OBO\index{OBO@\spxentry{OBO}|spxpagem}\phantomsection\label{\detokenize{glossary:term-OBO}}}] \leavevmode
\sphinxAtStartPar
Open Biomedical Ontologies.  A collection of ontologies, and a set of principles
for developing ontologies that fit together.

\item[{OWL\index{OWL@\spxentry{OWL}|spxpagem}\phantomsection\label{\detokenize{glossary:term-OWL}}}] \leavevmode
\sphinxAtStartPar
Web Ontology Language.  A W3C standard for representing ontologies.

\item[{Ontology\index{Ontology@\spxentry{Ontology}|spxpagem}\phantomsection\label{\detokenize{glossary:term-Ontology}}}] \leavevmode
\sphinxAtStartPar
A precise exposition declaring entities, their properties and relationships.

\item[{RO\index{RO@\spxentry{RO}|spxpagem}\phantomsection\label{\detokenize{glossary:term-RO}}}] \leavevmode
\sphinxAtStartPar
\sphinxhref{http://www.ontobee.org/ontology/RO}{Relation Ontology (RO)} is used with Basic Formal Ontology (BFO)
to represent relations between entities.  Object properties are often sub\sphinxhyphen{}properties
of properties in RO.

\item[{ROR\index{ROR@\spxentry{ROR}|spxpagem}\phantomsection\label{\detokenize{glossary:term-ROR}}}] \leavevmode
\sphinxAtStartPar
\sphinxhref{http://ror.org}{Research Organization Registry}.  An open (CC0), curated,
collection of facts about the
research organizations of the world.

\item[{Term\index{Term@\spxentry{Term}|spxpagem}\phantomsection\label{\detokenize{glossary:term-Term}}}] \leavevmode
\sphinxAtStartPar
The fundamental entry in an ontology. A term may be a class, an annotation property,
an object property, or a datatype property.

\item[{VIVO\index{VIVO@\spxentry{VIVO}|spxpagem}\phantomsection\label{\detokenize{glossary:term-VIVO}}}] \leavevmode
\sphinxAtStartPar
Software, ontologies, and community for representing scholarship.

\item[{VIVO 1\index{VIVO 1@\spxentry{VIVO 1}|spxpagem}\phantomsection\label{\detokenize{glossary:term-VIVO-1}}}] \leavevmode
\sphinxAtStartPar
The \sphinxhref{https://github.com/vivo-ontologies/vivo-ontology}{VIVO Ontology} as
implemented in VIVO beginning with VIVO version 1.6.

\end{description}


\chapter{List of Tables}
\label{\detokenize{index:list-of-tables}}\begin{itemize}
\item {} 
\sphinxAtStartPar
{\hyperref[\detokenize{organizations:table-1}]{\sphinxcrossref{\DUrole{std,std-ref}{Table 1 Types of Organizations}}}}

\item {} 
\sphinxAtStartPar
{\hyperref[\detokenize{organizations:table-2}]{\sphinxcrossref{\DUrole{std,std-ref}{Table 2 Dispositions}}}}

\item {} 
\sphinxAtStartPar
{\hyperref[\detokenize{organizations:table-3}]{\sphinxcrossref{\DUrole{std,std-ref}{Table 3 Qualities}}}}

\item {} 
\sphinxAtStartPar
{\hyperref[\detokenize{identifiers:table-4}]{\sphinxcrossref{\DUrole{std,std-ref}{Table 4 Identifiers}}}}

\item {} 
\sphinxAtStartPar
{\hyperref[\detokenize{classes:table-5}]{\sphinxcrossref{\DUrole{std,std-ref}{Table 5 Classes}}}}

\item {} 
\sphinxAtStartPar
{\hyperref[\detokenize{annotation-properties:table-6}]{\sphinxcrossref{\DUrole{std,std-ref}{Table 6 Annotation Properties}}}}

\item {} 
\sphinxAtStartPar
{\hyperref[\detokenize{object-properties:table-7}]{\sphinxcrossref{\DUrole{std,std-ref}{Table 7 Object Properties}}}}

\item {} 
\sphinxAtStartPar
{\hyperref[\detokenize{datatype-properties:table-8}]{\sphinxcrossref{\DUrole{std,std-ref}{Table 8 Datatype Properties}}}}

\item {} 
\sphinxAtStartPar
{\hyperref[\detokenize{named-individuals:table-9}]{\sphinxcrossref{\DUrole{std,std-ref}{Table 9 Named Individuals}}}}

\item {} 
\sphinxAtStartPar
{\hyperref[\detokenize{addresses:table-10}]{\sphinxcrossref{\DUrole{std,std-ref}{Table 10 Terms used to represent addresses}}}}

\item {} 
\sphinxAtStartPar
{\hyperref[\detokenize{annotation-properties:table-11}]{\sphinxcrossref{\DUrole{std,std-ref}{Table 11 Common Annotation Properties}}}}

\item {} 
\sphinxAtStartPar
{\hyperref[\detokenize{annotation-properties:table-12}]{\sphinxcrossref{\DUrole{std,std-ref}{Table 12 Curation Status}}}}

\item {} 
\sphinxAtStartPar
{\hyperref[\detokenize{datetimes:table-13}]{\sphinxcrossref{\DUrole{std,std-ref}{Table 13 Terms used to represent dates and times}}}}

\item {} 
\sphinxAtStartPar
{\hyperref[\detokenize{locations:table-14}]{\sphinxcrossref{\DUrole{std,std-ref}{Table 14 Terms used to represent locations}}}}

\item {} 
\sphinxAtStartPar
{\hyperref[\detokenize{object-properties:table-15}]{\sphinxcrossref{\DUrole{std,std-ref}{Table 15 Common Object Properties}}}}

\item {} 
\sphinxAtStartPar
{\hyperref[\detokenize{vivo-to-org:table-16}]{\sphinxcrossref{\DUrole{std,std-ref}{Table 16 Translating VIVO types to Organizational Ontology assertions}}}}

\item {} 
\sphinxAtStartPar
{\hyperref[\detokenize{ror-to-org:table-17}]{\sphinxcrossref{\DUrole{std,std-ref}{Table 17 Translating ROR types to VIVO Organizational assertions}}}}

\item {} 
\sphinxAtStartPar
{\hyperref[\detokenize{schema-to-org:table-18}]{\sphinxcrossref{\DUrole{std,std-ref}{Table 18 Translating schema.org types to VIVO Organizational assertions}}}}

\end{itemize}


\chapter{List of Figures}
\label{\detokenize{index:list-of-figures}}\begin{itemize}
\item {} 
\sphinxAtStartPar
{\hyperref[\detokenize{organizations:figure-1}]{\sphinxcrossref{\DUrole{std,std-ref}{Figure 1 Representation of organizations}}}}

\item {} 
\sphinxAtStartPar
{\hyperref[\detokenize{organizations:figure-2}]{\sphinxcrossref{\DUrole{std,std-ref}{Figure 2 Types of Organizations}}}}

\item {} 
\sphinxAtStartPar
{\hyperref[\detokenize{datetimes:figure-3}]{\sphinxcrossref{\DUrole{std,std-ref}{Figure 3 Representation of dates and times}}}}

\item {} 
\sphinxAtStartPar
{\hyperref[\detokenize{associations:figure-4}]{\sphinxcrossref{\DUrole{std,std-ref}{Figure 4 Representation of memberships}}}}

\item {} 
\sphinxAtStartPar
{\hyperref[\detokenize{associations:figure-5}]{\sphinxcrossref{\DUrole{std,std-ref}{Figure 5 Representation of employment}}}}

\end{itemize}
\begin{itemize}
\item {} 
\sphinxAtStartPar
\DUrole{xref,std,std-ref}{genindex}

\end{itemize}

\begin{sphinxthebibliography}{Reynolds}
\bibitem[Harmse2018]{object-properties:harmse2018}
\sphinxAtStartPar
Harmse, Henrietta, A Common Misconception regarding OWL Properties,
blog post,
\sphinxurl{https://henrietteharmse.com/2018/06/22/a-common-misconception-regarding-owl-properties/}
\bibitem[VOIG2019a]{notes-and-sources:voig2019a}
\sphinxAtStartPar
VIVO Ontology Interest Group (2021) Early Thoughts on the representation
of organizations in VIVO.  on\sphinxhyphen{}line.  \sphinxurl{http://bit.ly/2EhMdPq}
\bibitem[Arp2015]{notes-and-sources:arp2015}
\sphinxAtStartPar
Arp, Smith, and Spear (2015) Building Ontologies with Basic Formal Ontology.
MIT Press. ISBN 978\sphinxhyphen{}0262527811.  248 pages.
\bibitem[VOIG2019]{notes-and-sources:voig2019}
\sphinxAtStartPar
VIVO Ontology Interest Group (January 2019) VIVO Ontology Version 2.
on\sphinxhyphen{}line.
\sphinxurl{http://bit.ly/2R8gYuI}
\bibitem[VOIG2019b]{notes-and-sources:voig2019b}
\sphinxAtStartPar
VIVO Ontology Interest Group (February 2019) Early Thoughts on VIVO
Subsumption Hierarchy.  on\sphinxhyphen{}line.  \sphinxurl{http://bit.ly/2Ekg7m6}
\bibitem[VOIG2019c]{notes-and-sources:voig2019c}
\sphinxAtStartPar
VIVO Ontology Interest Group (February 2019) Early Thoughts on VIVO
Related Domains.  on\sphinxhyphen{}line.  \sphinxurl{http://bit.ly/2Jn3MTV}
\bibitem[VOIG2019d]{notes-and-sources:voig2019d}
\sphinxAtStartPar
VIVO Ontology Interest Group (March 2019) Early Thoughts on Ontologies
Used in VIVO.  on\sphinxhyphen{}line.  \sphinxurl{https://bit.ly/3fWdL0K}
\bibitem[Wikipedia]{notes-and-sources:wikipedia}
\sphinxAtStartPar
Wikipedia (2021) we site.  \sphinxurl{http://wikipedia.org}
\bibitem[Wiktionary]{notes-and-sources:wiktionary}
\sphinxAtStartPar
Wiktionary (2021) web site. \sphinxurl{http://wiktionary.org}
\bibitem[Wikidata]{notes-and-sources:wikidata}
\sphinxAtStartPar
Wikidata (2021) web site. \sphinxurl{http://wikidata.org}
\bibitem[Ong2017]{notes-and-sources:ong2017}
\sphinxAtStartPar
Ong E, Xiang Z, Zhao B, Liu Y, Lin Y, Zheng J, Mungall C, Courtot M,
Ruttenberg A, He Y. Ontobee: A linked ontology data server to support ontology term
dereferencing, linkage, query, and integration. Nucleic Acid Research. 2017
Jan 4;45(D1):D347\sphinxhyphen{}D352. PMID: 27733503. PMCID: PMC5210626.
\bibitem[Musen2015]{notes-and-sources:musen2015}
\sphinxAtStartPar
Musen, M.A. The Protégé project: A look back and a look forward. AI
Matters.
Association of Computing Machinery Specific Interest Group in Artificial Intelligence,
1(4), June 2015. \sphinxurl{https://doi.org/10.1145/2757001.2757003}.
\bibitem[Hannah2012]{notes-and-sources:hannah2012}
\sphinxAtStartPar
Hannah, J, Chen, C, Crow, WA, et al. (2012) Simplifying MIREOT: A
MIREOT Protege Plugin. International Semantic Web Conference,
\sphinxurl{http://ceur-ws.org/Vol-914/paper\_48.pdf}
\bibitem[Jackson2019]{notes-and-sources:jackson2019}
\sphinxAtStartPar
Jackson, R.C., Balhoff, J.P., Douglass, E. et al. ROBOT: A Tool
for Automating Ontology Workflows. BMC Bioinformatics 20, 407 (2019).
\sphinxurl{https://doi.org/10.1186/s12859-019-3002-3}
\bibitem[vivo2013]{notes-and-sources:vivo2013}
\sphinxAtStartPar
Conlon, M. Mitchell, S. (2018) VIVO Ontology for Researcher Discovery.
Ontology.
\sphinxurl{https://bioportal.bioontology.org/ontologies/VIVO}
\bibitem[Reynolds2014]{notes-and-sources:reynolds2014}
\sphinxAtStartPar
Reynolds, Dave (ed) (2014) The Organization Ontology.
Ontology.  \sphinxurl{https://www.w3.org/TR/vocab-org/}
\bibitem[GRID]{notes-and-sources:grid}
\sphinxAtStartPar
Digital Science, (2021) GRID Global Research Identifier Database.
\sphinxurl{https://grid.ac}
\bibitem[ROR21]{notes-and-sources:ror21}
\sphinxAtStartPar
ROR Community (2021) Research Organization Registry. Database.
\sphinxurl{https://ror.org}
\bibitem[schema.org]{notes-and-sources:schema-org}
\sphinxAtStartPar
W3C Schema.org Community Group (2021) Schema.org. Website.
\sphinxurl{https://schema.org}
\end{sphinxthebibliography}



\renewcommand{\indexname}{Index}
\printindex
\end{document}